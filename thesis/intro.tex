\chapter{Introduction}

Most fluid systems in nature contain more than one species of particles, and it is therefore important to understand the behavior and dynamics of these kinds of particle systems. They are described by the Navier-Stokes equations commonly used for single-component fluids, but with moving boundary conditions. This would be hard to solve explicitly and it would furthermore become problematic to analyse the properties of the system.

%Instead different models have been proposed and .... by e.g. Maxey and Riley (1983). 

In order to analyse the dynamics of inertial particles the equation of motion should be be expressed as an ordinary differential because then the tools of dynamical systems theory are accessible. The Maxey-Riley equation is one of those equations. 


This thesis concerns the investigation of small finite-size particles in a fluid, where the density of the 
particles differs from that of the fluid. 


The forming of rain droplets in clouds is not fully understood, and more sophisticated models are needed in order to take the great size of the rain droplets into account. 

\section{Outline}
