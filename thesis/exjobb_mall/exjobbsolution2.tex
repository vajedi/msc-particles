%Till John:\\
We will here try to solve the previously derived equations of motion for some particular cases.
We begin with the 8-dimensional D2-membrane, because we can compare the result to \cite{pioline}. 


\section{The $d=8$ $D2$, const $\alpha\mbox{, }\beta$ case}
Here we will let $\alpha$ and $\beta$ be independent of $\omega$ and $f$ in the equations of motion \Eqnref{solution_8d_eom}
The second relation tells us that the 2 scalars $p_r = \lambda \W_{rs}*h^s$ are constants. These (......... will be identified with charges later on in (),,, or has been identified with charges earlier in () .........) 
Together with the first equation of motion we get the following relations
\begin{align}
*h^r &= \frac{1}{\lambda}\W^{rs}p_s = \frac{1}{\lambda}p^r\nn\\
\frac{1}{\lambda} &= \frac{|*h|}{|p|}, \hspace{1cm}\mbox{where }|*h|=\sqrt{*h^r\W_{rs}*h^s}\mbox{, }|p| = {\sqrt{p_mp^m}}\nn\\
|*h| &= \sqrt{1+\Phi}\eqnlab{solution_lambda_identities}
\end{align}
which allows us to rewrite all $\lambda$ and $*h$ dependence in terms of $\Phi$ and $p$ in the third and fourth equations.
The equations of motion for $a^m$ becomes  
\begin{align}
d& \Big{[}\frac{|p|}{\sqrt{1+\Phi}}*k^m + 2\alpha \omega^{rm} p_r \Big{]} - {4 \over 3} F^{rm}p_r\nn\\
&= d\left[\frac{|p|}{\sqrt{1+\Phi}}*k^m + 2\lp\alpha+\frac{2}{3}\rp\omega^{rm}p_r\right] = 0
\end{align}
where we have used the Bianchi identities \eqnref{}(........ ??? .........) to move in all terms under the exterior derivative. 
Integrate to get
\begin{align}
*k^m = \bigg[\gamma^m -2\lp\alpha+\frac{2}{3}\rp p_r\omega^{rm}\bigg]\frac{\sqrt{1+\Phi}}{|p|}
\eqnlab{solution_8d_km_orig}
\end{align}
where $\gamma^m$ is a 1-form such that $d\gamma^m = 0$.
The equation of motion for $\phi^{rm}$ becomes 
\begin{align}
d &\Big{[}\frac{|p|}{\sqrt{1+\Phi}}*j_{rm} +2\alpha f_m p_r + 2\beta \epsilon_{mnp}(\epsilon_{sr}\W_{ut} + 2\epsilon_{st}\W_{ur})\omega^{un} \wedge \omega^{tp}p^{s}\Big{]}\nn\\
& + {\lambda \over 2}\epsilon_{mnp}\epsilon_{rs} *k^n \wedge F^{sp} +\frac{2}{3} H_m p_r + \alpha \epsilon_{mnp}\epsilon_{rs} p_{t} F^{sp} \wedge \omega^{tn}\nn\\
& = d \Big{[}\frac{|p|}{\sqrt{1+\Phi}}*j_{rm} +2\alpha f_m p_r + 2\beta \epsilon_{mnp}(\epsilon_{sr}\W_{ut} + 2\epsilon_{st}\W_{ur})\omega^{un} \wedge \omega^{tp}p^{s}\Big{]}\nn\\
& + \epsilon_{mnp}\epsilon_{rs} \Big[\frac{\gamma^n}{2} -\frac{2}{3} p_t\omega^{tn}\Big] \wedge F^{sp} +\frac{2}{3}\lp -df_m + \frac{1}{2}\epsilon_{mnp}\epsilon_{ts}\omega^{sp}\wedge F^{tn}\rp p_r \nn\\
& = d \Big{[}\frac{|p|}{\sqrt{1+\Phi}}*j_{rm} +2\lp\alpha-\frac{1}{3}\rp f_m p_r - \half\epsilon_{mnp}\epsilon_{rs}\gamma^n\wedge \omega^{sp}\nn\\
& + \epsilon_{mnp}\omega^{sn} \wedge \omega^{tp}\Big\{ \lp -2\beta + \frac{1}{3}\rp \epsilon_{rs}p_t + 6\beta\epsilon_{ut}p^{u}\W_{sr}\Big\}\Big{]} = 0
\end{align}
where we have used \Eqnref{solution_8d_km_orig} for $*k^m$, the Bianchi identities \Eqnref{}(........ ??? .........) and $\Eqnref{conven_2d_epsilon_rel}$ to move $SL(2,\rr)$-indices.
Integration and hodge dualisation gives
\begin{align}
j_{rm} & = \bigg\{ 
2\lp\alpha-\frac{1}{3}\rp *f_m p_r - \half\epsilon_{mnp}\epsilon_{rs}*\lp\gamma^n\wedge \omega^{sp}\rp + *\delta_{rm} \\
\eqnlab{solution_8d_jrm_orig}
& + \epsilon_{mnp}*\lp\omega^{sn} \wedge \omega^{tp}\rp\bigg[ \lp -2\beta + \frac{1}{3}\rp \epsilon_{rs}p_t + 6\beta\epsilon_{ut}p^{u}\W_{sr}\bigg]\nn
\bigg\}\frac{\sqrt{1+\Phi}}{|p|}
\end{align}
where $\delta_{rm}$ is a 2-form such that $d\delta_{rm} = 0$.
It is hard to find meaningful relations to make the integration forms $\gamma$ and $\delta$ useful.
One way would be to put restrictions on the background fields, e.g. letting $F^{rm} = 0$ (we could make it with the weaker assumption $p_rF^{rm} = 0$, but that would not lead to a U-duality invariant formulation, c.f. \cite{artikeln}).
In such a background we could solve the equations without introducing the parameter functions $\alpha$ and $\beta$ at all. 
If we let $\alpha = \beta = \delta_{rm} = 0$ and $\gamma = \tilde\gamma p_r\omega^{rm}$ we get the integrated equations of motion
\begin{align}
*k^m &= \bigg[\lp\tilde\gamma - \frac{4}{3}\rp p_r\omega^{rm}\bigg]\frac{\sqrt{1+\Phi}}{|p|}\\
j_{rm} & = \bigg\{ 
-\frac{2}{3}*f_m p_r + \epsilon_{mnp}*\lp\omega^{sn} \wedge \omega^{tp}\rp\bigg[ \lp \frac{1}{3} - \frac{\tilde\gamma}{2}\rp \epsilon_{rs}p_t\bigg]
\bigg\}\frac{\sqrt{1+\Phi}}{|p|}.\nn
\end{align}
If we let $\tilde\gamma = 2/3$ and make the variable substitutions 
\begin{align}
v^m & =\frac{p_r}{|p|}\omega^{rm} = \hat p_r\omega^{rm} = \hat p\cdot\omega^{m}\nn\\
u^m & =*f^m 
\eqnlab{solution_8d_vartrans_easy}
\end{align}
such that
\begin{align}
*k^m &= *\frac{\partial \Phi}{\partial f_m} = *\lp\frac{\partial \Phi}{\partial u_{\alpha n}}\frac{\partial u_{\alpha n}}{\partial f_m}\rp = \half d\xi^\delta\epsilon_{\delta\gamma\beta}\lp \frac{\partial \Phi}{\partial u_m^{\alpha}} \frac{1}{2}2\epsilon^{\alpha\beta\gamma}\rp = -\frac{\partial \Phi}{\partial u_m}\nn\\
j_{rm} &= \frac{\partial \Phi}{\partial \omega^{rm}} = \frac{\partial \Phi}{\partial v^n_\alpha}\frac{\partial v^n_\alpha}{\partial \omega^{rm}} = \frac{\partial \Phi}{\partial v^m}\hat p_r
\end{align}
we get
\begin{align}
\frac{\partial\Phi}{\partial u_m} &= \frac{2}{3} v^{m}\sqrt{1+\Phi}\nn\\
\frac{\partial\Phi}{\partial v^m}\hat p_r & = -\frac{2}{3}u_m \sqrt{1+\Phi}\hat p_r
\end{align}
which are the same equations (up to a variable reascaling) obtained in \cite{artikeln} with $\alpha = -1/6$, $\beta = 1/6$ and $F=0$ so that $\hat p$ and $\omega$ are aligned.
Note that we derived these equations without introducing $\alpha$ and $\beta$ at all. Introducing $\alpha$ now would just mean changing the multiplying factor $2/3$ to $(-2\alpha+2/3)$ in the first equation and $-2/3$ to $(-2\alpha-2/3)$ in the second equation and introducing $\beta$ would add such an $\omega\we\omega$ term we removed with our choice of $\tilde\gamma$.  
So, we have seen that we are, with the proper constraint on the background fields, not forced to introduce the parameter functions $\alpha$ and $\beta$. 
We will try to solve these equations later on in \Secref{csolution_8d_const_old} and instead continue by solving the equations for arbitrary backgrounds. 

% in a similar manner with the background field constraint $F$ arbitrary and $H=0$. 
% This is easily done by making the same variable transformation as before and letting the integration forms $\gamma = 0$ and $\delta_{rm} = f_m$






%Series expand $\Phi$ to lowest order (there should be no constant term in $\Phi$ since (......... Eftersom dualitetsrelationen skall vara med en etta??? .........))
%\begin{equation}
%\Phi = a_1 f_{\beta\alpha m}f^{\beta\alpha m} + a_2\omega_{\alpha rm}\omega^{\alpha rm} = a_1 f^2 + a_2\omega^2
%\end{equation}
%giving in the relations for $k^m$ and $j_{rm}$ to order 1 in the 1-form fields
%\begin{align}
%2a_1*f^m &= \bigg[\gamma^m -2\lp\alpha+\frac{2}{3}\rp p_r\omega^{rm}\bigg]\frac{1}{|p|}\nn\\
%2a_2\omega_{rm} & = \bigg\{ 
%2\lp\frac{1}{3}-\alpha\rp *f_m p_r - \half\epsilon_{mnp}\epsilon_{rs}*\lp\gamma^n\wedge \omega^{sp}\rp + *\delta_{rm} \nn\\
%& + \epsilon_{mnp}*\lp\omega^{sn} \wedge \omega^{tp}\rp\bigg[ \lp -2\beta + \frac{1}{3}\rp \epsilon_{rs}p_t + 6\beta\epsilon_{ut}p^{u}\W_{sr}\bigg]
%\bigg\}\frac{1}{|p|}
%\end{align}

%By inspection we see that these equations are fulfilled for the following cases of parameter combinations (in some places we put the current appearance of the expanded duality relations)
%\actr % Set arabic counter
%\begin{enumerate}
%% The second relation with all terms
%% 2a_2\omega_{rm}|p| & = 2\lp 1/3-\alpha\rp *f_m p_r - 1/2\epsilon_{mnp}\epsilon_{rs}*\lp\gamma^n\wedge \omega^{sp}\rp + *\delta_{rm} \nn\\
%% & + \epsilon_{mnp}*\lp\omega^{sn} \wedge \omega^{tp}\rp\big[ \lp -2\beta + 1/3\rp \epsilon_{rs}p_t + 6\beta\epsilon_{ut}p^{u}\W_{sr}\big]
%  \item $a_1=0$
%  \begin{enumerate}
%    \item $\gamma^m=0$, $\alpha=-2/3$
%        \begin{align}
%    2a_2\omega_{rm}|p| & = 2*f_m p_r + *\delta_{rm} \nn\\
%    & + \epsilon_{mnp}*\lp\omega^{sn} \wedge \omega^{tp}\rp\big[ \lp -2\beta + 1/3\rp \epsilon_{rs}p_t + 6\beta\epsilon_{ut}p^{u}\W_{sr}\big]
%    \end{align}
%        \begin{enumerate}
%          \item $a_2=0$
%      \begin{align}
%          0 & = 2*f_m p_r + *\delta_{rm} \nn\\
%      & + \epsilon_{mnp}*\lp\omega^{sn} \wedge \omega^{tp}\rp\big[ \lp -2\beta + 1/3\rp \epsilon_{rs}p_t + 6\beta\epsilon_{ut}p^{u}\W_{sr}\big]
%      \end{align}
%        \end{enumerate}
%        
%    \item $\gamma^m=\lp 2\alpha + 4/3\rp p_r\omega^{rm}$
%    \begin{align}
%    2a_2\omega_{rm}|p| & = 2\lp 1/3-\alpha\rp *f_m p_r + *\delta_{rm} \nn\\
%    & + \epsilon_{mnp}*\lp\omega^{sn} \wedge \omega^{tp}\rp\big[ -\lp 2\beta + \alpha + 1/3\rp \epsilon_{rs}p_t + 6\beta\epsilon_{ut}p^{u}\W_{sr}\big]
%    \end{align}
%        \begin{enumerate}
%          \item $a_2=0$
%      \begin{align}
%      0 & = 2\lp 1/3-\alpha\rp *f_m p_r + *\delta_{rm} \nn\\
%      & + \epsilon_{mnp}*\lp\omega^{sn} \wedge \omega^{tp}\rp\big[ -\lp 2\beta + \alpha + 1/3\rp \epsilon_{rs}p_t + 6\beta\epsilon_{ut}p^{u}\W_{sr}\big]
%      \end{align}
%      \begin{enumerate}
%            \item $\alpha=1/3$
%        \begin{align}
%        0 & = *\delta_{rm} \nn\\
%        & + \epsilon_{mnp}*\lp\omega^{sn} \wedge \omega^{tp}\rp\big[ -\lp 2\beta + \alpha + 1/3\rp \epsilon_{rs}p_t + 6\beta\epsilon_{ut}p^{u}\W_{sr}\big]
%        \end{align}

%      \end{enumerate}
%        \end{enumerate}
%  \end{enumerate}

%  \item 
%  \begin{enumerate}
%    \item 
%  \end{enumerate}
%\end{enumerate}

%Serieexpansion f;r det riktiga:\\
%$\gamma = \delta = 0$\\
%$u_m = b*f_m + c_{(st)}\epsilon_{mnp}*(\omega^{sn}\we\omega^{tp}) = v_m + ... = p^r\omega_{rm} + ...$\\
%i.e. $*f_m = \frac{1}{b}p^r\omega_{rm} - \frac{1}{b}c_{(st)}\epsilon_{mnp}*(\omega^{sn}\we\omega^{tp}) $\\

%\begin{align}
%&2a_1\lp \frac{1}{b}p^r\omega_{rm} - \frac{1}{b}c_{(st)}\epsilon_{mnp}*(\omega^{sn}\we\omega^{tp}) \rp = -2\lp\alpha+\frac{2}{3}\rp p^r\omega_{rm}\frac{1}{|p|}\nn\\
%&2a_2\omega_{rm} = \bigg\{ 
%2\lp\frac{1}{3}-\alpha\rp \lp \frac{1}{b}p^r\omega_{rm} - \frac{1}{b}c_{(st)}\epsilon_{mnp}*(\omega^{sn}\we\omega^{tp}) \rp p_r \nn\\
%& + \epsilon_{mnp}*\lp\omega^{sn} \wedge \omega^{tp}\rp\bigg[ \lp -2\beta + \frac{1}{3}\rp \epsilon_{rs}p_t + 6\beta\epsilon_{ut}p^{u}\W_{sr}\bigg]
%\bigg\}\frac{1}{|p|}
%\end{align}


From now on we put $\gamma$ and $\delta$ to $0$.
The first thing we note about the equations \eqnref{solution_8d_km_orig} and \eqnref{solution_8d_jrm_orig} is that if multiply the latter equation with one of the two charge vectors $\hat p_\parallel = p_r/|p|$ and $\hat p_\perp = p^r\epsilon_{rs}/|p|$ each of the $\omega$:s will be contracted with one of these vectors on the forms $\omega_\parallel^m = \hat p_\parallel\cdot\omega^m = p_r/|p|\omega^{rm}$, which is the projection of $\omega^m$ in the $\hat p$ direction and $\omega_\perp^m = \hat p_\perp\cdot\omega^m = p^r\epsilon_{rs}/|p|\omega^{sm}$, which is the projection of $\omega$ in the direction orthogonal to $\hat p$, since $\hat p_\parallel\cdot\hat p_\perp = 0$.
It would be nice to make a variable substitution such that the dependence of the new variables on $\omega$ is either it's parallel or orthogonal projection on $p$, so that the $r$ and $m$ indices in \eqnref{solution_8d_jrm_orig} decouple, letting us project the equation on $\hat p$ and $\hat p_\perp$ to get conditions on the parameters $\alpha$ and $\beta$. 
We try the following substitution according to the combinations of $p$, $\omega$ and $f$ entering the equations 
\begin{align}
v^m&=ap_r\omega^{rm}\nn\\
u_m&=b*f_m + c_{(st)}\epsilon_{mnp}*(\omega^{sn}\we\omega^{tp})
\eqnlab{solution_8d_vartrans}
\end{align}
where $a$, $b$ and $c_{(st)}$ are constants constructed using combinations of $p_r$ and $\epsilon_{rs}$.
The variations of $\Phi(u(\omega,f),v(\omega))$ becomes
\begin{align}
*k^m &= *\frac{\partial \Phi}{\partial f_m} = *\lp\frac{\partial \Phi}{\partial u_{\alpha n}}\frac{\partial u_{\alpha n}}{\partial f_m}\rp = \half d\xi^\delta\epsilon_{\delta\gamma\beta}\lp \frac{\partial \Phi}{\partial u_m^{\alpha}} \frac{b}{2}2\epsilon^{\alpha\beta\gamma}\rp = -b\frac{\partial \Phi}{\partial u_m}\nn\\
j_{rm} &= \frac{\partial \Phi}{\partial \omega^{rm}} = \frac{\partial \Phi}{\partial u_{\alpha n}}\frac{\partial u_{\alpha n}}{\partial \omega^{rm}} + \frac{\partial \Phi}{\partial v^n_\alpha}\frac{\partial v^n_\alpha}{\partial \omega^{rm}}\nn\\
& = \frac{\partial \Phi}{\partial u_{n\alpha}}\frac{\partial}{\partial \omega^{rm}}\lp c_{(st)}\epsilon_{nn'p}\epsilon_{\alpha\beta\gamma}\omega^{\beta sn'}\omega^{\gamma tp}\rp + \frac{\partial \Phi}{\partial v^m}ap_r\nn\\
& = -2c_{(rt)}\epsilon_{mnp}\epsilon_{\alpha\beta\gamma}\frac{\partial \Phi}{\partial u_{n\alpha}}\omega^{\gamma tp}d\xi^\beta + \frac{\partial \Phi}{\partial v^m}ap_r 
\end{align}
and thus the duality relations are
\begin{align}
\frac{\partial \Phi}{\partial u_m} = 2\lp\alpha+\frac{2}{3}\rp v^m\frac{\sqrt{1+\Phi}}{ab|p|}
\eqnlab{solution_8d_km_trans}
\end{align}
and
\begin{align}
a&p_r\frac{\partial \Phi}{\partial v^m} 
= \bigg\{
\frac{2}{b}\lp\alpha-\frac{1}{3}\rp\lp u_m - c_{(st)}\epsilon_{mnp}*(\omega^{sn}\we\omega^{tp})\rp p_r\nn\\
& + \epsilon_{mnp}*\lp\omega^{sn} \wedge \omega^{tp}\rp\bigg[ \lp -2\beta + \frac{1}{3}\rp \epsilon_{rs}p_t + 6\beta\epsilon_{ut}p^{u}\W_{sr}\bigg]\nn\\
& + \frac{4}{b}\lp\alpha+\frac{2}{3}\rp c_{(rt)}\epsilon_{mnp}\epsilon_{\alpha\beta\gamma}p_s\omega^{\alpha sn}\omega^{\gamma tp}d\xi^\beta 
\bigg\}\frac{\sqrt{1+\Phi}}{|p|}\nn\\
&= \bigg\{
\frac{2}{b}\lp\alpha-\frac{1}{3}\rp u_m p_r
+ \epsilon_{mnp}*(\omega^{sn}\we\omega^{tp})\bigg[- \frac{2}{b}\lp\alpha-\frac{1}{3}\rp c_{(st)} p_r\nn\\
& - \frac{4}{b}\lp\alpha+\frac{2}{3}\rp c_{(rt)}p_s + 6\beta p^{u}\epsilon_{ut}\W_{sr} + \lp - 2\beta + \frac{1}{3}\rp \epsilon_{rs}p_t  
\bigg]\bigg\}\frac{\sqrt{1+\Phi}}{|p|}
\end{align}
where we have used $\frac{\partial \Phi}{\partial u_m}$ from the \Eqnref{solution_8d_km_trans}.

We now decompose the equation $e_r$ as 
\begin{align}
e_r = (e\cdot\hat p_\parallel)\hat p_\parallel + (e\cdot\hat p_\perp)\hat p_\perp,   
\end{align}  
where $\hat p_\parallel = \hat p = p^r/|p|$ and $\hat p_\perp = p_s\epsilon^{sr}/|p|$ as usual.
Multiplying the equation $e_r$ with $\hat p_\perp$ gives it's projection along the $\hat p_\perp$ direction 
\begin{align}
0 = 
\epsilon_{mnp}*(\omega^{sn}\we\omega^{tp})\bigg[&
 - \frac{4}{b}\lp\alpha+\frac{2}{3}\rp p_v\epsilon^{vr}c_{(rt)}p_s + 6\beta p^{u}\epsilon_{ut}p_v\epsilon^{vr}\W_{sr}\nn\\
& + \lp - 2\beta + \frac{1}{3}\rp p_v\epsilon^{vr}\epsilon_{rs}p_t  
\bigg]\frac{\sqrt{1+\Phi}}{|p|^2}
\end{align}
where we have used that $p_r\epsilon^{rs} p_s = |p|^2\hat p_\perp\cdot\hat p_\parallel = 0$.
For this equation to hold the factor multiplying $*(\omega\we\omega)$ must be zero and gives conditions for $c$ and $\beta$. The only symmetric combinations we can come up with for $c_{rs}$ are $p^u\epsilon_{u(r}p_{s)}$, $\W_{rs}$ and $p_rp_s$. The only term that cancels the $\epsilon$ is the first so we let  
\begin{align}
c_{(rs)}=cp^u\epsilon_{u(r}p_{s)} = c|p|^2\hat p_\perp\otimes\hat p_\parallel
\end{align}
which gives 
\begin{align}
0 &= - \frac{2}{b}\lp\alpha+\frac{2}{3}\rp c|p|^2p_{t}p_s + 12\delta_{[ut]}^{vr}\beta p^{u}p_v\W_{sr} - \lp - 2\beta + \frac{1}{3}\rp p_sp_t\nn\\  
& = - \frac{2}{b}\lp\alpha+\frac{2}{3}\rp c|p|^2p_{t}p_s + 6\beta |p|^2\W_{st} - 4\beta p_{s}p_t - \frac{1}{3}p_sp_t
\end{align}
We cannot rewrite the $|p|^2\W_{st}$ term as $p_sp_t$ so we set $\beta=0$, which gives
\begin{align}
c = -\frac{b}{6|p|^2}\lp\alpha+\frac{2}{3}\rp^{-1} 
\end{align}

Multiplying the equation $e_r$ with $\hat p_\parallel$ gives it's projection along the $\hat p_\parallel$ direction 
\begin{align}
a|p|\frac{\partial \Phi}{\partial v^m} 
&= \bigg\{
\frac{2}{b}\lp\alpha-\frac{1}{3}\rp u_m 
+ \epsilon_{mnp}*(\omega^{sn}\we\omega^{tp})\bigg[\nn\\
& \frac{1}{3}\bigg\{ \lp\alpha-\frac{1}{3}\rp\lp\alpha+\frac{2}{3}\rp^{-1} + 2  
\bigg\}p^u\epsilon_{us}p_{t} \bigg]\bigg\}\sqrt{1+\Phi}
\end{align}
This $SL(2,\rr)$ scalar equation $(e\cdot\hat p_\parallel)$ is the same for the two values of $r$ in $\hat p_\parallel$ meaning the starting $2\cdot 3 + 3 = 9$ equations in the $SL(2,\rr)$ and $SL(3,\rr)$ indices has come down to $1\cdot 3+3=6$ equations. 
According to (.........  ..........) there should only be three scalar fields in the theory so we are left with what appears to be 6 equations of motion for 3 scalars.
To completely remove the $p$ dependence in the duality relations we first choose $a=1/|p|$ meaning $v^m = \omega_\parallel^m$, i.e. $v^m$ is the projection of $\omega^m$ along the constant $\hat p_\parallel$ direction.
Since there could be no $p_sp_t = |p|^2\hat p_\parallel\otimes \hat p_\parallel$ term in $c_{(rs)}$ we cannot rewrite both the $\omega$:s in terms of $v$ and thus we have to choose $\alpha = -1/3$ so the $*(\omega\we\omega)$ term becomes zero. 
We also choose $b = 1$.
The duality relations thus becomes
\begin{align}
\frac{\partial \Phi}{\partial u_m} &= \frac{2}{3} v^m\sqrt{1+\Phi}\nn\\
\frac{\partial \Phi}{\partial v^m} &= -\frac{4}{3} u_m \sqrt{1+\Phi}
\eqnlab{solution_equations_8d_final}
\end{align}
which is our final form of the duality relations for the $d=8$ $D2$ case with $\alpha$ and $\beta$ constant. We will try and solve these in \Secref{csolution_8d_const}.

To see whether these equations are reasonable or not we series expand $\Phi$ to third order
\begin{align}
\Phi &= a_0 u_\alpha^mu^\alpha_m + a_1 u_\alpha^mv^\alpha_m + a_2 v_\alpha^mv^\alpha_m\nn\\
& + \epsilon_{mnp}\epsilon^{\alpha\beta\gamma}\lp a_3u_\alpha^mu_\beta^nu_\gamma^p + a_4u_\alpha^mu_\beta^nv_\gamma^p + a_5u_\alpha^mv_\beta^nv_\gamma^p + a_6v_\alpha^mv_\beta^nv_\gamma^p \rp.
\end{align}
Since there should only be 3 scalars in the theory and we have 6 scalars in the 1-forms $u$ and $v$, there must be a relation $u = u(v)$ such that as to get the indices right 
\begin{align}
u_\alpha^m = b_1v_\alpha^m + b_2\epsilon_{\alpha\beta\gamma}\epsilon^{mnp} v^\beta_nv^\gamma_p + \Ordo(v^3). 
\end{align}
The series expansion of the duality equations to second order in $u$ and $v$ thus becomes
\begin{align}
2 a_0 u^\alpha_m + a_1 v^\alpha_m + \epsilon_{mnp}\epsilon^{\alpha\beta\gamma}\lp 3 a_3u_\beta^nu_\gamma^p + 2a_4u_\beta^nv_\gamma^p + a_5v_\beta^nv_\gamma^p \rp &= \frac{2}{3} v_m^\alpha\nn\\
 a_1 v^\alpha_m + a_2 v^\alpha_m + \epsilon_{mnp}\epsilon^{\alpha\beta\gamma}\lp a_4u_\beta^nu_\gamma^p + 2a_5u_\beta^nv_\gamma^p + 3a_6v_\beta^nv_\gamma^p \rp &= -\frac{4}{3} u_m^\alpha
\end{align}
and using $u = u(v)$ to second order in $v$
\begin{align}
0 &= 2a_0b_1v^\alpha_m + 2a_0b_2\epsilon^{\alpha\beta\gamma}\epsilon_{mnp} v_\beta^nv_\gamma^p + a_1v^\alpha_m + \epsilon_{mnp}\epsilon^{\alpha\beta\gamma}\lp 3 a_3b_1^2v^\beta_nv^\gamma_p + 2a_4b_1v^\beta_nv_\gamma^p + a_5v_\beta^nv_\gamma^p \rp - \frac{2}{3} v_m^\alpha \nn\\
0 &= a_1 v^\alpha_m + a_2 v^\alpha_m + \epsilon_{mnp}\epsilon^{\alpha\beta\gamma}\lp a_4b_1^2v^\beta_nv^\gamma_p + 2a_5b_1v^\beta_nv_\gamma^p + 3a_6v_\beta^nv_\gamma^p \rp + \frac{4}{3}b_1v^\alpha_m + \frac{4}{3}b_2\epsilon^{\alpha\beta\gamma}\epsilon_{mnp} v_\beta^nv_\gamma^p
\end{align}
from which we can read of
\begin{align}
0 &= 2a_0b_1 + a_1 - \frac{2}{3}\nn\\
0 &= 2a_0b_2 + 3a_3b_1^2 + 2a_4b_1 + a_5
\end{align}
from the first equation and
\begin{align}
0 &= a_1 + a_2 + \frac{4}{3}b_1\nn\\
0 &= a_4b_1^2 + 2a_5b_1 + 3a_6 + \frac{4}{3}b_2
\end{align}
from the second equation.
Demanding the equations to be equal and thus contain the same information we (......... Ja, vad�? Det kommer ju alltid vara mycket fler coefficienter �n villkor, vilket g�r att vi inte kan s�ga n�got om h�gre ordningars expansioner .........)







\section{The $d=9$ $D1$, const $\alpha$ case}
We restate the equations of motion found earlier with $\alpha$ constant
\begin{align}
&1 + \Phi - *f_m*f_n\M^{mn}=0\\
&d \Big{[}\lambda \M^{mn}*f_n\Big{]}=0\\
&d\Big{[}\lambda *j_{1m} + 2 \lambda \epsilon_{mm'} \alpha \omega^2 *f_{n'} \M^{m'n'}\Big{]}=0\\
&d\Big{[}\lambda*j_2 - 2\lambda \epsilon_{mn}\alpha \omega^{1n} *f_{n'}\M^{mn'}\Big{]} - 2\lambda \epsilon_{mn} F^{1n}*f_{n'}\M^{mn'}=0,
\end{align}
As in the $d=8$ $D2$ case we can identify (......... How? .........) the charges $p^m=\lambda \M^{mn}*f_n$ and together with the Bianchi identity $d\omega^{1m}=-F^{1m}$ we can rewrite the last two equations as
\begin{align}
&d\Big{[}\frac{|p|}{\sqrt{1+\Phi}}*j_{1m} + 2 \epsilon_{mm'} \alpha \omega^2 p^{m'}\Big{]}=0\\
&d\Big{[}\frac{|p|}{\sqrt{1+\Phi}}*j_2 - 2\epsilon_{mn}\lp\alpha -1\rp\omega^{1n} p^m \Big{]}=0
\end{align}
Integration gives 
\begin{align}
&*j_{1m} = \lbp - 2 \epsilon_{mm'} \alpha \omega^2 p^{m'} + \gamma_{1m} \rbp\frac{\sqrt{1+\Phi}}{|p|}\\
&*j_2 = \lbp 2\epsilon_{mn}\lp\alpha -1\rp\omega^{1n} p^m + \gamma_{2} \rbp\frac{\sqrt{1+\Phi}}{|p|}
\end{align}
where $\gamma$ are 1-forms such that $d\gamma=0$. We will use $\gamma = 0$. Make the variable substitution
\begin{align}
v &= \frac{1}{|p|}\epsilon_{mn}p^m\omega^{1n} = \hat p_\perp\cdot\omega^1 = \omega^1_\perp\nn\\
u &= *\omega^2
\end{align}
giving the variations
\begin{align}
j_{1m} &= \frac{\partial \Phi}{\partial \omega^{1m}} = \frac{\partial \Phi}{\partial v_\alpha}\frac{\partial v_\alpha}{\partial \omega^{1m}} = \frac{\partial \Phi}{\partial v}p_{\perp m}\nn\\
*j_2 &= *\frac{\partial \Phi}{\partial \omega^2} = *\lp\frac{\partial \Phi}{\partial u_{\alpha}}\frac{\partial u_{\alpha}}{\partial \omega^2}\rp = d\xi^\gamma\epsilon_{\gamma\beta}\lp \frac{\partial \Phi}{\partial u^{\alpha}} \epsilon^{\alpha\beta}\rp = \frac{\partial \Phi}{\partial u}
\end{align}
which in turn gives the duality relations
\begin{align}
\frac{\partial \Phi}{\partial v}\hat p_\perp &= - 2 \hat p_\perp\alpha u\sqrt{1+\Phi}\\
\frac{\partial \Phi}{\partial u} &= 2\lp\alpha -1\rp v \sqrt{1+\Phi}
\end{align}
According to (.........  ..........) there should only be one scalar field in the theory, but we have 3 equations for scalar field strengths. The first two, corresponding to the two components in $\hat p_\perp$ in the first relation, are obviously the same. 
We can further choose the coefficient $\alpha$ so that the remaining equations becomes the same as we did in the $d=8$ $D2$ case. (.......... Det g;r vi inte alls l'ngre, kolla upp med en serieutveckling .........) 
We thus choose $\alpha=1/2$ and get the equivalent equations
\begin{align}
\frac{\partial \Phi}{\partial v} &= - \sqrt{1+\Phi}u\\
\frac{\partial \Phi}{\partial u} &= \sqrt{1+\Phi}v
\end{align}
which is the final form of the duality relations in the $d=9$ $D1$ case with $\alpha$ and $\beta$ constant. We will try and solve these in \Secref{csolution_9d_const}.











\section{The $d=9$ $D1$, $\alpha$ case}
We consider the same equations as in the previous section for $\alpha$ a nonconstant function of $\omega^{1m}$ and $\omega^2$.
The new equations are
\begin{align}
&d\Big{[}\lambda *j_{1m} + 2 \lambda {\{} \epsilon_{mm'} \alpha \omega^2 - \epsilon_{m'n}*{\partial \alpha \over \partial \omega^{1m}}*(\omega^{1n} \wedge \omega^2){\}}*f_{n'} \M^{m'n'}\Big{]}=0\nn\\
&d\Big{[}\lambda*j_2 - 2\lambda \epsilon_{mn}{\{}\lp\alpha -1\rp\omega^{1n} + *{\partial \alpha \over \partial \omega^2 }*(\omega^{1n} \wedge \omega^2){\}}*f_{n'}\M^{mn'}\Big{]}=0,
\end{align}
or after integration and with $p$ (without integration forms)
\begin{align}
*j_{1m} &= 2 {\{} \alpha \hat p_{\perp m} \omega^2 + *{\partial \alpha \over \partial \omega^{1m}}*(\omega^{1}_\perp \wedge \omega^2){\}}\sqrt{1+\Phi}\nn\\
*j_2 &= 2{\{}\lp\alpha -1\rp\omega^{1}_\perp + *{\partial \alpha \over \partial \omega^2 }*(\omega^{1}_\perp \wedge \omega^2){\}}\sqrt{1+\Phi}.
\end{align}

Hum, g�r n�t smart nu d�...

%To get a nicer expression without hodge dualities and $p$ we define
%\begin{align}
%v^m &= \omega^{1m}\nn\\
%u^m &= q^m*\omega^2,\hspace{1cm}\Rightarrow\omega^2=\frac{q_m}{q^2}*u^m 
%\end{align}
%where $q^m$ is introduced to get an $SL(2,\rr)$-index on $u$ and is constructed by some combination of the constant charges $p^m$ to remove them from the equations.
%\begin{align}
%\frac{\partial\Phi}{\partial v^m} &= - 2\epsilon_{mn}{\{}\alpha \frac{q_p}{q^2}u^p  + \frac{\partial \alpha}{\partial v^{n'}}\frac{q_p}{q^2} v_\alpha^{n'}u^{\alpha p}) + *\gamma{\}}\frac{p^{n}}{|p|}\sqrt{1+\Phi}\\
%\frac{\partial\Phi}{\partial u^m}q_m &= 2\epsilon_{mn}{\{}\lp\alpha-1\rp v^{n} + \frac{\partial\alpha}{\partial u^p}q_p\frac{q_{n'}}{q^2} v_\alpha^{n}u^{\alpha {n'}}) + \delta^n{\}}\frac{p^m}{|p|}\sqrt{1+\Phi}
%\end{align}
%thus we see that we should choose
%\begin{align}
%q_m =  
%\end{align}
%to get
%\begin{align}
%\frac{\partial\Phi}{\partial v^m} &= - 2\epsilon_{mn}{\{}\alpha \frac{q_p}{q^2}u^p  + \frac{\partial \alpha}{\partial v^{n'}}\frac{q_p}{q^2} v_\alpha^{n'}u^{\alpha p}) + *\gamma{\}}\frac{p^{n}}{|p|}\sqrt{1+\Phi}\\
%\frac{\partial\Phi}{\partial u^m}q_m &= 2\epsilon_{mn}{\{}\lp\alpha-1\rp v^{n} + \frac{\partial\alpha}{\partial u^p}q_p\frac{q_{n'}}{q^2} v_\alpha^{n}u^{\alpha {n'}}) + \delta^n{\}}\frac{p^m}{|p|}\sqrt{1+\Phi}
%\end{align}

%The theory should only contain 2 scalars (.......... ????? ........).
%There are 2 scalars in $\omega^{1m}$ and one scalar in $\omega^2$. We must therefore relate $\omega^2$ to $\omega^{1m}$ somehow.  


%Dualize and multiply the first relation by $*\omega^{2\alpha}$
%\begin{align}
%*\omega^{2\alpha}j_{1\alpha m} &= - 2 *\omega^{2\alpha}{\{}\alpha *\omega_\alpha^2 + {\partial \alpha \over \partial \omega^{1\alpha n'}}*(\omega^{1n'} \wedge \omega^2) + *\gamma{\}}\epsilon_{mn}*f_{m'} \M^{m'n}
%\end{align}
%which means we can express $\epsilon_{mn}*f_{m'} \M^{m'n}$ explicitly as a function of $\omega^2$ and $\omega^{1m}$ and inserting it in the second relation we get 
%\begin{align}
%*j_{2\alpha } *\omega^{2\alpha}{\{}\alpha *\omega_\alpha^2 + {\partial \alpha \over \partial \omega^{1\alpha n'}}*(\omega^{1n'} \wedge \omega^2) + *\gamma_{\alpha m}{\}} &= - {\{}\lp\alpha-1\rp \omega_\alpha ^{1n} + *{\partial \alpha \over \partial \omega^{2\alpha}}*(\omega^{1n} \wedge \omega^2) + \delta^n{\}}*\omega^{2\alpha}j_{1\alpha m} 
%\end{align}

%We want to express $\omega^2_\alpha$ as a function of $\omega^{1m}_\alpha$. Since $\omega^2_\alpha$ does not have an $SL(2,\rr)$ index it must be contracted somehow in the construction.  








 
