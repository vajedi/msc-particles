\chapter{Nan}

The motions of suspended particles in a turbulent fluid are complex and 
approximations must be made. A very convenient one would be to consider 
inertaless point particles, in which case the equation of motion would be
given by
\begin{equation}
\dvec{r}(t) = \vec{u}(\vec{r}(t),t).
\label{eq:inertialess}
\end{equation}
The particles would hence follow the flow and at every point take on the 
velocity of the fluid. Particles which satisfy this equation of motion are 
\emph{advected} and are called \emph{passive tracers}. This approximation 
has many applications in fluid dynamics and in the early days, 
\eq{inertialess} was used predominantly. However, in order to describe 
more complex behavior like ... and ..., this is not enough.

$\gamma$ is the damping rate. The Stokes number $\St = 1/\gamma\tau$ is the 
(inverse?) intensity of the damping. When $\St \rightarrow 0$ 
the particles are completely 
advected and behave like point particles with no inertia, and the equation 
of motion is then given by \eq{inertialess}. 

\emph{Vorticity} is the measure of how fast fluid elements rotate about themselves. 
The curl of the velocity field is the vorticity field and is denoted 
$\vec{\omega} = \nabla \times \vec{u}$. Vorticity is a local measure and 
does is not necessarily nonzero even if the fluid rotates at a large scale. 

A vortex is a region of high vorticity.

% http://www.scribd.com/doc/81564634/Aurelie-Goater-Dispersion-of-Heavy-Particles-in-an-Isolated-Pancake-Like-Vortex
% Equation of motion for a small rigid sphere in a nonuniform flow

The particle Reynolds number is given by
\begin{equation}
\Re _p = \frac{d| \vec{\nu}_p-\vec{u}|}{\nu}.
\end{equation}

\section{The Maxey-Riley equation}

% Bra k�lla: http://www.lec.csic.es/~julyan/PDFs/79_2010_Springer.pdf

- For small spherical rigid particle advected by a (smooth) flow.
- Valid for small particles at low particle Reynolds numbers Re$_p$.
- Velocity difference across the particle must be small

The Maxey-Riley equation is given by
\begin{align}
m_p\dvec{v} = & m_f \frac{\D}{\D t}\vec{u}(\vec{r}(t),t)- 
\frac{1}{2}m_f\left(\dvec{v}-\frac{\D}{\D t}\left[\vec{u}(\vec{r}(t),t)
+\frac{1}{10}a^2\nabla^2\vec{u}(\vec{r}(t),t)\right]\right) \nonumber \\
& -6\pi a \rho_f \nu \vec{q}(t) + (m_p-m_f)\vec{g}-6\pi a^2\rho_f \nu 
\int\limits_0^t \frac{\d \tau}{\sqrt{\pi\nu (t-\tau)}} 
\frac{\d \vec{q}(\tau)}{\d \tau},
\end{align}
where
\begin{equation}
\vec{q}(t) = \vec{v}(t) - \vec{u}(\vec{r}(t),t)-\frac{1}{6}a^2\nabla^2\vec{u}
\end{equation}
$m_p$ is the particle mass, $a$ the particle radius, $m_f$ the mass of 
the fluid displaced by the particle, $\rho_f$ is the density 
of the fluid, $\nu$ the viscosity of the fluid. The first term 
on the right-hand side of the Maxey-Riley is the force exerted by a fluid 
element in position $\vec{r}(t)$ and corresponds to the force by the force of 
the undisturbed fluid. The second term is due to the \emph{added-mass effect}, 
which results from the fact that the particle displaces a certain amount of fluid 
along its trajectory, which makes the particle appear to have additional mass. 
The third and the fourth terms result from the viscosity and the buoyancy force, 
respectively, of the fluid, which represent the Stokes' drag. The factor of 
$a^2\nabla^2 \vec{u}$ is due to the spatial variation of the velocity field 
across the particle, and the terms containing this are called \emph{Fax�n 
corrections}. 

The integral is the \emph{Basset-Boussinesq history term}, which accounts for 
the fact that the vorticity diffuses away from the particle due to viscosity. 

\begin{equation}
\frac{\D\vec{u}}{\D t} = \frac{\partial \vec{u}}{\partial t} + 
(\vec{u}\cdot \nabla)\vec{u}.
\end{equation}
\begin{equation}
\frac{\d\vec{u}}{\d t} = \frac{\partial \vec{u}}{\partial t} + 
(\vec{v}\cdot \nabla)\vec{u}.
\end{equation}
