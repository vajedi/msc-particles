\chapter{Kaluza-Klein reduction on $T^2$, $T^3$ and $T^4$}
\label{sec:reduct}
In the formulation of the membrane dynamics, every background $n$-form gauge 
field will couple to its corresponding ($n$-1)-form gauge field on the 
world-volume through a universal coupling outlined in the next section. Our 
first task is thus to derive the background fields by reducing 11-dimensional 
supergravity to 9-, 8- and 7-dimensional maximal supergravity. By using the coupling mentioned 
above the membrane dynamics will automatically be covariant with respect to the 
global symmetries of the background it couples to, and we will therefore try to write the Bianchi 
identities and the duality relations in a form where the covariance with respect to the symmetry groups 
given in table 2.3 is manifest. The dimensional reduction of \eqnref{sugra_11dim} will be done the standard 
Kaluza-Klein way on a 2-, 3- and 4-torus. The $T^3$ case can also be found in for example \cite{artikeln}.
We start from the derived Kaluza-Klein Ansatz \eqnref{sugra_kk_ansatz} for the vielbeins 
\begin{equation}
{\hat e}{_{M}}^A = \toto{ e{_{\mu}}^a }{ A_{\mu}^{1m}e{_{m}}^i
  }{ 0 }{ e{_{m}}^i}, \hspace{0.5cm} {\hat e}{_{A}}^M = \toto{ e{_{a}}^\mu }{
  -e{_{a}}^{\mu}A_{\mu}^{1m} }{ 0 }{ e{_{i}}^m},
\eqnlab{reduct_kk_ansatz}
\end{equation}
\[
dx^M = (dx^\mu,dx^m),
\]
where ${\hat e}{_M}^A {\hat e}{_{NA}} = {\hat g}_{MN}$ and the fields
depends on $x^\mu$ only. Hats are used for 11-dimensional
fields. The Ansatz is of course independent of how many dimensions that are compactified. This gives
\begin{align}
{\hat e}^A &= dx^M {\hat e}{_M}^A = (dx^\mu e{_\mu}^a,dx^\mu A_\mu^{1m} e{_{m}}^i + dx^m e{_{m}}^i) \nonumber \\
&= (e^a,A^{1i} + e^i) = ({\hat e}^a,{\hat e}^i),
\eqnlab{reduct_viel_red}
\end{align}
where the index 1 on $A_{\mu}^{1m}$ separates the internal vector from 
the vector that will arise from the compactified 3-form. Using this we can 
calculate the contribution of the 11-dimensional Einstein term to the lower-dimensional supergravities.

\section{The Einstein term}
The calculation of the Einstein term is somewhat lengthy and can be
found in section \secref{ricci}. Here we start with the result
\begin{align}
\int d^{11}x \sqrt{|\hat{g}|} \hat{R} =& \int d^{d}x \sqrt{|g|} e^{-\varphi} \Big{[}R - {1 \over 4}G_{mn}F_{ab}^{1m}F^{1n \hspace{0.05cm} ab} - 2D_a (e{_i}^{m} \partial^ae{_m}^i) \nonumber \\
&- e^{n(i}({\partial}_a e{_n}^{j)})e{_i}^m(\partial^a e_{mj}) -e{_i}^m({\partial}^a e{_m}^i) e{_j}^n({\partial}_a e{_n}^j)\Big{]},
\label{compR}
\end{align}
where we have used that
\begin{equation}
\sqrt{\det\hat{g}_{MN}} = \det e{_M}^A = \det e{_\mu}^a\det e{_m}^i = \sqrt{\det g_{\mu \nu}} \sqrt{\det G_{mn}},
\end{equation}
where $G_{mn}$ is the metric on $T^D$ constructed from the D-bein $e{_m}^i$, and also the definitions
\begin{equation}
\sqrt{\mbox{det}G_{mn}}=e^{-\varphi}, \hspace{1cm} F_{ab}^{1m}=dA_{ab}^{1m}.
\eqnlab{reduct_f1def}
\end{equation}
Note also that we have put both the prefactor $2 {\kappa}_{11}^2$ in \eqnref{sugra_11dim} and vol($T^D$)$=\int d^Dy$ equal to 1, where vol($T^D$) is the volume of the internal torus.
The term with the covariant derivative can be integrated by parts to give
\begin{align}
&-2 \int d^dx \sqrt{|g|} e^{-\varphi} D_a (e{_i}^m \partial^a e{_m}^i) = -2 \int d^dx e^{-\varphi} {\partial}_a (\sqrt{|g|}e{_i}^m \partial^a e{_m}^i) \nonumber \\
&=2\int d^dx\sqrt{|g|}({\partial}_a e^{-\varphi})e{_i}^m \partial^a e{_m}^i = 2\int d^dx \sqrt{|g|} e^{-\varphi} (-{\partial}_a \varphi)\mbox{Tr}(e^{-1}\partial e) \nonumber \\
&=2 \int d^dx \sqrt{|g|} e^{-\varphi} ({\partial}_a \varphi) ({\partial}^a \varphi),
\label{t1}
\end{align}
where we have used \eqnref{conven_divergence} and \eqnref{conven_trlog}. The last term in (\ref{compR}) can be rewritten as
\begin{equation}
-\int d^dx \sqrt{|g|} e^{-\varphi}e{_i}^m({\partial}^a e{_m}^i) e{_j}^n({\partial}_a e{_n}^j) = -\int d^dx \sqrt{|g|} e^{-\varphi} ({\partial}_a \varphi) ({\partial}^a \varphi),
\label{t2}
\end{equation}
where we have again used \eqnref{conven_divergence}. And finally the fourth term in (\ref{compR}) can be rewritten as
\begin{align}
&-e^{n(i}({\partial}_a e{_n}^{j)})e{_j}^m(\partial^a e_{mi}) \nonumber \\
&= - {1 \over 2}e^{ni}({\partial}_a e{_n}^{j})e{_j}^m(\partial^a e_{mi})- {1 \over 2}e^{ni}({\partial}_a e{_n}^{j})e{_i}^m(\partial^a e_{mj})\nonumber \\
&=- {1 \over 2}G^{mn}e{_m}^{i}({\partial}_a e{_n}^{j})G^{pq}e_{jq}(\partial^a e_{pi})- {1 \over 2}G^{mn}({\partial}\bd{a} e\bd{n}\bu{k})(\partial\bu{a} e\bd{ml}){\delta}_{k}^j{\delta}_{j}^l \nonumber \\
&=- {1 \over 4}G^{mn}G^{pq} \Big{(}e{_m}^{i}({\partial}_a e{_n}^{j})e_{jq}(\partial^a e_{pi})+e{_n}^{i}({\partial}_a e{_m}^{j})e_{jp}(\partial^a e_{qi})\Big{)}\nonumber \\
&\hspace{0.47cm}-{1 \over 2}G^{mn}e^{jp}e_{pk}({\partial}_a e{_n}^{k})e{^l}_q e{^q}_j(\partial^a e_{ml})\nonumber \\
&= - {1 \over 4}G^{mn}G^{pq}\Big{(}e_{jq}({\partial}_a e{_n}^{j})e_{mi}(\partial^a e{_p}^i)+e_{jn}({\partial}_a e{_q}^{j})e_{ip}(\partial^a e{_m}^i)\nonumber \\
&\phantom{=}+e_{jq}({\partial}_a e{_n}^{j})e_{ip}(\partial^a e{_m}^i)+e_{jn}({\partial}_a e{_q}^{j})e_{im}(\partial^a e{_p}^i)\Big{)}\nonumber \\
&=-{1 \over 4}G^{mn}G^{pq}\Big{(}e_{jq}({\partial}_a e{_n}^j)+e_{jn}({\partial}_a e{_q}^j)\Big{)}\Big{(}e_{mi}(\partial^a e{_p}^i)+e_{pi}(\partial^a e{_m}^i)\Big{)}\nonumber \\
&=-{1 \over 4}G^{mn}({\partial}_a G_{qn})G^{pq}(\partial^a G_{mp})=-{1 \over 4}\mbox{Tr}(G^{-1} \partial G)^2.
\label{t3}
\end{align}
Inserting (\ref{t1}), (\ref{t2}) and (\ref{t3}) in (\ref{compR}) gives
\begin{align}
\int d^{11}x \sqrt{|\hat{g}|} \hat{R} =& \int d^9x \sqrt{|g|} e^{-\varphi}\Big{[}R + ({\partial}_a \varphi) ({\partial}^a \varphi) \nonumber \\
&-{1 \over 4}G\bd{mn}F_{ab}^{1m}F^{1n \hspace{0.05cm} ab} - {1 \over 4} \mbox{Tr}(G^{-1}\partial G)^2\Big{]}.
\label{finalaction}
\end{align}
The next step is to rescale the action to obtain an action in the Einstein frame. But before we perform the rescaling we note that only the index values 
separates equation (\ref{finalaction}) from the corresponding equation in 8 and 7 dimensions. Therefore we can make a general variable change according to
\begin{equation}
g{_{\mu \nu}} = e^{-2s\varphi}g_{\mu \nu}^E, \hspace{1cm} \M_{mn} = {G_{mn} \over (\mbox{det}G)^{1/D}},
\label{varchange2}
\end{equation}
where
\begin{equation}
s={1 \over 2-d}, \hspace{0.5cm} d=\mbox{dim}g{_{\mu \nu}}, \hspace{0.5cm} D=\mbox{dim}G{_{mn}}.
\end{equation}
The D $\times$ D matrix $\M_{mn}$ provides a parametrisation of the $SL(2,{\mathbb R})/$ \newline $SO(2,{\mathbb R})$
coset space in 9 dimensions, $SL(3,{\mathbb R})/SO(3,{\mathbb R})$ in 8 dimensions and $SL(4,{\mathbb R})/SO(4,{\mathbb R})$ in 7 dimensions. 
The short calculation
\begin{equation}
G_{mn}=(\mbox{det}G)^{1/D}\M_{mn}=e^{-2\varphi /D}\M_{mn} \hspace{0.5cm} \Rightarrow \hspace{0.5cm} \mbox{det}\M_{mn}=1,
\end{equation}
shows that all parametrisations are unimodular, a necessary condition for $SL$-groups. 
The first variable change in (\ref{varchange2}) gives
\begin{equation}
\mbox{det}g_{\mu \nu} = e^{-2ds\varphi}\mbox{det}g_{\mu \nu}^E.
\end{equation}
The terms in (\ref{finalaction}) will now be contracted with $g^E$ and $\M$ instead of $g$ and $G$ and thus change as
\begin{align}
(\partial \varphi)^2 &\rightarrow e^{2s\varphi}(\partial \varphi)^2 \hspace{0.2cm} \mbox{from}\hspace{0.2cm} g^{\mu \nu} \nonumber \\
G_{mn}F^{1m}F^{1n} &\rightarrow  e^{4s\varphi} e^{-2\varphi/D} \M_{mn}F^{1m}F^{1n} \hspace{0.2cm} \mbox{from}\hspace{0.2cm} (g^{\mu \nu})^2 \hspace{0.2cm}\mbox{and}\hspace{0.2cm} G_{mn} \nonumber \\
\mbox{Tr}(G^{-1} \partial G)^2 &\rightarrow e^{2s\varphi} \mbox{Tr}(e^{2\varphi/D}\M^{-1} \partial (e^{-2\varphi/D}\M))^2\hspace{0.2cm} \mbox{from}\hspace{0.2cm} g^{\mu \nu}\mbox{,}\hspace{0.2cm}G_{mn} \nonumber \\
& \hspace{0.65cm} \mbox{and }G^{mn}\nonumber,
\end{align}
altogether yielding
\begin{align}
\int d^{11}x \sqrt{|\hat{g}|} \hat{R} =& \int d^dx \sqrt{|g^E|}\Big{[}e^{-(ds+1)\varphi}R + e^{(-ds+2s-1)\varphi}(\partial \varphi)^2 \nonumber \\
&-{1 \over 4}e^{(-ds+4s-1-2/D)\varphi}\M_{mn}F^{1m}F^{1n} \nonumber \\
&-{1 \over 4}e^{(-ds+2s-1)\varphi} \mbox{Tr}(e^{2\varphi/D}\M^{-1} \partial (e^{-2\varphi/D}\M))^2\Big{]}.
\label{B1}
\end{align}
We can write the expression inside the trace as
\begin{align}
&\Big{(}e^{2\varphi/D}\M^{-1} \partial (e^{-2\varphi/D}\M)\Big{)}^2 \nonumber \\
&=\Big{(}e^{2\varphi/D}\M^{-1}\Big\{ -\frac{2}{D}\partial \varphi e^{-2\varphi/D}\M + e^{- 2\varphi/D } \partial \M\Big\}\Big{)}^2 \nonumber \\
&=\Big{(}-\frac{2}{D}\partial \varphi + \M^{-1} \partial \M\Big{)}^2 \nonumber \\
&=\frac{4}{D^2}(\partial \varphi)^2 - \frac{4}{D}(\partial \varphi)\M^{-1}\partial \M +(\M^{-1}\partial \M)^2
\end{align}
and we note that, from $\eqnref{conven_trlog}$
\begin{equation}
\mbox{Tr}(\M^{-1} \partial \M)=\partial(\ln \mbox{det}\M)=0.
\end{equation}
We also need to investigate how the change of variables affects the Ricci scalar. For that we use (see equation \eqnref{RicciE})
\begin{equation}
R = e^{2s\varphi}(R_E-s^2(d-1)(d-2)(\partial \varphi)^2),
\end{equation}
which inserted in (\ref{B1}) gives the action in the Einstein frame as
\begin{align}
\int d^{11}x \sqrt{|\hat{g}|} \hat{R} =& \int d^dx \sqrt{|g^E|}\Big{[}R^E - {9 \over D(d-2)}(\partial \varphi)^2 \nonumber \\
&-{1 \over 4}e^{-{18 \over D(d-2)}\varphi }\M_{mn}F^{1m}F^{1n}-{1 \over 4}\mbox{Tr}(\M^{-1}\partial \M)^2\Big{]}.
\label{B2}
\end{align}

\section{Field strengths and  Bianchi identities}
Let us now turn the attention to the 4-form field strength $\hat{G}=d\hat{C}$. Before we reduce the field strength 
we note that the curved 2-dimensional (or 3- or 4-dimensional) basis
\begin{equation}
\hat{e}^m = dx^m + A^{1m}, \hspace{1cm} A^{1m}=dx^{\mu}A_{\mu}^{1m},
\end{equation}
is invariant under general coordinate transformations $\delta x^m = -\lambda^m(x)$ of the internal torus as is seen from
\begin{align}
\delta e^m &= \delta(dx^m + dx^{\mu}A_{\mu}^{1m}) = \delta dx^m + \delta dx^{\mu}A_{\mu}^{1m} = d\delta x^m + dx^{\mu}(\delta A_{\mu}^{1m}) \nonumber \\
&= -d\lambda^m + dx^{\mu}({\partial}_{\mu}\lambda^m)=-d\lambda^m + d\lambda^m =0,
\end{align}
with $A_{\mu}^{1m}$ transforming under local U(1) gauge transformations
\begin{equation}
\delta A_{\mu}^{1m} = {\partial}_{\mu}\lambda^m
\end{equation}
which was shown in section \secref{sugra_kk}.
We now want to expand the 11-dimensional 3-form $\hat{C}$ into lower-dimensional components, where the fields 
preserve the $\lambda^m$ invariance. This is achieved by expanding in the basis above and by using the fact that 
$\hat{C}$ is manifestly invariant \eqnref{sugra_gauge11}. Note that when
integrating the 2, 3 or 4 compact directions in \eqnref{sugra_11dim}, only terms with
2, 3 or 4 $dx^m$'s will be non-zero. Hence $dx^m$ appears as $e^m$ from
the action's point of view, and in our expansion below we will
only have terms with 2 or less $e^m$'s in the $T^2$ case, 3 or less in
the $T^3$ case and so on.

\subsubsection{The $T^2$ case}
First we reduce the 11-dimensional relation
\begin{equation}
\delta \hat{C}=d \hat{\chi}
\label{tja}
\end{equation}
for the $T^2$-case according to the procedure we outlined above. The 11-dimensional 3-form becomes
\begin{align}
\hat{C} &= {1 \over {3!}} C_{PNM}dx^M \wedge dx^N \wedge dx^P \nonumber \\
&= {1 \over {3!}}\Big{(}C_{\rho \nu \mu}dx^{\mu} \wedge dx^{\nu} \wedge dx^{\rho} + 3C_{p \nu \mu}dx^{\mu} \wedge dx^{\nu} \wedge dx^{p} \nonumber \\
&\phantom{=} + 3 C_{p n \mu}dx^{\mu} \wedge dx^{n} \wedge dx^{p} +C_{pnm}dx^{m} \wedge dx^{n} \wedge dx^{p}\Big{)}\nonumber \\
&= \Big{\{ \mbox{The last term vanish because} \hspace{0.2cm}m,n,p=1,2\Big{\}}}\nonumber \\
&= C + {3 \cdot 2 \over 6} \underbrace{({1 \over 2}C_{\nu \mu}dx^{\mu} \wedge dx^{\nu})_p}_{B_{p}} \wedge dx^p + {3 \over 6} \underbrace{(C_{\mu}dx^{\mu})_{pn}}_{-A_{pn}^2 = A_{np}^2} \wedge dx^n \wedge dx^p \nonumber \\
&= C + B_{m} \wedge \hat{e}^m + {1 \over 2}{\epsilon}_{mn}A^2 \wedge \hat{e}^m \wedge \hat{e}^n,
\end{align}
where we have used \eqnref{reduct_viel_red} to see that
\begin{align}
d\hat e^m = d(A^{1m} + dx^m)) = F^{1m}.
\end{align}
Repeating this for the 2-form gauge parameter gives (superspace conventions, see appendix \chref{conven}):
\begin{align}
d\hat{\chi} &= d({1 \over 2}\chi_{NM}dx^M \wedge dx^N) \nonumber \\
&= {1 \over 2}d(\chi_{\nu \mu}dx^{\mu} \wedge dx^{\nu} + 2\chi_{n \mu}dx^{\mu} \wedge dx^n + \chi_{nm}dx^{m} \wedge dx^n)\nonumber \\
&= d(\chi' - {\chi'}_m \wedge \hat{e}^m + {1 \over 2}{\chi'}^2 {\epsilon}_{mn} \hat{e}^m \wedge \hat{e}^n)\nonumber \\
&= \underbrace{d\chi'-{\chi'}_m \wedge F^{1m}}_{\delta C} + \underbrace{(d{\chi'}_m + {\epsilon}_{mn}{\chi'}^2 \wedge F^{1n})}_{\delta B_{m}'} \wedge \hat{e}^m \nonumber \\
&\phantom{=}+{1 \over 2} {\epsilon}_{mn} \underbrace{d{\chi'}^2}_{\delta A^2} \wedge \hat{e}^m \wedge \hat{e}^n,
\label{tja2}
\end{align}
where we also have indicated which field transformations the terms are connected to. The $\lambda^m$ invariant field strengths can now be calculated as
\begin{align}
\hat{G}=d\hat{C} &= \underbrace{dC + B_{m} \wedge F^{1m}}_{G} + \underbrace{(-dB_{m} + A^2 {\epsilon}_{mn} \wedge F^{1n})}_{-H_m} \wedge \hat{e}^m \nonumber \\
&\phantom{=}+ {1 \over 2}{\epsilon}_{mn} \underbrace{dA^2}_{F^2} \wedge \hat{e}^m \wedge \hat{e}^n,
\label{kalle1}
\end{align}
hence
\begin{align}
G&=dC + B_{m} \wedge F^{1m}, \nonumber \\
H_m&= dB_{m} - {\epsilon}_{mn} F^{1n} \wedge A^2, \nonumber \\
F^2&=dA^2.
\end{align}
The 9-dimensional field strengths thus satisfies the following Bianchi identities:
\begin{align}
dG&=d(dC + B_{m} \wedge F^{1m})=H_m \wedge F^{1m}, \nonumber \\
dH_m&=d(dB_m - {\epsilon}_{mn} F^{1n} \wedge A^2) = - {\epsilon}_{mn} F^{1n} \wedge F^{2},\nonumber \\
dF^2&=d^2 A^2 = 0.
\end{align}

\subsubsection{The $T^3$ case}
The calculation of the 8-dimensional field strengths is very similar to the calculation of the 9-dimensional ones.
By repeating the procedure above we get
\begin{align}
\hat{C} &= {1 \over {3!}} C_{PNM}dx^M \wedge dx^N \wedge dx^P \nonumber \\
&= {1 \over {3!}}\Big{(}C_{\rho \nu \mu}dx^{\mu} \wedge dx^{\nu} \wedge dx^{\rho} + 3C_{p \nu \mu}dx^{\mu} \wedge dx^{\nu} \wedge dx^{p} \nonumber \\
&\phantom{=} + 3 C_{p n \mu}dx^{\mu} \wedge dx^{n} \wedge dx^{p} +C_{pnm}dx^{m} \wedge dx^{n} \wedge dx^{p}\Big{)}\nonumber \\
&= C' + B_{m}' \wedge \hat{e}^m + {1 \over 2}A^{2p} \wedge {\epsilon}_{mnp}\hat{e}^m \wedge \hat{e}^n +{a \over 6}{\epsilon}_{mnp}\hat{e}^m \wedge \hat{e}^n \wedge \hat{e}^p,
\label{olle}
\end{align}
where $a$ is a scalar. The reduction of the right side of (\ref{tja}) becomes
\begin{align}
d\hat{\chi}=&\underbrace{d\chi'-{\chi'}_m \wedge F^{1m}}_{\delta C'}+ \underbrace{(d{\chi'}_m + {\epsilon}_{mnp}{\chi'}^{2p} \wedge F^{1n})}_{\delta B_{m}'} \wedge \hat{e}^m \nonumber \\
&+{1 \over 2} \underbrace{d{\chi'}^{2p}}_{\delta A^{2p}} \wedge {\epsilon}_{mnp} \hat{e}^m \wedge \hat{e}^n.
\end{align}
From $\hat C$ we now get
\begin{align}
\hat{G}=d\hat{C}&= d(C' + B_{m}' \wedge \hat{e}^m + {1 \over 2}A^{2p} \wedge {\epsilon}_{mnp}\hat{e}^m \wedge \hat{e}^n + {a \over 6}{\epsilon}_{mnp}\hat{e}^m \wedge \hat{e}^n \wedge \hat{e}^p) \nonumber \\
&= \underbrace{dC' + B_{m}' \wedge F^{1m}}_{G} + \underbrace{(-dB_{m}' + A^{2p} {\epsilon}_{mnp} \wedge F^{1n})}_{-H_m} \wedge \hat{e}^m \nonumber \\
&\phantom{=}+ {1 \over 2} \underbrace{(dA^{2p} + aF^{1p})}_{F'^{2p}} {\epsilon}_{mnp} \wedge \hat{e}^m \wedge \hat{e}^n - {1 \over 6}(da){\epsilon}_{mnp}\hat{e}^m \wedge \hat{e}^n \wedge \hat{e}^p.
\label{kalle2}
\end{align}
So, to summarize we have the field strengths
\begin{align}
G &= dC' + B_{m}' \wedge F^{1m}, \nonumber \\
H_m &= dB_{m}' -  {\epsilon}_{mnp} F^{1n}\wedge A^{2p}, \nonumber \\
F'^{2m} &= dA^{2m} + aF^{1m} = F^{2m} + aF^{1m},
\label{hej}
\end{align}
satisfying the Bianchi identities
\begin{align}
dG &= d(dC' + B_{m}' \wedge F^{1m}) = dB'_m \wedge F^{1m} \nonumber \\
&= (H_m + {\epsilon}_{mnp} F^{1n} \wedge A^{2p}) \wedge F^{1m} = \Big{\{} {\epsilon}_{mnp} F^{1n} \wedge A^{2p} \wedge F^{1m} \nonumber \\
&\phantom{=}= -{\epsilon}_{mnp} F^{1n} \wedge A^{2p} \wedge F^{1m} =0 \Big{\}}= H_m \wedge F^{1m}, \nonumber \\
dH_m &= -{\epsilon}_{mnp} F^{1n} \wedge F^{2p}, \nonumber \\
dF'^{2m} &= da \wedge F^{1m}.
\label{dhf}
\end{align}

\subsubsection{The $T^4$ case}
To get the 7-dimensional field strengths we start by reducing $\hat{C}$ in exactly the same way as above, giving
\begin{align}
\hat{C} &= {1 \over {3!}} C_{PNM}dx^M \wedge dx^N \wedge dx^P \nonumber \\
&= C' + B_{m}' \wedge \hat{e}^m + {1 \over 2}A^{2pq} \wedge {\epsilon}_{mnpq}\hat{e}^m \wedge \hat{e}^n \nonumber \\
&\phantom{=} +{a^q \over 6}{\epsilon}_{mnpq}\hat{e}^m \wedge \hat{e}^n \wedge \hat{e}^p,
\end{align}
where $A^{2pq}$ is antisymmetric in $p$ and $q$. The gauge parameter becomes
\begin{align}
d\hat{\chi}=&\underbrace{d\chi'-{\chi'}_m \wedge F^{1m}}_{\delta C'}+ \underbrace{(d{\chi'}_m + {\epsilon}_{mnpq}{\chi'}^{2pq} \wedge F^{1n})}_{\delta B_{m}'} \wedge \hat{e}^m \nonumber \\
&+{1 \over 2} \underbrace{d{\chi'}^{2pq}}_{\delta A^{2pq}} {\epsilon}_{mnpq} \wedge \hat{e}^m \wedge \hat{e}^n,
\end{align}
and finally we get $\hat{G}$ as
\begin{align}
\hat{G}=d\hat{C}=& \underbrace{dC' + B_{m}' \wedge F^{1m}}_{G} + \underbrace{(-dB_{m}' + A^{2pq} {\epsilon}_{mnpq} \wedge F^{1n})}_{-H_m} \wedge \hat{e}^m \nonumber \\
&+ {1 \over 2} \underbrace{(dA^{2pq} + a^{[q} F^{1p]})}_{F'^{2pq}} {\epsilon}_{mnpq} \wedge \hat{e}^m \wedge \hat{e}^n \nonumber \\
&- {1 \over 6}(da^q){\epsilon}_{mnpq}\hat{e}^m \wedge \hat{e}^n \wedge \hat{e}^p.
\label{fieldsT4}
\end{align}
We end up with field strengths very similar to the 8-dimensional ones
\begin{align}
G &= dC' + B_{m}' \wedge F^{1m}, \nonumber \\
H_m &= dB_{m}' -  {\epsilon}_{mnpq} A^{2pq}\wedge F^{1n} , \nonumber \\
F'^{2mn} &= dA^{2mn} + a^{[n} F^{1m]} = F^{2mn} - a^{[m}F^{1n]},
\eqnlab{reduct_7d_fprim}
\end{align}
satisfying the Bianchi identities
\begin{align}
dG &= H_m \wedge F^{1m}, \nonumber \\
dH_m &= -{\epsilon}_{mnpq} F^{2pq} \wedge F^{1n}, \nonumber \\
dF'^{2mn} &= - da^{[m} \wedge F^{1n]}.
\eqnlab{reduct_7d_bianchi}
\end{align}

\section{The Chern-Simons term}
Having already reduced the 11-dimensional 3-form $\hat{C}$ and its 4-form field strength $\hat{G}$, it is not a difficult task to reduce the Chern-Simons term.
Again we will only keep terms with 2 or less $e^m$'s in the $T^2$ case, 3 or less in the $T^3$ case and so on.

\subsubsection{The 9-dimensional Chern-Simons term}
\begin{align}
&\phantom{=}{1 \over 6} \int \hat{G} \wedge \hat{G} \wedge \hat{C} \nonumber \\
&={1 \over 6} \int \Big{[} G \wedge G \wedge \epsilon_{mn} {1 \over 2}A^2 \wedge \hat{e}^m \wedge \hat{e}^n - 2 G \wedge H_m \wedge \hat{e}^m \wedge B_n \wedge \hat{e}^n \nonumber \\
&\phantom{=} + 2 G \wedge {1 \over 2} \epsilon_{mn} F^2 \wedge \hat{e}^m \wedge \hat{e}^n \wedge C + H_m \wedge \hat{e}^m \wedge H_n \wedge \hat{e}^n \wedge C \Big{]} \nonumber \\
&= \Big{\{} \hat{e}^m \wedge \hat{e}^n = d^2 x \sqrt{|G|}{\varepsilon}^{mn} = d^2 x \epsilon^{mn},\hspace{0.2cm} {\epsilon}_{mn} {\epsilon}^{mn} = 2 \Big{\}} \nonumber \\
&= {1 \over 6} \int \Big{[} G \wedge G \wedge A^2 - 2\epsilon^{mn} G \wedge H_m \wedge B_n +2 G \wedge F^2 \wedge C \nonumber \\
&\phantom{=} - \epsilon^{mn} H_m \wedge H_n \wedge C \Big{]}
\end{align}

\subsubsection{The 8-dimensional Chern-Simons term}
\begin{align}
&\phantom{=}{1 \over 6} \int \hat{G} \wedge \hat{G} \wedge \hat{C} \nonumber \\
&={1 \over 6} \int \Big{[} {\epsilon}_{mnp} G \wedge G {a \over 6} +{\epsilon}_{mnq} G \wedge H_p \wedge A^{2q} \nonumber \\
&\phantom{=}+ {\epsilon}_{mnq} G \wedge F'^{2q} \wedge B'_p + {1 \over 3} {\epsilon}_{mnp} G \wedge (da) \wedge C' \nonumber \\
&\phantom{=}- H_m \wedge H_n \wedge B'_p +  {\epsilon}_{mnq} H_p \wedge F'^{2q} \wedge C' \Big{]} \wedge \hat{e}^m \wedge \hat{e}^n \wedge \hat{e}^p \nonumber \\
&=\Big{\{} \hat{e}^m \wedge \hat{e}^n \wedge \hat{e}^p = d^3 x {\epsilon}^{mnp},\hspace{0.2cm} {\epsilon}_{mnp} {\epsilon}^{mnq} = 2\delta^{p}_q ,\hspace{0.2cm} {\epsilon}_{mnp}{\epsilon}^{mnp} = 6 \Big{\}} \nonumber \\
&={1 \over 6} \int \Big{[} G \wedge G a + 2G \wedge H_m \wedge A^{2m} + 2G \wedge F'^{2m} \wedge B'_m \nonumber \\
&\phantom{=}+ 2G \wedge (da) \wedge C' - {\epsilon}^{mnp}H_m \wedge H_n \wedge B'_p \nonumber \\
&\phantom{=}+2H_m \wedge F'^{2m} \wedge C' \Big{]},
\label{kjd}
\end{align}

\subsubsection{The 7-dimensional Chern-Simons term}
\begin{align}
&\phantom{=}{1 \over 6} \int \hat{G} \wedge \hat{G} \wedge \hat{C} \nonumber \\
&={1 \over 6} \int \Big{[} 2 \epsilon_{mnpq'} G \wedge H_q {a^{q'} \over 6} +2 \epsilon_{mnp'q'} \epsilon_{m'n'pq} G \wedge {1 \over 2} F'^{2p'q'} \wedge {1 \over 2} A^{2m'n'} \nonumber \\
&\phantom{=}-2 \epsilon_{mnpq'} G \wedge {1 \over 6}(da^{q'}) \wedge B'_q +\epsilon_{mnp'q'} H_p \wedge H_q \wedge {1 \over 2} A^{2p'q'} \nonumber \\
&\phantom{=}+2 \epsilon_{mnp'q'} H_q \wedge {1 \over 2}F'^{2p'q'} \wedge B'_p +2 \epsilon_{mnpq'} H_q \wedge {1 \over 6}(da^{q'}) \wedge C' \nonumber \\
&\phantom{=} +\epsilon_{mnp'q'} \epsilon_{m'n'pq} {1 \over 2}F'^{2p'q'} \wedge {1 \over 2}F'^{2m'n'} \wedge C' \Big{]} \wedge \hat{e}^m \wedge \hat{e}^n \wedge \hat{e}^p \wedge \hat{e}^q \nonumber \\
&=\Big{\{} \hat{e}^m \wedge \hat{e}^n \wedge \hat{e}^p \wedge \hat{e}^q= d^4 x {\epsilon}^{mnpq},\hspace{0.2cm} {\epsilon}_{mnp'q'} {\epsilon}^{mnpq} = 4\delta^{ \phantom{[} p \phantom{'} q}_{[p'q']} ,\nonumber \\
&\phantom{=}{\epsilon}_{mnpq'}{\epsilon}^{mnpq} = 6 \delta^q_{q'} \Big{\}} \nonumber \\
&={1 \over 6} \int \Big{[} 2G \wedge H_m a^{m} +2\epsilon_{mnpq}G \wedge F'^{2mn} \wedge A^{2pq} \nonumber \\
&\phantom{=}-2G \wedge (da^{m}) \wedge B'_m +2H_m \wedge H_n \wedge A^{2mn} \nonumber \\
&\phantom{=}+4 B'_m \wedge H_n \wedge F'^{2mn} +2 H_m \wedge (da^{m}) \wedge C' \nonumber \\
&\phantom{=}+ \epsilon_{mnpq} F'^{2mn} \wedge F'^{2pq} \wedge C' \Big{]}
\end{align}

\section{The $\hat{G}^2$ term and the symmetries of the actions}
The $\hat{G}^2$ term in \eqnref{sugra_11dim} will be reduced in the same way as the field strength in the previous section, i.e. by expanding the 4-form $\hat{G}$ 
in lower dimensional field strengths. We will also clarify how the remaining symmetries in 9-, 8-dimensional 
supergravity according to table 2.3 (i.e ${\mathbb R}^+$ for $T^2$ and $SL(2,{\mathbb R})$ for $T^3$) can be seen in the actions. 
(The ${\mathbb R}^+$ symmetry in 9-dimensional supergravity origins from $GL(2) \sim SL(2,{\mathbb R}) \times {\mathbb R}^+$.) Last but not least 
we will write the 7-dimensional Lagrangian in an $SL(5,{\mathbb R})$ covariant way.

\subsection{The 9-dimensional action}
We start by writing
\begin{equation}
\hat{G}={1 \over 4!}\hat{G}_{M_{1} \cdots M_{4}}dx^{M_{4}} \wedge \cdots \wedge dx^{M_{1}},
\label{kalle}
\end{equation}
and by comparing (\ref{kalle}) with (\ref{kalle1}) we find
\begin{align}
\hat{G}_{\mu \nu \rho \sigma}&=G_{\mu \nu \rho \sigma} \hspace{1.1cm} \mbox{1 combination}, \nonumber \\
\hat{G}_{\mu \nu \rho m}&=-H_m \hspace{1.15cm} \mbox{4 combinations}, \nonumber \\
\hat{G}_{\mu \nu mn}&=F^2 {\epsilon}_{mn} \hspace{1cm} \mbox{6 combinations},
\end{align}
hence
\begin{align}
{1 \over 48}\hat{G}_{M_{1} \cdots M_{4}} \hat{G}^{M_{1} \cdots M_{4}}&={1 \over 48}G^2 + {1 \over 12}H_m H^m + {1 \over 8}F^2 F^2 {\epsilon}_{mn} {\epsilon}^{mn} \nonumber \\
&= {1 \over 48}G^2 + {1 \over 12}H_m H^m + {1 \over 4}F^2 F^2.
\end{align}
When going to the Einstein frame we need to scale the fields as
\begin{align}
\sqrt{|g|} e^{-\varphi} &\rightarrow \sqrt{|g_E|}e^{-(ds+1)\varphi}, \nonumber \\
G^2 &\rightarrow e^{8s \varphi} G^2, \nonumber \\
H_m H^m &\rightarrow e^{6s\varphi}e^{2\varphi/D}\M^{mn}H_m H_n, \nonumber \\
F^2F^2 &\rightarrow e^{4s \varphi}F^2F^2, \nonumber \\
{\epsilon}_{mn} {\epsilon}^{mn} &\rightarrow e^{4\varphi/D} M^{pm} M^{qn} {\epsilon}_{mn} {\epsilon}_{pq}.
\label{ertyui}
\end{align}
The $G^2$ term in the Einstein frame then becomes
\begin{align}
S_{\hat G^2} &= - \int d^{11}x \sqrt{|\hat{g}|} \Big{[} \frac{1}{48} \hat G^2 \Big{]} \nonumber \\
& = - \int d^9x \sqrt{|g_E|}e^{-(ds+1)\varphi}\Big{[}\frac{1}{48}e^{8s\varphi}G^2+\frac{1}{12}e^{(6s+2/D)\varphi}H_mH_n\M^{mn} \nonumber \\
& \phantom{=}+ \frac{1}{4}e^{(4s+4/D)\varphi}(F^2)^2\Big{]}.
\end{align}
For the $T^2$ case we have $D=2$, $d=9$, $s=-1/7$ and we get the $G^2$ term as  
\begin{equation}
S_{\hat G^2} = - \int d^9x \sqrt{|g_E|}\left[\frac{1}{48}e^{-\frac{6}{7}\varphi}G^2+\frac{1}{12}e^{\frac{3}{7}\varphi}H_mH_n\M^{mn}+\frac{1}{4}e^{\frac{12}{7}\varphi}(F^2)^2\right].
\end{equation}
By writing the $G^2$ term together with the Einstein term and making a redefinition $\varphi \rightarrow (\sqrt{7}/3)\varphi$ (to get a factor 1/2 in front of the kinetic $(\partial\varphi)^2$ term) we get
\begin{align}
&\phantom{=}\int d^{11}x \sqrt{|\hat{g}|} (\hat{R}-{1 \over 48}\hat{G}^2) \nonumber \\
&=\int d^9x \sqrt{|g_E|} \Big{[} R_E - {1 \over 2}(\partial \varphi)^2-{1 \over 4}e^{-\frac{3}{\sqrt 7}\varphi}\M_{mn}F^{1m}F^{1n} \nonumber \\
&\phantom{=}-{1 \over 4}\mbox{Tr}(\M^{-1}\partial \M)^2 - {1 \over 48}e^{-\frac{2}{\sqrt 7}\varphi}G^2 - {1 \over 12}e^{\frac{1}{\sqrt 7}\varphi}H_mH_n\M^{mn} \nonumber \\
&\phantom{=}- {1 \over 4}e^{\frac{4}{\sqrt 7}\varphi}\M_{mn}(F^2)^2\Big{]}.
\label{tjolahopp}
\end{align}
A symmetric metric of unit determinant $\M$ parametrises the coset space SL(2,$\rr$)/SO(2), as was seen in subsection \ssecref{example_sl2so2}.
By performing an SL(2,$\rr$) transformation
\begin{align}
&\M_{mn}\rightarrow\Lambda{_m}^p \M_{pq} \Lambda{^q}_n,\;\; F^{1m}\rightarrow F^{1n}(\Lambda^{-1}){_n}^m,\;\; \nonumber \\
&H_m\rightarrow \Lambda{_m}^nH_m,\;\; \Lambda \in SL(2,\rr)
\end{align}  
to the terms in the Lagrangian, the SL(2,$\rr$) scalars
\begin{align}
&F^{1T}\M F^{1}\rightarrow F^{1T}\Lambda^{-1}\Lambda \M
\Lambda^T\Lambda^{T-1}F^1 = F^{1T}\M F^{1}\nonumber\\
&H^{T}\M^{-1}H\rightarrow H^{T}\Lambda^T(\Lambda \M \Lambda^T)^{-1}\Lambda H = H^{T}\M^{-1}H 
\end{align}
are invariant and also the scalar kinetic term
\begin{align}
&\frac{1}{4}\mbox{Tr}(\M^{-1}\partial \M)^2 = -\frac{1}{4}\mbox{Tr}(\partial \M^{-1}\partial \M) \nonumber \\
&\rightarrow %\frac{1}{4}\mbox{Tr}(\partial((\Lambda \M\Lambda^T)^{-1})\partial(\Lambda \M\Lambda^T)) = 
-\frac{1}{4}\mbox{Tr}(\Lambda^{T-1}\partial \M^{-1} \Lambda^{-1}\Lambda \partial \M \Lambda^T) = \frac{1}{4}\mbox{Tr}(\M^{-1}\partial \M)^2,
\eqnlab{sugra_kinetic_transform}
\end{align}
where we have used \eqnref{conven_kinetic} and that $\Lambda$ is constant so $\partial\Lambda = 0$, is invariant. 
The Lagrangian is also invariant under an $R^+$ scaling symmetry.
To see this we assign, to each field in the Lagrangian, a scale transformation $\lambda^\alpha$, where $\lambda$ is an arbitrary real constant and the exponent $\alpha$ is different for each different field.
By choosing the $\alpha$:s according to table (\ref{reduct_weights}) and by noting that $e^{x\varphi}$ scales as $2x\sqrt 7$, one can easily check that the Lagrangian (\ref{tjolahopp}) is invariant under the ${\rr}^+$ symmetry.
Consequently we have found that the Lagrangian is invariant under $\rr^+\times$SL(2,$\rr$)/SO(2)$\sim$GL(2,$\rr$)/SO(2) transformations, which is consistent with the expected data for d=9 in table 2.3.

\begin{table}
\begin{center}
\begin{tabular}{l|l l l l l l l}
Field & $g_{\mu\nu}$ & $A_\mu^{1m}$ & $A_\mu^2$ & $B_{\mu\nu m}$ & $C_{\mu\nu\rho}$ & $\phi^m$ & $e^\varphi$\\ \hline
$\rr^+$ scaling $\alpha$ & 0 & $3$ & $-4$ & $-1$ & $2$ & 0 & $2\sqrt 7$
\end{tabular}
\label{reduct_weights}
\caption{The scaling exponents of the ${\rr}^+$ symmetry for the different fields. $\phi^m$ denotes the 2 coset scalars in $\M$.} 
\end{center}
\end{table}

\subsection{The 8-dimensional action}

Again we start with
\begin{equation}
\hat{G}={1 \over 4!}\hat{G}_{M_{1} \cdots M_{4}}dx^{M_{4}} \wedge \cdots \wedge dx^{M_{1}},
\end{equation}
to identify the terms in (\ref{kalle2}) with
\begin{align}
\hat{G}_{\mu \nu \rho \sigma}&=G_{\mu \nu \rho \sigma} \hspace{2.03cm} \mbox{1 combination}, \nonumber \\
\hat{G}_{\mu \nu \rho m}&=-H_m \hspace{2.1cm} \mbox{4 combinations}, \nonumber \\
\hat{G}_{\mu \nu mn}&=F'^{2p} {\epsilon}_{mnp} \hspace{1.55cm} \mbox{6 combinations}, \nonumber \\
\hat{G}_{\mu mnp}&=-({\partial}_{\mu}a){\epsilon}_{mnp} \hspace{1cm} \mbox{4 combinations},
\end{align}
which gives
\begin{align}
{1 \over 48}\hat{G}_{M_{1} \cdots M_{4}} \hat{G}^{M_{1} \cdots M_{4}}&={1 \over 48}G^2 + {1 \over 12}H_m H^m + {1 \over 8}F'^{2p} F'{^2}_q {\epsilon}_{mnp} {\epsilon}^{mnq} \nonumber \\
&\phantom{=}+ {1 \over 12}({\partial}_{\mu}a)({\partial}^{\mu}a){\epsilon}_{mnp}{\epsilon}^{mnp} \nonumber \\
&= \Big{\{} {\epsilon}_{mnp} {\epsilon}^{mnq} = 2\delta^{p}_q ,\hspace{0.2cm} {\epsilon}_{mnp}{\epsilon}^{mnp} = 6 \Big{\}} \nonumber \\
&= {1 \over 48}G^2 + {1 \over 12}H_m H^m + {1 \over 4}F'^{2p} F'{^2}_p + {1 \over 2}({\partial}_{\mu}a)({\partial}^{\mu}a).
\end{align}
To get the action in the Einstein frame we can use (\ref{ertyui}) together with
\begin{align}
F'^2_m F'^{2m} &\rightarrow e^{4s \varphi}e^{-2\varphi /D}\M_{mn}F'^{2m}F'^{2n}, \nonumber \\
(\partial a)^2 &\rightarrow e^{2s\varphi} (\partial a)^2, \nonumber \\
{\epsilon}_{mnp}{\epsilon}^{mnp} &\rightarrow e^{6\varphi /D}\M^{mq}\M^{nr}\M^{ps} {\epsilon}_{mnp}{\epsilon}_{qrs},
\end{align}
where we have used (\ref{varchange2}). We now have $D=3$, $d=8$ and $s=-1/6$, and we get the action in the Einstein frame as
\begin{align}
S_{\hat G^2} =& \int d^8x \sqrt{|g|} \Big{[} {1 \over 48}e^{-\varphi}G^2 + {1 \over 12}H_mH_n\M^{mn} \nonumber \\
&+ {1 \over 4}e^{\varphi}\M_{mn}F'^{2m}F'^{2n} + {1 \over 2}e^{2\varphi}(\partial a)^2 \Big{]}.
\end{align}
Writing the Einstein term together with the $G^2$ term gives
\begin{align}
&\phantom{=}\int d^{11}x \sqrt{|\hat{g}|} (\hat{R}-{1 \over 48}\hat{G}^2) \nonumber \\
&= \int d^8x \sqrt{|g_E|} \Big{[} R_E - {1 \over 2}(\partial \varphi)^2-{1 \over 4}e^{-\varphi}\M_{mn}F^{1m}F^{1n} -{1 \over 4}\mbox{Tr}(\M^{-1}\partial \M)^2 \nonumber \\
&\phantom{=}- {1 \over 48}e^{-\varphi}G^2 - {1 \over 12}H_mH_n\M^{mn} - {1 \over 4}e^{\varphi}\M_{mn}F'^{2m}F'^{2n} - {1 \over 2}e^{2\varphi}(\partial a)^2 \Big{]}.
\label{hej2}
\end{align}

\subsubsection{SL(2) $\times$ SL(3) parametrisation}

We now want to formulate the action in a more $SL(2,{\mathbb R}) \times
SL(3,{\mathbb R})$ covariant way (see table 2.3). 
We define the metric
\begin{equation}
\W= {1 \over \mbox{Im} (\tau) }\toto{|\tau|^2 }{ \mbox{Re}(\tau) }{ \mbox{Re}(\tau) }{ 1} = e^{\varphi}\toto{a^2 + e^{-2\varphi} }{ a }{ a }{ 1},
\end{equation}
with
\begin{equation}
\tau = a + \mbox{i} e^{-\varphi}
\end{equation}
parametrising the $SL(2,{\mathbb R})/SO(2)$ coset. We get
\begin{equation}
\W^{-1}={1 \over \mbox{Im} (\tau) }\toto{ 1}{ -\mbox{Re}(\tau) }{ -\mbox{Re}(\tau) }{ |\tau|^2}=e^{\varphi}\toto{ 1}{ -a }{ -a }{ a^2 + e^{-2\varphi}}
\end{equation}
and
\begin{equation}
\partial \W=(\partial \varphi) \W + e^{\varphi}\toto{ 2a\partial a - 2\partial \varphi e^{-2\varphi}}{ \partial a }{ \partial a }{ 0},
\end{equation}
hence
\begin{align}
\partial \W \W^{-1}=& \hspace{0.15cm} \id \partial \varphi+ \nonumber \\
&  + e^{2\varphi}\toto{a\partial a - 2\partial \varphi e^{-2\varphi} }{ -a^2 \partial a + 2a\partial \varphi e^{-2\varphi} + \partial a e^{-2\varphi} }{ \partial a }{ -a \partial a} \nonumber \\
=& \hspace{0.15cm} e^{2\varphi}\toto{a\partial a - \partial \varphi e^{-2\varphi} }{ -a^2 \partial a + 2a\partial \varphi e^{-2\varphi} + \partial a e^{-2\varphi} }{ \partial a }{ -a \partial a + \partial \varphi e^{-2\varphi}}.
\end{align}
Finally we get
\begin{align}
\mbox{Tr}(\partial \W \W^{-1})^2 &= e^{4 \varphi}\Big{(}2(a\partial a - \partial \varphi e^{-2\varphi})^2 + 2(-a^2(\partial a)^2 + 2a\partial a\partial \varphi e^{-2 \varphi} \nonumber \\
&\phantom{=}+ (\partial a)^2 e^{-2 \varphi}) \Big{)}= 2\Big{(}(\partial \varphi)^2 + (\partial a)^2 e^{2 \varphi}\Big{)}.
\label{hej3}
\end{align}
Now look at the term
\begin{align}
F^{rm}F^{sn}\W_{rs} &= (a^2 e^{\varphi} + e^{-\varphi})F^{1m}F^{1n} + 2ae^{\varphi}F^{1m}F^{2n} +e^{\varphi}F^{2m}F^{2n} \nonumber \\
&= e^{\varphi}(aF^{1m}F'^{2n} + F'^{2m}F^{2n}) + e^{-\varphi}F^{1m}F^{1n} \nonumber \\
&= e^{\varphi}F'^{2m}F'^{2n} + e^{-\varphi} F^{1m}F^{1n},
\label{hej4}
\end{align}
where we have repeatedly used (\ref{hej}). Inserting (\ref{hej3}) and (\ref{hej4}) in (\ref{hej2}) yields
\begin{align}
&\int d^{11}x \sqrt{|\hat{g}|} (\hat{R}-{1 \over 48}\hat{G}^2) \nonumber \\
&= \int d^8x \sqrt{|g_E|} \Big{[} R_E - {1 \over 4}\mbox{Tr}(\partial \W \W^{-1})^2 -{1 \over 4}\mbox{Tr}(\M^{-1}\partial \M)^2 \nonumber \\
&\phantom{=}- {1 \over 4}\M_{mn}F^{rm}F^{sn}\W_{rs} - {1 \over 12}H_mH_n\M^{mn}- {1 \over 48}e^{-\varphi}G^2 \Big{]}.
\label{tjena}
\end{align}
We can now explicitly verify that the action (apart from the $G^2$ term that will be treated later) is invariant under the
following $SL(2,{\mathbb R})$ and $SL(3,{\mathbb R})$ transformations:
\begin{equation}
\W \rightarrow \Lambda \W \Lambda^{T}, \hspace{0.5cm} F^{m} \rightarrow (\Lambda^{T})^{-1}F^m, \hspace{0.5cm} \Lambda \in SL(2),
\end{equation}
\begin{equation}
\M \rightarrow R\M R^T, \hspace{0.5cm} H_m \rightarrow R{_m}^n H_m, \hspace{0.5cm} F^m \rightarrow F^n (R^{-1}){_n}^m, \hspace{0.5cm} R \in SL(3).
\label{sl3trans}
\end{equation}
To see that term 4 and 5 in (\ref{tjena}) are inert we only need to do the short calculations
\begin{equation}
H^T \M^{-1}H \rightarrow H^T R^T R^{-T}\M^{-1}R^{-1}RH = H^T \M^{-1}H,
\end{equation}
\begin{align}
F^T(\M \otimes \W)F \rightarrow & \hspace{0.15cm} F^T R^{-1} R\M R^T R^{-T}F \otimes F^T \Lambda^{-1} \Lambda \W {\Lambda}^{T} {\Lambda}^{-T}F \nonumber \\
&=F^T(\M \otimes \W)F,
\end{align}
whilst for term 2 and 3 we can reuse the shown transformation symmetry of the scalar kinetics in \eqnref{sugra_kinetic_transform}.

\subsubsection{The $G$-doublet}

We see in \ref{tjena} that there is no $SL(2,{\mathbb R})$ symmetry for the $G^2$-term. This is because the $SL(2,{\mathbb R})$ 
symmetry is not a symmetry of the whole action, but we will show below that by using a 4-form coming from the dual of $\hat{G}$ 
we can create an $SL(2,{\mathbb R})$-doublet and derive $SL(2,{\mathbb R})$ covariant Bianchi identities for $H$ and $G$.
But before we do this we will first redefine the potentials to get the field strengths written in an $SL(2,{\mathbb R})$ 
invariant way. The way that we have defined $H_m$ in (\ref{dhf}) is clearly not $SL(2,{\mathbb R})$ covariant, 
a fact which is easily remedied by the redefinition
\begin{equation}
B'_m = B_m - {1 \over 2}{\epsilon}_{mnp}A^{1n} \wedge A^{2p},
\end{equation}
which implies that
\begin{align}
H_m=dB'_m &= dB_m - {1 \over 2}{\epsilon}_{mnp}(F^{1n} \wedge A^{2p} - F^{2n} \wedge A^{1p}) \nonumber \\
&=dB_m - {1 \over 2}{\epsilon}_{mnp}{\epsilon}_{rs}F^{rn} \wedge A^{sp},
\end{align}
where $r,s = 1,2$. This gives the $SL(2,{\mathbb R})$ covariant Bianchi identity
\begin{equation}
dH_m=-{1 \over 2}{\epsilon}_{mnp}{\epsilon}_{rs}F^{rn} \wedge F^{sp}.
\end{equation}
It is also convenient to redefine $C'$ as
\begin{equation}
C'=C - {1 \over 3}A^{1m} \wedge B_m + {1 \over 6}{\epsilon}_{mnp}A^{1m} \wedge A^{2n} \wedge A^{1p},
\end{equation}
which gives
\begin{align}
G &= dC -{1 \over 3}dA^{1m} \wedge B_m - {1 \over 3}A^{1m} \wedge dB_m +{1 \over 6}{\epsilon}_{mnp}dA^{1m} \wedge A^{2n} \wedge A^{1p} \nonumber \\
&\phantom{=}-{1 \over 6}{\epsilon}_{mnp}A^{1m} \wedge dA^{2n} \wedge A^{1p} + {1 \over 6}{\epsilon}_{mnp}A^{1m} \wedge A^{2n} \wedge dA^{1p} \nonumber \\
&\phantom{=}+(B_m - {1 \over 2}{\epsilon}_{mnp}A^{1n} \wedge A^{2p})\wedge F^{1m} \nonumber \\
&=dC + {2 \over 3}B_m \wedge F^{1m} - {1 \over 3}A^{1m} \wedge dB_m \nonumber \\
&\phantom{=}+({1 \over 6}+{1 \over 6}-{1 \over 2}){\epsilon}_{mnp}A^{1m} \wedge A^{2n} \wedge F^{1p} + {1 \over 6}{\epsilon}_{mnp}A^{1m} \wedge A^{1n} \wedge F^{2p} \nonumber \\
&=dC + {2 \over 3}B_m \wedge F^{1m} \nonumber \\
& \phantom{=}+ {1 \over 3}A^{1m} \wedge \Big{(}dB_m + {1 \over 2}{\epsilon}_{mnp}(A^{1n} \wedge F^{2p} - A^{2n} \wedge F^{1p}) \Big{)} \nonumber \\
&=dC + {2 \over 3}B_m \wedge F^{1m} + {1 \over 3}A^{1m} \wedge H_m
\end{align}
Note that these redefinitions do not affect the corresponding Bianchi identities.
Unfortunately they do affect the invariance under reparametrisations along the internal 3-torus as one can immediately see in the definitions above.
This invariance is lost for the $B$ and $C$ fields, an unavoidable consequence of imposing manifest $SL(2,{\mathbb R})$ covariance.

Next it is finally time to deal with the $G^2$ term. We will try to write $G$ as a $SL(2,{\mathbb R})$ doublet by using the dual to $\hat{G}$.
The dual is calculated by $\hat{G}_7=*\hat{G}$, where $*$ is the Hodge dual operator (see \eqnref{conven_hodge_form}) and hence $\hat{G}_7$ is a 7-form.
When reducing the 11-dimensional 7-form we will receive a 4-form which is dual to the 4-form $G$. Expanding $\hat{G}_7$ gives
\begin{align}
\hat{G}_7 &= {1 \over 7!} \hat{G}_{7M{_7} \cdots M_{1}} dx^{M_{1}} \wedge \cdots \wedge dx^{M_{7}} \nonumber \\
&={1 \over 7!}\Big{(}G_{7 {\mu}_7 \cdots {\mu}_1} dx^{{\mu}_{1}} \wedge \cdots \wedge dx^{{\mu}_{7}} + 7G_{7 m_7 {\mu}_6 \cdots {\mu}_1} dx^{{\mu}_{1}} \wedge \cdots \wedge dx^{{\mu}_{6}} \wedge dx^{m_7} \nonumber \\
&\phantom{=} +{7 \cdot 6 \over 2!}G_{7 m_7 m_6 {\mu}_5 \cdots {\mu}_1}dx^{{\mu}_{1}} \wedge \cdots \wedge dx^{{\mu}_{5}} \wedge dx^{m_6} \wedge dx^{m_7} \nonumber \\
&\phantom{=} +{7 \cdot 6 \cdot 5 \over 3!}G_{7 m_7 m_6 m_5 {\mu}_4 \cdots {\mu}_1}dx^{{\mu}_{1}} \wedge \cdots \wedge dx^{{\mu}_{4}} \wedge dx^{m_5} \wedge dx^{m_6} \wedge dx^{m_7} \Big{)} \nonumber \\
\intertext{}
&= \underbrace{{1 \over 7!}G_{7 {\mu}_7 \cdots {\mu}_1} dx^{{\mu}_{1}} \wedge \cdots \wedge dx^{{\mu}_{7}}}_{G_7} + \underbrace{({1 \over 6!}G_{7{\mu}_6 \cdots {\mu}_1} dx^{{\mu}_{1}} \wedge \cdots \wedge dx^{{\mu}_{6}})_{m}}_{-G_{6m}} \wedge dx^m \nonumber \\
&\phantom{=} +{1 \over 2!} \underbrace{({1 \over 5!}G_{7{\mu}_5 \cdots {\mu}_1} dx^{{\mu}_{1}} \wedge \cdots \wedge dx^{{\mu}_{5}})_{nm}}_{-G_{5nm}} \wedge dx^m \wedge dx^n \nonumber \\
&\phantom{=} +{1 \over 3!} \underbrace{({1 \over 4!}G_{7{\mu}_4 \cdots {\mu}_1}dx^{{\mu}_{1}} \wedge \cdots \wedge dx^{{\mu}_{4}})_{pnm}}_{G'{\epsilon}_{pnm}} \wedge dx^m \wedge dx^n  \wedge dx^p \nonumber \\
&= G_7 - G_{6m} \wedge \hat{e}^m + {1 \over 2!} G_{5mn} \wedge \hat{e}^m \wedge \hat{e}^n - {1 \over 3!}G' \wedge {\epsilon}_{mnp} \hat{e}^m \wedge \hat{e}^n \wedge \hat{e}^p,
\label{g7}
\end{align}
where the 4-form $G'$ is dual to $G$ and the signs has been chosen so that the corresponding expansion of $\hat C_{(6)}$ has positive signs. To form the $SL(2,{\mathbb R})$ doublet we need the Bianchi identity for $\hat{G}_7$
\begin{equation}
d(*\hat{G})=d{\hat{G}_7}=-{1 \over 2}\hat{G} \wedge \hat{G},
\label{bianchig7}
\end{equation}
which was derived in \eqnref{bianchig7}. We can now use (\ref{bianchig7}) together with (\ref{g7}) to get
\begin{align}
d\hat{G}_7 &= \cdots +{1 \over 3!} dG' \wedge {\epsilon}_{mnp}
\hat{e}^m \wedge \hat{e}^n \wedge \hat{e}^p + \cdots = dG' d^Dy + \cdots\nn\\
& = -{1 \over 2}\hat{G} \wedge \hat{G}
= -{1 \over 2}\Big( \cdots -{2 \over 3!} G \wedge da \wedge
{\epsilon}_{mnp}\hat{e}^m \wedge \hat{e}^n \wedge \hat{e}^p \nonumber
\\
&\phantom{=}-{2 \over 2!} H_q \wedge \hat{e}^q \wedge F'^{2p} {\epsilon}_{mnp}\wedge \hat{e}^m \wedge \hat{e}^n + \cdots\Big) \nonumber \\
&= \big( G \wedge da + H_m\wedge F'^{2m}\big) d^Dy + \cdots
\end{align}
We thus read off the Bianchi identity for $G'$ as
\begin{equation}
dG'=G \wedge da + H_m \wedge F'^m = G \wedge da + H_m \wedge F^{2m} + H_m \wedge F^{1m}a.
\end{equation}
The next step is to make the definition
\begin{equation}
G' = -e^{-\varphi}(*_8 G),
\end{equation}
where $*_8$ is the 8-dimensional Hodge dual operator. This way of defining $G'$ is however not 
suitable for the construction of the $SL(2,{\mathbb R})$ doublet. Instead we make another definition
\begin{equation}
\tilde{G}=G'-aG=-e^{-\varphi}(*_8 G)-aG,
\end{equation}
with the Bianchi identity
\begin{equation}
d\tilde{G}=H_m \wedge F^{2m}.
\end{equation}
Hence, with the definitions $G^1 = G$, $G^2 = \tilde{G}$, $r = 1,2$ and $dA^{2m}=F^{2m}$, we get
\begin{equation}
dG^r=H_m \wedge F^{rm},
\end{equation}
and we also define
\begin{equation}
G^r=dC^r+{2 \over 3}B_m \wedge F^{rm} - {1 \over 3}A^{rm} \wedge H_m.
\end{equation}
Finally, by looking at the components of the dual of $G$ we find
\begin{align}
*_8G^1&=e^{-\varphi}G'=e^{-\varphi}(G^2+aG^1)=-\W_{21}G^1-\W_{22}G^2 \nonumber \\
&=-\W_{2t}G^t, \\
*_8G^2&=*_8(G'-aG^1)=e^{-\varphi}G^1+ae^{\varphi}G^2+a^2e^{\varphi}G^1=\W_{11}G^1+\W_{12}G^2 \nonumber \\
&=\W_{1t}G^t, \\
\Rightarrow & *_8G^r=-\epsilon^{rs}\W_{st}G^t.
\end{align}
The redefinitions of the potentials also affects the transformations that leaves the field strengths invariant. 
The new gauge transformations becomes:
\begin{align}
&\delta A^{rm}=d\chi^{rm}, \hspace{1cm} \delta B_m =d\chi_m -{1 \over 2}\epsilon_{mnp}\epsilon_{rs}A^{rn} \wedge d\chi^{sp}, \nonumber \\
&\delta C^r = d\chi^r -{2 \over 3}A^{rm} \wedge d\chi_m +{1 \over 3}B_m \wedge d\chi^{rm} + {1 \over 6}\epsilon_{mnp}\epsilon_{st}A^{rm} \wedge A^{sn} \wedge d\chi^{tp},
\end{align}
and the calculations
\begin{align}
\delta F^{rm}&=d \delta A^{rm} = d^2 \chi^{rm}=0, \\
\delta H_m&=d\delta B_m - {1 \over 2}\epsilon_{mnp}\epsilon_{rs}F^{rn} \wedge \delta A^{sp} \nonumber \\
&={1 \over 2}\epsilon_{mnp}\epsilon_{rs}(F^{rn}\wedge d\chi^{sp}-F^{rn}\wedge d\chi^{sp})=0, \\
\intertext{}
\delta G^r&=d \delta C^r + {2 \over 3} \delta B_m \wedge F^{rm} - {1 \over 3}\delta A^{rm} \wedge H_m =-{2 \over 3}F^{rm} \wedge d\chi_m \nonumber \\
&\phantom{=}-{1 \over 3}dB_m \wedge d\chi^{rm}-{1 \over 6}\epsilon_{mnp}\epsilon_{st}d(A^{rm} \wedge A^{sn}) \wedge d\chi^{tp} \nonumber \\
&\phantom{=}+{2 \over 3} d\chi_m \wedge F^{rm}-{1 \over 6}\epsilon_{mnp}\epsilon_{st}A^{sn} \wedge d\chi^{tp} \wedge F^{rm} +{1 \over 3}H_m \wedge d\chi^{rm} \nonumber \\
&=-{1 \over 6}\epsilon_{mnp}\epsilon_{st}\Big{(}A^{rm} \wedge F^{sn} \wedge d\chi^{tp} - A^{sm} \wedge F^{rn} \wedge d\chi^{tp} \nonumber \\
&\phantom{=}+ F^{sn} \wedge A^{tp} \wedge d\chi^{rm}\Big{)} \nonumber \\
&= \Big{\{} \mbox{Use } \epsilon_{mnp}\epsilon_{st}(A^{rm} \wedge F^{sn} \wedge d\chi^{tp} + F^{sn} \wedge A^{tp} \wedge d\chi^{rm}) \nonumber \\
&\phantom{=}= -\epsilon_{mnp}\epsilon_{st}(A^{sm} \wedge F^{rn} \wedge d\chi^{tp} + 2F^{rm} \wedge A^{sn} \wedge d\chi^{tp}), \nonumber \\
&\phantom{=}\mbox{(calculate each component separate)}\Big{\}} \nonumber \\
&={1 \over 3}\epsilon_{mnp}\epsilon_{st}(A^{sm} \wedge F^{rn} + A^{sn} \wedge F^{rm})\wedge d\chi^{tp}=0,
\end{align}
shows that the field strengths are invariant.






\subsection{The 7-dimensional action}

We will calculate the $\hat{G}^2$-term for $T^4$ in exactly the same way as for $T^2$ and $T^3$, hence we start with (\ref{fieldsT4}) to identify $\hat{G}^2$ with
\begin{align}
&\phantom{=}{1 \over 48}\hat{G}_{M_{1} \cdots M_{4}} \hat{G}^{M_{1} \cdots M_{4}} \nonumber \\
&= {1 \over 48}G^2 + {1 \over 12}H_m H^m + {1 \over 8}F'^{2}_{pq}F'^{2p'q'}{\epsilon}_{mnp'q'}{\epsilon}^{mnpq} \nonumber \\
&\phantom{=}+{1 \over 12}({\partial}^{\mu} a^q)({\partial}_{\mu}a^{q'})G_{q'q''}{\epsilon}_{mnpq}{\epsilon}^{mnpq''} \nonumber \\
&= \Big{\{} {\epsilon}_{mnp'q'}{\epsilon}^{mnpq}=4 \delta^{ \phantom{[} p \phantom{'} q}_{[p'q']}, {\epsilon}_{mnpq}{\epsilon}^{mnpq''}=3! \delta^{q''}_{q} \Big{\}} \nonumber \\
&= {1 \over 48}G^2 + {1 \over 12}H_m H^m + {1 \over 2}F'^{2}_{mn}F'^{2mn} + {1 \over 2}({\partial}^{\mu} a^{q'})({\partial}_{\mu}a^{q})G_{qq'}.\phantom{iiiiii}
\end{align}
Switching to Einstein frame requires the scaling
\begin{align} 
F'^2_{mn} F'^{2mn} &\rightarrow e^{4s \varphi}e^{-4\varphi /D}\M_{mm'}\M_{nn'}F'^{2m'n'}F'^{2mn}, \nonumber \\
({\partial}_{\mu} a^q)({\partial}^{\mu} a^{q'})G_{qq'} &\rightarrow e^{2s\varphi} e^{-2\varphi /D} ({\partial}_{\mu} a^q)({\partial}^{\mu} a^{q'}) \M_{qq'}, \nonumber \\
{\epsilon}_{mnpq}{\epsilon}^{mnpq} &\rightarrow e^{8\varphi /D}\M^{mm'}\M^{nn'}\M^{pp'}\M^{qq'} {\epsilon}_{mnpq}{\epsilon}_{m'n'p'q'},\phantom{iiii}
\end{align}
together with (\ref{ertyui}). Recognizing $D=4$, $d=7$ and $s=-1/5$ we find
\begin{align}
S_{\hat G^2} =& \int d^7x \sqrt{|g|} \Big{[} {1 \over 48}e^{-{6 \over 5}\varphi}G^2 + {1 \over 12}e^{-{3 \over 10}\varphi}H_mH_n\M^{mn} \nonumber \\
&+ {1 \over 2}e^{{3 \over 5}\varphi}\M_{mm'}\M_{nn'}F'^{2m'n'}F'^{2mn} + {1 \over 2}e^{{3 \over 2}\varphi}({\partial}_{\mu}a^{q})({\partial}^{\mu}a^{q'})\M_{qq'} \Big{]}
\end{align}
and writing $\hat{G}^2$ together with $R$ and a redefinition $\varphi \rightarrow (\sqrt{10}/3) \varphi$ gives
\begin{align}
&\int d^{11}x \sqrt{|\hat{g}|} (\hat{R}-{1 \over 48}\hat{G}^2) \nonumber \\
&= \int d^7x \sqrt{|g_E|} \Big{[} R_E - {1 \over 2}(\partial \varphi)^2-{1 \over 4}e^{-{3 \over \sqrt{10}}\varphi}\M_{mn}F^{1m}F^{1n} \nonumber \\
&\phantom{=} -{1 \over 4}\mbox{Tr}(\M^{-1}\partial \M)^2 - {1 \over 48}e^{{-4 \over \sqrt{10}}\varphi}G^2 - {1 \over 12}e^{-{1 \over \sqrt{10}} \varphi}H_mH_n\M^{mn} \nonumber \\
&\phantom{=}- {1 \over 2}e^{{2 \over \sqrt{10}} \varphi}\M_{mm'}\M_{nn'}F'^{2m'n'}F'^{2mn} - {1 \over 2}e^{{\sqrt{10} \over 2} \varphi}({\partial}_{\mu} a^q)({\partial}^{\mu} a^{q'})\M_{qq'} \Big{]}.
\eqnlab{reduct_7d}
\end{align}


\subsubsection{SL(5)/SO(5) parametrisation}
From table 2.3 we find that the U-duality coset in d=7 is \coset{5}. We thus want to rewrite the Lagrangian \eqnref{reduct_7d} in an SL(5,$\rr$) covariant way with the help of an \coset{5} parametrisation matrix $\W$. 
It makes sense to start with the scalars.

\subsubsection{The scalar terms}
The scalar part of the Lagrangian \eqnref{reduct_7d} is 
\begin{equation}
\Lagr_{scalar}^{d=7} = -\frac{1}{2}(\partial\varphi)^2-\frac{1}{4}\tr(\M^{-1}\partial \M)^2-\frac{1}{2}e^{\frac{\sqrt{10}}{2}\varphi}\partial a^m\M_{mn}\partial a^n.
\eqnlab{sl5so5_scalar_lagr}
\end{equation}
We note that there is a total of 14 scalar fields out of which 4 are dilatons ($\varphi$ and three in $\M$) and 10 are axions ($a\times 4$ and six lives in $\M$), in this Lagrangian, which is precisely the number and types of scalars in $\W$, as we saw in subsection \ssecref{ex_sl5so5}.  
So the game is now to include all the scalars in $\W$ and hope that it is possible to rewrite the scalar Lagrangian as one single kinetic scalar term
\begin{equation}
\Lagr_{scalar}^{d=7} = -\frac{1}{4}\tr(\W^{-1}\partial \W)^2 = \frac{1}{4}\tr(\partial\W^{-1}\partial \W).
\end{equation}
By splitting the coset metrics $\M$ and $\W$ into upper diagonal vielbeins
\begin{align}
\M &= \N\N^T,&\M\in \mbox{\coset{4}}\nonumber\\
\W &= \V\V^T,&\W\in \mbox{\coset{5}}
\end{align}
we can rewrite the Lagrangian as
\begin{align}
\Lagr_{scalar}^{d=7} &= -\frac{1}{4}\tr\left(\W^{-1}\partial \W\right)^2 = -\frac{1}{4}\tr\left(\V^{T-1}\V^{-1}(\partial\V\V^T+\V\partial\V^T)\right)^2\nonumber\\
&=-\frac{1}{4}\tr\left(\V^{-1}\partial\V\V^{-1}\partial\V+(\V^{-1}\partial\V\V^{-1}\partial\V)^T+2\partial\V^T\V^{T-1}\V^{-1}\partial\V \right)\nonumber\\
&=-\frac{1}{2}\tr\left(\V^{-1}\partial\V\V^{-1}\partial\V+\V^{-1}\partial\V(\V^{-1}\partial\V)^T\right)\nonumber\\
&=-\frac{1}{2}\tr\left(\V^{-1}\partial\V(\V^{-1}\partial\V+(\V^{-1}\partial\V)^T)\right).
\eqnlab{scalar_kinetic_simp}
\end{align}
From the information in \eqnref{sl5so5_scalar_lagr} we deduce that $\V$ must contain the dilaton $\varphi$ and the four axions $a^m$ together with the \coset{4} vielbein $\N$.
We note that the Lagrangian \eqnref{sl5so5_scalar_lagr} is invariant under a redefinition $\V\rightarrow\V^{T-1}$ coming from the symmetry between $\M$ and $\M^{-1}$ in the $\tr(\partial M^{-1}\partial M)$-term.
We thus make the Ansatz
\begin{equation} 
\U=\toto{\T f(a^m,\varphi)}{g(a^m,\varphi)}{0}{h(a^m,\varphi)}=
\setlength{\unitlength}{.4mm}
\left(\begin{array}{l}\cr\mbox{}\end{array}\right.
\begin{picture}(23,30)(8,10)% (size)(offset)
\put(0,0){
\path(0,7)(20,7)(20,27)(0,27)(0,7)
\path(0,0)(20,0)(20,5)(0,5)(0,0)
\path(22,7)(27,7)(27,27)(22,27)(22,7)
\path(22,0)(27,0)(27,5)(22,5)(22,0)
\put(10,17){\makebox(0,0){\tiny 4$\times$4}}
\put(10,2.5){\makebox(0,0){\tiny 1$\times$4}}
\put(35,17){\makebox(0,0){\tiny 4$\times$1}}
\put(35,2.5){\makebox(0,0){\tiny 1$\times$1}}
\path(25,17)(29,17)
\path(25,2.5)(29,2.5)
}
\end{picture}
\left.\begin{array}{l}\cr\mbox{}\end{array}\right)
\end{equation}
where f and h are scalar functions of the fields, g is a 4-dimensional vector function of the fields, $\U$ is either $\V$ or $\V^{T-1}$, which can be chosen for convenience later and $\T$ is similarly either $\N$ or $\N^{T-1}$. 
We have thus reduced the problem to the level of a four piece jigsaw puzzle, so all that is left is to fit the pieces together.

The condition $\det\U=1$ and the knowledge $\det\T=1$ gives a relationship between f and h
\begin{equation}
1=\det\U=hf^4\det\T=hf^4,
\end{equation}
so $h$ will henceforth be replaced by $f^{-4}$.
The inverse of $\U$ is easily found to be
\begin{equation}
\U^{-1}=\toto{f^{-1}\T^{-1}}{-f^{3}\T^{-1}g}{0}{f^4}
\end{equation}
and
\begin{equation}
\partial\U=\toto{f\partial\T+\partial f\T}{\partial g}{0}{-4f^{-5}\partial f}
\end{equation}
giving
\begin{equation}
\U^{-1}\partial\U = \toto{\T^{-1}\partial\T+f^{-1}\partial f}{f^{-1}\T^{-1}\partial g+4f^{-2}\partial f\T^{-1}g}{0}{-4f^{-1}\partial f}. 
\end{equation}
The upper left part of the matrix in the trace in \eqnref{scalar_kinetic_simp} becomes
\begin{align}
(\U^{-1}&\partial\U(\U^{-1}\partial\U+(\U^{-1}\partial\U)^T)_{mn}=\T^{-1}\partial\T\T^{-1}\partial\T+\T^{-1}\partial\T(\T^{-1}\partial%%@
\T)^T\nonumber\\
&+3f^{-1}\partial f \T^{-1}\partial\T+f^{-1}\partial f (\T^{-1}\partial\T)^T+2f^{-2}(\partial f)^2\id_{4\times4}\nonumber\\
&+f^{-2}\T^{-1}\partial g\partial g^T\T^{T-1}+4f^{-3}\partial f\T^{-1}\partial g g^T\T^{T-1}\nonumber\\
&+4f^{-3}\partial f\T^{-1}g\partial g^T\T^{T-1}+16f^{-4}(\partial f)^2\T^{-1}gg^t\T^{T-1}
\end{align}
and the lower right part becomes
\begin{align}
(\U^{-1}&\partial\U(\U^{-1}\partial\U+(\U^{-1}\partial\U)^T)_{55}=32f^{-2}(\partial f)^2.
\end{align}
Take the trace and remember that $\tr(\T^{-1}\partial\T)=0$ from \eqnref{conven_parinv}
\begin{align}
\Lagr_{scalar}^{d=7} &=-\frac{1}{2}\tr\left(\U^{-1}\partial\U(\U^{-1}\partial\U+(\U^{-1}\partial\U)^T)\right)\\
&=-\frac{1}{2}\tr\left(\T^{-1}\partial\T(\T^{-1}\partial\T+(\T^{-1}\partial\T)^T)\right)-\frac{1}{2}f^{-2}\partial g^T\T^{T-1}\T^{-1}\partial g\nonumber\\
&-4f^{-3}\partial fg^T\T^{T-1}\T^{-1}\partial g
-8f^{-4}(\partial f)^2g^t\T^{T-1}\T^{-1}g-20f^{-2}(\partial f)^2.\nonumber
\end{align}
The first term is independent on the choice of $\T$ and can be written as $\Lagr_0=-\tr(\M^{-1}\partial\M)^2/4$ in both cases. Since the scalars in the Lagrangian \eqnref{sl5so5_scalar_lagr} are multiplied with $\M$ and not $\M^{-1}$, we choose to use $\U=\V^{T-1}$ so $\T=\N^{T-1} \Leftrightarrow \T^{T-1}\T^{-1}=\M$.
Refine the Ansatz by letting
\begin{align}
f&=e^{x\varphi},\hspace{1cm} &g&=a^me^{y\varphi},\nonumber\\
\partial f &= x\partial\varphi e^{x\varphi}, &\partial g &=y\partial\varphi a^me^{y\varphi} + \partial a^me^{y\varphi},
\end{align}
where x and y are real parameters to be adjusted later.
This means
\begin{align}
\Lagr&_{scalar}^{d=7}-\Lagr_0 = \left\{-\frac{1}{2}(y\partial\varphi a^m + \partial a^m)\M_{mn}(y\partial\varphi a^n + \partial a^n)\right.\nonumber\\ 
&-4x\partial\varphi a^m\M_{mn}(y\partial\varphi a^n + \partial a^n)-8x^2(\partial\varphi)^2a^m\M_{mn}a^n\bigg\}e^{2y\varphi-2x\varphi}\nonumber\\
&-20x^2(\partial\varphi)^2.
\end{align}
Comparing the last term to $-(\partial\varphi)^2/2$ in \eqnref{sl5so5_scalar_lagr} gives $x=\pm 1/(2\sqrt{10})$ and comparing the exponent $2(y-x)\varphi$ to the exponent in 
\begin{align}
-\frac{1}{2}e^{\frac{\sqrt{10}}{2}\varphi}\partial a^m\M_{mn}\partial a^n
\end{align}
from \eqnref{sl5so5_scalar_lagr}, gives $y=(10\pm 2)/(4\sqrt{10})$. Choosing the minus sign, so $x=-1/(2\sqrt{10})$ and $y=2/\sqrt{10}$, removes all unwanted terms and we have   
\begin{equation}
-\frac{1}{4}\tr(\W^{-1}\partial \W)^2 = -\frac{1}{2}(\partial\varphi)^2-\frac{1}{4}\tr(\M^{-1}\partial \M)^2-\frac{1}{2}e^{\frac{\sqrt{10}}{2}\varphi}\partial a^m\M_{mn}\partial a^n
\eqnlab{sl5so5_scalar_kinetic}
\end{equation}
just as we wanted. $\W$ is explicitly given by
\begin{align}
\W&=\V\V^T = \U^{T-1}\U^{-1} = \toto{f^{-1}\N}{0}{-f^{3}g^T\N}{f^4}\toto{f^{-1}\N^{T}}{-f^{3}\N^{T}g}{0}{f^4}\nonumber\\ 
&= \toto{f^{-2}\M}{-f^{2}\M g}{-f^{2}g^T\M}{f^6g^T\M g+f^8}=e^{\frac{1}{\sqrt{10}}\varphi}\toto{\M_{mn}}{-a_m}{-a_n}{a_m a^m+e^{\frac{-5}{\sqrt{10}}\varphi}}.
\eqnlab{sl5so5_metric}
\end{align}
To see that the metric indeed is a parametrisation of \coset{5} we first note that if $\W$ belongs to \coset{5} then so does $\W^{-1}$. This means we can compare the vielbein $\U$ to the vielbein $\V$ \eqnref{ex_sl5so5_v} found in the \coset{5} example.
We find that after doing the following field redefinitions in \eqnref{ex_sl5so5_v}
\begin{align}
&\tilde\tau_{10}=a_1,\hspace{.5cm}\tilde\tau_9=a_2,\hspace{.5cm}\tilde\tau_7=a_3,\hspace{.5cm}\tau_4=a_4,\hspace{.5cm}\phi_1=\phi_1'+\frac{1}{\sqrt{10}}\varphi,\nonumber\\
&\phi_2=\phi_2'+\frac{2}{\sqrt{10}}\varphi,\hspace{.5cm}\phi_3=\phi_3'+\frac{3}{\sqrt{10}}\varphi,\hspace{.5cm}\phi_4=\frac{4}{\sqrt{1%%@
0}}\varphi
\end{align}
the vielbeins are identical and hence $\W\in$\coset{5}.

\subsubsection{The 1-form terms}
We now want to use the found metric $\W$ to write the 1-form gauge potentials in an \coset{5} covariant way. 
The 1-form part of the Lagrangian \eqnref{reduct_7d} is 
\begin{equation}
\Lagr_{A}^{d=7} = -\frac{1}{4}e^{-\frac{3}{\sqrt{10}}\varphi}F^{1m}\M_{mn}F^{1n}-\frac{1}{2}e^{\frac{2}{\sqrt{10}}\varphi}F'^{2[mn]}\M_{mm'}\M_{nn'}F'^{2[m'n']}.
\eqnlab{sl5so5_1form_lagr}
\end{equation}
Let r,s,... run from 1 to 5 and decompose
\begin{align}
F^{[rs]}\W_{rr'}\W_{ss'}F^{[r's']} &= F^{[mn]}\W_{mm'}\W_{nn'}F^{[m'n']}+4F^{[mn]}\W_{m5}\W_{nn'}F^{[5n']}\nonumber\\
& \phantom{=} +2F^{[5n]}\W_{55}\W_{nn'}F^{[5n']}+2F^{[5n]}\W_{5m'}\W_{n5}F^{[m'5]}.
\eqnlab{sl5so5_1form_expansion}
\end{align}
Define the 10 field strengths of $F^{[rs]}$ to contain the 4 $F^{1m}$ field strengths and the 6 $F^{'2[mn]}$ field strengths according to  
\begin{align}
F^{5n} &= \frac{1}{2}F^{1n} = -F^{n5},\nonumber\\
F^{[mn]} &= F^{2[mn]} = F^{'2[mn]} + a^{[m}F^{1n]},
\label{dsdg}
\end{align}
where we have used the relation for $F'^{2mn}$ in \eqnref{reduct_7d_fprim}.
Inserting this together with the components of $\W$ into \eqnref{sl5so5_1form_expansion} gives
\begin{align}
F&^{[rs]}\W_{rr'}\W_{ss'}F^{[r's']} = \bigg\{F^{2[mn]}\M_{mm'}\M_{nn'}F^{2[m'n']}-2F^{2[mn]}a_m\M_{nn'}F^{1n'}\nonumber\\ 
&+\left.\frac{1}{2}F^{1m}\left(a_pa^p\M_{mn}+e^{-\frac{5}{\sqrt{10}}\varphi}\M_{mn}-a_ma_n\right)F^{1n}\right\}e^{\frac{2}{\sqrt{10}}\varphi}\nonumber\\
&=\bigg\{F'^{2[mn]}\M_{mm'}\M_{nn'}F'^{2[m'n']}+(2-2)F'^{2[mn]}a_m\M_{nn'}F^{1n'}\nonumber\\ 
&+\left.F^{1m}\left(\frac{1}{2}e^{-\frac{5}{\sqrt{10}}\varphi}\M_{mn}+(a_pa^p\M_{mn}-a_ma_n)(\frac{1}{2}+\frac{1}{2}-1)\right)F^{1n}\right\}e^{\frac{2}{\sqrt{10}}\varphi}\nonumber\\
&=e^{\frac{2}{\sqrt{10}}\varphi}F'^{2[mn]}\M_{mm'}\M_{nn'}F'^{2[m'n']}+\frac{1}{2}e^{-\frac{3}{\sqrt{10}}\varphi}F^{1m}\M_{mn}F^{1n}.
\end{align}
We thus see that we can rewrite the 1-form Lagrangian \eqnref{sl5so5_1form_lagr} as
\begin{equation}
\Lagr_{A}^{d=7} = -\frac{1}{2}F^{[rs]}\W_{rr'}\W_{ss'}F^{[r's']}
\eqnlab{sl5so5_1form_term}
\end{equation}
which is invariant under SL(5,$\rr$) transformations.

\subsubsection{Higher order terms}


We now have the action
\begin{align}
S^{d=7} = \int d^7x \sqrt{|g_E|} &\Big{[} R_E - \frac{1}{4}\tr(\W^{-1}\partial \W)^2 - {1 \over 2}F^{[rs]}\W_{rr'}\W_{ss'}F^{[r's']}\nonumber\\ 
& - {1 \over 12}e^{-{1 \over \sqrt{10}} \varphi}H_mH_n\M^{mn} - {1 \over 48}e^{{-4 \over \sqrt{10}}\varphi}G^2 \Big{]}.
\label{asdfghjkl}
\end{align}
The next step is to write the $H$ and $G$ terms in an \coset{5} invariant way. To accomplish this we will use that the 
dual of $G$ in 7 dimensions is a 3-form and hence we can write it together with with the four 3-forms $H_m$ by using the 5-dimensional metric $\W$ 
derived above. We start by noting that
\begin{equation}
-{1 \over 48}\int d^7x \sqrt{|g_E|} e^{{-4 \over \sqrt{10}}\varphi}G^2 = -{1 \over 2}\int e^{{-4 \over \sqrt{10}}\varphi} *G \wedge G.
\end{equation}
Inserting this in the action and varying with respect to $C$ gives
\begin{align}
\delta S &=-{1 \over 2}\int e^{{-4 \over \sqrt{10}}\varphi}( \delta G \wedge *G + G \wedge \delta*G ) + \delta S_{CS} \nonumber \\
& =-\int e^{{-4 \over \sqrt{10}}\varphi}( d \delta C \wedge *G ) + \delta S_{CS} \nonumber \\
& =- \int \delta C \wedge d(e^{{-4 \over \sqrt{10}}\varphi}*G) + \delta S_{CS},
\eqnlab{kjskj}
\end{align}
where $\delta S_{CS}$ is the variation of the Chern-Simons term. We can also calculate the dual of $\hat{G}$ and expand it in the same way 
as in (\ref{g7}), yielding
\begin{equation}
\hat{G_7}= \cdots +{1 \over 3!}G'^q \epsilon_{mnpq} \hat{e}^m \wedge \hat{e}^n \wedge \hat{e}^p - {1 \over 4!}H' \epsilon_{mnpq} \hat{e}^m \wedge \hat{e}^n \wedge \hat{e}^p \wedge \hat{e}^q, 
\end{equation}
where we have ignored the forms of higher order than four. Via \eqnref{bianchig7} we get
\begin{equation}
dH'=H_m \wedge da^m + {1 \over 2}\epsilon_{mnpq}F'^{2mn} \wedge F'^{2pq}.
\eqnlab{sadf}
\end{equation}
If we would have calculated $\delta S_{CS}$ in \eqnref{kjskj} we would have found the equation of motion for $e^{{-4 \over \sqrt{10}}\varphi}*G$ 
to be the right hand side of \eqnref{sadf}. (This of course have to be the case since the Bianchi identity for $H'$ is derived from the $\hat{C}$-e.o.m.) 
This means that we can swap the equation of motion and the Bianchi identity and define
\begin{equation}
H'=e^{{-4 \over \sqrt{10}}\varphi}*G.
\end{equation}
The next step is to rewrite $G^2$ in (\ref{asdfghjkl}) as $4(*G)^2$. This gives, if we only consider the $H$ and $G$ terms, the action
\begin{align}
S_{H,G} &= \int d^7x \sqrt{|g_E|} \Big{[} -{1 \over 12}e^{-{1 \over \sqrt{10}} \varphi}H_mH_n\M^{mn} - {4 \over 48}e^{{-4 \over \sqrt{10}}\varphi}(*G)^2 \Big{]} \nonumber \\
&= -{1 \over 12} \int d^7x \sqrt{|g_E|} \Big{[} e^{-{1 \over \sqrt{10}} \varphi}H_mH_n\M^{mn} + e^{{4 \over \sqrt{10}}\varphi}(H')^2 \Big{]}.
\end{align}
If we define
\begin{equation}
H_5 = H' - H_m a^m
\label{h5}
\end{equation}
and
\begin{equation}
H_r = (H_m,H_5),
\end{equation}
we get
\begin{align}
-{1 \over 12} \int & d^7x \sqrt{|g_E|} H_r H_s \W^{rs}  \nonumber \\
=& -{1 \over 12} \int d^7x \sqrt{|g_E|} \Big{[} e^{-{1 \over \sqrt{10}} \varphi}H_mH_n\M^{mn} + e^{{4 \over \sqrt{10}} \varphi}a^m a^n H_m H_n \nonumber \\
& +2e^{{4 \over \sqrt{10}} \varphi}H_mH_5a^m +e^{{4 \over \sqrt{10}} \varphi}H_5H_5 \Big{]} \nonumber \\
=& -{1 \over 12} \int d^7x \sqrt{|g_E|} \Big{[} e^{-{1 \over \sqrt{10}} \varphi}H_mH_n\M^{mn} + e^{{4 \over \sqrt{10}} \varphi}\Big{(}a^m a^n H_m H_n \nonumber \\
& +2H_ma^m(H'-H_na^n) + (H'-H_ma^m)(H'-H_na^n) \Big{)}\Big{]} \nonumber \\
=& -{1 \over 12} \int d^7x \sqrt{|g_E|} \Big{[} e^{-{1 \over \sqrt{10}} \varphi}H_mH_n\M^{mn} + e^{{4 \over \sqrt{10}} \varphi}H'^2 \Big{]},
\end{align}
and hence the complete action (apart from the Chern-Simons term) for 7-dimensional supergravity becomes
\begin{align}
S^{d=7} = \int d^7x \sqrt{|g_E|} &\Big{[} R_E - \frac{1}{4}\tr(\W^{-1}\partial \W)^2 - {1 \over 2}F^{[rs]}\W_{rr'}\W_{ss'}F^{[r's']}\nonumber\\ 
& - {1 \over 12}H_r H_s \W^{rs} \Big{]}.
\end{align}
From \eqnref{reduct_7d_bianchi}, \eqnref{sadf} and (\ref{h5}) we get the Bianchi identity
\begin{align}
dH_5 &=dH'-H_m \wedge da^m -dH_ma^m \nonumber \\
&= {1 \over 2}\epsilon_{mnpq}F'^{2mn} \wedge F'^{2pq} + \epsilon_{mnpq}a^mF^{1n}\we F^{2pq},
\end{align}
and we also have, from \eqnref{reduct_7d_bianchi} once again
\begin{equation}
dH_q=\epsilon_{mnpq}F^{1m} \wedge F^{2np}=2\epsilon_{mnpq}F^{5m} \wedge F^{2np}.
\end{equation}
These can now be combined to the single SL(5,$\rr$) covariant Bianchi identity
\begin{equation}
dH_v={1 \over 2}\epsilon_{rstuv}F^{rs} \wedge F^{tu},
\eqnlab{SL5bianchiH}
\end{equation}
as is proved by calculating the components
\begin{align}
{1 \over 2}\epsilon&_{rstu5}F^{rs} \wedge F^{tu} \nonumber \\
&={1 \over 2}\epsilon_{mnpq5}F^{mn} \wedge F^{pq} ={1 \over 2}\epsilon_{mnpq5}(F'^{2mn} + a^m F^{1n}) \wedge (F'^{2pq} + a^p F^{1q}) \nonumber \\
&=\Big{\{} \epsilon_{mnpq5}a^m F^{1n} \wedge a^p F^{1q} =0\Big{\}} \nonumber \\
&={1 \over 2}\epsilon_{mnpq5}(F'^{2mn} \wedge F'^{2pq} + 2 a^m F^{1n} \wedge F'^{2pq}) \nonumber \\
&={1 \over 2}\epsilon_{mnpq5}(F'^{2mn} \wedge F'^{2pq} + 2 a^m F^{1n} \wedge F^{2pq})=dH_5,
\end{align}
and
\begin{align}
{1 \over 2}\Big{(}& \epsilon_{5mnpq}F^{5m} \wedge F^{np} + \epsilon_{m5npq}F^{m5} \wedge F^{np} + \epsilon_{mn5pq}F^{mn} \wedge F^{5p} \nonumber \\
&+ \epsilon_{mnp5q}F^{mn} \wedge F^{p5} \Big{)} = {1 \over 2}\Big{(}4\epsilon_{mnpq5}F^{5m} \wedge F^{np}\Big{)}=dH_q.
\end{align}
Before we finish this chapter we want to make the rest of the Bianchi identities and the gauge transformations SL(5,$\rr$) covariant. 
With $F^{rs}$ defined as in (\ref{dsdg}) we see that we get
\begin{equation}
dF^{rs}=0,
\end{equation}
and also that the field strength is invariant under the transformation
\begin{equation}
\delta A^{rs} = d\chi^{rs}.
\end{equation}
Next we turn the attention to the variation of the $B$-field. First we combine $B_5$, the (not exact) potential to $H_5$, 
with $B'_m$ to form $B_v=(B'_m,B_5)$. Then we note that it is possible, even without a redinition of $B'_m$, to write
\begin{equation}
H_v=dB_v + {1 \over 2}\epsilon_{rstuv}F^{rs} \wedge A^{tu}.
\end{equation}
Differentiating this immediately gives \eqnref{SL5bianchiH}. $H_v$ is thus invariant under the SL(5,$\rr$) covariant transformation
\begin{equation}
\delta B_v = d\chi_v + {1 \over 2}\epsilon_{rstuv}A^{rs} \wedge d\chi^{tu}.
\end{equation}
The last Bianchi identity that needs to be made covariant is the one for $G$. Noting that $F^{1m}=2F^{5m}$, the B.I. for $G$ turns into
\begin{equation}
dG=2H_m \wedge F^{5m}.
\end{equation}
Redefining $C'$ as
\begin{equation}
C'=C-2A^{5m} \wedge B_m + 2\epsilon_{mnpq}A^{5m} \wedge A^{pq} \wedge A^{5n},
\end{equation}
gives $G$ as
\begin{equation}
G=dC+B_m \wedge F^{5m} - A^{5m} \wedge H_m.
\end{equation}
Hence we can combine $G$ with the four 4-forms originating from the dual of $\hat{G}$ to form
\begin{equation}
G^r=dC^r + B_s \wedge F^{rs} - A^{rs} \wedge H_s.
\label{awedc}
\end{equation}
This is consistent with
\begin{equation}
dG^r=2H_s \wedge F^{rs},
\end{equation}
as is shown by the calculation
\begin{align}
dG^{r'}&=dB_v \wedge F^{r'v} -A^{r'v} \wedge dH_v + F^{r'v} \wedge H_v \nonumber \\
&=2H_v \wedge F^{r'v} - {1 \over 2}\epsilon_{rstuv}F^{rs} \wedge (A^{tu} \wedge F^{r'v} + A^{r'v} \wedge F^{tu}).
\end{align}
The vanishing of the second term above can be shown by first setting $r'=5$ according to
\begin{align}
\epsilon_{rstuv}F^{rs} \wedge (&A^{tu} \wedge F^{5v} + A^{5v} \wedge F^{tu}) \nonumber \\
&=\epsilon_{rstuq}F^{rs} \wedge (A^{tu} \wedge F^{5q} + A^{5q} \wedge F^{tu}) \nonumber \\
&=\epsilon_{mn5pq}F^{mn} \wedge (A^{5p} \wedge F^{5q} + A^{5q} \wedge F^{5p}) \nonumber \\
&\phantom{=}+\epsilon_{rspuq}F^{rs} \wedge (A^{pu} \wedge F^{5q} + A^{5q} \wedge F^{pu}) \nonumber \\
&=2\epsilon_{mn5pq}F^{mn} \wedge (A^{5p} \wedge F^{5q} + A^{5q} \wedge F^{5p}) \nonumber \\
&\phantom{=}+\epsilon_{rsnpq}F^{rs} \wedge (A^{np} \wedge F^{5q} + A^{5q} \wedge F^{np}) \nonumber \\
&=2\epsilon_{mn5pq}F^{mn} \wedge (A^{5p} \wedge F^{5q} + A^{5q} \wedge F^{5p}) \nonumber \\
&\phantom{=}+2\epsilon_{m5npq}F^{m5} \wedge (A^{np} \wedge F^{5q} + A^{5q} \wedge F^{np}) \nonumber \\
&=4\epsilon_{mn5pq}F^{mn} \wedge (A^{5p} \wedge F^{5q} + A^{5q} \wedge F^{5p})=0,
\end{align}
where $r,s,t,u,v$ are 5-dimensional indices and $m,n,p,q$ are 4-dimensional indices as usual.
Doing the same calculation but this time with $r'=m$ would show that the term is zero, and hence we receive the desired Bianchi identity. With $G^r$ 
defined as in (\ref{awedc}) the variation of it becomes
\begin{align}
\delta G^{r'} &= d \delta C^{r'} + \delta B_v \wedge F^{r'v} - \delta A^{r'v} \wedge H_v \nonumber \\
&=d\delta C^{r'} + (d\chi_v + {1 \over 2}\epsilon_{rstuv}A^{rs} \wedge d\chi^{tu}) \wedge F^{r'v} - d\chi^{r'v} \wedge H_v \nonumber \\
&=d(\delta C^{r'} + d\chi_v \wedge A^{r'v} + {1 \over 2}\epsilon_{rstuv}A^{rs} \wedge d\chi^{tu} \wedge A^{r'v} - d\chi^{r'v} \wedge B_v) \nonumber \\
&\phantom{=}-{1 \over 2}\epsilon_{rstuv}(F^{rs} \wedge d\chi^{tu} \wedge A^{r'v} + d\chi^{r'v} \wedge F^{rs} \wedge A^{tu}) \nonumber \\
&=d(\delta C^{r'} + d\chi_v \wedge A^{r'v} + {1 \over 2}\epsilon_{rstuv}A^{rs} \wedge d\chi^{tu} \wedge A^{r'v} - d\chi^{r'v} \wedge B_v),
\end{align}
where one again have to calculate the components of the $F \wedge A$ terms to see that they cancel out.
If we want $G^r$ to be invariant we see that the variation of $C^r$ must be
\begin{equation}
\delta C^{r'}=d\chi^{r'} - A^{r's} \wedge d\chi_s + B_s \wedge d\chi^{r's} - {1 \over 2}\epsilon_{rstuv}A^{r'r} \wedge A^{st} \wedge d\chi^{uv}.
\end{equation}

This concludes the third chapter. After all this calculational work, a short summary might be needed. In this chapter 
we have constructed the complete bosonic action for 9-, 8- and 7-dimensional supergravity with the various symmetries clearly expressed. 
We have also constructed U-duality covariant Bianchi identities as well as U-duality covariant gauge transformations. 
These will all play a very important role in the rest of this thesis work.





