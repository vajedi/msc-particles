\chapter{Introduction}
Successful unification have always meant great progress in the history of physics. Famous examples are Maxwell's unification 
of electricity and magnetism and Einstein's unification of space and time in the special theory of relativity.  
Today, the major topic in theoretical physics is the search for ways to combine gravity and quantum mechanics in a theory 
 modestly called \textit{a theory for everything}. The first signs of success in this area was the emerge of 
string theory in the early 1970's. Actually, bosonic string theory was invented already in the late 1960's to explain strong 
interactions in hadron physics. But this attempt of describing the strong nuclear force ran into some serious theoretical 
difficulties, one of them being the requirement of an unwanted zero mass particle with two units of spin. This difficulty 
was however turned into a virtue by the identification of the unwanted particle with the graviton, and string theory with a theory 
for quantum gravity. The next step in the development of string theory was the incorporation of fermions, creating 
the supersymmetric superstring theory. It also stood clear early on that in order to have consistent string theory, the number 
of spacetime dimensions must be 10, not 4. String theory evolved over the years and although some serious anomalies were shown 
to cancel out by Green and Schwarz in the first superstring revolution 1984 there were still some disturbing problems that 
needed to be solved in the early 1990's. First of all, there was not only one but five different self-consistent superstring 
theories. Another problem was that supersymmetry permits a supersymmetric field theory in 11 dimensions making the 10-dimensional 
spacetime demanded by string theory somewhat questionable. A promising solution to this was proposed by Witten in the second superstring 
revolution in 1994. According to Witten all five superstring theories corresponds to different extremal regions of the coupling 
constants arising after compactification of an 11-dimensional non-perturbative theory called M-theory. They are also connected to each 
other through various dualities (S- and T-duality). At the low energy limit, M-theory is believed to be described by 11-dimensional supergravity.

In absence of perturbation theory, the duality relations between the different areas of the coupling constants are essential to 
get further insight of the properties of M-theory. For example, in the case where the strong coupling of one theory is connected by 
a duality relation to the weak coupling of another theory (S-duality), non-perturbative effects and strong coupling dynamics in the 
first theory can be handled with perturbative methods in the second theory. However, to actually prove such a duality is 
often extremely difficult since the duality involves a map between a weak and a strong coupled theory and hence only the perfect 
knowledge of non-perturbative effects could firmly establish the duality. This has not been done in M-theory today but still 
there is some evidence supporting the dualities, in form of BPS states. For a state to be BPS, there is a condition that the mass 
must be equal to the charge(s). This means that even at strong coupling the mass is bound to the charge, or in other words, the 
mass-charge relation do not suffer from any quantum corrections. Thus one can analyze the properties of the BPS state at weak coupling 
and simply extrapolate the results to the strong coupling area. 

In 1994 it was shown that S-duality and T-duality fit into a larger group of duality symmetries, called U-duality. For example, 
the duality group of type II string theories compactified on a $d$-dimensional torus is a discrete subgroup to the symmetry group 
of (10-$d$)-dimensional supergravity. Furthermore, U-duality maps all the fundamental string states to any non-perturbative object that 
can be predicted by the theory. In the dimensionally reduced supergravity, BPS states appears as solitonic generalized black hole 
solutions. Some of these solutions can be considered to be winding modes of higher dimensional objects called $p$-branes and because 
of the U-duality mapping to fundamental strings, $p$-branes can also be treated as fundamental. The existence of fundamental BPS states 
with U-duality symmetry strongly implies that U-duality is a true symmetry of M-theory.

This thesis work is an attempt to outline dynamics for a $D_2$-brane in $D=8$ and a $D_1$-brane in $D=9$ that is fully 
covariant under the U-duality group. This is done via modified world-volume field strengths that couple all fields on the brane 
to background fields, and thereby encode all interactions. Supergravity backgrounds are constructed through Kaluza-Klein 
reduction on tori. Comments are made on a similar formulation of the dynamics for a $D_3$-brane in $D=7$. The thesis is closely 
related to \cite{artikeln}.


\section{Outline}
First of all, this thesis is written in two parts, \textbf{Dimensional Reduction} and \textbf{Dynamics}. The first chapter 
in the Dimensional Reduction part (i.e chapter 2) is \textbf{Supergravity}, where we introduce supergravity and some of its properties. The symmetries of the action and 
the degrees of freedom are reviewed together with the field equations. We also present the concept of Kaluza-Klein reduction of supergravity, 
which will play an important role in the following chapters. The last sections deals with U-duality, the symmetry group 
of maximal supergravity. The purpose of this opening chapter is to prepare the reader for the rest of the thesis.

In chapter 3, \textbf{Kaluza-Klein reduction on $T^2$, $T^3$ and $T^4$}, we as the name implies reduce 11-dimensional 
supergravity by compactifying it on 2-, 3- and 4-dimensional tori. The field content 
of the lower-dimensional supergravity theories are deduced and the Bianchi identities are calculated. A major part 
of this chapter is the constructing of U-duality covariant Bianchi identities by using dual fields. The results of the 
calculations will be the supergravity backgrounds for the branes in chapter 5. 

The first chapter of the second part is \textbf{Brane dynamics}. This chapter contains theory necessary for the understanding 
of chapter 5, with the focus being mainly on the general theory of branes. We motivate a special way of formulating the world-volume 
field strengths which gives a coupling between the brane and the background crucial for the final work of finding U-duality 
covariant membrane dynamics. The form of the action to be used is also introduced.

Chapter 5, \textbf{The equations of motion}, is the chapter of detailed calculations. After some work we get implicit equations of 
motions and duality relations, and the goal is to find an action that fulfil these relations. A lot of the calculations 
are done by implementing a solution Ansatz in a computer program, and the results of these calculations can be found in chapter 6, \textbf{Computer solutions}.

The final chapter, \textbf{Conclusions}, contains (you guessed it) the conclusions of this thesis together with some 
brief comments connecting our results with results from other relevant articles.

There are three appendices attached to this thesis, the first one giving an introduction to differential forms as well 
as some useful formulas. To make some of the texts more readable, calculations longer than a page have been shifted to the 
second appendix. Finally, the third appendix contains programming code needed for the handling of the very large numbers appearing in some 
of the series expansions.

