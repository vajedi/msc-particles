\chapter{Conclusions}
In this thesis we have made an attempt to derive the empiric covariant brane action \eqnref{dynamics_final_action} for some various cases in different dimensions. 
In particular we have tried to connect the 8-dimensional $D_2$ membrane action with manifest U-duality invariance to the action found using automorphic functions in \cite{pioline}. For this end we have followed the work done in \cite{artikeln} tightly.
Our main results are that the equations \eqnref{csolution_equations_8D_w0_alpha}, derived using the background field constraint $F=0$ considered in \cite{artikeln}, can be solved for general coefficients \eqnref{csolution_dualit_general_alpha} and \eqnref{csolution_phi_general_alpha} and that there might exist a consistent solution (of order higher than 7) which solves the equations \eqnref{csolution_general_duality} of a general background using \eqnref{csolution_duality_u_final} and \eqnref{csolution_phi_final} without the introduction of any additional parameters (the equations are derived and solved using $p=(1,0)$, so if we found relations and solution to these equations, we would expect them to be generalizable to general values on $p$).  

The implications of the first result would be that if we let $\alpha$ be a parameter of the theory, the simple identification $p^2 = y^2+x_0^2$ between our charges and the "extra variables" found in \cite{pioline} would be erroneous.  
Since this result comes from a solution of equations that are not representing the general case, we cannot draw too big conclusions though. It could of course also be that the values of all parameters will be locked when solving the equations for a general background, using the correct relations between $u$, $v$ and $w$ with the possibility of general $\alpha$ in the $F=0$ case.
Also, using $\kappa$-symmetry might lock the values of any parameters once and for all, but that is not obvious unless done.
Anyway, it is obvious that the significance of all free parameters is still unclear.
According to \eqnref{solution_8d_eom_a} and \eqnref{solution_8d_eom_phi} the inclusion of the gauge invariant field strength $\bar h$ (which consists of an odd number of $\omega$) in \eqnref{solution_8d_field_strengths} will effectively add odd terms in $v$ to \eqnref{solution_8d_eom_a}, even terms in $v$ to \eqnref{solution_8d_eom_phi} and terms of mixed order in $v$ to both equations, coming from the relations $u(v)$ and $w(v)$.   
We could try to solve these equations, using an expansion $\bar h$, but to be honest, that much arbitrariness is more than we can bear.

All solutions found have been of order 6 or lower, which should be interpreted as a result of shortage in our equation solving stamina, rather than a indication that the {\it{real}} solution is of this order.   
Knowing what the relations between $u$, $v$ and $w$ look like, we would expect much higher ansatz orders to be solvable.
Although we have derived the duality equations \eqnref{csolution_9d_general_duality} to solve for the $d=9$ $D_1$ case, we have mainly considered the $d=8$ membrane case.
Solving the equations of the $D_1$ case, which have some similarities to the $d=8$ membrane equations, could be one way to work around the problem and get a hint of what the relations between $u$, $v$ and $w$ should look like also in the membrane case.
Since we are using less polynomial variables for the ansatz in the $D_1$ case, we expect these equations to be somewhat simpler to solve (at least if the order of the ansatz is not greater than in the previous case).

One weakness of our approach in solving the equations is that we have assumed $u$ and $w$ to be explicit functions of $v$, with no more motivation than that $v$ is the most occurrent of the variables on the right hand side for the parameter free equations and thus gives the easiest equations to solve a priori.   
Because of the symmetry between $v$ and $w$ a better assumption might have been to use relations $v(u)$ and $w(u)$.  
Another improvement could be to include more types of polynomial variables in the ansatz but a bigger search space usually increases the complexity of the problem at hand.
We can of course neither give a guarantee that the program we have written to expand the duality equations are completely bug free. Even though it produces expansions which have been consistent in all checks made, there always exists more or less harmful bugs in computer programs (of course trying to expand all the equations by hand would produce way more errors and could probably neither be done in a lifetime).





