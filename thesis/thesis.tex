% Kommandon:
%
% Textrader i en align: \intertext{text}
% Mjuk avstavning: \-
% Använd \intertext{} för att avstava ekvationer

%********************************************************************
%
%    Made by:
%      Simon Vajedi
%
%        Created: 2012-10-03
%  Latest revision: 
%         Status: Draft      
%
%  Description/notes:
%        Thesis work :
%               Dynamics of inertial particles in random flows.
%        
%********************************************************************


\documentclass[a4paper,11pt,twoside,openright]{UGthesis}
\usepackage{amssymb}
\usepackage[swedish,english]{babel}
\usepackage[utf8]{inputenc}
\usepackage[T1]{fontenc}
\usepackage{enumerate}
\usepackage{epic}
\usepackage{eepic}
\usepackage{fancyheadings}
\usepackage[dvips]{graphicx}
%\usepackage{graphpap}
\usepackage{ifthen}
\usepackage{subfigure}
\usepackage{textcomp}
\usepackage{textfit}

\usepackage{amsmath}
\usepackage{multicol}

\usepackage[errorshow]{tracefnt}
%\usepackage{makeidx}
%\usepackage{fancybox}
%\usepackage{psboxit}

%\include{psfig}

%\input MHmacros

%************************************
%         New commands.
%************************************

%** Differential forms
\newcommand{\we}{\wedge}
\newcommand{\dxmu}{dx^\mu}
\newcommand{\dxnu}{dx^\nu}
\newcommand{\la}{\lambda}
\newcommand{\La}{\Lambda}
\newcommand{\si}{\sigma}
\newcommand{\ta}{\tau}
\newcommand{\al}{\alpha}
\newcommand{\ka}{\kappa}
\newcommand{\om}{\omega}
\newcommand{\be}{\beta}
\newcommand{\ga}{\gamma}
\newcommand{\da}{\dagger}

\newcommand{\wils}{\langle W(\Sigma)\rangle}
\newcommand{\corr}[1]{\left\langle #1 \right\rangle}
\newcommand{\com}[1]{\lbrack #1 \rbrack}
\newcommand{\expo}{\textrm{exp}}
\newcommand{\reg}{\textrm{reg}}
\newcommand{\ren}{\textrm{ren}}
\newcommand{\xnp}{x_\textrm{np}}
\newcommand{\non}{\nonumber}
\newcommand{\R}{\mathcal{R}}
\newcommand{\Or}{\mathcal{O}}
\newcommand{\Lagr}{\mathcal{L}}
\newcommand{\D}{\mathfrak{D}}
\newcommand{\bein}{\textsf{e}}

%** derivatives
\newcommand{\pa}{\partial}
\newcommand{\pamd}{\partial_\mu}
\newcommand{\pamu}{\partial^\mu}
\newcommand{\pand}{\partial_\nu}
\newcommand{\panu}{\partial^\nu}
\newcommand{\pard}{\partial_\rho}
\newcommand{\paru}{\partial^\rho}
\newcommand{\patd}{\partial_\tau}
\newcommand{\patu}{\partial^\tau}
\newcommand{\pald}{\partial_\lambda}
\newcommand{\palu}{\partial^\lambda}
\newcommand{\pafrac}[2]{\frac{\pa #1}{\pa #2}}
\newcommand{\defrac}[2]{\frac{\d #1}{\d #2}}
\newcommand{\Dfrac}[2]{\frac{\D #1}{\D #2}}

%** integrals
\newcommand{\intphi}{\int_0^{2\pi}}
\newcommand{\inttheta}{\int_0^{\pi}}

%** Dirac notation
\newcommand{\ket}[1]{| #1 \rangle}
\newcommand{\bra}[1]{\langle #1 |}
\newcommand{\bracket}[3]{\langle #1 |#2| #3 \rangle}
\newcommand{\braket}[2]{\langle #1 | #2 \rangle}

\newcommand{\matlab}{\textsc{Matlab~}}
\newcommand{\beq}{\begin{equation}}
\newcommand{\eeq}{\end{equation}}
\newcommand{\beqa}{\begin{eqnarray}}
\newcommand{\eeqa}{\end{eqnarray}}
\newcommand{\bal}{\begin{align}}
\newcommand{\eal}{\end{align}}

\newcommand\ffam{\sffamily}
\newcommand\fser{\bfseries}
\newcommand\fsh{\upshape}

\newcommand\blankpage{\thispagestyle{empty}\mbox{}\newpage}

\newcommand\rctr{\renewcommand{\theenumi}{\roman{enumi}}}
\newcommand\actr{\renewcommand{\theenumi}{\arabic{enumi}}}
\newcommand\rctrii{\renewcommand{\theenumii}{\roman{enumii}}}
\newcommand\actrii{\renewcommand{\theenumii}{\arabic{enumii}}}


\newcommand\nn{\nonumber}

\newcommand\benu{\begin{enumerate}}
\newcommand\eenu{\end{enumerate}}
\newcommand\bit{\begin{itemize}}
\newcommand\eit{\end{itemize}}

\newcommand{\ctxt}[2]{\put(#1){\makebox(0,0){#2}}}
\newcommand{\ltxt}[2]{\put(#1){\makebox(0,0)[l]{#2}}}

\newcommand{\chrf}[3]{\left\{ \hspace{-2mm}
\begin{array}{ccc}
#1 \\ [-1.2mm]#2 ~ #3
\end{array}
\hspace{-2mm} \right\}}


%%%%%%%%%%%%%%%%%%%%%%%%%%%%%%%%%%%%
%
% Labeling and refering
%
%%%%%%%%%%%%%%%%%%%%%%%%%%%%%%%%%%%%
\newcommand{\eqlbl}[1]{\label{eq:#1}}
\newcommand{\figlbl}[1]{\label{fig:#1}}
\newcommand{\tablbl}[1]{\label{tab:#1}}
\renewcommand{\eqref}[1]{(\ref{eq:#1})}
\newcommand{\figref}[1]{\ref{fig:#1}}
\newcommand{\tabref}[1]{\ref{tab:#1}}
\newcommand{\eq}[1]{Eq.~(\ref{eq:#1})}
\newcommand{\eqs}[1]{Eqs.~(\ref{eq:#1})}
\newcommand{\tab}[1]{Table~\ref{tab:#1}}
\newcommand{\tabs}[1]{Tables~\ref{tab:#1}}
\newcommand{\fig}[1]{Fig.~\ref{fig:#1}}
\newcommand{\figs}[1]{Figs.~\ref{fig:#1}}

\newcommand{\chllbl}[1]{\label{ch:#1}}
\newcommand{\seclbl}[1]{\label{sec:#1}}
\newcommand{\sseclbl}[1]{\label{ssec:#1}}
\newcommand{\chref}[1]{\ref{ch:#1}}
\newcommand{\secref}[1]{\ref{sec:#1}}
\newcommand{\ssecref}[1]{\ref{ssec:#1}}
\newcommand{\chap}[1]{Chapter~\ref{ch:#1}}
\newcommand{\chaps}[1]{Chapter~\ref{ch:#1}}
\renewcommand{\sec}[1]{Section~\ref{sec:#1}}
\newcommand{\secs}[1]{Sections.~\ref{sec:#1}}
\newcommand{\ssec}[1]{Subsec.~\ref{ssec:#1}}
\newcommand{\ssecs}[1]{Subsecs.~\ref{ssec:#1}}

%%%%%%%%%%%%%%%%%%%%%%%%%%%%%%%%%%%%
%
% Document
%
%%%%%%%%%%%%%%%%%%%%%%%%%%%%%%%%%%%%

\RequirePackage{hyperref}

% Ordinary nth order (\Od) and 1st order (\OD) derivative
% Ex: \Od{3}{f}{x}  \OD{f}{x}
\newcommand{\Od}[3]{\frac{d^{#1}#2}{d#3^{#1}}}
\newcommand{\OD}[2]{\frac{d#1}{d#2}}

% Partial nth order (\Pd) and 1st order (\PD) derivative 
% Ex: \Pd{2}{f}{y} \PD{f}{y}
\newcommand{\Pd}[3]{\frac{\partial^{#1}#2}{\partial#3^{#1}}}
\newcommand{\PD}[2]{\frac{\partial#1}{\partial#2}}

% Fourier and Laplace transform symbols
% Ex: \Fourier \Laplace
\newcommand{\Fourier}{\mathcal{F}}
\newcommand{\Laplace}{\mathcal{F}}

% Real and Complex 
% Ex: \Real \Complex
\newcommand{\Real}{\mathbf{R}}
\newcommand{\Complex}{\mathbf{C}}

% Real and imaginary parts (nicer than \Re and \Im)
% Ex \real \imag
\newcommand{\real}{\mathrm{Re}\ }
\newcommand{\imag}{\mathrm{Im}\ }

% My definitions
% Titel
\newcommand{\titel}{Dynamics of inertial particles in random flows}
% Indexes next to each other
% One up/down
\newcommand{\od}[1]{{}_{#1}}
\newcommand{\ou}[1]{{}^{#1}}
% Both up/down
\newcommand{\bd}[1]{{}_{#1}^{\phantom{#1}}}
\newcommand{\bu}[1]{{}^{#1}_{\phantom{#1}}}
\newcommand{\bb}[2]{{}^{#1}_{#2}}
% Lowered text
\newcommand{\subtext}[1]{_{\mbox{\tiny{#1}}}}
% Raised text
\newcommand{\suptext}[1]{^{\mbox{\tiny{#1}}}}
% Omega with 3 indices
\newcommand{\Om}[2]{\Omega\od{#1}\ou{#2}}
\newcommand{\Omh}[2]{\hat\Omega\od{#1}\ou{#2}}
% Delta with small indices
\newcommand{\delmu}{\delta_{{_{x^\mu}}}}
\newcommand{\delm}{\delta_{{_{x^m}}}}
% 2 by 2 matrix
\newcommand{\toto}[4]{\begin{pmatrix}#1 & #2 \cr #3 & #4\end{pmatrix}}
% 3 by 3 matrix
\newcommand{\tete}[9]{\pmatrix{ #1 & #2 & #3\cr #4 & #5 & #6\cr #7 & #8 & #9}}
% SL(n,R)/SO(n) coset
\newcommand{\coset}[1]{SL(#1,$\rr$)/SO(#1)}
% Real, complex space
\newcommand{\rr}{\mathbb{R}}
\newcommand{\cc}{\mathbb{C}}
% Matrices M,N,T,U,V,W
\newcommand{\M}{\mathcal{M}}
\newcommand{\N}{\mathcal{N}}
\newcommand{\U}{\mathcal{U}}
\newcommand{\T}{\mathcal{T}}
\newcommand{\V}{\mathcal{V}}
\newcommand{\W}{\mathcal{W}}
% Lie group g,h
\newcommand{\lga}{\mathbf{a}} % (Not finished)
\newcommand{\lgb}{\mathbf{b}} % (Not finished)
\newcommand{\lgd}{\mathbf{d}} % (Not finished)
\newcommand{\g}{\mathbf{g}} % (Not finished)
\newcommand{\h}{\mathbf{h}} % (Not finished)
\newcommand{\lgk}{\mathbf{k}} % (Not finished)
\newcommand{\lgn}{\mathbf{n}} % (Not finished)
\newcommand{\lgo}{\mathbf{o}} % (Not finished)
\newcommand{\lgu}{\mathbf{u}} % (Not finished)
% Identity matrix
\newcommand{\id}{\mathbb{I}} % (Not finished)
% Zero matrix
\newcommand{\mzero}{\mathbb{O}}
% Trace
\newcommand{\tr}{\mbox{Tr}}
% Parentesis
\newcommand{\lp}{\left(}
\newcommand{\rp}{\right)}
\newcommand{\lbp}{\left\{}
\newcommand{\rbp}{\right\}}
% 1/2
\newcommand{\half}{\frac{1}{2}}
% Binom that isnt too darn high
\newcommand{\Binom}[2]{{}_{#1}\mathcal{C}_{#2}}
% Ordo
\newcommand{\Ordo}{\mathcal{O}}
% Invariant with a star
\newcommand{\US}[1]{U_{#1\star}}
\newcommand{\VS}[1]{V_{#1\star}}
\newcommand{\WS}[1]{W_{#1\star}}
% Triple v
\newcommand{\VVV}{WV}

% n/m fractions
% Ex: \nfrac{2}{3}
\newcommand{\nfrac}[2]{{}^{#1}\!\!/\!{}_{#2}}

%*******************************************************************
%
%  My own commands.
%               -Simon Vajedi
%
%*******************************************************************

\renewcommand{\vec}[1]{\boldsymbol{#1}}
\newcommand{\dvec}[1]{\dot{\vec{#1}}}
\newcommand{\ddvec}[1]{\ddot{\vec{#1}}}
\renewcommand{\d}{\mathrm{d}}
\renewcommand{\D}{\mathrm{D}}
\newcommand{\St}{\mathrm{St}}
\newcommand{\Ku}{\mathrm{Ku}}
\renewcommand{\Re}{\mathrm{Re}}

% detta gör texten bredare
\addtolength{\textwidth}{2cm}
\addtolength{\hoffset}{-1cm}

%**************************************************
%
%       Biblio defs.
%
%**************************************************

\newcommand{\article}[4]{
\bibitem{#1}
#2, (#3) {\em #4}
}


% And here the document begins!
\begin{document}
%%%%%%%%%%%%%%%%%%%%%%%%%%%%%%%%%%%%
%
% Cover
%
%%%%%%%%%%%%%%%%%%%%%%%%%%%%%%%%%%%%

%\include{cover}

%\blankpage

\pagenumbering{roman}
\setcounter{page}{1}

%%%%%%%%%%%%%%%%%%%%%%%%%%%%%%%%%%%%%
%
% Title page
%
%%%%%%%%%%%%%%%%%%%%%%%%%%%%%%%%%%%%%

\thispagestyle{empty}

%\vspace*{-1cm}
%\vspace*{4mm}

\begin{center}
  {\fsh\ffam\fser Thesis for the degree of Master of Science in Engineering Physics}
\end{center}


\vspace*{0.5cm}

\begin{center}
{\upshape\sffamily\bfseries\LARGE \titel}
\end{center}

\vspace*{2mm}
\begin{center}
        \rule{110mm}{2pt}
\end{center}

\vspace*{4mm}
\begin{center}
  {\fsh\ffam\fser\Large Simon Vajedi}\\
\end{center}
\vfill

\begin{center}
%       \includegraphics[width=4cm]{/usr/local/lib/cthlogo.eps} 
\scalebox{1.3941}{\includegraphics*[1.8cm,1.8cm][4.8cm,4.6cm]{pics/ChalmGUmarke.epsi}}
%\includegraphics[width=9cm]{pics/ChalmGUmarke.epsi} 
%       \epsffile{/home/tfe/gran/tex/figures/AvancezM70mm.eps}
%       \hspace*{2cm}
%       \includegraphics[width=4cm]{/usr/local/lib/gulogo.eps} 
\end{center}

\vfill
\begin{center}
        {\ffam\fsh Applied Physics\\*[1mm]
        Chalmers University of Technology\\*[2mm]
        Göteborg, Sweden 2013}
\end{center}


%%%%%%%%%%%%%%%%%%%%%%%%%%%%%%%%%%%%
%
% Tryckort-sida
%
%%%%%%%%%%%%%%%%%%%%%%%%%%%%%%%%%%%%

\mbox{}\thispagestyle{empty}\newpage
\vspace*{165mm}

%{\ffam
%       {\center ISBN  ??-????-???-? \\ ISSN ????-???? \\}
%       \hspace*{20mm}
%       {\center Bibliotekets reproservice\\ Göteborg 2003 \\}
%}

\blankpage

%%%%%%%%%%%%%%%%%%%%%%%%%%%%%%%%%%%
%
% Abstract
%
%%%%%%%%%%%%%%%%%%%%%%%%%%%%%%%%%%%

\thispagestyle{empty}
\begin{center}
        {\ffam 
        {\fser\Large \titel}\\
%       {\fser\large A study of conformally invariant actions}\\[6mm]
%       {\fsh\ffam\fser\Large On Various Aspects of $p$-branes}\\[4mm]
\vspace{3mm}
        {\normalsize Simon Vajedi  \\
        Applied Physics \\
        Chalmers University of Technology\\
        SE-412 96 Göteborg, Sweden} \\[7mm]}
\end{center}

\centerline{\ffam\fser Abstract}
\medskip
\normalsize
\noindentIn this thesis we outline U-duality covariant dynamics for a $D_1$-brane in 9-dimensional supergravity and for a 
$D_2$-brane in 8-dimensional supergravity. By introducing modified field strengths, all world-volume fields couple 
to the background fields in a manifestly U-duality symmetric way. Our methods produce an action involving a function 
$\Phi$ which can only be deduced by demanding it to fulfill certain duality relations. This is achieved (to some extent) 
by implementing the equations in a computer program. Furthermore, solving the field equations for some of the potentials 
produces integration parameters which can be identified as the brane charges.
The thesis also contains an introduction to supergravity, Kaluza-Klein reduction and U-duality, as well as the complete 
construction of bosonic 9-, 8- and 7-dimensional supergravity.





\vfill

\newpage

%%%%%%%%%%%%%%%%%%%%%%%%%%%%%%%%%%%%%%%%%
%
% Acknowledgments
%
%%%%%%%%%%%%%%%%%%%%%%%%%%%%%%%%%%%%%%%%%
\thispagestyle{plain}
\vspace*{4cm}

\centerline{\ffam\fser\Large Acknowledgments}
\medskip
\smallskip

\normalsize
\noindentWe would like to express our gratitude to our supervisor Bengt E.W. Nilsson, for guidance and for introducing us to the 
fascinating world of M-theory. A special thanks for never losing your patience although, at times, it seemed 
like the work with this thesis could go on forever. The authors would also like to thank each other for good work and 
cooperation and for the complete understanding of the importance of breaks.



 

%%%%%%%%%%%%%%%%%%%%%%%%%%%%%%%%%%%%%%%%%%%%%
%
% Table of contents
%
%%%%%%%%%%%%%%%%%%%%%%%%%%%%%%%%%%%%%%%%%%%%%
%\pagestyle{plain}
\cleardoublepage
\tableofcontents
\pagestyle{empty}

\cleardoublepage
\pagestyle{fancy}
\renewcommand{\chaptermark}[1]{\markboth{Chapter \thechapter\ \ \ #1}{#1}}
\renewcommand{\sectionmark}[1]{\markright{\thesection\ \ #1}}
\lhead[\fancyplain{}{\sffamily\thepage}]%
  {\fancyplain{}{\sffamily\rightmark}}
\rhead[\fancyplain{}{\sffamily\leftmark}]%
  {\fancyplain{}{\sffamily\thepage}}
\cfoot{}
\setlength\headheight{14pt}

\rctr

\setcounter{page}{1}
\pagenumbering{arabic}

%%%%%%%%%%%%%%%%%%%%%%%%%%%%%%%%%%%%%%%%%%%%
%
% Text
%
%%%%%%%%%%%%%%%%%%%%%%%%%%%%%%%%%%%%%%%%%%%%
%\baselineskip=14.56pt

\chapter{Introduction}

Most fluid systems in nature contain more than one species of particles, and it is therefore important to understand the behavior and dynamics of these kinds of particle systems. They are described by the Navier-Stokes equations commonly used for single-component fluids, but with moving boundary conditions. This would be hard to solve explicitly and it would furthermore become problematic to analyse the properties of the system.

%Instead different models have been proposed and .... by e.g. Maxey and Riley (1983). 

In order to analyse the dynamics of inertial particles the equation of motion should be be expressed as an ordinary differential because then the tools of dynamical systems theory are accessible. The Maxey-Riley equation is one of those equations. 


This thesis concerns the investigation of small finite-size particles in a fluid, where the density of the 
particles differs from that of the fluid. 


The forming of rain droplets in clouds is not fully understood, and more sophisticated models are needed in order to take the great size of the rain droplets into account. 

\section{Outline}

%\part{Dimensional Reduction}
%\chapter{Colloidal systems/Random flows}
\chapter{Nature of Flows} 

compressible vs incompr.
viscuous
reynolds number
laminar vs turbulent flow
newtonian vs nonNewtonian

\section{Dynamics of Fluids}

tracers (active tracer, passive tracer). 
define $\vec u, \vec r, t$

\subsection{Material Derivative}

To measure changes of an arbitrary material property $\alpha(\vec r, t)$ of a fluid which depends on time $t$ and position $\vec r(t) \equiv (x(t), y(t), z(t))$, it is possible to measure $\alpha$ locally at a fixed point in space. That constitutes the \emph{Eulerian derivative} $\partial\alpha / \partial t$. It is also possible to follow the flow of the fluid and measure the property changes along fluid trajectories. It would then be necessary to take the derivative with respect to all variables. 
%Consider an arbitrary material property $\alpha(\vec r, t)$ of the fluid which depends on time $t$ and position $\vec r(t) \equiv (x(t), y(t), z(t))$. 
A small change $\d \alpha$ during time $\d t$ will then be given by
%\beq
%\d \alpha = \pafrac{\alpha}{t}\d t + \pafrac{\alpha}{x}\d x + \pafrac{\alpha}{y}\d y + \pafrac{\alpha}{z}\d z,
%\eeq
%and the change of $\alpha$ during time $\d t$ is 
\beq
\defrac{ \alpha}{t} = \pafrac{\alpha}{t} + \pafrac{\alpha}{x}\defrac{x}{t} + \pafrac{\alpha}{y}\defrac{y}{t} + \pafrac{\alpha}{z}\defrac{z}{t}.
\eqlbl{materchange}
\eeq
Since the velocity of the fluid is $\vec u = \left(\defrac{x}{t},\defrac{y}{t},\defrac{z}{t}\right)$, \eq{materchange} can be written as
\beq
\defrac{ \alpha}{t} = \pafrac{\alpha}{t} +( \vec u \cdot \nabla) \alpha.
\eqlbl{materderiv}
\eeq
This is the \emph{material derivative} with respect to $\alpha$. $\alpha$ could denote any property of the fluid, e.g. pressure, temperature, density, momentum, and the list goes on. Notice that the first term on the right-hand side is the Eulerian derivative, and the second term accounts for spatial variations of $\alpha$. It is in the field of fluid dynamics common to denote the material derivative by $\D/\D t$. 

%To express the time derivative of the fluid it is in the field of fluid dynamics conventional to replace the operator $\d/\d t$ with $\D/\D t$, and this is done to better differentiate between particle and fluid properties. 

A very useful property is the acceleration of a fluid element at position $\vec r$, which is obtained by setting $\alpha \equiv \vec u$ in \eq{materderiv}:%\footnote{$\nabla \vec u$ is the ...}
% bra källa: http://www.scribd.com/doc/106526009/138/A-7-Covariant-Derivatives-of-Tensors
\beq
\Dfrac{\vec u}{t} = \pafrac{\vec u}{t} + (\vec u \cdot \nabla) \vec u.
\eqlbl{fluidacc}
\eeq
%This acceleration will be used to describe motion and forces of the fluid. 
%This acceleration will be used extensively in this thesis to model the motions of the particles and the surrounding fluid. 
We could also insert $\alpha \equiv \rho_f$ in \eq{materderiv}, where $\rho_f$ is the density of the fluid, to obtain
\beq
\defrac{ \rho_f}{t} = \pafrac{ \rho_f}{t} + \vec u \cdot \nabla  \rho_f.
\eqlbl{materrho}
\eeq
Using the \emph{mass continuity equation} 
\beq
\pafrac{\rho_f}{t}+\nabla \cdot (\rho_f \vec u) = \pafrac{\rho_f}{t}+\nabla \rho_f\cdot \vec u + \rho_f\nabla \cdot \vec u = 0,
\eeq
\eq{materrho} transforms into
\beq
\defrac{ \rho_f}{t} =-\nabla \rho_f\cdot \vec u - \rho_f\nabla \cdot \vec u  + \vec u \cdot \nabla  \rho_f =  - \rho_f\nabla \cdot \vec u.
\eeq
An \emph{incompressible flow} is defined as having constant density, which implies that $\d  \rho_f / \d t = 0$ and consequently 
\beq
%\pafrac{ \rho_f}{t} + \vec u \cdot \nabla  \rho_f = 0.
\nabla \cdot \vec u = 0.
\eeq
This condition applies to all incompressible flows. 

\subsection{The Navier-Stokes Equations}

The Navier-Stokes equations are very useful and have many applications in many different areas. 

The motion of fluids is governed by the Navier-Stokes equations. These equations are derived using Newton's second law as well as the conservation laws for momentum, mass and energy. Furthermore, it is assumed that the fluid is a continuum. Denote $\vec u(\vec r, t)$ the velocity of the fluid at position $\vec r(t) \equiv (x(t), y(t), z(t))$ and time $t$. The general form of the equations is
%Considering the form of the fluid acceleration given in \eq{fluidacc} the Navier-Stokes equations should be of the form
\beq
\rho_f \Dfrac{\vec u}{t} = \nabla \cdot \vec \sigma + \vec f,
\eeq
where $\vec f$ is the sum of all body forces, and $\vec \sigma$ is the stress tensor given by
\beq
\vec \sigma = 
\begin{pmatrix}
\sigma_{xx} & \sigma_{xy} & \sigma_{xz} \\
\sigma_{yx} & \sigma_{yy} & \sigma_{yz} \\
\sigma_{zx} & \sigma_{zy} & \sigma_{zz} 
\end{pmatrix}.
\eeq
%where $\rho_f$ is the density of the fluid and $\vec f$ the forces per unit volume exerted on the fluid parcel. 

\section{Turbulence}

Turbulent systems are chaotic and irregular, motion of fluids.  

\section{Simulating flows}

\chapter{Colloidal Systems}


% - Different cases/parameters to consider (large particles, inertialess....)
% - The models to describe these cases
% - The maxey-riley equation and why it's useful

Most fluid systems in nature contain more than one species of particles, and it is therefore important to understand the behavior and dynamics of these kinds of particle systems. They are described by the Navier-Stokes equations commonly used for single-component fluids, but with moving boundary conditions. This would be hard to solve explicitly and it would furthermore become problematic to analyse the properties of the system.

%Instead different models have been proposed and .... by e.g. Maxey and Riley (1983). 

In order to analyse the dynamics of inertial particles the equation of motion should be be expressed as an ordinary differential because then the tools of dynamical systems theory are accessible. The Maxey-Riley equation is one of those equations. 


This thesis concerns the investigation of small finite-size particles in a fluid, where the density of the 
particles differs from that of the fluid. 


The forming of rain droplets in clouds is not fully understood, and more sophisticated models are needed in order to take the great size of the rain droplets into account.



%A lot of research has been done to make the right kind of approximations to find a sufficiently accurate, and simple, model for different fluid systems. Different fluid systems can be very different in nature, Not all fluids behave the same way, so a good model for one system is not necessarily good for the next. In this section we will study different models that have been used in this field. The equation of motion that will be treated with extra care in this thesis is the Maxey-Riley equation, but first the simple advective model.

%First we will examine a very simple model which has been used extensively in numerous occasions. the dynamics of point particles with no inertia. The approximation is that the particles have no size and no mass. This approximation is valid in many situations and has been used extensively in various applications. 
\section{Equations of Motion}

In order to describe the dynamics of particles in a turbulent fluid it is important to construct a model with appropriate approximations. The Navier-Stokes equations with moving boundary conditions could in principle be used to describe the motion of the particles. These equations are, however, very difficult to solve for turbulent systems, both analytically and numerically. It is necessary to construct more simple models that are easier to analyse but sufficiently accurate to describe the relevant phenomena. These models may be very illuminating as new interesting properties can be discovered of the system. 

%Depending on the system, different models are more or less accurate. Many mistakes have bee

%olika approximationer/ekvationer för olika fall

%Thus, a model that best describes a system is not necessarily the ideal one. Simple models which can be evaluated analytically may be 

\subsection{Advective Model}

Let us first consider point particles with no inertia. This approximation is valid when the particle mass and size are negligible. The equation of motion is in this model given by
\begin{equation}
\dvec{r}(t) = \vec u(\vec{r}(t),t),
\eqlbl{inertialess}
\end{equation}
where $\vec r (t)$ is the particle position at time $t$ and $\vec u $ is the fluid velocity. The dots over variables denote time derivatives. As evident from this equation, the particle will in this model follow the flow completely and at every point take on the velocity of the fluid. This phenomenon is called \emph{advection}.

This model is valid in numerous cases and has been used extensively in applications, especially in the early days. However, it does not account for many interesting properties of ....

%Thus, inertial particles Particles which satisfy this equation of motion are 
%\emph{advected} and are called \emph{passive tracers}. This approximation 
%has many applications in fluid dynamics and in the early days, 
%\eq{inertialess} was used predominantly, neglecting the mass and size 
%of the particles. However, in order to describe 
%more complex behavior like clustering and path coalescence, this is not enough.

\subsection{Stokes' Law}
\sseclbl{stokeslaw}

%Suspended particles with finite size have inertia, and the result is that 
%the particles do not only follow the flow of the surrounding fluid. They 
%may coalesce and cluster, and ... This is 
%how rain cloud can form and .... In order to analyse these interesting 
%behaviors of inertial particles more sophisticated models are necessary.

To improve upon the simple advective model, let us consider finite-size particles. These particles have inertia, and the result is that the particles do not only follow the flow of the surrounding fluid. To account for particle inerta, consider a spherical particle with radius $a$ and mass $m_p$. Now, as depicted in Fig 2.1, the velocity of the flow close to the particle surface differs from the flow velocity a little distance away. Let us neglect this effect and instead consider the relative velocity between the particle and the fluid as $\vec w = \vec u(\vec r, t) - \dvec r$. The particle Reynolds number is then defined to be Re$_p \equiv L_0 |\vec w| / \nu $. 

The fricitonal force, commonly referred to as Stokes' drag, is
\beq
\vec F = 6 \pi a \nu \rho_f \vec w,
\eeq
and, assuming that this is the dominant force determining the motion of the particle, the equation of motion becomes 
\beq
\ddvec r = \gamma [\vec u (\vec r ,t) - \dvec r],
\eqlbl{stokeslaw}
\eeq
where $\gamma = 6 \pi a \nu \rho_f/m$ is the damping rate. 
This equation is hard to evaluate because the fluid velocity $\vec u$ depends on $\vec r$. 

This model assumes that the major forces are due to viscuous drag, and that the particle size is small so that the velocity field changes negligibly across the particle. Moreover, the interaction between particles is not taken into account, and effects of the inertia of the displaced fluid parcels are neglected. We will next consider corrections to the Stokes' law \eqref{stokeslaw} using the Maxey-Riley equation.

%The Stokes number $\St = 1/\gamma\tau$ is the (inverse?) intensity of the damping. When $\St \rightarrow 0$ the particles are completely advected and behave like point particles with no inertia, and the equation of motion is then given by \eq{inertialess}. 

%\emph{Vorticity} is the measure of how fast fluid elements rotate about themselves. The curl of the velocity field is the vorticity field and is denoted $\vec{\omega} = \nabla \times \vec{u}$. Vorticity is a local measure and does is not necessarily nonzero even if the fluid rotates at a large scale. A vortex is a region of high vorticity.

% http://www.scribd.com/doc/81564634/Aurelie-Goater-Dispersion-of-Heavy-Particles-in-an-Isolated-Pancake-Like-Vortex
% Equation of motion for a small rigid sphere in a nonuniform flow

\subsection{The Maxey-Riley Equation}

% Bra k�lla: http://www.lec.csic.es/~julyan/PDFs/79_2010_Springer.pdf

%- For small spherical rigid particle advected by a (smooth) flow.

%- Valid for small particles at low particle Reynolds numbers Re$_p$.

%- Velocity difference across the particle must be small

%Intro historia om Maxey-Riley equation, how it come to and evolved why it is important and useful, what can you describe with it? 

The Maxey-Riley equation is valid for small, spherical and rigid particles advected by a smooth flow, i.e. flows that are predominantly laminar. The particles should have small Reynolds numbers Re$_p$, and the velocity difference across 
the particle must be small. 

%The Maxey-Riley equation for a particle of density $\rho_p$ and radius $a$ is given by

%\begin{align}
%m_p\dvec{v} = & m_f \frac{\D}{\D t}\vec{u}(\vec{r}(t),t)- \frac{1}{2}m_f\left(\dvec{v}-\frac{\D}{\D t}\left[\vec{u}(\vec{r}(t),t)+\frac{1}{10}a^2\nabla^2\vec{u}(\vec{r}(t),t)\right]\right) \nonumber \\& -6\pi a \rho_f \nu \vec{q}(t) + (m_p-m_f)\vec{g}-6\pi a^2\rho_f \nu \int\limits_0^t \frac{\d \tau}{\sqrt{\pi\nu (t-\tau)}} \frac{\d \vec{q}(\tau)}{\d \tau},
%\end{align}

We will consider one term at a time. 

\subsubsection{Buoyancy Force}

Previously we have neglected the gravitational force and assumed that the dynamics of the particles are mainly influenced by the fluid, but how would rain droplets fall if not for gravity? The bouyancy force is the correction of the differences of density between the particles and the surrounding fluid. It has the simple form
\beq
\vec f_b = (\rho_p - \rho_f) \vec g ,
\eqlbl{bouyancyforce}
\eeq
%  $m_f$ is the mass of the fluid parcel displaced by the sphere, and
where $\vec g$ is the gravitational acceleration and %$V_p$ the volume of the particle. 
$\vec f_b$ is the force per unit volume. If the particle density is less than that of the fluid they are referred to as \emph{bubbles}. [lite om bubbles]. 

There are also systems with \emph{neutrally buoyant particles}, e.g. plankton in the oceans, where $\rho_p = \rho_f$. [lite om fysiken i detta fall] ]However, this thesis will mainly focus on systems where $\rho_p > \rho_f$.

%The first term on the right-hand side of the Maxey-Riley equation is the bouyancy force due to gravity. When $\rho_p < \rho_f$ the particles are lighter and hence referred to as \emph{bubbles}. [lite om bubbles] 

%However, this thesis will mainly focus on systems where the particle density is greater than that of the fluid. This is the case of aerosols and cumulus clouds. 

%This thesis will mainly focus on \emph{particle suspension}, where the particle density is greater than that of the fluid. In the reverse? case, the particles are lighter and hence referred to as \emph{bubbles}.  

\subsubsection{Force of the Undisturbed Flow}

The effects of the undisturbed fluid is evaluated in position $\vec r$, at the center of the particle sphere. The force is the same as the flow would exert on a fluid element of the same size as the particle, and it is applied in the direction of the trajectory of the \emph{fluid element}, not the particle trajectory. Using Newton's second law and the fluid acceleration given in \eq{fluidacc} the force becomes
\beq
\vec f_f = \rho_f \Dfrac{\vec u}{t}.
\eeq

%To understand which direction the fluid is moving, let us first consider an arbitrary material property $\alpha(\vec r, t)$ of the fluid which depends on time $t$ and position $\vec r(t) \equiv (x(t), y(t), z(t))$. A small change $\d \alpha$ will then be given by
%\beq
%\d \alpha = \pafrac{\alpha}{t}\d t + \pafrac{\alpha}{x}\d x + \pafrac{\alpha}{y}\d y + \pafrac{\alpha}{z}\d z.
%\eeq
%Thus the change of $\alpha$ during time $\d t$ is 
%\beq
%\defrac{ \alpha}{t} = \pafrac{\alpha}{t} + \pafrac{\alpha}{x}\defrac{x}{t} + \pafrac{\alpha}{y}\defrac{y}{t} + \pafrac{\alpha}{z}\defrac{z}{t}.
%\eqlbl{materchange}
%\eeq
%Since the velocity of the fluid is $\vec u = \left(\defrac{x}{t},\defrac{y}{t},\defrac{z}{t}\right)$, \eq{materchange} can be written as
%\beq
%\defrac{ \alpha}{t} = \pafrac{\alpha}{t} + \vec u \cdot \nabla \alpha.
%\eeq
%This is the \emph{material derivative}, and $\alpha$ could denote any property of the fluid, e.g. pressure, temperature, density, and the list goes on. To express the time derivative of the fluid it is in fluid dynamics conventional to replace the operator $\d/\d t$ with $\D/\D t$, and this is done to better differentiate between particle and fluid properties. 

%The acceleration of the fluid element is obtained by setting $\alpha \equiv \vec u$:
%\beq
%\Dfrac{\vec u}{t} = \pafrac{\vec u}{t} + \vec u \cdot \nabla \vec u
%\eeq
%This is the acceleration of the undisturbed fluid in position $\vec r$. The force exerted on the particle by the undisturbed fluid is thus 
%\beq 
%\vec F_f = \rho_f V_p \Dfrac{\vec u}{t},
%\eeq
%and is evaluated along the trajectory of the fluid element rather than the particle trajectory.

%The second term is the force The derivative 
%\begin{equation}
%\frac{\D\vec u}{\D t} = \frac{\partial \vec{u}}{\partial t} + \vec u \cdot \nabla \vec u
%\end{equation}
%is the \emph{convective derivative} and is evaluated along the trajectory of the fluid element rather than the particle trajectory. 

\subsubsection{Added Inertia}

An accelerating particle moves an amount of fluid as it travels along its trajectory, because it cannot be in the same position as fluid particles simultaneously. An exception would be if the particle is inertialess and completely follows the flow. This gives rise to a drag force which causes the particle to lose kinetic energy, and the result is that the particle appears to have additional inertia. This phenomenon is commonly referred to as the \emph{added-mass effect}. 
%It is a fricitonal force and in practice slows the particle down. 
The added-mass term has the form
\beq 
\vec f_m = -\frac{\rho_f }{2}\left\{\ddvec r - \Dfrac{}{t}\left[\vec u + \frac{1}{10}a^2\nabla^2\vec u\right]\right\}.
\eqlbl{addedmassforce}
\eeq
The minus sign accounts for the frictional nature of the force. The term $a^2\nabla^2\vec u / 10$ is the \emph{Faxen correction} and is due to spatial variations of the velocity field. This term has mostly mainly been neglected for practical reasons, but also because the radius $a$ of the particle is usually small so that the velocity field of the fluid does not change significantly across the particle. It is easy to verify using \eq{addedmassforce} that if the particle is inertialess and follows the flow the particle and the fluid at position $\vec r$ would accelerate towards the same direction and $\vec f _m  = 0$, i.e. the particle displaces no fluid. 

\subsubsection{Stokes' Drag}

Stokes' drag is the frictional force due to the viscosity of the fluid. This was discussed in \ssec{stokeslaw}, but now a Faxen correction is included to take the flow variations into account. The form of Stokes' drag is then
\beq
\vec f_{d}=- \frac{9 \rho _f \nu }{2 a^2}\vec Q,
\eeq
where
\begin{equation}
\vec Q = \dvec r - \vec u-\frac{1}{6}a^2\nabla^2\vec u.
\end{equation}
%It does makes sense that large viscosity would make the particle slower in the fluid, so we expect Stokes' drag to be proportional to $-\nu$. 

\subsubsection{Basset-Boussinesq History Term}

The integral is the \emph{Basset-Boussinesq history force}, which accounts for the fact that the vorticity diffuses away from the particle due to viscosity. wiki: This force is usually neglected, but it can be large if the particle accelerate at a high rate. 

% is the force exerted by a fluid element in position $\vec{r}(t)$ and corresponds to the force by the force of the undisturbed fluid. The second term is due to the \emph{added-mass effect}, which results from the fact that the particle displaces a certain amount of fluid along its trajectory, which makes the particle appear to have additional mass. The third and the fourth terms result from the viscosity and the buoyancy force, respectively, of the fluid, which represent the Stokes drag. The factor of $a^2\nabla^2 \vec{u}$ is due to the spatial variation of the velocity field across the particle, and the terms containing this are called \emph{Faxen corrections}.

\subsubsection{The Maxey-Riley Equation}
The final form of the Maxey-Riley equation is obtained by summing all the forces so far. The equation of motion becomes
\begin{align}
\rho_p \ddvec r = &(\rho _p - \rho _f) \vec g + \rho _f 
\frac{\D \vec u}{\D t} - \frac{\rho_f}{2}\left\{\ddvec r - 
\frac{\D}{\D t}\left[\vec u + \frac{1}{10}a^2\nabla^2 \vec u \right] 
\right\} \nn \\ &- \frac{9 \rho _f \nu}{2 a^2}\left\{\vec Q + 
a \int_0^t \d \tau \frac{\dvec Q (\tau )}{\sqrt{\pi \nu (t-\tau )}}
\right\}.
\end{align}
%Here dots over variables denote time derivatives. Note that $\vec r\equiv \vec r(t)$, $\rho_f$ is the fluid density, $\nu$ the kinematic viscosity of the fluid, and $\vec g$ the gravitational acceleration. 
*The conditions where MR holds. (small particles)
*I will consider corrections for larger particles (faxen corrections will change)

%\chapter{Kaluza-Klein reduction on $T^2$, $T^3$ and $T^4$}
\label{sec:reduct}
In the formulation of the membrane dynamics, every background $n$-form gauge 
field will couple to its corresponding ($n$-1)-form gauge field on the 
world-volume through a universal coupling outlined in the next section. Our 
first task is thus to derive the background fields by reducing 11-dimensional 
supergravity to 9-, 8- and 7-dimensional maximal supergravity. By using the coupling mentioned 
above the membrane dynamics will automatically be covariant with respect to the 
global symmetries of the background it couples to, and we will therefore try to write the Bianchi 
identities and the duality relations in a form where the covariance with respect to the symmetry groups 
given in table 2.3 is manifest. The dimensional reduction of \eqnref{sugra_11dim} will be done the standard 
Kaluza-Klein way on a 2-, 3- and 4-torus. The $T^3$ case can also be found in for example \cite{artikeln}.
We start from the derived Kaluza-Klein Ansatz \eqnref{sugra_kk_ansatz} for the vielbeins 
\begin{equation}
{\hat e}{_{M}}^A = \toto{ e{_{\mu}}^a }{ A_{\mu}^{1m}e{_{m}}^i
  }{ 0 }{ e{_{m}}^i}, \hspace{0.5cm} {\hat e}{_{A}}^M = \toto{ e{_{a}}^\mu }{
  -e{_{a}}^{\mu}A_{\mu}^{1m} }{ 0 }{ e{_{i}}^m},
\eqnlab{reduct_kk_ansatz}
\end{equation}
\[
dx^M = (dx^\mu,dx^m),
\]
where ${\hat e}{_M}^A {\hat e}{_{NA}} = {\hat g}_{MN}$ and the fields
depends on $x^\mu$ only. Hats are used for 11-dimensional
fields. The Ansatz is of course independent of how many dimensions that are compactified. This gives
\begin{align}
{\hat e}^A &= dx^M {\hat e}{_M}^A = (dx^\mu e{_\mu}^a,dx^\mu A_\mu^{1m} e{_{m}}^i + dx^m e{_{m}}^i) \nonumber \\
&= (e^a,A^{1i} + e^i) = ({\hat e}^a,{\hat e}^i),
\eqnlab{reduct_viel_red}
\end{align}
where the index 1 on $A_{\mu}^{1m}$ separates the internal vector from 
the vector that will arise from the compactified 3-form. Using this we can 
calculate the contribution of the 11-dimensional Einstein term to the lower-dimensional supergravities.

\section{The Einstein term}
The calculation of the Einstein term is somewhat lengthy and can be
found in section \secref{ricci}. Here we start with the result
\begin{align}
\int d^{11}x \sqrt{|\hat{g}|} \hat{R} =& \int d^{d}x \sqrt{|g|} e^{-\varphi} \Big{[}R - {1 \over 4}G_{mn}F_{ab}^{1m}F^{1n \hspace{0.05cm} ab} - 2D_a (e{_i}^{m} \partial^ae{_m}^i) \nonumber \\
&- e^{n(i}({\partial}_a e{_n}^{j)})e{_i}^m(\partial^a e_{mj}) -e{_i}^m({\partial}^a e{_m}^i) e{_j}^n({\partial}_a e{_n}^j)\Big{]},
\label{compR}
\end{align}
where we have used that
\begin{equation}
\sqrt{\det\hat{g}_{MN}} = \det e{_M}^A = \det e{_\mu}^a\det e{_m}^i = \sqrt{\det g_{\mu \nu}} \sqrt{\det G_{mn}},
\end{equation}
where $G_{mn}$ is the metric on $T^D$ constructed from the D-bein $e{_m}^i$, and also the definitions
\begin{equation}
\sqrt{\mbox{det}G_{mn}}=e^{-\varphi}, \hspace{1cm} F_{ab}^{1m}=dA_{ab}^{1m}.
\eqnlab{reduct_f1def}
\end{equation}
Note also that we have put both the prefactor $2 {\kappa}_{11}^2$ in \eqnref{sugra_11dim} and vol($T^D$)$=\int d^Dy$ equal to 1, where vol($T^D$) is the volume of the internal torus.
The term with the covariant derivative can be integrated by parts to give
\begin{align}
&-2 \int d^dx \sqrt{|g|} e^{-\varphi} D_a (e{_i}^m \partial^a e{_m}^i) = -2 \int d^dx e^{-\varphi} {\partial}_a (\sqrt{|g|}e{_i}^m \partial^a e{_m}^i) \nonumber \\
&=2\int d^dx\sqrt{|g|}({\partial}_a e^{-\varphi})e{_i}^m \partial^a e{_m}^i = 2\int d^dx \sqrt{|g|} e^{-\varphi} (-{\partial}_a \varphi)\mbox{Tr}(e^{-1}\partial e) \nonumber \\
&=2 \int d^dx \sqrt{|g|} e^{-\varphi} ({\partial}_a \varphi) ({\partial}^a \varphi),
\label{t1}
\end{align}
where we have used \eqnref{conven_divergence} and \eqnref{conven_trlog}. The last term in (\ref{compR}) can be rewritten as
\begin{equation}
-\int d^dx \sqrt{|g|} e^{-\varphi}e{_i}^m({\partial}^a e{_m}^i) e{_j}^n({\partial}_a e{_n}^j) = -\int d^dx \sqrt{|g|} e^{-\varphi} ({\partial}_a \varphi) ({\partial}^a \varphi),
\label{t2}
\end{equation}
where we have again used \eqnref{conven_divergence}. And finally the fourth term in (\ref{compR}) can be rewritten as
\begin{align}
&-e^{n(i}({\partial}_a e{_n}^{j)})e{_j}^m(\partial^a e_{mi}) \nonumber \\
&= - {1 \over 2}e^{ni}({\partial}_a e{_n}^{j})e{_j}^m(\partial^a e_{mi})- {1 \over 2}e^{ni}({\partial}_a e{_n}^{j})e{_i}^m(\partial^a e_{mj})\nonumber \\
&=- {1 \over 2}G^{mn}e{_m}^{i}({\partial}_a e{_n}^{j})G^{pq}e_{jq}(\partial^a e_{pi})- {1 \over 2}G^{mn}({\partial}\bd{a} e\bd{n}\bu{k})(\partial\bu{a} e\bd{ml}){\delta}_{k}^j{\delta}_{j}^l \nonumber \\
&=- {1 \over 4}G^{mn}G^{pq} \Big{(}e{_m}^{i}({\partial}_a e{_n}^{j})e_{jq}(\partial^a e_{pi})+e{_n}^{i}({\partial}_a e{_m}^{j})e_{jp}(\partial^a e_{qi})\Big{)}\nonumber \\
&\hspace{0.47cm}-{1 \over 2}G^{mn}e^{jp}e_{pk}({\partial}_a e{_n}^{k})e{^l}_q e{^q}_j(\partial^a e_{ml})\nonumber \\
&= - {1 \over 4}G^{mn}G^{pq}\Big{(}e_{jq}({\partial}_a e{_n}^{j})e_{mi}(\partial^a e{_p}^i)+e_{jn}({\partial}_a e{_q}^{j})e_{ip}(\partial^a e{_m}^i)\nonumber \\
&\phantom{=}+e_{jq}({\partial}_a e{_n}^{j})e_{ip}(\partial^a e{_m}^i)+e_{jn}({\partial}_a e{_q}^{j})e_{im}(\partial^a e{_p}^i)\Big{)}\nonumber \\
&=-{1 \over 4}G^{mn}G^{pq}\Big{(}e_{jq}({\partial}_a e{_n}^j)+e_{jn}({\partial}_a e{_q}^j)\Big{)}\Big{(}e_{mi}(\partial^a e{_p}^i)+e_{pi}(\partial^a e{_m}^i)\Big{)}\nonumber \\
&=-{1 \over 4}G^{mn}({\partial}_a G_{qn})G^{pq}(\partial^a G_{mp})=-{1 \over 4}\mbox{Tr}(G^{-1} \partial G)^2.
\label{t3}
\end{align}
Inserting (\ref{t1}), (\ref{t2}) and (\ref{t3}) in (\ref{compR}) gives
\begin{align}
\int d^{11}x \sqrt{|\hat{g}|} \hat{R} =& \int d^9x \sqrt{|g|} e^{-\varphi}\Big{[}R + ({\partial}_a \varphi) ({\partial}^a \varphi) \nonumber \\
&-{1 \over 4}G\bd{mn}F_{ab}^{1m}F^{1n \hspace{0.05cm} ab} - {1 \over 4} \mbox{Tr}(G^{-1}\partial G)^2\Big{]}.
\label{finalaction}
\end{align}
The next step is to rescale the action to obtain an action in the Einstein frame. But before we perform the rescaling we note that only the index values 
separates equation (\ref{finalaction}) from the corresponding equation in 8 and 7 dimensions. Therefore we can make a general variable change according to
\begin{equation}
g{_{\mu \nu}} = e^{-2s\varphi}g_{\mu \nu}^E, \hspace{1cm} \M_{mn} = {G_{mn} \over (\mbox{det}G)^{1/D}},
\label{varchange2}
\end{equation}
where
\begin{equation}
s={1 \over 2-d}, \hspace{0.5cm} d=\mbox{dim}g{_{\mu \nu}}, \hspace{0.5cm} D=\mbox{dim}G{_{mn}}.
\end{equation}
The D $\times$ D matrix $\M_{mn}$ provides a parametrisation of the $SL(2,{\mathbb R})/$ \newline $SO(2,{\mathbb R})$
coset space in 9 dimensions, $SL(3,{\mathbb R})/SO(3,{\mathbb R})$ in 8 dimensions and $SL(4,{\mathbb R})/SO(4,{\mathbb R})$ in 7 dimensions. 
The short calculation
\begin{equation}
G_{mn}=(\mbox{det}G)^{1/D}\M_{mn}=e^{-2\varphi /D}\M_{mn} \hspace{0.5cm} \Rightarrow \hspace{0.5cm} \mbox{det}\M_{mn}=1,
\end{equation}
shows that all parametrisations are unimodular, a necessary condition for $SL$-groups. 
The first variable change in (\ref{varchange2}) gives
\begin{equation}
\mbox{det}g_{\mu \nu} = e^{-2ds\varphi}\mbox{det}g_{\mu \nu}^E.
\end{equation}
The terms in (\ref{finalaction}) will now be contracted with $g^E$ and $\M$ instead of $g$ and $G$ and thus change as
\begin{align}
(\partial \varphi)^2 &\rightarrow e^{2s\varphi}(\partial \varphi)^2 \hspace{0.2cm} \mbox{from}\hspace{0.2cm} g^{\mu \nu} \nonumber \\
G_{mn}F^{1m}F^{1n} &\rightarrow  e^{4s\varphi} e^{-2\varphi/D} \M_{mn}F^{1m}F^{1n} \hspace{0.2cm} \mbox{from}\hspace{0.2cm} (g^{\mu \nu})^2 \hspace{0.2cm}\mbox{and}\hspace{0.2cm} G_{mn} \nonumber \\
\mbox{Tr}(G^{-1} \partial G)^2 &\rightarrow e^{2s\varphi} \mbox{Tr}(e^{2\varphi/D}\M^{-1} \partial (e^{-2\varphi/D}\M))^2\hspace{0.2cm} \mbox{from}\hspace{0.2cm} g^{\mu \nu}\mbox{,}\hspace{0.2cm}G_{mn} \nonumber \\
& \hspace{0.65cm} \mbox{and }G^{mn}\nonumber,
\end{align}
altogether yielding
\begin{align}
\int d^{11}x \sqrt{|\hat{g}|} \hat{R} =& \int d^dx \sqrt{|g^E|}\Big{[}e^{-(ds+1)\varphi}R + e^{(-ds+2s-1)\varphi}(\partial \varphi)^2 \nonumber \\
&-{1 \over 4}e^{(-ds+4s-1-2/D)\varphi}\M_{mn}F^{1m}F^{1n} \nonumber \\
&-{1 \over 4}e^{(-ds+2s-1)\varphi} \mbox{Tr}(e^{2\varphi/D}\M^{-1} \partial (e^{-2\varphi/D}\M))^2\Big{]}.
\label{B1}
\end{align}
We can write the expression inside the trace as
\begin{align}
&\Big{(}e^{2\varphi/D}\M^{-1} \partial (e^{-2\varphi/D}\M)\Big{)}^2 \nonumber \\
&=\Big{(}e^{2\varphi/D}\M^{-1}\Big\{ -\frac{2}{D}\partial \varphi e^{-2\varphi/D}\M + e^{- 2\varphi/D } \partial \M\Big\}\Big{)}^2 \nonumber \\
&=\Big{(}-\frac{2}{D}\partial \varphi + \M^{-1} \partial \M\Big{)}^2 \nonumber \\
&=\frac{4}{D^2}(\partial \varphi)^2 - \frac{4}{D}(\partial \varphi)\M^{-1}\partial \M +(\M^{-1}\partial \M)^2
\end{align}
and we note that, from $\eqnref{conven_trlog}$
\begin{equation}
\mbox{Tr}(\M^{-1} \partial \M)=\partial(\ln \mbox{det}\M)=0.
\end{equation}
We also need to investigate how the change of variables affects the Ricci scalar. For that we use (see equation \eqnref{RicciE})
\begin{equation}
R = e^{2s\varphi}(R_E-s^2(d-1)(d-2)(\partial \varphi)^2),
\end{equation}
which inserted in (\ref{B1}) gives the action in the Einstein frame as
\begin{align}
\int d^{11}x \sqrt{|\hat{g}|} \hat{R} =& \int d^dx \sqrt{|g^E|}\Big{[}R^E - {9 \over D(d-2)}(\partial \varphi)^2 \nonumber \\
&-{1 \over 4}e^{-{18 \over D(d-2)}\varphi }\M_{mn}F^{1m}F^{1n}-{1 \over 4}\mbox{Tr}(\M^{-1}\partial \M)^2\Big{]}.
\label{B2}
\end{align}

\section{Field strengths and  Bianchi identities}
Let us now turn the attention to the 4-form field strength $\hat{G}=d\hat{C}$. Before we reduce the field strength 
we note that the curved 2-dimensional (or 3- or 4-dimensional) basis
\begin{equation}
\hat{e}^m = dx^m + A^{1m}, \hspace{1cm} A^{1m}=dx^{\mu}A_{\mu}^{1m},
\end{equation}
is invariant under general coordinate transformations $\delta x^m = -\lambda^m(x)$ of the internal torus as is seen from
\begin{align}
\delta e^m &= \delta(dx^m + dx^{\mu}A_{\mu}^{1m}) = \delta dx^m + \delta dx^{\mu}A_{\mu}^{1m} = d\delta x^m + dx^{\mu}(\delta A_{\mu}^{1m}) \nonumber \\
&= -d\lambda^m + dx^{\mu}({\partial}_{\mu}\lambda^m)=-d\lambda^m + d\lambda^m =0,
\end{align}
with $A_{\mu}^{1m}$ transforming under local U(1) gauge transformations
\begin{equation}
\delta A_{\mu}^{1m} = {\partial}_{\mu}\lambda^m
\end{equation}
which was shown in section \secref{sugra_kk}.
We now want to expand the 11-dimensional 3-form $\hat{C}$ into lower-dimensional components, where the fields 
preserve the $\lambda^m$ invariance. This is achieved by expanding in the basis above and by using the fact that 
$\hat{C}$ is manifestly invariant \eqnref{sugra_gauge11}. Note that when
integrating the 2, 3 or 4 compact directions in \eqnref{sugra_11dim}, only terms with
2, 3 or 4 $dx^m$'s will be non-zero. Hence $dx^m$ appears as $e^m$ from
the action's point of view, and in our expansion below we will
only have terms with 2 or less $e^m$'s in the $T^2$ case, 3 or less in
the $T^3$ case and so on.

\subsubsection{The $T^2$ case}
First we reduce the 11-dimensional relation
\begin{equation}
\delta \hat{C}=d \hat{\chi}
\label{tja}
\end{equation}
for the $T^2$-case according to the procedure we outlined above. The 11-dimensional 3-form becomes
\begin{align}
\hat{C} &= {1 \over {3!}} C_{PNM}dx^M \wedge dx^N \wedge dx^P \nonumber \\
&= {1 \over {3!}}\Big{(}C_{\rho \nu \mu}dx^{\mu} \wedge dx^{\nu} \wedge dx^{\rho} + 3C_{p \nu \mu}dx^{\mu} \wedge dx^{\nu} \wedge dx^{p} \nonumber \\
&\phantom{=} + 3 C_{p n \mu}dx^{\mu} \wedge dx^{n} \wedge dx^{p} +C_{pnm}dx^{m} \wedge dx^{n} \wedge dx^{p}\Big{)}\nonumber \\
&= \Big{\{ \mbox{The last term vanish because} \hspace{0.2cm}m,n,p=1,2\Big{\}}}\nonumber \\
&= C + {3 \cdot 2 \over 6} \underbrace{({1 \over 2}C_{\nu \mu}dx^{\mu} \wedge dx^{\nu})_p}_{B_{p}} \wedge dx^p + {3 \over 6} \underbrace{(C_{\mu}dx^{\mu})_{pn}}_{-A_{pn}^2 = A_{np}^2} \wedge dx^n \wedge dx^p \nonumber \\
&= C + B_{m} \wedge \hat{e}^m + {1 \over 2}{\epsilon}_{mn}A^2 \wedge \hat{e}^m \wedge \hat{e}^n,
\end{align}
where we have used \eqnref{reduct_viel_red} to see that
\begin{align}
d\hat e^m = d(A^{1m} + dx^m)) = F^{1m}.
\end{align}
Repeating this for the 2-form gauge parameter gives (superspace conventions, see appendix \chref{conven}):
\begin{align}
d\hat{\chi} &= d({1 \over 2}\chi_{NM}dx^M \wedge dx^N) \nonumber \\
&= {1 \over 2}d(\chi_{\nu \mu}dx^{\mu} \wedge dx^{\nu} + 2\chi_{n \mu}dx^{\mu} \wedge dx^n + \chi_{nm}dx^{m} \wedge dx^n)\nonumber \\
&= d(\chi' - {\chi'}_m \wedge \hat{e}^m + {1 \over 2}{\chi'}^2 {\epsilon}_{mn} \hat{e}^m \wedge \hat{e}^n)\nonumber \\
&= \underbrace{d\chi'-{\chi'}_m \wedge F^{1m}}_{\delta C} + \underbrace{(d{\chi'}_m + {\epsilon}_{mn}{\chi'}^2 \wedge F^{1n})}_{\delta B_{m}'} \wedge \hat{e}^m \nonumber \\
&\phantom{=}+{1 \over 2} {\epsilon}_{mn} \underbrace{d{\chi'}^2}_{\delta A^2} \wedge \hat{e}^m \wedge \hat{e}^n,
\label{tja2}
\end{align}
where we also have indicated which field transformations the terms are connected to. The $\lambda^m$ invariant field strengths can now be calculated as
\begin{align}
\hat{G}=d\hat{C} &= \underbrace{dC + B_{m} \wedge F^{1m}}_{G} + \underbrace{(-dB_{m} + A^2 {\epsilon}_{mn} \wedge F^{1n})}_{-H_m} \wedge \hat{e}^m \nonumber \\
&\phantom{=}+ {1 \over 2}{\epsilon}_{mn} \underbrace{dA^2}_{F^2} \wedge \hat{e}^m \wedge \hat{e}^n,
\label{kalle1}
\end{align}
hence
\begin{align}
G&=dC + B_{m} \wedge F^{1m}, \nonumber \\
H_m&= dB_{m} - {\epsilon}_{mn} F^{1n} \wedge A^2, \nonumber \\
F^2&=dA^2.
\end{align}
The 9-dimensional field strengths thus satisfies the following Bianchi identities:
\begin{align}
dG&=d(dC + B_{m} \wedge F^{1m})=H_m \wedge F^{1m}, \nonumber \\
dH_m&=d(dB_m - {\epsilon}_{mn} F^{1n} \wedge A^2) = - {\epsilon}_{mn} F^{1n} \wedge F^{2},\nonumber \\
dF^2&=d^2 A^2 = 0.
\end{align}

\subsubsection{The $T^3$ case}
The calculation of the 8-dimensional field strengths is very similar to the calculation of the 9-dimensional ones.
By repeating the procedure above we get
\begin{align}
\hat{C} &= {1 \over {3!}} C_{PNM}dx^M \wedge dx^N \wedge dx^P \nonumber \\
&= {1 \over {3!}}\Big{(}C_{\rho \nu \mu}dx^{\mu} \wedge dx^{\nu} \wedge dx^{\rho} + 3C_{p \nu \mu}dx^{\mu} \wedge dx^{\nu} \wedge dx^{p} \nonumber \\
&\phantom{=} + 3 C_{p n \mu}dx^{\mu} \wedge dx^{n} \wedge dx^{p} +C_{pnm}dx^{m} \wedge dx^{n} \wedge dx^{p}\Big{)}\nonumber \\
&= C' + B_{m}' \wedge \hat{e}^m + {1 \over 2}A^{2p} \wedge {\epsilon}_{mnp}\hat{e}^m \wedge \hat{e}^n +{a \over 6}{\epsilon}_{mnp}\hat{e}^m \wedge \hat{e}^n \wedge \hat{e}^p,
\label{olle}
\end{align}
where $a$ is a scalar. The reduction of the right side of (\ref{tja}) becomes
\begin{align}
d\hat{\chi}=&\underbrace{d\chi'-{\chi'}_m \wedge F^{1m}}_{\delta C'}+ \underbrace{(d{\chi'}_m + {\epsilon}_{mnp}{\chi'}^{2p} \wedge F^{1n})}_{\delta B_{m}'} \wedge \hat{e}^m \nonumber \\
&+{1 \over 2} \underbrace{d{\chi'}^{2p}}_{\delta A^{2p}} \wedge {\epsilon}_{mnp} \hat{e}^m \wedge \hat{e}^n.
\end{align}
From $\hat C$ we now get
\begin{align}
\hat{G}=d\hat{C}&= d(C' + B_{m}' \wedge \hat{e}^m + {1 \over 2}A^{2p} \wedge {\epsilon}_{mnp}\hat{e}^m \wedge \hat{e}^n + {a \over 6}{\epsilon}_{mnp}\hat{e}^m \wedge \hat{e}^n \wedge \hat{e}^p) \nonumber \\
&= \underbrace{dC' + B_{m}' \wedge F^{1m}}_{G} + \underbrace{(-dB_{m}' + A^{2p} {\epsilon}_{mnp} \wedge F^{1n})}_{-H_m} \wedge \hat{e}^m \nonumber \\
&\phantom{=}+ {1 \over 2} \underbrace{(dA^{2p} + aF^{1p})}_{F'^{2p}} {\epsilon}_{mnp} \wedge \hat{e}^m \wedge \hat{e}^n - {1 \over 6}(da){\epsilon}_{mnp}\hat{e}^m \wedge \hat{e}^n \wedge \hat{e}^p.
\label{kalle2}
\end{align}
So, to summarize we have the field strengths
\begin{align}
G &= dC' + B_{m}' \wedge F^{1m}, \nonumber \\
H_m &= dB_{m}' -  {\epsilon}_{mnp} F^{1n}\wedge A^{2p}, \nonumber \\
F'^{2m} &= dA^{2m} + aF^{1m} = F^{2m} + aF^{1m},
\label{hej}
\end{align}
satisfying the Bianchi identities
\begin{align}
dG &= d(dC' + B_{m}' \wedge F^{1m}) = dB'_m \wedge F^{1m} \nonumber \\
&= (H_m + {\epsilon}_{mnp} F^{1n} \wedge A^{2p}) \wedge F^{1m} = \Big{\{} {\epsilon}_{mnp} F^{1n} \wedge A^{2p} \wedge F^{1m} \nonumber \\
&\phantom{=}= -{\epsilon}_{mnp} F^{1n} \wedge A^{2p} \wedge F^{1m} =0 \Big{\}}= H_m \wedge F^{1m}, \nonumber \\
dH_m &= -{\epsilon}_{mnp} F^{1n} \wedge F^{2p}, \nonumber \\
dF'^{2m} &= da \wedge F^{1m}.
\label{dhf}
\end{align}

\subsubsection{The $T^4$ case}
To get the 7-dimensional field strengths we start by reducing $\hat{C}$ in exactly the same way as above, giving
\begin{align}
\hat{C} &= {1 \over {3!}} C_{PNM}dx^M \wedge dx^N \wedge dx^P \nonumber \\
&= C' + B_{m}' \wedge \hat{e}^m + {1 \over 2}A^{2pq} \wedge {\epsilon}_{mnpq}\hat{e}^m \wedge \hat{e}^n \nonumber \\
&\phantom{=} +{a^q \over 6}{\epsilon}_{mnpq}\hat{e}^m \wedge \hat{e}^n \wedge \hat{e}^p,
\end{align}
where $A^{2pq}$ is antisymmetric in $p$ and $q$. The gauge parameter becomes
\begin{align}
d\hat{\chi}=&\underbrace{d\chi'-{\chi'}_m \wedge F^{1m}}_{\delta C'}+ \underbrace{(d{\chi'}_m + {\epsilon}_{mnpq}{\chi'}^{2pq} \wedge F^{1n})}_{\delta B_{m}'} \wedge \hat{e}^m \nonumber \\
&+{1 \over 2} \underbrace{d{\chi'}^{2pq}}_{\delta A^{2pq}} {\epsilon}_{mnpq} \wedge \hat{e}^m \wedge \hat{e}^n,
\end{align}
and finally we get $\hat{G}$ as
\begin{align}
\hat{G}=d\hat{C}=& \underbrace{dC' + B_{m}' \wedge F^{1m}}_{G} + \underbrace{(-dB_{m}' + A^{2pq} {\epsilon}_{mnpq} \wedge F^{1n})}_{-H_m} \wedge \hat{e}^m \nonumber \\
&+ {1 \over 2} \underbrace{(dA^{2pq} + a^{[q} F^{1p]})}_{F'^{2pq}} {\epsilon}_{mnpq} \wedge \hat{e}^m \wedge \hat{e}^n \nonumber \\
&- {1 \over 6}(da^q){\epsilon}_{mnpq}\hat{e}^m \wedge \hat{e}^n \wedge \hat{e}^p.
\label{fieldsT4}
\end{align}
We end up with field strengths very similar to the 8-dimensional ones
\begin{align}
G &= dC' + B_{m}' \wedge F^{1m}, \nonumber \\
H_m &= dB_{m}' -  {\epsilon}_{mnpq} A^{2pq}\wedge F^{1n} , \nonumber \\
F'^{2mn} &= dA^{2mn} + a^{[n} F^{1m]} = F^{2mn} - a^{[m}F^{1n]},
\eqnlab{reduct_7d_fprim}
\end{align}
satisfying the Bianchi identities
\begin{align}
dG &= H_m \wedge F^{1m}, \nonumber \\
dH_m &= -{\epsilon}_{mnpq} F^{2pq} \wedge F^{1n}, \nonumber \\
dF'^{2mn} &= - da^{[m} \wedge F^{1n]}.
\eqnlab{reduct_7d_bianchi}
\end{align}

\section{The Chern-Simons term}
Having already reduced the 11-dimensional 3-form $\hat{C}$ and its 4-form field strength $\hat{G}$, it is not a difficult task to reduce the Chern-Simons term.
Again we will only keep terms with 2 or less $e^m$'s in the $T^2$ case, 3 or less in the $T^3$ case and so on.

\subsubsection{The 9-dimensional Chern-Simons term}
\begin{align}
&\phantom{=}{1 \over 6} \int \hat{G} \wedge \hat{G} \wedge \hat{C} \nonumber \\
&={1 \over 6} \int \Big{[} G \wedge G \wedge \epsilon_{mn} {1 \over 2}A^2 \wedge \hat{e}^m \wedge \hat{e}^n - 2 G \wedge H_m \wedge \hat{e}^m \wedge B_n \wedge \hat{e}^n \nonumber \\
&\phantom{=} + 2 G \wedge {1 \over 2} \epsilon_{mn} F^2 \wedge \hat{e}^m \wedge \hat{e}^n \wedge C + H_m \wedge \hat{e}^m \wedge H_n \wedge \hat{e}^n \wedge C \Big{]} \nonumber \\
&= \Big{\{} \hat{e}^m \wedge \hat{e}^n = d^2 x \sqrt{|G|}{\varepsilon}^{mn} = d^2 x \epsilon^{mn},\hspace{0.2cm} {\epsilon}_{mn} {\epsilon}^{mn} = 2 \Big{\}} \nonumber \\
&= {1 \over 6} \int \Big{[} G \wedge G \wedge A^2 - 2\epsilon^{mn} G \wedge H_m \wedge B_n +2 G \wedge F^2 \wedge C \nonumber \\
&\phantom{=} - \epsilon^{mn} H_m \wedge H_n \wedge C \Big{]}
\end{align}

\subsubsection{The 8-dimensional Chern-Simons term}
\begin{align}
&\phantom{=}{1 \over 6} \int \hat{G} \wedge \hat{G} \wedge \hat{C} \nonumber \\
&={1 \over 6} \int \Big{[} {\epsilon}_{mnp} G \wedge G {a \over 6} +{\epsilon}_{mnq} G \wedge H_p \wedge A^{2q} \nonumber \\
&\phantom{=}+ {\epsilon}_{mnq} G \wedge F'^{2q} \wedge B'_p + {1 \over 3} {\epsilon}_{mnp} G \wedge (da) \wedge C' \nonumber \\
&\phantom{=}- H_m \wedge H_n \wedge B'_p +  {\epsilon}_{mnq} H_p \wedge F'^{2q} \wedge C' \Big{]} \wedge \hat{e}^m \wedge \hat{e}^n \wedge \hat{e}^p \nonumber \\
&=\Big{\{} \hat{e}^m \wedge \hat{e}^n \wedge \hat{e}^p = d^3 x {\epsilon}^{mnp},\hspace{0.2cm} {\epsilon}_{mnp} {\epsilon}^{mnq} = 2\delta^{p}_q ,\hspace{0.2cm} {\epsilon}_{mnp}{\epsilon}^{mnp} = 6 \Big{\}} \nonumber \\
&={1 \over 6} \int \Big{[} G \wedge G a + 2G \wedge H_m \wedge A^{2m} + 2G \wedge F'^{2m} \wedge B'_m \nonumber \\
&\phantom{=}+ 2G \wedge (da) \wedge C' - {\epsilon}^{mnp}H_m \wedge H_n \wedge B'_p \nonumber \\
&\phantom{=}+2H_m \wedge F'^{2m} \wedge C' \Big{]},
\label{kjd}
\end{align}

\subsubsection{The 7-dimensional Chern-Simons term}
\begin{align}
&\phantom{=}{1 \over 6} \int \hat{G} \wedge \hat{G} \wedge \hat{C} \nonumber \\
&={1 \over 6} \int \Big{[} 2 \epsilon_{mnpq'} G \wedge H_q {a^{q'} \over 6} +2 \epsilon_{mnp'q'} \epsilon_{m'n'pq} G \wedge {1 \over 2} F'^{2p'q'} \wedge {1 \over 2} A^{2m'n'} \nonumber \\
&\phantom{=}-2 \epsilon_{mnpq'} G \wedge {1 \over 6}(da^{q'}) \wedge B'_q +\epsilon_{mnp'q'} H_p \wedge H_q \wedge {1 \over 2} A^{2p'q'} \nonumber \\
&\phantom{=}+2 \epsilon_{mnp'q'} H_q \wedge {1 \over 2}F'^{2p'q'} \wedge B'_p +2 \epsilon_{mnpq'} H_q \wedge {1 \over 6}(da^{q'}) \wedge C' \nonumber \\
&\phantom{=} +\epsilon_{mnp'q'} \epsilon_{m'n'pq} {1 \over 2}F'^{2p'q'} \wedge {1 \over 2}F'^{2m'n'} \wedge C' \Big{]} \wedge \hat{e}^m \wedge \hat{e}^n \wedge \hat{e}^p \wedge \hat{e}^q \nonumber \\
&=\Big{\{} \hat{e}^m \wedge \hat{e}^n \wedge \hat{e}^p \wedge \hat{e}^q= d^4 x {\epsilon}^{mnpq},\hspace{0.2cm} {\epsilon}_{mnp'q'} {\epsilon}^{mnpq} = 4\delta^{ \phantom{[} p \phantom{'} q}_{[p'q']} ,\nonumber \\
&\phantom{=}{\epsilon}_{mnpq'}{\epsilon}^{mnpq} = 6 \delta^q_{q'} \Big{\}} \nonumber \\
&={1 \over 6} \int \Big{[} 2G \wedge H_m a^{m} +2\epsilon_{mnpq}G \wedge F'^{2mn} \wedge A^{2pq} \nonumber \\
&\phantom{=}-2G \wedge (da^{m}) \wedge B'_m +2H_m \wedge H_n \wedge A^{2mn} \nonumber \\
&\phantom{=}+4 B'_m \wedge H_n \wedge F'^{2mn} +2 H_m \wedge (da^{m}) \wedge C' \nonumber \\
&\phantom{=}+ \epsilon_{mnpq} F'^{2mn} \wedge F'^{2pq} \wedge C' \Big{]}
\end{align}

\section{The $\hat{G}^2$ term and the symmetries of the actions}
The $\hat{G}^2$ term in \eqnref{sugra_11dim} will be reduced in the same way as the field strength in the previous section, i.e. by expanding the 4-form $\hat{G}$ 
in lower dimensional field strengths. We will also clarify how the remaining symmetries in 9-, 8-dimensional 
supergravity according to table 2.3 (i.e ${\mathbb R}^+$ for $T^2$ and $SL(2,{\mathbb R})$ for $T^3$) can be seen in the actions. 
(The ${\mathbb R}^+$ symmetry in 9-dimensional supergravity origins from $GL(2) \sim SL(2,{\mathbb R}) \times {\mathbb R}^+$.) Last but not least 
we will write the 7-dimensional Lagrangian in an $SL(5,{\mathbb R})$ covariant way.

\subsection{The 9-dimensional action}
We start by writing
\begin{equation}
\hat{G}={1 \over 4!}\hat{G}_{M_{1} \cdots M_{4}}dx^{M_{4}} \wedge \cdots \wedge dx^{M_{1}},
\label{kalle}
\end{equation}
and by comparing (\ref{kalle}) with (\ref{kalle1}) we find
\begin{align}
\hat{G}_{\mu \nu \rho \sigma}&=G_{\mu \nu \rho \sigma} \hspace{1.1cm} \mbox{1 combination}, \nonumber \\
\hat{G}_{\mu \nu \rho m}&=-H_m \hspace{1.15cm} \mbox{4 combinations}, \nonumber \\
\hat{G}_{\mu \nu mn}&=F^2 {\epsilon}_{mn} \hspace{1cm} \mbox{6 combinations},
\end{align}
hence
\begin{align}
{1 \over 48}\hat{G}_{M_{1} \cdots M_{4}} \hat{G}^{M_{1} \cdots M_{4}}&={1 \over 48}G^2 + {1 \over 12}H_m H^m + {1 \over 8}F^2 F^2 {\epsilon}_{mn} {\epsilon}^{mn} \nonumber \\
&= {1 \over 48}G^2 + {1 \over 12}H_m H^m + {1 \over 4}F^2 F^2.
\end{align}
When going to the Einstein frame we need to scale the fields as
\begin{align}
\sqrt{|g|} e^{-\varphi} &\rightarrow \sqrt{|g_E|}e^{-(ds+1)\varphi}, \nonumber \\
G^2 &\rightarrow e^{8s \varphi} G^2, \nonumber \\
H_m H^m &\rightarrow e^{6s\varphi}e^{2\varphi/D}\M^{mn}H_m H_n, \nonumber \\
F^2F^2 &\rightarrow e^{4s \varphi}F^2F^2, \nonumber \\
{\epsilon}_{mn} {\epsilon}^{mn} &\rightarrow e^{4\varphi/D} M^{pm} M^{qn} {\epsilon}_{mn} {\epsilon}_{pq}.
\label{ertyui}
\end{align}
The $G^2$ term in the Einstein frame then becomes
\begin{align}
S_{\hat G^2} &= - \int d^{11}x \sqrt{|\hat{g}|} \Big{[} \frac{1}{48} \hat G^2 \Big{]} \nonumber \\
& = - \int d^9x \sqrt{|g_E|}e^{-(ds+1)\varphi}\Big{[}\frac{1}{48}e^{8s\varphi}G^2+\frac{1}{12}e^{(6s+2/D)\varphi}H_mH_n\M^{mn} \nonumber \\
& \phantom{=}+ \frac{1}{4}e^{(4s+4/D)\varphi}(F^2)^2\Big{]}.
\end{align}
For the $T^2$ case we have $D=2$, $d=9$, $s=-1/7$ and we get the $G^2$ term as  
\begin{equation}
S_{\hat G^2} = - \int d^9x \sqrt{|g_E|}\left[\frac{1}{48}e^{-\frac{6}{7}\varphi}G^2+\frac{1}{12}e^{\frac{3}{7}\varphi}H_mH_n\M^{mn}+\frac{1}{4}e^{\frac{12}{7}\varphi}(F^2)^2\right].
\end{equation}
By writing the $G^2$ term together with the Einstein term and making a redefinition $\varphi \rightarrow (\sqrt{7}/3)\varphi$ (to get a factor 1/2 in front of the kinetic $(\partial\varphi)^2$ term) we get
\begin{align}
&\phantom{=}\int d^{11}x \sqrt{|\hat{g}|} (\hat{R}-{1 \over 48}\hat{G}^2) \nonumber \\
&=\int d^9x \sqrt{|g_E|} \Big{[} R_E - {1 \over 2}(\partial \varphi)^2-{1 \over 4}e^{-\frac{3}{\sqrt 7}\varphi}\M_{mn}F^{1m}F^{1n} \nonumber \\
&\phantom{=}-{1 \over 4}\mbox{Tr}(\M^{-1}\partial \M)^2 - {1 \over 48}e^{-\frac{2}{\sqrt 7}\varphi}G^2 - {1 \over 12}e^{\frac{1}{\sqrt 7}\varphi}H_mH_n\M^{mn} \nonumber \\
&\phantom{=}- {1 \over 4}e^{\frac{4}{\sqrt 7}\varphi}\M_{mn}(F^2)^2\Big{]}.
\label{tjolahopp}
\end{align}
A symmetric metric of unit determinant $\M$ parametrises the coset space SL(2,$\rr$)/SO(2), as was seen in subsection \ssecref{example_sl2so2}.
By performing an SL(2,$\rr$) transformation
\begin{align}
&\M_{mn}\rightarrow\Lambda{_m}^p \M_{pq} \Lambda{^q}_n,\;\; F^{1m}\rightarrow F^{1n}(\Lambda^{-1}){_n}^m,\;\; \nonumber \\
&H_m\rightarrow \Lambda{_m}^nH_m,\;\; \Lambda \in SL(2,\rr)
\end{align}  
to the terms in the Lagrangian, the SL(2,$\rr$) scalars
\begin{align}
&F^{1T}\M F^{1}\rightarrow F^{1T}\Lambda^{-1}\Lambda \M
\Lambda^T\Lambda^{T-1}F^1 = F^{1T}\M F^{1}\nonumber\\
&H^{T}\M^{-1}H\rightarrow H^{T}\Lambda^T(\Lambda \M \Lambda^T)^{-1}\Lambda H = H^{T}\M^{-1}H 
\end{align}
are invariant and also the scalar kinetic term
\begin{align}
&\frac{1}{4}\mbox{Tr}(\M^{-1}\partial \M)^2 = -\frac{1}{4}\mbox{Tr}(\partial \M^{-1}\partial \M) \nonumber \\
&\rightarrow %\frac{1}{4}\mbox{Tr}(\partial((\Lambda \M\Lambda^T)^{-1})\partial(\Lambda \M\Lambda^T)) = 
-\frac{1}{4}\mbox{Tr}(\Lambda^{T-1}\partial \M^{-1} \Lambda^{-1}\Lambda \partial \M \Lambda^T) = \frac{1}{4}\mbox{Tr}(\M^{-1}\partial \M)^2,
\eqnlab{sugra_kinetic_transform}
\end{align}
where we have used \eqnref{conven_kinetic} and that $\Lambda$ is constant so $\partial\Lambda = 0$, is invariant. 
The Lagrangian is also invariant under an $R^+$ scaling symmetry.
To see this we assign, to each field in the Lagrangian, a scale transformation $\lambda^\alpha$, where $\lambda$ is an arbitrary real constant and the exponent $\alpha$ is different for each different field.
By choosing the $\alpha$:s according to table (\ref{reduct_weights}) and by noting that $e^{x\varphi}$ scales as $2x\sqrt 7$, one can easily check that the Lagrangian (\ref{tjolahopp}) is invariant under the ${\rr}^+$ symmetry.
Consequently we have found that the Lagrangian is invariant under $\rr^+\times$SL(2,$\rr$)/SO(2)$\sim$GL(2,$\rr$)/SO(2) transformations, which is consistent with the expected data for d=9 in table 2.3.

\begin{table}
\begin{center}
\begin{tabular}{l|l l l l l l l}
Field & $g_{\mu\nu}$ & $A_\mu^{1m}$ & $A_\mu^2$ & $B_{\mu\nu m}$ & $C_{\mu\nu\rho}$ & $\phi^m$ & $e^\varphi$\\ \hline
$\rr^+$ scaling $\alpha$ & 0 & $3$ & $-4$ & $-1$ & $2$ & 0 & $2\sqrt 7$
\end{tabular}
\label{reduct_weights}
\caption{The scaling exponents of the ${\rr}^+$ symmetry for the different fields. $\phi^m$ denotes the 2 coset scalars in $\M$.} 
\end{center}
\end{table}

\subsection{The 8-dimensional action}

Again we start with
\begin{equation}
\hat{G}={1 \over 4!}\hat{G}_{M_{1} \cdots M_{4}}dx^{M_{4}} \wedge \cdots \wedge dx^{M_{1}},
\end{equation}
to identify the terms in (\ref{kalle2}) with
\begin{align}
\hat{G}_{\mu \nu \rho \sigma}&=G_{\mu \nu \rho \sigma} \hspace{2.03cm} \mbox{1 combination}, \nonumber \\
\hat{G}_{\mu \nu \rho m}&=-H_m \hspace{2.1cm} \mbox{4 combinations}, \nonumber \\
\hat{G}_{\mu \nu mn}&=F'^{2p} {\epsilon}_{mnp} \hspace{1.55cm} \mbox{6 combinations}, \nonumber \\
\hat{G}_{\mu mnp}&=-({\partial}_{\mu}a){\epsilon}_{mnp} \hspace{1cm} \mbox{4 combinations},
\end{align}
which gives
\begin{align}
{1 \over 48}\hat{G}_{M_{1} \cdots M_{4}} \hat{G}^{M_{1} \cdots M_{4}}&={1 \over 48}G^2 + {1 \over 12}H_m H^m + {1 \over 8}F'^{2p} F'{^2}_q {\epsilon}_{mnp} {\epsilon}^{mnq} \nonumber \\
&\phantom{=}+ {1 \over 12}({\partial}_{\mu}a)({\partial}^{\mu}a){\epsilon}_{mnp}{\epsilon}^{mnp} \nonumber \\
&= \Big{\{} {\epsilon}_{mnp} {\epsilon}^{mnq} = 2\delta^{p}_q ,\hspace{0.2cm} {\epsilon}_{mnp}{\epsilon}^{mnp} = 6 \Big{\}} \nonumber \\
&= {1 \over 48}G^2 + {1 \over 12}H_m H^m + {1 \over 4}F'^{2p} F'{^2}_p + {1 \over 2}({\partial}_{\mu}a)({\partial}^{\mu}a).
\end{align}
To get the action in the Einstein frame we can use (\ref{ertyui}) together with
\begin{align}
F'^2_m F'^{2m} &\rightarrow e^{4s \varphi}e^{-2\varphi /D}\M_{mn}F'^{2m}F'^{2n}, \nonumber \\
(\partial a)^2 &\rightarrow e^{2s\varphi} (\partial a)^2, \nonumber \\
{\epsilon}_{mnp}{\epsilon}^{mnp} &\rightarrow e^{6\varphi /D}\M^{mq}\M^{nr}\M^{ps} {\epsilon}_{mnp}{\epsilon}_{qrs},
\end{align}
where we have used (\ref{varchange2}). We now have $D=3$, $d=8$ and $s=-1/6$, and we get the action in the Einstein frame as
\begin{align}
S_{\hat G^2} =& \int d^8x \sqrt{|g|} \Big{[} {1 \over 48}e^{-\varphi}G^2 + {1 \over 12}H_mH_n\M^{mn} \nonumber \\
&+ {1 \over 4}e^{\varphi}\M_{mn}F'^{2m}F'^{2n} + {1 \over 2}e^{2\varphi}(\partial a)^2 \Big{]}.
\end{align}
Writing the Einstein term together with the $G^2$ term gives
\begin{align}
&\phantom{=}\int d^{11}x \sqrt{|\hat{g}|} (\hat{R}-{1 \over 48}\hat{G}^2) \nonumber \\
&= \int d^8x \sqrt{|g_E|} \Big{[} R_E - {1 \over 2}(\partial \varphi)^2-{1 \over 4}e^{-\varphi}\M_{mn}F^{1m}F^{1n} -{1 \over 4}\mbox{Tr}(\M^{-1}\partial \M)^2 \nonumber \\
&\phantom{=}- {1 \over 48}e^{-\varphi}G^2 - {1 \over 12}H_mH_n\M^{mn} - {1 \over 4}e^{\varphi}\M_{mn}F'^{2m}F'^{2n} - {1 \over 2}e^{2\varphi}(\partial a)^2 \Big{]}.
\label{hej2}
\end{align}

\subsubsection{SL(2) $\times$ SL(3) parametrisation}

We now want to formulate the action in a more $SL(2,{\mathbb R}) \times
SL(3,{\mathbb R})$ covariant way (see table 2.3). 
We define the metric
\begin{equation}
\W= {1 \over \mbox{Im} (\tau) }\toto{|\tau|^2 }{ \mbox{Re}(\tau) }{ \mbox{Re}(\tau) }{ 1} = e^{\varphi}\toto{a^2 + e^{-2\varphi} }{ a }{ a }{ 1},
\end{equation}
with
\begin{equation}
\tau = a + \mbox{i} e^{-\varphi}
\end{equation}
parametrising the $SL(2,{\mathbb R})/SO(2)$ coset. We get
\begin{equation}
\W^{-1}={1 \over \mbox{Im} (\tau) }\toto{ 1}{ -\mbox{Re}(\tau) }{ -\mbox{Re}(\tau) }{ |\tau|^2}=e^{\varphi}\toto{ 1}{ -a }{ -a }{ a^2 + e^{-2\varphi}}
\end{equation}
and
\begin{equation}
\partial \W=(\partial \varphi) \W + e^{\varphi}\toto{ 2a\partial a - 2\partial \varphi e^{-2\varphi}}{ \partial a }{ \partial a }{ 0},
\end{equation}
hence
\begin{align}
\partial \W \W^{-1}=& \hspace{0.15cm} \id \partial \varphi+ \nonumber \\
&  + e^{2\varphi}\toto{a\partial a - 2\partial \varphi e^{-2\varphi} }{ -a^2 \partial a + 2a\partial \varphi e^{-2\varphi} + \partial a e^{-2\varphi} }{ \partial a }{ -a \partial a} \nonumber \\
=& \hspace{0.15cm} e^{2\varphi}\toto{a\partial a - \partial \varphi e^{-2\varphi} }{ -a^2 \partial a + 2a\partial \varphi e^{-2\varphi} + \partial a e^{-2\varphi} }{ \partial a }{ -a \partial a + \partial \varphi e^{-2\varphi}}.
\end{align}
Finally we get
\begin{align}
\mbox{Tr}(\partial \W \W^{-1})^2 &= e^{4 \varphi}\Big{(}2(a\partial a - \partial \varphi e^{-2\varphi})^2 + 2(-a^2(\partial a)^2 + 2a\partial a\partial \varphi e^{-2 \varphi} \nonumber \\
&\phantom{=}+ (\partial a)^2 e^{-2 \varphi}) \Big{)}= 2\Big{(}(\partial \varphi)^2 + (\partial a)^2 e^{2 \varphi}\Big{)}.
\label{hej3}
\end{align}
Now look at the term
\begin{align}
F^{rm}F^{sn}\W_{rs} &= (a^2 e^{\varphi} + e^{-\varphi})F^{1m}F^{1n} + 2ae^{\varphi}F^{1m}F^{2n} +e^{\varphi}F^{2m}F^{2n} \nonumber \\
&= e^{\varphi}(aF^{1m}F'^{2n} + F'^{2m}F^{2n}) + e^{-\varphi}F^{1m}F^{1n} \nonumber \\
&= e^{\varphi}F'^{2m}F'^{2n} + e^{-\varphi} F^{1m}F^{1n},
\label{hej4}
\end{align}
where we have repeatedly used (\ref{hej}). Inserting (\ref{hej3}) and (\ref{hej4}) in (\ref{hej2}) yields
\begin{align}
&\int d^{11}x \sqrt{|\hat{g}|} (\hat{R}-{1 \over 48}\hat{G}^2) \nonumber \\
&= \int d^8x \sqrt{|g_E|} \Big{[} R_E - {1 \over 4}\mbox{Tr}(\partial \W \W^{-1})^2 -{1 \over 4}\mbox{Tr}(\M^{-1}\partial \M)^2 \nonumber \\
&\phantom{=}- {1 \over 4}\M_{mn}F^{rm}F^{sn}\W_{rs} - {1 \over 12}H_mH_n\M^{mn}- {1 \over 48}e^{-\varphi}G^2 \Big{]}.
\label{tjena}
\end{align}
We can now explicitly verify that the action (apart from the $G^2$ term that will be treated later) is invariant under the
following $SL(2,{\mathbb R})$ and $SL(3,{\mathbb R})$ transformations:
\begin{equation}
\W \rightarrow \Lambda \W \Lambda^{T}, \hspace{0.5cm} F^{m} \rightarrow (\Lambda^{T})^{-1}F^m, \hspace{0.5cm} \Lambda \in SL(2),
\end{equation}
\begin{equation}
\M \rightarrow R\M R^T, \hspace{0.5cm} H_m \rightarrow R{_m}^n H_m, \hspace{0.5cm} F^m \rightarrow F^n (R^{-1}){_n}^m, \hspace{0.5cm} R \in SL(3).
\label{sl3trans}
\end{equation}
To see that term 4 and 5 in (\ref{tjena}) are inert we only need to do the short calculations
\begin{equation}
H^T \M^{-1}H \rightarrow H^T R^T R^{-T}\M^{-1}R^{-1}RH = H^T \M^{-1}H,
\end{equation}
\begin{align}
F^T(\M \otimes \W)F \rightarrow & \hspace{0.15cm} F^T R^{-1} R\M R^T R^{-T}F \otimes F^T \Lambda^{-1} \Lambda \W {\Lambda}^{T} {\Lambda}^{-T}F \nonumber \\
&=F^T(\M \otimes \W)F,
\end{align}
whilst for term 2 and 3 we can reuse the shown transformation symmetry of the scalar kinetics in \eqnref{sugra_kinetic_transform}.

\subsubsection{The $G$-doublet}

We see in \ref{tjena} that there is no $SL(2,{\mathbb R})$ symmetry for the $G^2$-term. This is because the $SL(2,{\mathbb R})$ 
symmetry is not a symmetry of the whole action, but we will show below that by using a 4-form coming from the dual of $\hat{G}$ 
we can create an $SL(2,{\mathbb R})$-doublet and derive $SL(2,{\mathbb R})$ covariant Bianchi identities for $H$ and $G$.
But before we do this we will first redefine the potentials to get the field strengths written in an $SL(2,{\mathbb R})$ 
invariant way. The way that we have defined $H_m$ in (\ref{dhf}) is clearly not $SL(2,{\mathbb R})$ covariant, 
a fact which is easily remedied by the redefinition
\begin{equation}
B'_m = B_m - {1 \over 2}{\epsilon}_{mnp}A^{1n} \wedge A^{2p},
\end{equation}
which implies that
\begin{align}
H_m=dB'_m &= dB_m - {1 \over 2}{\epsilon}_{mnp}(F^{1n} \wedge A^{2p} - F^{2n} \wedge A^{1p}) \nonumber \\
&=dB_m - {1 \over 2}{\epsilon}_{mnp}{\epsilon}_{rs}F^{rn} \wedge A^{sp},
\end{align}
where $r,s = 1,2$. This gives the $SL(2,{\mathbb R})$ covariant Bianchi identity
\begin{equation}
dH_m=-{1 \over 2}{\epsilon}_{mnp}{\epsilon}_{rs}F^{rn} \wedge F^{sp}.
\end{equation}
It is also convenient to redefine $C'$ as
\begin{equation}
C'=C - {1 \over 3}A^{1m} \wedge B_m + {1 \over 6}{\epsilon}_{mnp}A^{1m} \wedge A^{2n} \wedge A^{1p},
\end{equation}
which gives
\begin{align}
G &= dC -{1 \over 3}dA^{1m} \wedge B_m - {1 \over 3}A^{1m} \wedge dB_m +{1 \over 6}{\epsilon}_{mnp}dA^{1m} \wedge A^{2n} \wedge A^{1p} \nonumber \\
&\phantom{=}-{1 \over 6}{\epsilon}_{mnp}A^{1m} \wedge dA^{2n} \wedge A^{1p} + {1 \over 6}{\epsilon}_{mnp}A^{1m} \wedge A^{2n} \wedge dA^{1p} \nonumber \\
&\phantom{=}+(B_m - {1 \over 2}{\epsilon}_{mnp}A^{1n} \wedge A^{2p})\wedge F^{1m} \nonumber \\
&=dC + {2 \over 3}B_m \wedge F^{1m} - {1 \over 3}A^{1m} \wedge dB_m \nonumber \\
&\phantom{=}+({1 \over 6}+{1 \over 6}-{1 \over 2}){\epsilon}_{mnp}A^{1m} \wedge A^{2n} \wedge F^{1p} + {1 \over 6}{\epsilon}_{mnp}A^{1m} \wedge A^{1n} \wedge F^{2p} \nonumber \\
&=dC + {2 \over 3}B_m \wedge F^{1m} \nonumber \\
& \phantom{=}+ {1 \over 3}A^{1m} \wedge \Big{(}dB_m + {1 \over 2}{\epsilon}_{mnp}(A^{1n} \wedge F^{2p} - A^{2n} \wedge F^{1p}) \Big{)} \nonumber \\
&=dC + {2 \over 3}B_m \wedge F^{1m} + {1 \over 3}A^{1m} \wedge H_m
\end{align}
Note that these redefinitions do not affect the corresponding Bianchi identities.
Unfortunately they do affect the invariance under reparametrisations along the internal 3-torus as one can immediately see in the definitions above.
This invariance is lost for the $B$ and $C$ fields, an unavoidable consequence of imposing manifest $SL(2,{\mathbb R})$ covariance.

Next it is finally time to deal with the $G^2$ term. We will try to write $G$ as a $SL(2,{\mathbb R})$ doublet by using the dual to $\hat{G}$.
The dual is calculated by $\hat{G}_7=*\hat{G}$, where $*$ is the Hodge dual operator (see \eqnref{conven_hodge_form}) and hence $\hat{G}_7$ is a 7-form.
When reducing the 11-dimensional 7-form we will receive a 4-form which is dual to the 4-form $G$. Expanding $\hat{G}_7$ gives
\begin{align}
\hat{G}_7 &= {1 \over 7!} \hat{G}_{7M{_7} \cdots M_{1}} dx^{M_{1}} \wedge \cdots \wedge dx^{M_{7}} \nonumber \\
&={1 \over 7!}\Big{(}G_{7 {\mu}_7 \cdots {\mu}_1} dx^{{\mu}_{1}} \wedge \cdots \wedge dx^{{\mu}_{7}} + 7G_{7 m_7 {\mu}_6 \cdots {\mu}_1} dx^{{\mu}_{1}} \wedge \cdots \wedge dx^{{\mu}_{6}} \wedge dx^{m_7} \nonumber \\
&\phantom{=} +{7 \cdot 6 \over 2!}G_{7 m_7 m_6 {\mu}_5 \cdots {\mu}_1}dx^{{\mu}_{1}} \wedge \cdots \wedge dx^{{\mu}_{5}} \wedge dx^{m_6} \wedge dx^{m_7} \nonumber \\
&\phantom{=} +{7 \cdot 6 \cdot 5 \over 3!}G_{7 m_7 m_6 m_5 {\mu}_4 \cdots {\mu}_1}dx^{{\mu}_{1}} \wedge \cdots \wedge dx^{{\mu}_{4}} \wedge dx^{m_5} \wedge dx^{m_6} \wedge dx^{m_7} \Big{)} \nonumber \\
\intertext{}
&= \underbrace{{1 \over 7!}G_{7 {\mu}_7 \cdots {\mu}_1} dx^{{\mu}_{1}} \wedge \cdots \wedge dx^{{\mu}_{7}}}_{G_7} + \underbrace{({1 \over 6!}G_{7{\mu}_6 \cdots {\mu}_1} dx^{{\mu}_{1}} \wedge \cdots \wedge dx^{{\mu}_{6}})_{m}}_{-G_{6m}} \wedge dx^m \nonumber \\
&\phantom{=} +{1 \over 2!} \underbrace{({1 \over 5!}G_{7{\mu}_5 \cdots {\mu}_1} dx^{{\mu}_{1}} \wedge \cdots \wedge dx^{{\mu}_{5}})_{nm}}_{-G_{5nm}} \wedge dx^m \wedge dx^n \nonumber \\
&\phantom{=} +{1 \over 3!} \underbrace{({1 \over 4!}G_{7{\mu}_4 \cdots {\mu}_1}dx^{{\mu}_{1}} \wedge \cdots \wedge dx^{{\mu}_{4}})_{pnm}}_{G'{\epsilon}_{pnm}} \wedge dx^m \wedge dx^n  \wedge dx^p \nonumber \\
&= G_7 - G_{6m} \wedge \hat{e}^m + {1 \over 2!} G_{5mn} \wedge \hat{e}^m \wedge \hat{e}^n - {1 \over 3!}G' \wedge {\epsilon}_{mnp} \hat{e}^m \wedge \hat{e}^n \wedge \hat{e}^p,
\label{g7}
\end{align}
where the 4-form $G'$ is dual to $G$ and the signs has been chosen so that the corresponding expansion of $\hat C_{(6)}$ has positive signs. To form the $SL(2,{\mathbb R})$ doublet we need the Bianchi identity for $\hat{G}_7$
\begin{equation}
d(*\hat{G})=d{\hat{G}_7}=-{1 \over 2}\hat{G} \wedge \hat{G},
\label{bianchig7}
\end{equation}
which was derived in \eqnref{bianchig7}. We can now use (\ref{bianchig7}) together with (\ref{g7}) to get
\begin{align}
d\hat{G}_7 &= \cdots +{1 \over 3!} dG' \wedge {\epsilon}_{mnp}
\hat{e}^m \wedge \hat{e}^n \wedge \hat{e}^p + \cdots = dG' d^Dy + \cdots\nn\\
& = -{1 \over 2}\hat{G} \wedge \hat{G}
= -{1 \over 2}\Big( \cdots -{2 \over 3!} G \wedge da \wedge
{\epsilon}_{mnp}\hat{e}^m \wedge \hat{e}^n \wedge \hat{e}^p \nonumber
\\
&\phantom{=}-{2 \over 2!} H_q \wedge \hat{e}^q \wedge F'^{2p} {\epsilon}_{mnp}\wedge \hat{e}^m \wedge \hat{e}^n + \cdots\Big) \nonumber \\
&= \big( G \wedge da + H_m\wedge F'^{2m}\big) d^Dy + \cdots
\end{align}
We thus read off the Bianchi identity for $G'$ as
\begin{equation}
dG'=G \wedge da + H_m \wedge F'^m = G \wedge da + H_m \wedge F^{2m} + H_m \wedge F^{1m}a.
\end{equation}
The next step is to make the definition
\begin{equation}
G' = -e^{-\varphi}(*_8 G),
\end{equation}
where $*_8$ is the 8-dimensional Hodge dual operator. This way of defining $G'$ is however not 
suitable for the construction of the $SL(2,{\mathbb R})$ doublet. Instead we make another definition
\begin{equation}
\tilde{G}=G'-aG=-e^{-\varphi}(*_8 G)-aG,
\end{equation}
with the Bianchi identity
\begin{equation}
d\tilde{G}=H_m \wedge F^{2m}.
\end{equation}
Hence, with the definitions $G^1 = G$, $G^2 = \tilde{G}$, $r = 1,2$ and $dA^{2m}=F^{2m}$, we get
\begin{equation}
dG^r=H_m \wedge F^{rm},
\end{equation}
and we also define
\begin{equation}
G^r=dC^r+{2 \over 3}B_m \wedge F^{rm} - {1 \over 3}A^{rm} \wedge H_m.
\end{equation}
Finally, by looking at the components of the dual of $G$ we find
\begin{align}
*_8G^1&=e^{-\varphi}G'=e^{-\varphi}(G^2+aG^1)=-\W_{21}G^1-\W_{22}G^2 \nonumber \\
&=-\W_{2t}G^t, \\
*_8G^2&=*_8(G'-aG^1)=e^{-\varphi}G^1+ae^{\varphi}G^2+a^2e^{\varphi}G^1=\W_{11}G^1+\W_{12}G^2 \nonumber \\
&=\W_{1t}G^t, \\
\Rightarrow & *_8G^r=-\epsilon^{rs}\W_{st}G^t.
\end{align}
The redefinitions of the potentials also affects the transformations that leaves the field strengths invariant. 
The new gauge transformations becomes:
\begin{align}
&\delta A^{rm}=d\chi^{rm}, \hspace{1cm} \delta B_m =d\chi_m -{1 \over 2}\epsilon_{mnp}\epsilon_{rs}A^{rn} \wedge d\chi^{sp}, \nonumber \\
&\delta C^r = d\chi^r -{2 \over 3}A^{rm} \wedge d\chi_m +{1 \over 3}B_m \wedge d\chi^{rm} + {1 \over 6}\epsilon_{mnp}\epsilon_{st}A^{rm} \wedge A^{sn} \wedge d\chi^{tp},
\end{align}
and the calculations
\begin{align}
\delta F^{rm}&=d \delta A^{rm} = d^2 \chi^{rm}=0, \\
\delta H_m&=d\delta B_m - {1 \over 2}\epsilon_{mnp}\epsilon_{rs}F^{rn} \wedge \delta A^{sp} \nonumber \\
&={1 \over 2}\epsilon_{mnp}\epsilon_{rs}(F^{rn}\wedge d\chi^{sp}-F^{rn}\wedge d\chi^{sp})=0, \\
\intertext{}
\delta G^r&=d \delta C^r + {2 \over 3} \delta B_m \wedge F^{rm} - {1 \over 3}\delta A^{rm} \wedge H_m =-{2 \over 3}F^{rm} \wedge d\chi_m \nonumber \\
&\phantom{=}-{1 \over 3}dB_m \wedge d\chi^{rm}-{1 \over 6}\epsilon_{mnp}\epsilon_{st}d(A^{rm} \wedge A^{sn}) \wedge d\chi^{tp} \nonumber \\
&\phantom{=}+{2 \over 3} d\chi_m \wedge F^{rm}-{1 \over 6}\epsilon_{mnp}\epsilon_{st}A^{sn} \wedge d\chi^{tp} \wedge F^{rm} +{1 \over 3}H_m \wedge d\chi^{rm} \nonumber \\
&=-{1 \over 6}\epsilon_{mnp}\epsilon_{st}\Big{(}A^{rm} \wedge F^{sn} \wedge d\chi^{tp} - A^{sm} \wedge F^{rn} \wedge d\chi^{tp} \nonumber \\
&\phantom{=}+ F^{sn} \wedge A^{tp} \wedge d\chi^{rm}\Big{)} \nonumber \\
&= \Big{\{} \mbox{Use } \epsilon_{mnp}\epsilon_{st}(A^{rm} \wedge F^{sn} \wedge d\chi^{tp} + F^{sn} \wedge A^{tp} \wedge d\chi^{rm}) \nonumber \\
&\phantom{=}= -\epsilon_{mnp}\epsilon_{st}(A^{sm} \wedge F^{rn} \wedge d\chi^{tp} + 2F^{rm} \wedge A^{sn} \wedge d\chi^{tp}), \nonumber \\
&\phantom{=}\mbox{(calculate each component separate)}\Big{\}} \nonumber \\
&={1 \over 3}\epsilon_{mnp}\epsilon_{st}(A^{sm} \wedge F^{rn} + A^{sn} \wedge F^{rm})\wedge d\chi^{tp}=0,
\end{align}
shows that the field strengths are invariant.






\subsection{The 7-dimensional action}

We will calculate the $\hat{G}^2$-term for $T^4$ in exactly the same way as for $T^2$ and $T^3$, hence we start with (\ref{fieldsT4}) to identify $\hat{G}^2$ with
\begin{align}
&\phantom{=}{1 \over 48}\hat{G}_{M_{1} \cdots M_{4}} \hat{G}^{M_{1} \cdots M_{4}} \nonumber \\
&= {1 \over 48}G^2 + {1 \over 12}H_m H^m + {1 \over 8}F'^{2}_{pq}F'^{2p'q'}{\epsilon}_{mnp'q'}{\epsilon}^{mnpq} \nonumber \\
&\phantom{=}+{1 \over 12}({\partial}^{\mu} a^q)({\partial}_{\mu}a^{q'})G_{q'q''}{\epsilon}_{mnpq}{\epsilon}^{mnpq''} \nonumber \\
&= \Big{\{} {\epsilon}_{mnp'q'}{\epsilon}^{mnpq}=4 \delta^{ \phantom{[} p \phantom{'} q}_{[p'q']}, {\epsilon}_{mnpq}{\epsilon}^{mnpq''}=3! \delta^{q''}_{q} \Big{\}} \nonumber \\
&= {1 \over 48}G^2 + {1 \over 12}H_m H^m + {1 \over 2}F'^{2}_{mn}F'^{2mn} + {1 \over 2}({\partial}^{\mu} a^{q'})({\partial}_{\mu}a^{q})G_{qq'}.\phantom{iiiiii}
\end{align}
Switching to Einstein frame requires the scaling
\begin{align} 
F'^2_{mn} F'^{2mn} &\rightarrow e^{4s \varphi}e^{-4\varphi /D}\M_{mm'}\M_{nn'}F'^{2m'n'}F'^{2mn}, \nonumber \\
({\partial}_{\mu} a^q)({\partial}^{\mu} a^{q'})G_{qq'} &\rightarrow e^{2s\varphi} e^{-2\varphi /D} ({\partial}_{\mu} a^q)({\partial}^{\mu} a^{q'}) \M_{qq'}, \nonumber \\
{\epsilon}_{mnpq}{\epsilon}^{mnpq} &\rightarrow e^{8\varphi /D}\M^{mm'}\M^{nn'}\M^{pp'}\M^{qq'} {\epsilon}_{mnpq}{\epsilon}_{m'n'p'q'},\phantom{iiii}
\end{align}
together with (\ref{ertyui}). Recognizing $D=4$, $d=7$ and $s=-1/5$ we find
\begin{align}
S_{\hat G^2} =& \int d^7x \sqrt{|g|} \Big{[} {1 \over 48}e^{-{6 \over 5}\varphi}G^2 + {1 \over 12}e^{-{3 \over 10}\varphi}H_mH_n\M^{mn} \nonumber \\
&+ {1 \over 2}e^{{3 \over 5}\varphi}\M_{mm'}\M_{nn'}F'^{2m'n'}F'^{2mn} + {1 \over 2}e^{{3 \over 2}\varphi}({\partial}_{\mu}a^{q})({\partial}^{\mu}a^{q'})\M_{qq'} \Big{]}
\end{align}
and writing $\hat{G}^2$ together with $R$ and a redefinition $\varphi \rightarrow (\sqrt{10}/3) \varphi$ gives
\begin{align}
&\int d^{11}x \sqrt{|\hat{g}|} (\hat{R}-{1 \over 48}\hat{G}^2) \nonumber \\
&= \int d^7x \sqrt{|g_E|} \Big{[} R_E - {1 \over 2}(\partial \varphi)^2-{1 \over 4}e^{-{3 \over \sqrt{10}}\varphi}\M_{mn}F^{1m}F^{1n} \nonumber \\
&\phantom{=} -{1 \over 4}\mbox{Tr}(\M^{-1}\partial \M)^2 - {1 \over 48}e^{{-4 \over \sqrt{10}}\varphi}G^2 - {1 \over 12}e^{-{1 \over \sqrt{10}} \varphi}H_mH_n\M^{mn} \nonumber \\
&\phantom{=}- {1 \over 2}e^{{2 \over \sqrt{10}} \varphi}\M_{mm'}\M_{nn'}F'^{2m'n'}F'^{2mn} - {1 \over 2}e^{{\sqrt{10} \over 2} \varphi}({\partial}_{\mu} a^q)({\partial}^{\mu} a^{q'})\M_{qq'} \Big{]}.
\eqnlab{reduct_7d}
\end{align}


\subsubsection{SL(5)/SO(5) parametrisation}
From table 2.3 we find that the U-duality coset in d=7 is \coset{5}. We thus want to rewrite the Lagrangian \eqnref{reduct_7d} in an SL(5,$\rr$) covariant way with the help of an \coset{5} parametrisation matrix $\W$. 
It makes sense to start with the scalars.

\subsubsection{The scalar terms}
The scalar part of the Lagrangian \eqnref{reduct_7d} is 
\begin{equation}
\Lagr_{scalar}^{d=7} = -\frac{1}{2}(\partial\varphi)^2-\frac{1}{4}\tr(\M^{-1}\partial \M)^2-\frac{1}{2}e^{\frac{\sqrt{10}}{2}\varphi}\partial a^m\M_{mn}\partial a^n.
\eqnlab{sl5so5_scalar_lagr}
\end{equation}
We note that there is a total of 14 scalar fields out of which 4 are dilatons ($\varphi$ and three in $\M$) and 10 are axions ($a\times 4$ and six lives in $\M$), in this Lagrangian, which is precisely the number and types of scalars in $\W$, as we saw in subsection \ssecref{ex_sl5so5}.  
So the game is now to include all the scalars in $\W$ and hope that it is possible to rewrite the scalar Lagrangian as one single kinetic scalar term
\begin{equation}
\Lagr_{scalar}^{d=7} = -\frac{1}{4}\tr(\W^{-1}\partial \W)^2 = \frac{1}{4}\tr(\partial\W^{-1}\partial \W).
\end{equation}
By splitting the coset metrics $\M$ and $\W$ into upper diagonal vielbeins
\begin{align}
\M &= \N\N^T,&\M\in \mbox{\coset{4}}\nonumber\\
\W &= \V\V^T,&\W\in \mbox{\coset{5}}
\end{align}
we can rewrite the Lagrangian as
\begin{align}
\Lagr_{scalar}^{d=7} &= -\frac{1}{4}\tr\left(\W^{-1}\partial \W\right)^2 = -\frac{1}{4}\tr\left(\V^{T-1}\V^{-1}(\partial\V\V^T+\V\partial\V^T)\right)^2\nonumber\\
&=-\frac{1}{4}\tr\left(\V^{-1}\partial\V\V^{-1}\partial\V+(\V^{-1}\partial\V\V^{-1}\partial\V)^T+2\partial\V^T\V^{T-1}\V^{-1}\partial\V \right)\nonumber\\
&=-\frac{1}{2}\tr\left(\V^{-1}\partial\V\V^{-1}\partial\V+\V^{-1}\partial\V(\V^{-1}\partial\V)^T\right)\nonumber\\
&=-\frac{1}{2}\tr\left(\V^{-1}\partial\V(\V^{-1}\partial\V+(\V^{-1}\partial\V)^T)\right).
\eqnlab{scalar_kinetic_simp}
\end{align}
From the information in \eqnref{sl5so5_scalar_lagr} we deduce that $\V$ must contain the dilaton $\varphi$ and the four axions $a^m$ together with the \coset{4} vielbein $\N$.
We note that the Lagrangian \eqnref{sl5so5_scalar_lagr} is invariant under a redefinition $\V\rightarrow\V^{T-1}$ coming from the symmetry between $\M$ and $\M^{-1}$ in the $\tr(\partial M^{-1}\partial M)$-term.
We thus make the Ansatz
\begin{equation} 
\U=\toto{\T f(a^m,\varphi)}{g(a^m,\varphi)}{0}{h(a^m,\varphi)}=
\setlength{\unitlength}{.4mm}
\left(\begin{array}{l}\cr\mbox{}\end{array}\right.
\begin{picture}(23,30)(8,10)% (size)(offset)
\put(0,0){
\path(0,7)(20,7)(20,27)(0,27)(0,7)
\path(0,0)(20,0)(20,5)(0,5)(0,0)
\path(22,7)(27,7)(27,27)(22,27)(22,7)
\path(22,0)(27,0)(27,5)(22,5)(22,0)
\put(10,17){\makebox(0,0){\tiny 4$\times$4}}
\put(10,2.5){\makebox(0,0){\tiny 1$\times$4}}
\put(35,17){\makebox(0,0){\tiny 4$\times$1}}
\put(35,2.5){\makebox(0,0){\tiny 1$\times$1}}
\path(25,17)(29,17)
\path(25,2.5)(29,2.5)
}
\end{picture}
\left.\begin{array}{l}\cr\mbox{}\end{array}\right)
\end{equation}
where f and h are scalar functions of the fields, g is a 4-dimensional vector function of the fields, $\U$ is either $\V$ or $\V^{T-1}$, which can be chosen for convenience later and $\T$ is similarly either $\N$ or $\N^{T-1}$. 
We have thus reduced the problem to the level of a four piece jigsaw puzzle, so all that is left is to fit the pieces together.

The condition $\det\U=1$ and the knowledge $\det\T=1$ gives a relationship between f and h
\begin{equation}
1=\det\U=hf^4\det\T=hf^4,
\end{equation}
so $h$ will henceforth be replaced by $f^{-4}$.
The inverse of $\U$ is easily found to be
\begin{equation}
\U^{-1}=\toto{f^{-1}\T^{-1}}{-f^{3}\T^{-1}g}{0}{f^4}
\end{equation}
and
\begin{equation}
\partial\U=\toto{f\partial\T+\partial f\T}{\partial g}{0}{-4f^{-5}\partial f}
\end{equation}
giving
\begin{equation}
\U^{-1}\partial\U = \toto{\T^{-1}\partial\T+f^{-1}\partial f}{f^{-1}\T^{-1}\partial g+4f^{-2}\partial f\T^{-1}g}{0}{-4f^{-1}\partial f}. 
\end{equation}
The upper left part of the matrix in the trace in \eqnref{scalar_kinetic_simp} becomes
\begin{align}
(\U^{-1}&\partial\U(\U^{-1}\partial\U+(\U^{-1}\partial\U)^T)_{mn}=\T^{-1}\partial\T\T^{-1}\partial\T+\T^{-1}\partial\T(\T^{-1}\partial%%@
\T)^T\nonumber\\
&+3f^{-1}\partial f \T^{-1}\partial\T+f^{-1}\partial f (\T^{-1}\partial\T)^T+2f^{-2}(\partial f)^2\id_{4\times4}\nonumber\\
&+f^{-2}\T^{-1}\partial g\partial g^T\T^{T-1}+4f^{-3}\partial f\T^{-1}\partial g g^T\T^{T-1}\nonumber\\
&+4f^{-3}\partial f\T^{-1}g\partial g^T\T^{T-1}+16f^{-4}(\partial f)^2\T^{-1}gg^t\T^{T-1}
\end{align}
and the lower right part becomes
\begin{align}
(\U^{-1}&\partial\U(\U^{-1}\partial\U+(\U^{-1}\partial\U)^T)_{55}=32f^{-2}(\partial f)^2.
\end{align}
Take the trace and remember that $\tr(\T^{-1}\partial\T)=0$ from \eqnref{conven_parinv}
\begin{align}
\Lagr_{scalar}^{d=7} &=-\frac{1}{2}\tr\left(\U^{-1}\partial\U(\U^{-1}\partial\U+(\U^{-1}\partial\U)^T)\right)\\
&=-\frac{1}{2}\tr\left(\T^{-1}\partial\T(\T^{-1}\partial\T+(\T^{-1}\partial\T)^T)\right)-\frac{1}{2}f^{-2}\partial g^T\T^{T-1}\T^{-1}\partial g\nonumber\\
&-4f^{-3}\partial fg^T\T^{T-1}\T^{-1}\partial g
-8f^{-4}(\partial f)^2g^t\T^{T-1}\T^{-1}g-20f^{-2}(\partial f)^2.\nonumber
\end{align}
The first term is independent on the choice of $\T$ and can be written as $\Lagr_0=-\tr(\M^{-1}\partial\M)^2/4$ in both cases. Since the scalars in the Lagrangian \eqnref{sl5so5_scalar_lagr} are multiplied with $\M$ and not $\M^{-1}$, we choose to use $\U=\V^{T-1}$ so $\T=\N^{T-1} \Leftrightarrow \T^{T-1}\T^{-1}=\M$.
Refine the Ansatz by letting
\begin{align}
f&=e^{x\varphi},\hspace{1cm} &g&=a^me^{y\varphi},\nonumber\\
\partial f &= x\partial\varphi e^{x\varphi}, &\partial g &=y\partial\varphi a^me^{y\varphi} + \partial a^me^{y\varphi},
\end{align}
where x and y are real parameters to be adjusted later.
This means
\begin{align}
\Lagr&_{scalar}^{d=7}-\Lagr_0 = \left\{-\frac{1}{2}(y\partial\varphi a^m + \partial a^m)\M_{mn}(y\partial\varphi a^n + \partial a^n)\right.\nonumber\\ 
&-4x\partial\varphi a^m\M_{mn}(y\partial\varphi a^n + \partial a^n)-8x^2(\partial\varphi)^2a^m\M_{mn}a^n\bigg\}e^{2y\varphi-2x\varphi}\nonumber\\
&-20x^2(\partial\varphi)^2.
\end{align}
Comparing the last term to $-(\partial\varphi)^2/2$ in \eqnref{sl5so5_scalar_lagr} gives $x=\pm 1/(2\sqrt{10})$ and comparing the exponent $2(y-x)\varphi$ to the exponent in 
\begin{align}
-\frac{1}{2}e^{\frac{\sqrt{10}}{2}\varphi}\partial a^m\M_{mn}\partial a^n
\end{align}
from \eqnref{sl5so5_scalar_lagr}, gives $y=(10\pm 2)/(4\sqrt{10})$. Choosing the minus sign, so $x=-1/(2\sqrt{10})$ and $y=2/\sqrt{10}$, removes all unwanted terms and we have   
\begin{equation}
-\frac{1}{4}\tr(\W^{-1}\partial \W)^2 = -\frac{1}{2}(\partial\varphi)^2-\frac{1}{4}\tr(\M^{-1}\partial \M)^2-\frac{1}{2}e^{\frac{\sqrt{10}}{2}\varphi}\partial a^m\M_{mn}\partial a^n
\eqnlab{sl5so5_scalar_kinetic}
\end{equation}
just as we wanted. $\W$ is explicitly given by
\begin{align}
\W&=\V\V^T = \U^{T-1}\U^{-1} = \toto{f^{-1}\N}{0}{-f^{3}g^T\N}{f^4}\toto{f^{-1}\N^{T}}{-f^{3}\N^{T}g}{0}{f^4}\nonumber\\ 
&= \toto{f^{-2}\M}{-f^{2}\M g}{-f^{2}g^T\M}{f^6g^T\M g+f^8}=e^{\frac{1}{\sqrt{10}}\varphi}\toto{\M_{mn}}{-a_m}{-a_n}{a_m a^m+e^{\frac{-5}{\sqrt{10}}\varphi}}.
\eqnlab{sl5so5_metric}
\end{align}
To see that the metric indeed is a parametrisation of \coset{5} we first note that if $\W$ belongs to \coset{5} then so does $\W^{-1}$. This means we can compare the vielbein $\U$ to the vielbein $\V$ \eqnref{ex_sl5so5_v} found in the \coset{5} example.
We find that after doing the following field redefinitions in \eqnref{ex_sl5so5_v}
\begin{align}
&\tilde\tau_{10}=a_1,\hspace{.5cm}\tilde\tau_9=a_2,\hspace{.5cm}\tilde\tau_7=a_3,\hspace{.5cm}\tau_4=a_4,\hspace{.5cm}\phi_1=\phi_1'+\frac{1}{\sqrt{10}}\varphi,\nonumber\\
&\phi_2=\phi_2'+\frac{2}{\sqrt{10}}\varphi,\hspace{.5cm}\phi_3=\phi_3'+\frac{3}{\sqrt{10}}\varphi,\hspace{.5cm}\phi_4=\frac{4}{\sqrt{1%%@
0}}\varphi
\end{align}
the vielbeins are identical and hence $\W\in$\coset{5}.

\subsubsection{The 1-form terms}
We now want to use the found metric $\W$ to write the 1-form gauge potentials in an \coset{5} covariant way. 
The 1-form part of the Lagrangian \eqnref{reduct_7d} is 
\begin{equation}
\Lagr_{A}^{d=7} = -\frac{1}{4}e^{-\frac{3}{\sqrt{10}}\varphi}F^{1m}\M_{mn}F^{1n}-\frac{1}{2}e^{\frac{2}{\sqrt{10}}\varphi}F'^{2[mn]}\M_{mm'}\M_{nn'}F'^{2[m'n']}.
\eqnlab{sl5so5_1form_lagr}
\end{equation}
Let r,s,... run from 1 to 5 and decompose
\begin{align}
F^{[rs]}\W_{rr'}\W_{ss'}F^{[r's']} &= F^{[mn]}\W_{mm'}\W_{nn'}F^{[m'n']}+4F^{[mn]}\W_{m5}\W_{nn'}F^{[5n']}\nonumber\\
& \phantom{=} +2F^{[5n]}\W_{55}\W_{nn'}F^{[5n']}+2F^{[5n]}\W_{5m'}\W_{n5}F^{[m'5]}.
\eqnlab{sl5so5_1form_expansion}
\end{align}
Define the 10 field strengths of $F^{[rs]}$ to contain the 4 $F^{1m}$ field strengths and the 6 $F^{'2[mn]}$ field strengths according to  
\begin{align}
F^{5n} &= \frac{1}{2}F^{1n} = -F^{n5},\nonumber\\
F^{[mn]} &= F^{2[mn]} = F^{'2[mn]} + a^{[m}F^{1n]},
\label{dsdg}
\end{align}
where we have used the relation for $F'^{2mn}$ in \eqnref{reduct_7d_fprim}.
Inserting this together with the components of $\W$ into \eqnref{sl5so5_1form_expansion} gives
\begin{align}
F&^{[rs]}\W_{rr'}\W_{ss'}F^{[r's']} = \bigg\{F^{2[mn]}\M_{mm'}\M_{nn'}F^{2[m'n']}-2F^{2[mn]}a_m\M_{nn'}F^{1n'}\nonumber\\ 
&+\left.\frac{1}{2}F^{1m}\left(a_pa^p\M_{mn}+e^{-\frac{5}{\sqrt{10}}\varphi}\M_{mn}-a_ma_n\right)F^{1n}\right\}e^{\frac{2}{\sqrt{10}}\varphi}\nonumber\\
&=\bigg\{F'^{2[mn]}\M_{mm'}\M_{nn'}F'^{2[m'n']}+(2-2)F'^{2[mn]}a_m\M_{nn'}F^{1n'}\nonumber\\ 
&+\left.F^{1m}\left(\frac{1}{2}e^{-\frac{5}{\sqrt{10}}\varphi}\M_{mn}+(a_pa^p\M_{mn}-a_ma_n)(\frac{1}{2}+\frac{1}{2}-1)\right)F^{1n}\right\}e^{\frac{2}{\sqrt{10}}\varphi}\nonumber\\
&=e^{\frac{2}{\sqrt{10}}\varphi}F'^{2[mn]}\M_{mm'}\M_{nn'}F'^{2[m'n']}+\frac{1}{2}e^{-\frac{3}{\sqrt{10}}\varphi}F^{1m}\M_{mn}F^{1n}.
\end{align}
We thus see that we can rewrite the 1-form Lagrangian \eqnref{sl5so5_1form_lagr} as
\begin{equation}
\Lagr_{A}^{d=7} = -\frac{1}{2}F^{[rs]}\W_{rr'}\W_{ss'}F^{[r's']}
\eqnlab{sl5so5_1form_term}
\end{equation}
which is invariant under SL(5,$\rr$) transformations.

\subsubsection{Higher order terms}


We now have the action
\begin{align}
S^{d=7} = \int d^7x \sqrt{|g_E|} &\Big{[} R_E - \frac{1}{4}\tr(\W^{-1}\partial \W)^2 - {1 \over 2}F^{[rs]}\W_{rr'}\W_{ss'}F^{[r's']}\nonumber\\ 
& - {1 \over 12}e^{-{1 \over \sqrt{10}} \varphi}H_mH_n\M^{mn} - {1 \over 48}e^{{-4 \over \sqrt{10}}\varphi}G^2 \Big{]}.
\label{asdfghjkl}
\end{align}
The next step is to write the $H$ and $G$ terms in an \coset{5} invariant way. To accomplish this we will use that the 
dual of $G$ in 7 dimensions is a 3-form and hence we can write it together with with the four 3-forms $H_m$ by using the 5-dimensional metric $\W$ 
derived above. We start by noting that
\begin{equation}
-{1 \over 48}\int d^7x \sqrt{|g_E|} e^{{-4 \over \sqrt{10}}\varphi}G^2 = -{1 \over 2}\int e^{{-4 \over \sqrt{10}}\varphi} *G \wedge G.
\end{equation}
Inserting this in the action and varying with respect to $C$ gives
\begin{align}
\delta S &=-{1 \over 2}\int e^{{-4 \over \sqrt{10}}\varphi}( \delta G \wedge *G + G \wedge \delta*G ) + \delta S_{CS} \nonumber \\
& =-\int e^{{-4 \over \sqrt{10}}\varphi}( d \delta C \wedge *G ) + \delta S_{CS} \nonumber \\
& =- \int \delta C \wedge d(e^{{-4 \over \sqrt{10}}\varphi}*G) + \delta S_{CS},
\eqnlab{kjskj}
\end{align}
where $\delta S_{CS}$ is the variation of the Chern-Simons term. We can also calculate the dual of $\hat{G}$ and expand it in the same way 
as in (\ref{g7}), yielding
\begin{equation}
\hat{G_7}= \cdots +{1 \over 3!}G'^q \epsilon_{mnpq} \hat{e}^m \wedge \hat{e}^n \wedge \hat{e}^p - {1 \over 4!}H' \epsilon_{mnpq} \hat{e}^m \wedge \hat{e}^n \wedge \hat{e}^p \wedge \hat{e}^q, 
\end{equation}
where we have ignored the forms of higher order than four. Via \eqnref{bianchig7} we get
\begin{equation}
dH'=H_m \wedge da^m + {1 \over 2}\epsilon_{mnpq}F'^{2mn} \wedge F'^{2pq}.
\eqnlab{sadf}
\end{equation}
If we would have calculated $\delta S_{CS}$ in \eqnref{kjskj} we would have found the equation of motion for $e^{{-4 \over \sqrt{10}}\varphi}*G$ 
to be the right hand side of \eqnref{sadf}. (This of course have to be the case since the Bianchi identity for $H'$ is derived from the $\hat{C}$-e.o.m.) 
This means that we can swap the equation of motion and the Bianchi identity and define
\begin{equation}
H'=e^{{-4 \over \sqrt{10}}\varphi}*G.
\end{equation}
The next step is to rewrite $G^2$ in (\ref{asdfghjkl}) as $4(*G)^2$. This gives, if we only consider the $H$ and $G$ terms, the action
\begin{align}
S_{H,G} &= \int d^7x \sqrt{|g_E|} \Big{[} -{1 \over 12}e^{-{1 \over \sqrt{10}} \varphi}H_mH_n\M^{mn} - {4 \over 48}e^{{-4 \over \sqrt{10}}\varphi}(*G)^2 \Big{]} \nonumber \\
&= -{1 \over 12} \int d^7x \sqrt{|g_E|} \Big{[} e^{-{1 \over \sqrt{10}} \varphi}H_mH_n\M^{mn} + e^{{4 \over \sqrt{10}}\varphi}(H')^2 \Big{]}.
\end{align}
If we define
\begin{equation}
H_5 = H' - H_m a^m
\label{h5}
\end{equation}
and
\begin{equation}
H_r = (H_m,H_5),
\end{equation}
we get
\begin{align}
-{1 \over 12} \int & d^7x \sqrt{|g_E|} H_r H_s \W^{rs}  \nonumber \\
=& -{1 \over 12} \int d^7x \sqrt{|g_E|} \Big{[} e^{-{1 \over \sqrt{10}} \varphi}H_mH_n\M^{mn} + e^{{4 \over \sqrt{10}} \varphi}a^m a^n H_m H_n \nonumber \\
& +2e^{{4 \over \sqrt{10}} \varphi}H_mH_5a^m +e^{{4 \over \sqrt{10}} \varphi}H_5H_5 \Big{]} \nonumber \\
=& -{1 \over 12} \int d^7x \sqrt{|g_E|} \Big{[} e^{-{1 \over \sqrt{10}} \varphi}H_mH_n\M^{mn} + e^{{4 \over \sqrt{10}} \varphi}\Big{(}a^m a^n H_m H_n \nonumber \\
& +2H_ma^m(H'-H_na^n) + (H'-H_ma^m)(H'-H_na^n) \Big{)}\Big{]} \nonumber \\
=& -{1 \over 12} \int d^7x \sqrt{|g_E|} \Big{[} e^{-{1 \over \sqrt{10}} \varphi}H_mH_n\M^{mn} + e^{{4 \over \sqrt{10}} \varphi}H'^2 \Big{]},
\end{align}
and hence the complete action (apart from the Chern-Simons term) for 7-dimensional supergravity becomes
\begin{align}
S^{d=7} = \int d^7x \sqrt{|g_E|} &\Big{[} R_E - \frac{1}{4}\tr(\W^{-1}\partial \W)^2 - {1 \over 2}F^{[rs]}\W_{rr'}\W_{ss'}F^{[r's']}\nonumber\\ 
& - {1 \over 12}H_r H_s \W^{rs} \Big{]}.
\end{align}
From \eqnref{reduct_7d_bianchi}, \eqnref{sadf} and (\ref{h5}) we get the Bianchi identity
\begin{align}
dH_5 &=dH'-H_m \wedge da^m -dH_ma^m \nonumber \\
&= {1 \over 2}\epsilon_{mnpq}F'^{2mn} \wedge F'^{2pq} + \epsilon_{mnpq}a^mF^{1n}\we F^{2pq},
\end{align}
and we also have, from \eqnref{reduct_7d_bianchi} once again
\begin{equation}
dH_q=\epsilon_{mnpq}F^{1m} \wedge F^{2np}=2\epsilon_{mnpq}F^{5m} \wedge F^{2np}.
\end{equation}
These can now be combined to the single SL(5,$\rr$) covariant Bianchi identity
\begin{equation}
dH_v={1 \over 2}\epsilon_{rstuv}F^{rs} \wedge F^{tu},
\eqnlab{SL5bianchiH}
\end{equation}
as is proved by calculating the components
\begin{align}
{1 \over 2}\epsilon&_{rstu5}F^{rs} \wedge F^{tu} \nonumber \\
&={1 \over 2}\epsilon_{mnpq5}F^{mn} \wedge F^{pq} ={1 \over 2}\epsilon_{mnpq5}(F'^{2mn} + a^m F^{1n}) \wedge (F'^{2pq} + a^p F^{1q}) \nonumber \\
&=\Big{\{} \epsilon_{mnpq5}a^m F^{1n} \wedge a^p F^{1q} =0\Big{\}} \nonumber \\
&={1 \over 2}\epsilon_{mnpq5}(F'^{2mn} \wedge F'^{2pq} + 2 a^m F^{1n} \wedge F'^{2pq}) \nonumber \\
&={1 \over 2}\epsilon_{mnpq5}(F'^{2mn} \wedge F'^{2pq} + 2 a^m F^{1n} \wedge F^{2pq})=dH_5,
\end{align}
and
\begin{align}
{1 \over 2}\Big{(}& \epsilon_{5mnpq}F^{5m} \wedge F^{np} + \epsilon_{m5npq}F^{m5} \wedge F^{np} + \epsilon_{mn5pq}F^{mn} \wedge F^{5p} \nonumber \\
&+ \epsilon_{mnp5q}F^{mn} \wedge F^{p5} \Big{)} = {1 \over 2}\Big{(}4\epsilon_{mnpq5}F^{5m} \wedge F^{np}\Big{)}=dH_q.
\end{align}
Before we finish this chapter we want to make the rest of the Bianchi identities and the gauge transformations SL(5,$\rr$) covariant. 
With $F^{rs}$ defined as in (\ref{dsdg}) we see that we get
\begin{equation}
dF^{rs}=0,
\end{equation}
and also that the field strength is invariant under the transformation
\begin{equation}
\delta A^{rs} = d\chi^{rs}.
\end{equation}
Next we turn the attention to the variation of the $B$-field. First we combine $B_5$, the (not exact) potential to $H_5$, 
with $B'_m$ to form $B_v=(B'_m,B_5)$. Then we note that it is possible, even without a redinition of $B'_m$, to write
\begin{equation}
H_v=dB_v + {1 \over 2}\epsilon_{rstuv}F^{rs} \wedge A^{tu}.
\end{equation}
Differentiating this immediately gives \eqnref{SL5bianchiH}. $H_v$ is thus invariant under the SL(5,$\rr$) covariant transformation
\begin{equation}
\delta B_v = d\chi_v + {1 \over 2}\epsilon_{rstuv}A^{rs} \wedge d\chi^{tu}.
\end{equation}
The last Bianchi identity that needs to be made covariant is the one for $G$. Noting that $F^{1m}=2F^{5m}$, the B.I. for $G$ turns into
\begin{equation}
dG=2H_m \wedge F^{5m}.
\end{equation}
Redefining $C'$ as
\begin{equation}
C'=C-2A^{5m} \wedge B_m + 2\epsilon_{mnpq}A^{5m} \wedge A^{pq} \wedge A^{5n},
\end{equation}
gives $G$ as
\begin{equation}
G=dC+B_m \wedge F^{5m} - A^{5m} \wedge H_m.
\end{equation}
Hence we can combine $G$ with the four 4-forms originating from the dual of $\hat{G}$ to form
\begin{equation}
G^r=dC^r + B_s \wedge F^{rs} - A^{rs} \wedge H_s.
\label{awedc}
\end{equation}
This is consistent with
\begin{equation}
dG^r=2H_s \wedge F^{rs},
\end{equation}
as is shown by the calculation
\begin{align}
dG^{r'}&=dB_v \wedge F^{r'v} -A^{r'v} \wedge dH_v + F^{r'v} \wedge H_v \nonumber \\
&=2H_v \wedge F^{r'v} - {1 \over 2}\epsilon_{rstuv}F^{rs} \wedge (A^{tu} \wedge F^{r'v} + A^{r'v} \wedge F^{tu}).
\end{align}
The vanishing of the second term above can be shown by first setting $r'=5$ according to
\begin{align}
\epsilon_{rstuv}F^{rs} \wedge (&A^{tu} \wedge F^{5v} + A^{5v} \wedge F^{tu}) \nonumber \\
&=\epsilon_{rstuq}F^{rs} \wedge (A^{tu} \wedge F^{5q} + A^{5q} \wedge F^{tu}) \nonumber \\
&=\epsilon_{mn5pq}F^{mn} \wedge (A^{5p} \wedge F^{5q} + A^{5q} \wedge F^{5p}) \nonumber \\
&\phantom{=}+\epsilon_{rspuq}F^{rs} \wedge (A^{pu} \wedge F^{5q} + A^{5q} \wedge F^{pu}) \nonumber \\
&=2\epsilon_{mn5pq}F^{mn} \wedge (A^{5p} \wedge F^{5q} + A^{5q} \wedge F^{5p}) \nonumber \\
&\phantom{=}+\epsilon_{rsnpq}F^{rs} \wedge (A^{np} \wedge F^{5q} + A^{5q} \wedge F^{np}) \nonumber \\
&=2\epsilon_{mn5pq}F^{mn} \wedge (A^{5p} \wedge F^{5q} + A^{5q} \wedge F^{5p}) \nonumber \\
&\phantom{=}+2\epsilon_{m5npq}F^{m5} \wedge (A^{np} \wedge F^{5q} + A^{5q} \wedge F^{np}) \nonumber \\
&=4\epsilon_{mn5pq}F^{mn} \wedge (A^{5p} \wedge F^{5q} + A^{5q} \wedge F^{5p})=0,
\end{align}
where $r,s,t,u,v$ are 5-dimensional indices and $m,n,p,q$ are 4-dimensional indices as usual.
Doing the same calculation but this time with $r'=m$ would show that the term is zero, and hence we receive the desired Bianchi identity. With $G^r$ 
defined as in (\ref{awedc}) the variation of it becomes
\begin{align}
\delta G^{r'} &= d \delta C^{r'} + \delta B_v \wedge F^{r'v} - \delta A^{r'v} \wedge H_v \nonumber \\
&=d\delta C^{r'} + (d\chi_v + {1 \over 2}\epsilon_{rstuv}A^{rs} \wedge d\chi^{tu}) \wedge F^{r'v} - d\chi^{r'v} \wedge H_v \nonumber \\
&=d(\delta C^{r'} + d\chi_v \wedge A^{r'v} + {1 \over 2}\epsilon_{rstuv}A^{rs} \wedge d\chi^{tu} \wedge A^{r'v} - d\chi^{r'v} \wedge B_v) \nonumber \\
&\phantom{=}-{1 \over 2}\epsilon_{rstuv}(F^{rs} \wedge d\chi^{tu} \wedge A^{r'v} + d\chi^{r'v} \wedge F^{rs} \wedge A^{tu}) \nonumber \\
&=d(\delta C^{r'} + d\chi_v \wedge A^{r'v} + {1 \over 2}\epsilon_{rstuv}A^{rs} \wedge d\chi^{tu} \wedge A^{r'v} - d\chi^{r'v} \wedge B_v),
\end{align}
where one again have to calculate the components of the $F \wedge A$ terms to see that they cancel out.
If we want $G^r$ to be invariant we see that the variation of $C^r$ must be
\begin{equation}
\delta C^{r'}=d\chi^{r'} - A^{r's} \wedge d\chi_s + B_s \wedge d\chi^{r's} - {1 \over 2}\epsilon_{rstuv}A^{r'r} \wedge A^{st} \wedge d\chi^{uv}.
\end{equation}

This concludes the third chapter. After all this calculational work, a short summary might be needed. In this chapter 
we have constructed the complete bosonic action for 9-, 8- and 7-dimensional supergravity with the various symmetries clearly expressed. 
We have also constructed U-duality covariant Bianchi identities as well as U-duality covariant gauge transformations. 
These will all play a very important role in the rest of this thesis work.







%\part{Dynamics}
%\chapter{Brane dynamics}
\label{sec:dynamics}

In part one of this thesis we used several pages to calculate and explain the Bianchi identities and the duality relations that arise in compactified supergravity.
We also made an effort in trying to show the various symmetries in supergravity and how they manifest themselves in the equations. 
But the main purpose of this thesis is to investigate the dynamics of a brane embedded in the previously derived background.
Hence, our next step is to construct an action for a brane that couples to the all the background fields.
This is what we will do in the next chapter, by using field strengths roughly of the form "$f=da-A$", where $a$ is a world-volume field and $A$ is the pullback of a background field. 
In other words, every background field has its world-volume counterpart and there is a maximal limit to what rank of the background fields the brane couples to. 
This type of coupling is very useful as we will see in the next chapter, but still it needs some explanation and motivation and that is what we will try to provide here. 
The procedure was used in \cite{artikeln}, first outlined in \cite{valskriven} and developed and generalized in \cite{ref10},\cite{ref11},\cite{ref12},\cite{ref13}.

\section{Maurer-Cartan theory}
\label{sec:maurer}
The definition of a Lie group (see for example \cite{fuchs}) is a finite-dimensional differentiable manifold that also 
carries the algebraic structures of a group. A simple one-dimensional example is the circle. After identifying the 
identity element, any other point at angle $\theta$ from the identity acts by rotating the circle by the angle $\theta$ 
and hence obeying the group properties. Another example is the group $SO(n)$ of all rotations in an $n$-dimensional space. 
This can also be described as the group of all $n \times n$-matrices that are orthogonal and have determinant one. 
A consequence of combining group and manifold structures is that there always exists differentiable mappings of the Lie 
group manifold such that any group element can be mapped to any other group element. These kind of mappings are called left- or right-translations. 
This property makes it possible to construct the tangent space at some fixed group 
element and then by left- or right-translations transport a basis of the vector space to any other point of the manifold, and in this way obtain a basis 
in the tangent space of each point of the group manifold. This construction implies that on a Lie group manifold there exists 
global vector fields that vanish nowhere. They can be obtained by fixing a vector in some arbitrary tangent space (but usually at the identity element) and then 
transport it to the tangent space of any other element of the group. These vector fields are invariant under left- and right-translations respectively and the vector space of the invariant vector fields are, by construction, isomorphic to any tangent space. 
For any two vector fields $A$, $B$ on an arbitrary differentiable manifold we now define the bracket
\begin{equation}
[A,B]^a = \displaystyle\sum_{b=1}^{d}{B^b{\partial A^a \over \partial \xi^b}} - \displaystyle\sum_{b=1}^{d}{A^b{\partial B^a \over \partial \xi^b}},
\label{liebracket}
\end{equation}
where $\xi_a$ are the local coordinates. This definition is covariant under a change of the coordinate system, and hence the bracket is again a vector field. 
It can be shown that $[A,B]$ fulfills the Jacobi identity and it is manifestly bilinear and antisymmetric. In short, the vector space 
of all vector fields endowed by (\ref{liebracket}) becomes a Lie algebra. In the special case where the manifold is a Lie group it can be shown that the invariant 
vector fields mentioned above gets a natural structure of a finite-dimensional Lie subalgebra of the Lie algebra of all vector fields. The point of all this is to 
show that the tangent space at the unit element (or any other element) of a Lie group carries the structure of a Lie algebra, and that invariant vector fields 
on the manifold are used to identify it. \newline
Using invariant vector fields is perhaps the most common way to describe a Lie group and its Lie algebra, but there is an alternative way that uses differential forms 
on the manifold to capture information about the Lie algebra. This approach is called Maurer-Cartan theory and is often more appropriate 
when dealing with supergravity since the goal is to construct an action integral. Consider a manifold $M$ 
and its cotangent space CT($M$), CT($M$) being the vector space of 1-forms on the manifold $M$. The Maurer-Cartan forms $\xi^A$ 
is a basis to CT($M$) and hence $d\xi^A$ is a 2-form and can be expressed in the basis provided by $\xi^B \wedge \xi^C$,
\begin{equation}
d\xi^A=f{^A}_{BC}\xi^B \wedge \xi^C.
\end{equation}
If we can find a basis $\xi^A$ such that $f{^A}_{BC}$ are constants,
\begin{equation}
f{^A}_{BC}=-{1 \over 2}C{^A}_{BC},
\end{equation}
we get
\begin{equation}
d\xi^A+{1 \over 2}C{^A}_{BC}\xi^B \wedge \xi^C=0,
\label{mcstructure}
\end{equation}
and
\begin{equation}
d^2 \xi^A = -C{^A}_{BC}d\xi^B \wedge \xi^C = {1 \over 2}C{^A}_{BC}C{^B}_{DE}\xi^D \wedge \xi^E \wedge \xi^C = 0.
\label{mcjacobi}
\end{equation}
Then $M$ is a Lie group and (\ref{mcstructure}) are its Maurer-Cartan equations. The cotangent space is defined as the dual 
to the tangent space and hence the relation
\begin{equation}
\xi^A(A_B)=\delta^{A}_B,
\end{equation}
gives a connection between the the two different descriptions of a Lie group. As a matter of fact, equation (\ref{mcstructure}) 
implies (\ref{liebracket}) and vice versa, and equation (\ref{mcjacobi}) is equivalent to the Jacobi identity. To relate to potentials 
in supergravity theory one simply replaces the 1-forms $\xi^A$ with a set of 1-forms $\mu^A$ that do not satisfy (\ref{mcstructure}). The 2-forms
\begin{equation}
R^A = d\mu^A + {1 \over 2}C{^A}_{BC}\mu^B \wedge \mu^C
\label{mccurvature}
\end{equation}
expresses the deviation from the Maurer-Cartan equations and are called the curvatures of $\mu^A$. Equations of this type are used thoroughly 
in section \secref{ricci} when calculating the reduction of the Ricci scalar. The authors of \cite{maurer} created a natural generalization of 
the Maurer-Cartan equations to the case of $p$-forms, i.e. the exterior derivative of a $p$-form can be expressed as a polynomial consisting of the $p$-form with constant coefficients. 
They also made generalizations of (\ref{mcjacobi}) and (\ref{mccurvature}) to forms of higher order. The concepts discussed in this section will be used to create an action for the branes and the coupling between the branes and the background.

\section{Branes and their actions}
\label{sec:branes}
Since the goal of this thesis is to investigate branes and their dynamics in special backgrounds, we feel it is appropriate 
to give a short introduction to the theory of branes and that this is the right place to give it. A rather complete introduction can be found 
in \cite{bengtsson}, here we will only cover the most necessary parts needed for our further work. First of all we will try to 
make clear where all of the different types of branes comes from. The $p$-branes origins from the string theories, where it became 
clear that the theories could be extended to include objects with higher dimension than 1. The name is a generalization of the 
2-dimensional membrane to $p$-dimensional $p$-branes. The number $p$ indicates the branes number of spatial dimensions which means that a string can be referred to as a $1$-brane. Since the string theories 
are 10-dimensional with 9 spatial dimensions there can be no $p$-branes of higher order than 9.

A special type of $p$-branes are the $D_p$-branes. They are created when one assigns Dirichlet boundary conditions to an open string. 
For instance, if the string have Dirichlet boundary conditions in 3 spatial dimensions, it can only move in 6 spatial dimensions. 
This 6-dimensional manifold on which the endpoints of the string is fixed is called a $D_6$-brane. Note also that for a 
$D_9$-brane the string can move in all dimensions which implies that the string must have Neumann boundary conditions. When $D_p$-branes 
first was discovered it was considered as a curiosity, there was no real need for them since one might easily consider the strings to have Neumann boundary conditions. 
But this opinion had to be revised with the discovery of T-duality. It turned out that an open string with Neumann conditions got 
Dirichlet conditions when subject to T-dualisation, making $D_p$-branes essential for the theory. Another important feature is the fact 
that $D_p$-branes make it possible to study the excitations of the brane by using the renormalizable quantum field theory of the open string 
instead of the non-renormalizable world-volume theory of the $D_p$-brane itself. In this way it becomes possible to compute non-perturbative phenomena using perturbative methods. 

In M-theory we have two fundamental branes, M2 and M5. To see why it is easiest to compare with electromagnetic field theory in four dimensions. 
With the $2$-form field strength $F_2$ given, one gets the electric and the magnetic source equations by acting with $d$ on the dual of $F_2$ and $F_2$ itself 
respectively. In other words the electrically charged particle is dual to the magnetically charged particle. In 11-dimensional M-theory there is 
only the $4$-form field strength $\hat{G}=d\hat{C}$. Acting with $d$ on the dual of $\hat{G}$ would then give a $p$-brane 
that is electrically coupled to the gauge field $\hat{C}$. The $p$-brane will be localized in 8 of the 10 spatial dimensions and hence we can conclude that it is a $2$-brane (M2).
Acting with $d$ directly on $\hat{G}$ gives instead a brane that is localized in 5 dimensions (M5), which is magnetically coupled to $\hat{C}$. 
Another completely equivalent way of putting it is to say that M5 is electrically coupled to $C_6$, the dual of $\hat{C}$, and that 
M2 is magnetically coupled to $C_6$. The M2- and the M5-brane are closely related to the $D_p$-branes. For instance, if one dimension 
of the M2-brane is compactified it is S-dual to the fundamental string in type IIA string theory. If instead we reduce M-theory on a circle the M2-brane 
reduces to a $D_2$-brane. In the same way, if the M5-brane wraps around a circle it becomes a $D_4$-brane in type IIA. In figure \figref{njae} one 
finds the general picture of the different branes. The plot indicates elementary $p$-branes, i.e. 
$p$-brane solutions carrying an electrical charge for the field strength and solitonic branes, or $p$-branes carrying 
a magnetic charge. Finally, the plot also indicates the branes obtained 
by Kaluza-Klein reduction\cite{supergrav_pbranes_lectures}.

%*hep-th/9701088

\newcommand\kk{\circle*{1.5}}
\newcommand\soliton{\circle*{3}}
\newcommand\element{\circle{3}}

\setlength{\unitlength}{0.9mm}
\begin{figure}[h]
\begin{center}
\begin{picture}(90,120)(0,-5)
\put(0,0){\vector(0,1){110}}
\put(0,0){\vector(1,0){90}}
\put(0,10){\kk}
\put(0,40){\kk}
\put(30,40){\kk}
\put(10,50){\kk}
\put(30,50){\kk}
\multiput(0,60)(10,0){6}{\kk}
\multiput(0,70)(10,0){7}{\kk}
\put(0,80){\kk}
\multiput(20,80)(10,0){4}{\kk}
\put(70,80){\kk}
\multiput(20,90)(10,0){2}{\kk}
\multiput(50,90)(10,0){2}{\kk}
\put(30,100){\element}
\put(10,90){\element}
\put(0,90){\element}
\put(10,80){\element}
\put(0,50){\element}
\put(0,30){\element}
\put(0,20){\element}
\put(10,30){\element}
\put(10,40){\element}
\put(60,100){\soliton}
\put(70,90){\soliton}
\put(80,90){\soliton}
\put(60,80){\soliton}
\put(40,50){\soliton}
\put(20,30){\soliton}
\put(20,40){\soliton}
\put(10,20){\soliton}
\put(20,50){\makebox(0,0){$\otimes$}}
\put(40,90){\makebox(0,0){$\otimes$}}
\put(-5,0){\makebox(0,0){\footnotesize{$1$}}}
\put(-5,10){\makebox(0,0){\footnotesize{$2$}}}
\put(-5,20){\makebox(0,0){\footnotesize{$3$}}}
\put(-5,30){\makebox(0,0){\footnotesize{$4$}}}
\put(-5,40){\makebox(0,0){\footnotesize{$5$}}}
\put(-5,50){\makebox(0,0){\footnotesize{$6$}}}
\put(-5,60){\makebox(0,0){\footnotesize{$7$}}}
\put(-5,70){\makebox(0,0){\footnotesize{$8$}}}
\put(-5,80){\makebox(0,0){\footnotesize{$9$}}}
\put(-5,90){\makebox(0,0){\footnotesize{$10$}}}
\put(-5,100){\makebox(0,0){\footnotesize{$11$}}}
\put(-5,110){\makebox(0,0){$D$}}
\put(0,-5){\makebox(0,0){\footnotesize{$0$}}}
\put(10,-5){\makebox(0,0){\footnotesize{$1$}}}
\put(20,-5){\makebox(0,0){\footnotesize{$2$}}}
\put(30,-5){\makebox(0,0){\footnotesize{$3$}}}
\put(40,-5){\makebox(0,0){\footnotesize{$4$}}}
\put(50,-5){\makebox(0,0){\footnotesize{$5$}}}
\put(60,-5){\makebox(0,0){\footnotesize{$6$}}}
\put(70,-5){\makebox(0,0){\footnotesize{$7$}}}
\put(80,-5){\makebox(0,0){\footnotesize{$8$}}}
\put(90,-5){\makebox(0,0){\footnotesize{$d$}}}
\put(48,32){\element}
\put(63,32){\makebox(0,0){elementary}}
\put(48,27){\soliton}
\put(60.6,27){\makebox(0,0){solitonic}}
\put(48,22){\makebox(0,0){$\otimes$}}
\put(60.7,22){\makebox(0,0){self-dual}}
\put(48,17){\kk}
\put(65.4,17){\makebox(0,0){Kaluza-Klein}}
\end{picture}
\caption{Supergravity $p$-brane solutions ($p \leq (D-3)$). $D$ is the dimension of spacetime and $d$ is the world-volume dimension.}
\figlab{njae}
\end{center}
\end{figure}


\subsection{The point particle}
To get to the action we will use in the next chapter we start with the 
description of a classical point particle. As the particle moves in target space (with coordinates $X^0, X^1, \cdots, X^{D-1}$) 
it sweeps out a world-line in spacetime, parametrised by $\tau$. The infinitesimal path length then becomes
\begin{equation}
dl=(-ds^2)^{1/2}=(-dX^{\mu}dX^{\nu}\eta_{\mu \nu})^{1/2},
\end{equation}
where $\eta_{\mu \nu}$ is the Minkowski metric and hence the action becomes
\begin{equation}
S=-m\int dl=-m\int d\tau (-\dot{X^{\mu}}\dot{X_{\mu}})^{1/2}.
\end{equation}
Now consider the action
\begin{equation}
S'={1 \over 2}\int d\tau ({1 \over e}\dot{X^{\mu}}\dot{X_{\mu}} - e m^2),
\label{primedaction}
\end{equation}
where $e(\tau)$ is the world-line einbein. Varying $S'$ with respect to $e$ gives the equation of motion
\begin{equation}
e^2 m^2 + \dot{X^{\mu}}\dot{X_{\mu}} = 0,
\label{particleeom}
\end{equation}
which is solved by $e = m^{-1} (-\dot{X^{\mu}}\dot{X_{\mu}})^{1/2}$. Inserting this in (\ref{primedaction}) gives $S'=S$, showing that the two actions are equivalent. 
A big advantage with $S'$ compared to $S$ is that $S'$ allows a treatment of the massless, $m=0$, case. 
Another nice feature with $S'$ is that it does not use the awkward square root that $S$ does. \newline Creating an action 
for a bosonic membrane requires a similar procedure. For a 2-brane we parametrise the world-volume that the membrane sweeps out 
with the coordinates $\xi^i$, $i=(0,1,2)$. Here, $\xi^0$ is the time coordinate and the other two spatial coordinates.
The functions $X^{\mu}(\xi^i)$ describes the membrane's evolution in spacetime, giving the shape of the membrane's world-volume in target space.
There is an induced metric on the world-volume given by
\begin{equation}
h_{ij}=\partial_i X^{\mu} \partial_j X^{\nu} \eta_{\mu \nu}
\end{equation}
and by (just as in the particle case) writing the action as the total volume we get
\begin{equation}
S=-T\int dV = -T\int d^3 \xi (-\mbox{det}\partial_i X^{\mu} \partial_j X_{\mu})^{1/2},
\label{nambugoto}
\end{equation}
where the constant $T$ is the tension of the membrane. This is the Nambu-Goto action (or a generalization of it from the string case) for the membrane. 
But just as the action with the square root for the particle was inconvenient to work with, we would like an action more like 
(\ref{primedaction}) for the membrane. This can be created by using an independent world-volume metric $g_{ij}$,
\begin{equation}
S'=-{T \over 2}\int d^3 \xi \sqrt{-g} (g^{ij}\partial_i X^{\mu} \partial_j X_{\mu} -1),
\end{equation}
where $g=\mbox{det}(g_{ij})$ as usual. Calculating the equations of motion for this action we find
\begin{equation}
g_{ij}=\partial_i X^{\mu} \partial_j X_{\mu},
\end{equation}
i.e. the world-volume metric equals the one induced by the spacetime metric. We can also create an action totally analogous to (\ref{primedaction}) by writing it in the form
\begin{equation}
S''={1 \over 2}\int d^3 \xi [{1 \over V}\mbox{det}(\partial_i X^{\mu} \partial_j X_{\mu})-T^2V],
\end{equation}
where $V(\xi)$ is an independent world-volume density. Varying with respect to $V$ gives
\begin{equation}
V^2 T^2 + \mbox{det}(\partial_i X^{\mu} \partial_j X_{\mu})=0,
\end{equation}
and solving for $V$ and inserting in $S''$ gives $S'' = S$. Also note that that all our calculations so far can be done for general $p$-branes, i.e. the action for a $p$-brane becomes
\begin{equation}
S={1 \over 2}\int d^{p+1} \xi [{1 \over V}\mbox{det}(\partial_i X^{\mu} \partial_j X_{\mu})-T^2V].
\label{membraneaction}
\end{equation}

\subsection{Scale invariance}
\sseclab{scaleinvariance}
Investigating the actions (\ref{primedaction}) and (\ref{membraneaction}) we find that they are not scale invariant, i.e. not invariant under the transformations
\begin{equation}
x^{\mu} \rightarrow \lambda x^{\mu} \hspace{1cm} V \rightarrow \lambda^{2(p+1)}V,
\end{equation}
(or $e \rightarrow \lambda^2 e$ in the particle case). This motivated the authors of \cite{valskriven} to modify the actions slightly. Consider the action
\begin{equation}
S=\int d\tau {1 \over 2e}[\dot{X^{\mu}}\dot{X_{\mu}} - \dot{Y}^2],
\end{equation}
where $Y$ is the coordinate of an extra dimension. The $Y$ equation of motion becomes
\begin{equation}
\partial_{\tau}(e^{-1}\dot{Y})=0,
\end{equation}
which is solved by $\dot{Y} = me$ for arbitrary mass parameter $m$. Inserting this in the action gives (\ref{primedaction}). 
This new action is scale invariant, and the idea is that the mass of the particle can be viewed as its momentum in the extra dimension and that the scale invariance is broken when one chooses a particular solution of the equation of motion. 
To generalize this to the case of $p$-branes we have to introduce the independent world-volume $p$-form gauge potential $A$ and its $(p+1)$-form field strength $F = dA$.
The dual of $F$ is a scalar and hence we can write the action as
\begin{equation}
S=\int d^{p+1} \xi {1 \over 2V}[\mbox{det}(\partial_i X^{\mu} \partial_j X_{\mu})-(*F)^2].
\label{scaleaction}
\end{equation}
The equation of motion for $A$ becomes
\begin{equation}
\partial_{i}(V^{-1}*F)=0,
\end{equation}
and if we pick the solution $*F=TV$ we have again (\ref{membraneaction}). Noting that $A$ scales as
\begin{equation}
A_{i_1 \cdots i_p} \rightarrow \lambda^{p+1} A_{i_1 \cdots i_p},
\end{equation}
we see that (\ref{scaleaction}) is indeed scale invariant. This invariance is however broken when choosing a solution to the equation of motion (if $T \neq 0$) 
entirely analogous to the particle case.

\subsection{Super $p$-branes}
Let us now consider the supersymmetric extensions of the ideas above. For this we need to replace the scalar fields describing the embedding in the bosonic actions with 
\begin{equation}
Z^{M}(\xi)=(X^{\mu},\theta^{\alpha}),
\end{equation}
i.e. mappings from the world-volume of the $p$-brane to a superspace. The coordinates $X^{\mu}$ are the usual bosonic coordinates used above while $\theta^{\alpha}$ are fermionic coordinates. 
(For an introduction to supersymmetry, see for example \cite{bilal}) We also need the supervielbein $e{_M}^A$ where $M = \mu,\alpha$ are world-indices and $A = a,\alpha$ are tangent space indices. 
We can now create the most general supersymmetric one-forms
\begin{equation}
\Pi^A = Z^Me{_M}^A,
\end{equation}
which, due to how the supersymmetry generators acts on the superspace coordinates, explicitly becomes
\begin{equation}
\Pi^{\mu}=dX^{\mu}-i \bar{\theta} \Gamma^{\mu} d\theta, \hspace{1cm} \Pi^{\alpha} = d\theta^{\alpha},
\label{pullbackfields}
\end{equation}
where $\Gamma^{\mu}$ is spacetime Dirac matrices. The pullbacks of these forms to the world-volume becomes
\begin{equation}
\Pi_{i}^A=\partial_i Z^Me{_M}^A,
\end{equation}
and replacing the scalar fields in (\ref{nambugoto}) with the pullbacks gives the action
\begin{equation}
S=-T\int d^{p+1} \xi \sqrt{-\mbox{det}(\Pi_{i}^{\mu}\Pi_{j}^{\nu} \eta_{\mu \nu})},
\end{equation}
or, as usual, by introducing a world-volume density $V$
\begin{equation}
S={1 \over 2}\int d^{p+1} \xi [{1 \over V}\mbox{det}(\Pi_{i}^{\mu}\Pi_{j}^{\nu} \eta_{\mu \nu})-T^2V].
\label{superaction}
\end{equation}
But this is however not the whole story. For the action to be supersymmetric it must have an equal number of bosonic and fermionic degrees 
of freedom. For example, if $p = 2$ we have $11-3=8$ bosonic degrees of freedom (in 11-dimensional spacetime). Usually superspace is made up of 32 fermionic coordinates, hence 32 degrees of freedom. Going on shell reduces 
the number to 16, and often one uses $\kappa$-symmetry, a fermionic symmetry that reduces the degrees of freedom by a factor 2, to get to 8 fermionic degrees of freedom. 
Actually, demanding that $\kappa$-symmetry equalizes the number of bosonic and fermionic degrees of freedom puts constraints 
on in which dimensions the different $p$-branes can live in. All this requires an additional term to (\ref{superaction}), a so called Wess-Zumino term. 
We will not go into exactly how the Wess-Zumino term looks like here but just note that they lead to the appearance of a $p$-form topological 
charge that is central (i.e. commutes with all the generators) with respect to the global symmetry group of spacetime \cite{valskriven}. 
But in the same way as we in subsection \ssecref{scaleinvariance} interpreted the mass as the momentum in an extra dimension, the central charges have a natural interpretation 
as momentum extra dimensions. This suggests the possibility that a supersymmetrization of (\ref{scaleaction}) might 
provide some kind of "higher-dimensional" action that is manifestly supersymmetric, i.e. not in need of a Wess-Zumino term.
This turn out to be the case and the resulting action is
\begin{equation}
S=\int d^{p+1} \xi {1 \over 2V}\Big{[}\mbox{det}(\Pi_{i}^{\mu}\Pi_{j}^{\nu} \eta_{\mu \nu}) - (*F)^2\Big{]}.
\label{manifestsuperaction}
\end{equation}
Here $F$ is a supertranslation-invariant modified field strength for a world-volume $p$-form gauge potential $A$.
So, instead of using a Wess-Zumino term to make the action supersymmetric we have to put constraints on the potential $A$, i.e give it a non-trivial supersymmetric transformation 
that makes sure that (\ref{manifestsuperaction}) is supersymmetric.

\subsection{Modified field strengths}
As we mentioned in section \secref{maurer}, the Maurer-Cartan equations can be generalized to the case of $p$-forms. 
Hence, for a ($p+1$)-form $F$ we get
\begin{equation}
dF+h=0,
\label{freediff}
\end{equation}
where $h$ is a ($p+2$)-form constructed from the fields $\Pi^{\mu}$ and $\Pi^{\alpha}$. 
The important thing to note here is that if we want to write $h=db$, then $b$ can not be constructed just from $\Pi^{\mu}$ and $\Pi^{\alpha}$ but have to include $x^m$ and/or $\theta$ explicitly \cite{valskriven}.
Eq. (\ref{freediff}) can be solved by
\begin{equation}
F=dA-b,
\end{equation}
for a $p$-form $A$ and where $b$ is the potential for $h$. The modified field strength $F$ will now be supersymmetry invariant, i.e. 
can be used in an action of the form (\ref{manifestsuperaction}), if we ascribe the correct supersymmetry variation to $A$. 
This variation is of course dependent of $h$ and we will not try calculate it here, but just stress the point that $b$ can not be constructed from the fields in (\ref{pullbackfields}). 
\newline 
We have so far in this chapter only considered flat superspace, but the results can be generalized to curved space. In fact, the action remains that of (\ref{manifestsuperaction}) 
but the modified field strengths becomes of the type
\begin{equation}
F=dA-B,
\label{axnl}
\end{equation}
where $B$ is the potential of the supergravity ($p+2$)-form $H$. In this way we will get world-volume field strengths that couples to all 
the fields in the supergravity background. (In the next chapter world-volume fields and field strengths will be denoted with lower-case letters in opposite to background fields and field strengths to avoid confusion.)

\subsection{The final form of the action}
We will now continue with making some modifications to action (\ref{manifestsuperaction}). First we note that 
the independent world-volume scalar density $V$ give rise to the constraint
\begin{equation}
g-(*F)^2=0,
\eqnlab{laksoe}
\end{equation} 
where $g$ is the determinant of the metric. Any scalar density in front of the action would give the same constrain and hence we can define
\begin{equation}
{1 \over 2V}=-{\lambda \over \sqrt{-g}}.
\eqnlab{ertnml}
\end{equation}
We also define
\begin{equation}
*F = \sqrt{-g}*h,
\end{equation}
now using a lower-case letter for the field strength to indicate that it is a world-volume field strength. This altogether gives
\begin{equation}
S=\int d^{p+1} \xi \sqrt{-g} \lambda [1 + (*h)^2],
\end{equation}
which is the action we would use if there was only one world-volume field strength to deal with. But in our case 
every background tensor field of low enough rank to couple to the membrane will give rise to a field strength. 
Due to this we will use an action of the form
\begin{equation}
S=\int d^{p+1} \xi \sqrt{-g} \lambda [1 + \Phi + (*h)^2],
\eqnlab{dynamics_final_action}
\end{equation}
where $h$ now is the highest form field strength and $\Phi$ is a function of the other field strengths. The 
function $\Phi$ is yet unknown and we will use most of the next chapter to derive it. Actually, obtaining $\Phi$ 
is the main problem of the whole thesis.

\section{Electrically and magnetically charged branes}
\seclab{dynamics_charge}
In section \secref{branes} we mentioned the M2 brane, which is electrically coupled to $\hat C$, and the magnetically coupled M5 brane. 
To evolve this statement a little more we again turn to electromagnetic field theory. With the source equations given, the 
total electric or magnetic charge can be calculated by a surface integral surrounding the source, where the surface is infinitely big (Gauss's law).
Since the electrical charge is received by an integral over the dual of the field strength, (\ref{fieldstrengthdual}) 
gives the conserved electrical charge in 11 dimensions as
\begin{equation}
U=\int *\hat G = \int d\hat C_{(6)} - \half\hat G\we\hat C,
\end{equation}
where the integral of the 7-form integrand is over the boundary at infinity of an arbitrary infinite 8-dimensional 
spacelike subspace of $\hat D$ = 11 spacetime. This arbitrariness of the subspace indicates that $U$ in fact describes a 
whole set of conserved charges. The embedding of the 8-dimensional spacelike subspace into the 10-dimensional spatial 
hypersurface can be secified by a 2-form volume-element and accordingly the set of charges should properly be denoted by $U_{2}$ (where the 2 indicates a set of 2 charges, not the form of U).
The $\hat G \wedge \hat C$ term does however vanish for certain $p$-brane solutions and the $d\hat C_{(6)}$ term only give 
a contribution when the subspace is transverse to the (M)2-brane, leaving only one single charge\cite{supergrav_pbranes_lectures}.

In the same way one may find the magnetic charge by integrating over the field strength $\hat G$. But this time, by virtue of $d\hat G =0$, 
the charge should be conserved topologically. Because of the integrand being a 4-form, the corresponding integral
\begin{equation}
V=\int \hat G,
\end{equation}
is now being taken over the boundary at infinity of a 5-dimensional spacelike subspace. The embedding into the 10-dimensional 
hypersurface can be specified by a 5-form volume-element and hence the proper notation for $V$ is $V_{5}$ (i.e. a set of 5 charges). It is this 
magnetic charge that is carried by the solitonic M5-brane. But once again only one orientation of the subspace, the one transverse 
to the brane, gives a nonvanishing contribution to the charge.

The same reasoning holds for $p$-branes in reduced supergravity, i.e. the charges can be derived by integrating the field strengths, 
much in the same way as above. This is important to have in mind when we in the next chapter identify integration parameters 
from the equations of motion as membrane charges. For example, we have already deduced the relation $*F=TV$ and by using 
\eqnref{laksoe} and \eqnref{ertnml} we see that $*h$ is directly related to the tension (charge) as
\begin{equation}
\lambda *h =-2T.
\end{equation}
However, the number of world-volume fields are often more then one and the field strengths are typically of the type 
$f^r = da^r - B^r$. In the same way, $*h^r$ then becomes the set of charges with which the brane couples to the background field $B^r$, 
\cite{ref11}.

\section{The scalar fields as Goldstone modes}
It is clear that by using field strengths of the type $f=da-A$, i.e. field strengths where every background field 
has a world-volume counterpart, the brane dynamics will be covariant with respect to the global symmetries of the 
background supergravity to which the brane couples to. This is why we have put so much effort in trying to write the 
background fields and gauge transformations in an U-duality covariant way. The background coupling also indicates that
a background gauge transformation $\delta A= d\Lambda$, is accompanied by a shift in the world-volume potential, $\delta a= \Lambda$. 
This is directly related to the general nature of world-volume fields being Goldstone modes corresponding to
background "gauge-symmetries" \cite{goldstone_tensor_modes}. First of all, the theory have eight bosonic degrees of freedom. Any object placed in spacetime 
breaks translational symmetries and hence the scalar fields on the branes can be identified as Goldstone modes arising 
from the breaking of the symmetries. For example, the M2 brane in 11-dimensional spacetime breaks the translational 
symmetry in eight transverse directions giving eight Goldstone scalars. For the M5 brane on the other hand, there is only five 
transverse directions leaving three scalar fields to be realized on the brane. The same situation arises in our cases, 
i.e. a $D_1$- and a $D_2$-brane in 9- and 8-dimensional supergravity respectively. The $D_1$-brane has 7 transverse directions in 
a 9-dimensional background leaving 1 internal scalar field on the brane, while the number of transverse directions is 
5 for the $D_2$-brane in 8 dimensions which gives 3 internal scalar fields. Our formulation of the world-volume field strengths 
gives internal scalar fields that can be interpreted as these remaining Goldstone modes. However, the procedure often leads to a
situation where the number world-volume fields is larger than the number of physical Goldstone modes. Hence we have to limit the number 
of fields by using some self-duality condition. This puts constraints on $\Phi$ that we will use in our attempt to derive it.\\
\\

%*hep-th/9811145





%\chapter{The equations of motion}
\chlab{solutions}
\section{World-volume field strengths and the equations of motion}
Now it is finally time to write down the world-volume field strengths and calculate the equations of motion. As has been 
mentioned several times by now the field strengths will be of the form $da-A$, but we will soon see that extra terms need 
to be added in order to make them gauge invariant. Starting with a $D_1$-brane in 9-dimensional supergravity, 
the potentials $A^{1m}$, $A^2$, and $B_m$ 
%\footnote{All dilaton factors will be suppressed during these calculations, they can be reinstated at any time with exponent factors determined using $\kappa$ symmetry, c.f. \cite{}.}
will give rise to three field strengths according to
\begin{align}
\eqnlab{solution_9d_fieldstrengths}
\omega^{1m} &= d\phi^{1m}-A^{1m}, \nonumber \\
\omega^{2} &= d\phi^2 - A^2, \nonumber \\
f'_m &=da_m - B_m,
\end{align}
where $\phi^{1m}$ and $\phi^2$ are scalars and $a_m$ is a 1-form. It is easy to see that 
$\omega^{1m}$ and $\omega^{2}$ are gauge invariant if $\phi^{1m}$ and $\phi^2$ are invariant under the gauge transformations 
$\chi^{1m}$ and $\chi^2$ respectively. For $f'_m$ we get
\begin{align}
\delta f'_m &= d \delta a_m -\delta B_m = d \delta a_m - (d{\chi}_m + {\epsilon}_{mn}{\chi}^2 \wedge F^{1n}) \nonumber \\
&= d(\delta a_m -\chi_m) - \delta({\epsilon}_{mn}{\phi}^2 \wedge F^{1n}),
\end{align}
which implies $\delta a_m = \chi_m$ and that the field strength
\begin{equation}
\eqnlab{solution_9d_fm}
f'_m + {\epsilon}_{mn}{\phi}^2 \wedge F^{1n}= da_m - B_m + {\epsilon}_{mn}{\phi}^2 \wedge F^{1n}
\end{equation}
is gauge invariant. The corrected world volume 2-form field strength is thus 
\begin{equation}
\eqnlab{solution_9d_h}
f_m=da_m - B_m + {\epsilon}_{mn}{\phi}^2 \wedge F^{1n} + \bar f_m,
\end{equation}
where we have introduced an additional 2-form $\bar f_m(\omega^{1m},\omega^2)$, which is gauge invariant since it depends only on field strengths.
To be able to vary this term when deriving the equations of motion we let
% Ingen *\beta_m term kan skapas p.g.a bara en typ av index
\begin{align}
\bar f_m = \alpha_{mn}\omega^{1n} \wedge \omega^2
\end{align}
where $\alpha_{mn}$ is on the form $\alpha_1\W_{mn}+\alpha_2\epsilon_{mn}$ and $\alpha_1$ and $\alpha_2$ are scalar functions of $\omega^1$ and $\omega^2$, which seems to be the most general expression for $\bar f_m$ if we do not include dualities of the field strengths.
The significance of this term is not clear yet and although it may turn out to be meaningless we include it because we can.
We also note there to be an ambiguity related to redefinitions of the background potentials. 
The Bianchi identities for the background field strengths was previously found to be
\begin{align}
dF^{1m} = 0,\;\;\;\;\; dF^{2} = 0\;\;\;\;\;\mbox{ and } \;\;\;\;\;\; dH_{m} = -\epsilon_{mn}F^{1n}\we A^2.
\end{align}  
The first $2$ are closed forms, so $F$ can be written $F=dA$. The third is invariant under field redefinitions, e.g. $B_m = \tilde B_m + \beta\epsilon_{mn}A^{1n}\we A^2$, because $B_m$ enters the identity as $dH_m = d^2B_m + \cdots$.
The new potential $\tilde B_m$ must have a different gauge transformation
\begin{align}
\delta \tilde B_{m} &= d\tilde\chi_{m} + \lp\beta+1\rp\epsilon_{mn}\chi^2\we F^{1n} - \beta\epsilon_{mn}\chi^{1n}\we F^2,
\end{align}
leaving $H_m$ invariant.
Using $\tilde B_m$ to create our world volume field strengths as before, we find
\begin{align}
f_m &=da_m - \tilde B_m + \lp\beta+1\rp\epsilon_{mn}{\phi}^2 \wedge F^{1n} - \beta\epsilon_{mn}\phi^{1n}\we F^2
\end{align}
which cannot be taken back to the form in \eqnref{solution_9d_h} by a field redefinition of $a_m$.  
% (use $a_1=-a_2=-\beta$)
%\begin{align}
%a_m &= \tilde a_m + a_1\epsilon_{mn}A^{1n}\phi^2 + a_2\epsilon_{mn}\phi^{1n}A^2\nn\\
%da_m &= \tilde da_m + a_1\epsilon_{mn}A^{1n} (\omega^2 + A^2) + a_1\epsilon_{mn}F^{1n} \phi^2 + a_2\epsilon_{mn}A^2(\omega^{1n} + A^{1n}) + a_2\epsilon_{mn}\phi^{1n}F^2\nn\\
%&= \tilde da_m  - \beta\epsilon_{mn}A^{1n}\omega^2 + \beta\epsilon_{mn}A^2\omega^{1n} - \beta\epsilon_{mn}F^{1n} \phi^2 + \beta\epsilon_{mn}\phi^{1n}F^2\nn\\
%\end{align}
The $\beta$ terms will enter the equations of motion in a way very similar to a $\alpha_{mn}=\alpha\epsilon_{mn}$ term (proportional to $F$ when varied w.r.t. $\phi$).  
We will not consider terms of this type from now on and focus on artificially introduced terms such as the $\alpha$ term but we note that the two different types of term do not enter the equations of motion in the exact same way and thus some results found using parameters of $\alpha$ type may be incomplete or erroneous.

The first two lines in \eqnref{solution_9d_fieldstrengths} and \eqnref{solution_9d_h} thus gives the field strengths 
for a $D_1$-brane in 9-dimensional supergravity, obeying the following Bianchi identities 
\begin{align}
\eqnlab{solution_9d_bianchi}
d \omega^{1m}&=-F^{1m} \\
d \omega^{2}&=-F^2\\
d f_m &= - H_m + \epsilon_{mn}\lp\omega^{2}+2A^{2}\rp \wedge F^{1n} + d\bar f_m.
\end{align}
Doing the same calculations for a $D_2$-brane in 8-dimensional supergravity yields the field strengths
\begin{align}
\eqnlab{solution_8d_field_strengths}
\omega^{rm} &= d\phi^{rm} - A^{rm}, \nonumber \\
f_m &= da_m - B_m + {1 \over 2} \epsilon_{mnp} \epsilon_{rs} \phi^{rn} F^{sp}, \nonumber \\
h^r &= db^r - C^r - {1 \over 3} \phi^{rm}H_m - {2 \over 3}a_m \wedge F^{rm} + \bar h^r 
\end{align}
where $\bar h^r$ is a gauge invariant 3-form. We will expand it as
\begin{align}
\bar h^r = \alpha\ou{r}\od{s}\omega^{sm}\wedge f_m + \beta\ou{r}\od{s}\epsilon_{mnp}\W_{uv} \omega^{sm} \wedge \omega^{un} \wedge \omega^{vp} + *\gamma^r,
\end{align}
where $\alpha\ou{r}\od{s} = \alpha_1\delta^r_s + \alpha_2\epsilon\ou{r}\od{s}$, $\beta\ou{r}\od{s} = \beta_1\delta^r_s + \beta_2\epsilon\ou{r}\od{s}$ and $\alpha_1,\alpha_2,\beta_1$ and $\beta_2$ are constants. The last world volume scalar function $\gamma^r$ is dependent on $\omega^{rm}$ and $f_m$ and corresponds to the remaining gauge invariant 3-forms with one free $SL(2,\rr)$ index, e.g. forms like the $\alpha$ and $\beta$ terms, using $\alpha$ or $\beta$ as scalar functions of the field strengths or such terms with the $SL(3)$ indices contracted to arbitrary functions of the field strengths.
If we series expand $\gamma^r$ to third order in the fields $\omega^{rm}_\alpha$, $\lp*\omega^{rm}\rp^{\beta\alpha}$, $f_{\beta\alpha}^m$ and $\lp*f^m\rp^\alpha$ we find the following different nonzero components
\begin{align}
\eqnlab{solution_gamma_expand}
\gamma^r &= (\delta^r_{r'} + \epsilon\ou{r}\od{r'}) \omega^{r'm}_\alpha\lp \epsilon_{mnp}\varepsilon^{\alpha\beta\gamma}\omega^{sn}_\beta \omega^{p}_{s\gamma} + \lp*f_m\rp^\alpha + \epsilon_{mnp}\lp*f^n\rp_\beta f^{p\beta\alpha} \rp \nn\\
\end{align}
giving
\begin{align}
*\gamma^r =& \frac{1}{6}d\xi^\delta\we d\xi^\epsilon\we d\xi^\phi\varepsilon_{\delta\epsilon\phi}\gamma^r\\
% =& \frac{1}{6}d\xi^\delta\we d\xi^\epsilon\we d\xi^\phi\varepsilon_{\delta\epsilon\phi}\varepsilon^{\alpha\beta\gamma}\lp \epsilon_{mnp}\omega^{rm}_\alpha\omega^{sn}_\beta \omega^{p}_{s\gamma} + \frac{1}{2}\omega^{rm}_\alpha f_{m\gamma\beta}   + \frac{1}{2}\epsilon_{mnp}\omega^{rm}_{\alpha'} f^{n}_{\gamma\beta}f\ou{p}\od{\alpha}\ou{\alpha'}\rp\nn\\  
% =& -d\xi^\delta\we d\xi^\epsilon\we d\xi^\phi \lp \epsilon_{mnp}\omega^{rm}_{[\delta}\omega^{sn}_\epsilon \omega^{p}_{s\phi]} + \frac{1}{2}\omega^{rm}_{[\delta} f_{m\phi\epsilon]}   + \frac{1}{2}\epsilon_{mnp}\omega^{rm}_{\alpha} f^{n}_{\phi\epsilon}f\ou{p}\od{\delta}\ou{\alpha}\rp\nn\\  
=& -(\delta^r_{r'} + \epsilon\ou{r}\od{r'})\lp\epsilon_{mnp}\omega^{r'm}\we\omega^{sn}\we\omega^{p}_{s} + \omega^{r'm}\we f_{m} + \epsilon_{mnp} f^{n}\we d\xi^\beta f^{p}_{\beta\alpha}\omega^{r'm\alpha}\rp\nn  
\end{align}
i.e. the two first terms are already accounted for in the $\alpha$ and $\beta$ terms and thus $\gamma^r$ has order 3 of its lowest order term.


The field strengths are invariant under the gauge transformations
\begin{align}
\delta \phi^{rm} &= \chi^{rm}, \nonumber \\
\delta a_m &= \chi_m - {1 \over 2}\epsilon_{mnp}\epsilon_{rs}A^{rn}\chi^{sp}, \nonumber \\
\delta b^r &= \chi^r - {2 \over 3}A^{rm} \wedge \chi_m + {1 \over 3}B_m \chi^{rm} + {1 \over 6}\epsilon_{mnp}\epsilon_{st}A^{rm} \wedge A^{sn} \chi^{tp},
\end{align}
and the Bianchi identities becomes
\begin{align}
\eqnlab{solution_8d_bianchi}
&d \omega^{rm}=-F^{rm}, \\
&d f_m=-H_m + {1 \over 2} \epsilon_{mnp} \epsilon_{rs} F^{rn} \wedge \omega^{sp}.
\end{align}

Finally for a $D_3$-brane in 7-dimensional supergravity we get the world-volume field strengths
\begin{align}
\omega^{rs} &= d\phi^{rs} - A^{rs}, \nonumber \\
f_v &= da_v - B_v - {1 \over 2} \epsilon_{rstuv} \phi^{rs} F^{tu}, \nonumber \\
h^r &= db^r - C^r - \phi^{rs}H_s - a_s \wedge F^{rs}, 
\end{align}
which are invariant under the transformations
\begin{align}
\delta \phi^{rs} &= \chi^{rs}, \nonumber \\
\delta a_v &= \chi_v + {1 \over 2}\epsilon_{rstuv}A^{rs}\chi^{tu}, \nonumber \\
\delta b^{r'} &= \chi^{r'} - A^{r's} \wedge \chi_s + B_s \chi^{r's} - {1 \over 2}\epsilon_{rstuv}A^{r'r} \wedge A^{st} \chi^{uv},
\end{align}
We will not consider this case any more in this thesis.

\subsubsection{Variations of the actions}
Turning the attention to the action derived in chapter 4 we see that for a $D_1$-brane we get
\begin{equation}
S=\int d^2 \xi \sqrt{-g} \lambda [1 + \Phi(\omega^{1m},\omega^{2}) - *f_m*f_n\M^{mn}].
\end{equation} 
We will now continue with deriving the equations of motion from this action. Since we do not know the form of $\Phi$ 
yet, the equations will be implicit to begin with. Especially easy is the e.o.m. coming from $\lambda$,
\begin{equation}
1 + \Phi - *f_m*f_n\M^{mn}=0,
\end{equation}
which of course cannot be inserted back into the action. Varying with respect to $a_m$ gives
\begin{align}
\delta_{a_m}S=& \int * { \delta \Lagr\lp a_{m}\rp \over \delta (a_m) } \wedge \delta (a_m)=-\int 2\lambda *{\partial *f_{m'} \over \partial da_m} *f_n \M^{m'n} \wedge d\delta a_m \nonumber \\
=& -\int 2\lambda *f_n \M^{mn} \wedge d\delta a_m =0\nonumber \\
\Rightarrow& \hspace{0.3cm} d \Big{[}\lambda \M^{mn}*f_n\Big{]}=0,
\end{align}
where we have used \eqnref{conven_hodge_variation} and \eqnref{conven_deltaS}. Using the same formulas also gives
\begin{align}
\delta_{\phi^{1m}}S =& \int * { \delta \Lagr\lp \phi^{1m}\rp \over \delta (\phi^{1m}) } \wedge \delta \phi^{1m} \nonumber \\
=& \int (\lambda * {\partial \Phi \over \partial \omega^{1m}} - 2 \lambda * {\partial *f_{m'} \over \partial \omega^{1m}} *f_n \M^{m'n}) \wedge d \delta \phi^{1m} \nonumber \\
=&\int [\lambda *j_{1m} + 2 \lambda {\{} \alpha_{mm'} \omega^2 - *{\partial \alpha_{m'n} \over \partial \omega^{1m}}*(\omega^{1n} \wedge \omega^2){\}}*f_{n'} \M^{m'n'}] \nonumber \\
& \wedge d\delta \phi^{1m}=0 \nonumber \\
\Rightarrow \hspace{0.3cm}d&\Big{[}\lambda *j_{1m} + 2 \lambda {\{} \alpha_{mm'} \omega^2 - {\partial \alpha_{m'n} \over \partial \omega^{1m}}*(\omega^{1n} \wedge \omega^2){\}}*f_{n'} \M^{m'n'}\Big{]}=0,
\end{align}
where we have defined
\begin{equation}
{\partial \Phi \over \partial \omega^{1m}}=j_{1m}.
\end{equation}
Varying with respect to $\phi^2$ gives the last e.o.m. as
\begin{align}
\delta_{\phi^{2}}S &= \nonumber \\ 
=& \int * { \delta \Lagr\lp \phi^{2} \rp \over \delta (\phi^{2}) } \wedge \delta \phi^{2} =\int (\lambda * { \partial \Phi \over \partial \omega^2} -2\lambda*{\partial *f_m \over \partial \omega^2 } *f_n \M^{mn}) \wedge d \delta \phi^2 \nonumber \\
& -\int 2\lambda*{\partial *f_m \over \partial \phi^2 }*f_n \M^{mn} \wedge \delta \phi^2 \nonumber \\
=& \int[\lambda*j_2 - 2\lambda {\{}\alpha_{mn} \omega^{1n} + *{\partial \alpha_{mn} \over \partial \omega^2 }*(\omega^{1n} \wedge \omega^2){\}}*f_{n'}\M^{mn'}] \wedge d \delta \phi^2 \nonumber \\
&+\int 2\lambda \epsilon_{mn} F^{1n}*f_{n'}\M^{mn'} \wedge \delta \phi^2=0 \nonumber \\
\Rightarrow& \hspace{0.3cm} d\Big{[}\lambda*j_2 - 2\lambda {\{}\alpha_{mn} \omega^{1n} + *{\partial \alpha_{mn} \over \partial \omega^2 }*(\omega^{1n} \wedge \omega^2){\}}*f_{n'}\M^{mn'}\Big{]} \nonumber \\
& \hspace{0.3cm} -2\lambda \epsilon_{mn} F^{1n}*f_{n'}\M^{mn'}=0,
\end{align}
with
\begin{equation}
{\partial \Phi \over \partial \omega^{2}} =j_2.
\end{equation}
Repeating all this for the $D_2$-brane we get the following equations of motion
%\begin{align}
%\lambda:& \hspace{0.2cm} 1+\Phi-*h^r*h^s\W_{rs}=0 \\
%b^r:& \hspace{0.2cm} d \Big{[}\lambda \W_{rs}*h^s\Big{]}=0 \\
%a_m:& \hspace{0.2cm} d \Big{[}\lambda*k^m+2\lambda \Big{\{} \alpha\ou{r}\od{s} \omega^{sm} + *{\partial \alpha\ou{r}\od{s} \over \partial f_m} *(\omega^{sn} \wedge f_n) - *\frac{\partial\gamma^r}{\partial f_m}\nonumber \\
%& \hspace{0.2cm} + *{\partial \beta\ou{r}\od{t} \over \partial f_m} *(\epsilon_{m'np}\W_{uv} \omega^{tm'} \wedge \omega^{un} \wedge \omega^{vp}) \Big{\}} *h^s \W_{rs} \Big{]} \nonumber \\
%& \hspace{0.2cm} -{4 \over 3} \lambda F^{rm} *h^s \W_{rs} =0 \\
%\phi^{rm}:& \hspace{0.2cm} d \Big{[}\lambda *j_{rm} + 2\lambda \Big{\{} \alpha\ou{r'}\od{r} f_m - *{\partial \alpha\ou{r'}\od{t} \over \partial \omega^{rm}}*(\omega^{tn} \wedge f_n)  \nonumber \\
%& \hspace{0.2cm} - *{\partial \beta\ou{r'}\od{t} \over \partial \omega^{rm}} *(\epsilon_{m'np}\W_{uv} \omega^{tm'} \wedge \omega^{un} \wedge \omega^{vp}) + *\frac{\partial\gamma^{r'}}{\partial\omega^{rm}}\nonumber \\
%& \hspace{0.2cm} + \epsilon_{mnp}(\beta\ou{r'}\od{r}\W_{ut}+ 2\beta\ou{r'}\od{t}\W_{ur})\omega^{un} \wedge \omega^{tp} \Big{\}}*h^{s} \W_{r's} \Big{]} \nonumber \\
%& \hspace{0.2cm} + \lambda \Big{[}{1 \over 2}\epsilon_{mnp}\epsilon_{rs} *k^n \wedge F^{sp} +{2 \over 3} H_m*h^s\W_{sr} \nonumber \\
%& \hspace{0.2cm} + \epsilon_{mnp}\epsilon_{rs} *h^t \W_{r't} F^{sp} \wedge \Big{\{} \alpha\ou{r'}\od{u} \omega^{un} - *{\partial \alpha\ou{r'}\od{u} \over \partial f_n}*(\omega^{um'} \wedge f_{m'}) \nonumber \\
%& \hspace{0.2cm} - *{\partial \beta\ou{r'}\od{t'} \over \partial f_n} *(\epsilon_{m'n'p'}\W_{uv} \omega^{t'm'} \wedge \omega^{un'} \wedge \omega^{vp'}) \Big{\}}\Big{]}=0,
%\eqnlab{solution_8d_eom}
%\end{align}
\begin{align}
\eqnlab{solution_8d_eom}
\lambda:& \hspace{0.2cm} 1+\Phi-*h^r*h^s\W_{rs}=0 \\
b^r:& \hspace{0.2cm} d \Big{[}\lambda \W_{rs}*h^s\Big{]}=0 \\
a_m:& \hspace{0.2cm} d \Big{[}\lambda*k^m - 2\lambda*\frac{\partial *\bar h^r}{\partial f_m}*h^s \W_{rs} \Big{]} -{4 \over 3} \lambda F^{rm} *h^s \W_{rs} =0 \\
\phi^{rm}:& \hspace{0.2cm} d \Big{[}\lambda *j_{rm} - 2\lambda *\frac{\partial*\bar h^{r'}}{\partial\omega^{rm}}*h^{s} \W_{r's} \Big{]} + \lambda \Big{[}{1 \over 2}\epsilon_{mnp}\epsilon_{rs} *k^n \wedge F^{sp}\nn\\
& \hspace{0.2cm} +{2 \over 3} H_m*h^s\W_{sr} - \epsilon_{mnp}\epsilon_{rs} *h^t \W_{r't} F^{sp} \wedge *{\partial *\bar h^{r'} \over \partial f_n}\Big{]}=0,
\end{align}
where
\begin{equation}
k^{m} = {\partial \Phi \over \partial f_m}, \hspace{0.5cm} j_{rm} = {\partial \Phi \over \partial \omega^{rm}}.
\end{equation}

We can use the Bianchi identity \eqnref{solution_8d_bianchi} to move in all terms under the exterior derivative in the equation of motion for $a^m$
%\begin{align}
%d& \Big{[}\lambda*k^m +2\lambda \Big{\{} \alpha\ou{r}\od{s}\omega^{sm}+\frac{2}{3}\omega^{rm} + *{\partial \alpha\ou{r}\od{s} \over \partial f_m} *(\omega^{sn} \wedge f_n) \\
%& + *{\partial \beta\ou{r}\od{t} \over \partial f_m} *(\epsilon_{m'np}\W_{uv} \omega^{tm'} \wedge \omega^{un} \wedge \omega^{vp}) - *\frac{\partial\gamma^r}{\partial f_m}
%\Big{\}} *h^s \W_{rs} \Big{]}=0 \nn
%\eqnlab{solution_8d_eom_a}
%\end{align}
\begin{align}
\eqnlab{solution_8d_eom_a}
d& \Big{[}\lambda*k^m +2\lambda \Big{\{} \frac{2}{3}\omega^{rm} - *\frac{\partial*\bar h^r}{\partial f_m} \Big{\}} *h^s \W_{rs} \Big{]}=0 
\end{align}
If we act with an external derivative on the equation of motion for $\phi^{rm}$, use the equation of motion for $a^m$ to eliminate the $k^n$ term, use the Bianchi identities for $F^{rm}$, $H_m$ and $\omega^{rm}$ and at last \eqnref{conven_2d_epsilon_rel} we get
%\begin{align}
%d&\Big{[} {\lambda \over 2}\epsilon_{mnp}\epsilon_{rs} *k^n \wedge F^{sp} +\lambda{2 \over 3} H_m*h^s\W_{sr} \nn \\
%& +\lambda\epsilon_{mnp}\epsilon_{rs} *h^t \W_{r't} F^{sp} \wedge \Big{\{} \alpha\ou{r'}\od{s'} \omega^{s'n} - *{\partial \alpha\ou{r'}\od{s'} \over \partial f_n}*(\omega^{s'm'} \wedge f_{m'}) \nn \\
%& - *{\partial \beta\ou{r'}\od{s'} \over \partial f_n} *(\epsilon_{m'n'p'}\W_{uv} \omega^{s'm'} \wedge \omega^{un'} \wedge \omega^{vp'}) + *\frac{\partial\gamma^{r'}}{\partial f_n}\Big{\}}\Big{]}\nn\\
%& = d\Big{[} 
%\lambda{2 \over 3} H_m*h^s\W_{sr} \nonumber \\
%& +\lambda\epsilon_{mnp}\epsilon_{rs} *h^t \W_{r't} F^{sp} \wedge \Big{\{} -\frac{2}{3}\omega^{rn} - 2*{\partial \alpha\ou{r'}\od{s'} \over \partial f_n}*(\omega^{s'm'} \wedge f_{m'}) \nn\\
%& - 2*{\partial\ou{r'}\od{s'} \beta \over \partial f_n} *(\epsilon_{m'n'p'}\W_{uv} \omega^{s'm'} \wedge \omega^{un'} \wedge \omega^{vp'}) + 2*\frac{\partial\gamma^{r'}}{\partial f_n}\Big{\}}\Big{]}\nn\\
%& = -2\lambda\epsilon_{mnp}\epsilon_{rs} *h^t \W_{r't}F^{sp} \wedge d\Big{[}  *{\partial \alpha\ou{r'}\od{s'} \over \partial f_n}*(\omega^{s'm'} \wedge f_{m'}) \nonumber \\
%& + *{\partial \beta\ou{r'}\od{s'} \over \partial f_n} *(\epsilon_{m'n'p'}\W_{uv} \omega^{s'm'} \wedge \omega^{un'} \wedge \omega^{vp'}) -*\frac{\partial\gamma^{r'}}{\partial f_n}\Big{]} = 0,
%\eqnlab{solution_abc_constraint}
%\end{align}
\begin{align}
d&\Big{[} {\lambda \over 2}\epsilon_{mnp}\epsilon_{rs} *k^n \wedge F^{sp} +\lambda{2 \over 3} H_m*h^s\W_{sr} - \lambda\epsilon_{mnp}\epsilon_{rs} *h^t \W_{r't} F^{sp} \wedge *\frac{\partial*\bar h^{r'}}{\partial f_n}\Big{]}\nn\\
&= d\Big{[} \lambda{2 \over 3} H_m*h^s\W_{sr} - \frac{2}{3}\lambda\epsilon_{mnp}\epsilon_{rs} *h^t \W_{r't} F^{sp} \wedge\omega^{r'n}\Big{]} = 0,
\end{align}
meaning we can put all terms under an external derivative, making the equations integrable for general $\bar h^r$.
So, the equation of motion for $\phi^{rm}$ can be written 
\begin{align}
\eqnlab{solution_8d_eom_phi}
d&\Big{[}\lambda *j_{rm} - \lambda{2 \over 3}f_m *h^{s} \W_{rs} - 2\lambda*\frac{\partial*\bar h^{r'}}{\partial\omega^{rm}}*h^{s} \W_{r's}\nn\\
& + \lambda\frac{1}{3}*h^u\epsilon_{mnp}\W_{tu}\epsilon_{rs} \omega^{sn}\we\omega^{tp} \Big{]}=0,
\end{align}
where we have omitted a closed form $\Gamma^m$ coming from the integration of \eqnref{solution_8d_eom_a}, which will be included in the next section.

We will now try to solve the previously derived equations of motion for some particular cases.
We begin with the 8-dimensional $D_2$-membrane, because we can compare the result to the results found in \cite{artikeln,pioline}. 

\section{Duality equations of the $d=8$ $D2$ case}
The second equation of motion tells us that the 2 scalars $p_r = \lambda \W_{rs}*h^s$ are constants. These has been identified with charges earlier in section \secref{dynamics_charge}. 
Together with the first equation of motion we get the following relations
\begin{align}
\eqnlab{solution_lambda_identities}
*h^r &= \frac{1}{\lambda}\W^{rs}p_s = \frac{1}{\lambda}p^r\nn\\
\frac{1}{\lambda} &= \frac{|*h|}{|p|}, \hspace{1cm}\mbox{where }|*h|=\sqrt{*h^r\W_{rs}*h^s}\mbox{, }|p| = {\sqrt{p_mp^m}}\nn\\
|*h| &= \sqrt{1+\Phi}
\end{align}
which allows us to rewrite all $\lambda$ and $*h$ dependence in terms of $\Phi$ and $p$ in the third and fourth equations.
The equation of motion \eqnref{solution_8d_eom_a} for $a^m$ becomes (we will from now on insert $\bar h$ in terms of $\alpha$ and $\beta$ in the equations and use $\gamma^r=0$)  
\begin{align}
d& \Big{[}\frac{|p|}{\sqrt{1+\Phi}}*k^m + 2\lp\alpha_1 +\frac{2}{3}\rp\ou{r}\od{s} \omega^{sm} p_r + 2\alpha_2p^r\epsilon_{rs}\omega^{sm}\Big{]} = 0.
\end{align}
Next we define the unit vectors
\begin{align}
\hat p_\parallel &= \hat p = \frac{p_r}{|p|}\\ 
\hat p_\perp &= \frac{p^s\epsilon_{sr}}{|p|} 
\end{align}
forming an orthonormal $SL(2,\rr)$ basis since $\hat p_\perp\cdot\hat p_\parallel = p^s\epsilon_{sr}p^r/|p|^2 = 0$. 
We will also use the shorthand notation for the contraction of $\hat p_\parallel$ and $\hat p_\perp$ with an $SL(2,\rr)$ tensor $T^r$ as
\begin{align}
T_\parallel &= \hat p_\parallel\cdot T = p_rT^r/|p|\\
T_\perp &= \hat p_\perp\cdot T = p^s\epsilon_{sr}T^r/|p|.
\end{align}
Integrate the equations of motion to get
\begin{align}
\eqnlab{solution_8d_km_orig}
*k^m = \bigg[\Gamma^m - 2\lp\alpha_1+\frac{2}{3}\rp\omega_\parallel^m - 2\alpha_2\omega_\perp^m\bigg]\sqrt{1+\Phi}
\end{align}
where $\Gamma^m$ is a closed 1-form such that $d\Gamma^m = 0$.
The equation of motion for $\phi^{rm}$ becomes (including $\Gamma^m$) 
\begin{align}
d \Big{[}&\frac{|p|}{\sqrt{1+\Phi}}*j_{rm} + 2\alpha\ou{r'}\od{r} f_m p_{r'} + 2 \epsilon_{mnp}(\beta\ou{s}\od{r}\W_{ut} + 2\beta\ou{s}\od{t}\W_{ur})\omega^{un} \wedge \omega^{tp}p_{s}\Big{]}\nn\\
& + \half\epsilon_{mnp}\epsilon_{rs} *k^n \wedge F^{sp} +\frac{2}{3} H_m p_r + \alpha\ou{t}\od{t'} \epsilon_{mnp}\epsilon_{rs} p_{t} F^{sp} \wedge \omega^{t'n}\nn\\
=& d \Big{[}\frac{|p|}{\sqrt{1+\Phi}}*j_{rm} + 2\alpha\ou{r'}\od{r} f_m p_{r'} + 2 \epsilon_{mnp}(\beta\ou{s}\od{r}\W_{ut} + 2\beta\ou{s}\od{t}\W_{ur})\omega^{un} \wedge \omega^{tp}p_{s}\Big{]}\nn\\
& + \epsilon_{mnp}\epsilon_{rs} \Big[\frac{\Gamma^n}{2} -\frac{2}{3} p_t\omega^{tn}\Big] \wedge F^{sp} +\frac{2}{3}\lp -df_m + \frac{1}{2}\epsilon_{mnp}\epsilon_{ts}\omega^{sp}\wedge F^{tn}\rp p_r \nn\\
=& d \Big{[}\frac{|p|}{\sqrt{1+\Phi}}*j_{rm} + 2\lp\alpha_1-\frac{1}{3}\rp f_m p_r + 2\alpha_2f_m p^s\epsilon_{sr} - \half\epsilon_{mnp}\epsilon_{rs}\Gamma^n\wedge \omega^{sp}\nn\\
& + \epsilon_{mnp}\omega^{sn} \wedge \omega^{tp}\Big\{ 2\beta_1\W_{st}p_r + 4\beta_1\W_{sr}p_{t} + \lp -2\beta_2 + \frac{1}{3}\rp \epsilon_{rs}p_t\nn\\
&\hspace{3cm} + 6\beta_2\epsilon_{ut}p^{u}\W_{sr}\Big\}\Big{]} = 0
\end{align}
where we have used \eqnref{solution_8d_km_orig} to replace $*k^m$, the Bianchi identities \eqnref{solution_8d_bianchi} and equation $\eqnref{conven_2d_epsilon_rel}$ to move $SL(2,\rr)$-indices.
Integration and hodge dualisation gives
\begin{align}
\eqnlab{solution_8d_jrm_orig}
j_{rm} =& \bigg\{ 
2\lp\alpha_1-\frac{1}{3}\rp *f_m \hat p_{\parallel r} + 2\alpha_2*f_m \hat p_{\perp r} - \frac{1}{2|p|}\epsilon_{mnp}\epsilon_{rs}*\lp\Gamma^n\wedge \omega^{sp}\rp\nn\\
&+ \frac{1}{|p|}*\Delta_{rm} + \epsilon_{mnp}*\lp\omega^{sn} \wedge \omega^{tp}\rp\bigg[ 2\beta_1\W_{st}\hat p_{\parallel r} + 4\beta_1\W_{sr}\hat p_{\parallel t}\nn\\
&\hspace{2cm} + \lp -2\beta_2 + \frac{1}{3}\rp \epsilon_{rs}\hat p_{\parallel t} + 6\beta_2\W_{sr}\hat p_{\perp t}\bigg]\bigg\}\sqrt{1+\Phi}
\end{align}
where $\Delta_{rm}$ is a 2-form such that $d\Delta_{rm} = 0$.

\subsection{Series expansion of the duality equations}
\sseclab{solution_general_serie}
We will later see that we can create closed forms $\Gamma$ and $\Delta$ by imposing constraints on the background fields.
For now we will let them be zero and make a general expansion of the duality equations \eqnref{solution_8d_km_orig} and \eqnref{solution_8d_jrm_orig} to second order.
The $\gamma^r$ parameter function which was previously removed will not enter to these low orders\footnote{Actually, according to \eqnref{solution_gamma_expand} there is one third order term which will be of order 2 when varied w.r.t. $\omega$ and $f$ but its character is different from the other terms entering the equations, meaning it must most likely be zero anyway (alternatively we must use a different ansatz for $\Phi$ including terms of this type).}.
The first thing we note about the equations \eqnref{solution_8d_km_orig} and \eqnref{solution_8d_jrm_orig} is that if we multiply the latter equation with one of the two charge vectors $\hat p_\parallel$ or $\hat p_\perp$, all of the $\omega$:s, in the equations will be contracted on the forms $\omega_\parallel^m$, which is the projection of $\omega^m$ in the $\hat p$ direction or $\omega_\perp^m$, which is the projection of $\omega^m$ in the direction orthogonal to $\hat p$.
To rewrite $*(\omega\we\omega)$ in terms of $*(\omega_\parallel\we\omega_\parallel)$ and $*(\omega_\perp\we\omega_\perp)$ you can use the Pythagorean relation 
\begin{align}
\eqnlab{solution_pythgoras}
\omega^{n}_{\perp\beta}\omega^{p}_{\perp\gamma} = \hat p^{r'}\epsilon_{r'r}\omega^{rn}_{\beta}p_{s'}\epsilon^{s's}\omega^{p}_{\gamma s} = \omega^{rn}_{\beta}\omega^{p}_{\gamma r} - \omega^{n}_{\parallel\beta}\omega^{p}_{\parallel\gamma}.  
\end{align}
Hence, we will decompose the equation $e_r=\eqnref{solution_8d_jrm_orig}$ as 
\begin{align}
e_r = (e\cdot\hat p_\parallel)\hat p_\parallel + (e\cdot\hat p_\perp)\hat p_\perp,   
\end{align}  
which will give the 2 $SL(2,\rr)$ scalar equations $(e\cdot\hat p_\parallel)$ and $(e\cdot\hat p_\perp)$ (these must be fulfilled independent of each other since the multiplying vectors $\pup{{}}$ and $\puo{{}}$ points in different directions).
First we do the projection of $e_r$ in the $\hat p_\parallel$ direction
\begin{align}
\eqnlab{solution_8d_jrm_parallel}
e\cdot\hat p_\parallel &= 
\pup{r}j_{rm} = \bigg\{ 
2\lp\alpha_1-\frac{1}{3}\rp *f_m + \epsilon_{mnp}\bigg[ 2\beta_1*\lp\omega^{n}_\perp \wedge \omega^{p}_\perp\rp\nn\\
& + 6\beta_1*\lp\omega_\parallel^{n} \wedge \omega_\parallel^{p}\rp + \lp 4\beta_2 + \frac{1}{3}\rp*\lp\omega_\perp^{n} \wedge \omega_\parallel^{p}\rp\bigg]\bigg\}\sqrt{1+\Phi}
\end{align}
and then the projection of $e_r$ in the $\hat p_\perp$ direction
\begin{align}
\eqnlab{solution_8d_jrm_perp}
e\cdot\hat p_\perp &= 
\puo{r}j_{rm} = \bigg\{ 
2\alpha_2*f_m + \epsilon_{mnp}\bigg[ 4\beta_1*\lp\omega_\perp^{n} \wedge \omega_\parallel^{p}\rp\nn\\
& - \lp -2\beta_2 + \frac{1}{3}\rp*\lp\omega_\parallel^{n} \wedge \omega_\parallel^{p}\rp + 6\beta_2*\lp\omega_\perp^{n} \wedge \omega_\perp^{p}\rp\bigg]\bigg\}\sqrt{1+\Phi}
\end{align} 
Now we will series expand $\Phi$ to third order in $\omega$ and $*f$
\begin{align}
\Phi =& a_1\omega_{\alpha rm}\omega^{\alpha rm} + a_2 *f_{\alpha m}*f^{\alpha m}\nn\\
&+ a_3\varepsilon^{\alpha\beta\gamma}\epsilon_{mnp}\omega^{rm}_\alpha\omega^{n}_{\beta r}*f^{p}_\gamma + a_4\varepsilon^{\alpha\beta\gamma}\epsilon_{mnp}*f^{m}_\alpha*f^{n}_\beta*f^{p}_\gamma,
\end{align}
where $a_i$ are constant expansion coefficients. Note that we have omitted the constant term, we expect this to be zero to get the duality relations on the manifested form.
The variations becomes
\begin{align}
\frac{\partial\Phi}{\partial*f^m_\alpha} &= 2a_2 *f_m^\alpha + a_3\varepsilon^{\alpha\beta\gamma}\epsilon_{mnp}\omega^{rn}_{\beta}\omega^{p}_{\gamma r} + 3a_4\varepsilon^{\alpha\beta\gamma}\epsilon_{mnp}*f^{n}_\beta*f^{p}_\gamma\nn\\
\frac{\partial\Phi}{\partial\omega^{rm}_\alpha} &= 2a_1\omega^\alpha_{rm} + 2a_3\varepsilon^{\alpha\beta\gamma}\epsilon_{mnp}\omega^{n}_{\beta r}*f^{p}_\gamma,
\end{align}
giving (together with $\sqrt{1+\Phi}\approx 1 + \Ordo(\omega^2,*f^2)$) the equations \eqnref{solution_8d_km_orig}, \eqnref{solution_8d_jrm_parallel} and \eqnref{solution_8d_jrm_perp} to second order in the fields (with all terms moved to the same side the equality sign) 
\begin{align}
*k_m^\alpha: & 2a_2 *f_m^\alpha + a_3\varepsilon^{\alpha\beta\gamma}\epsilon_{mnp}\lp\omega^{n}_{\perp\beta}\omega^{p}_{\perp\gamma} + \omega^{n}_{\parallel\beta}\omega^{p}_{\parallel\gamma}\rp + 3a_4\varepsilon^{\alpha\beta\gamma}\epsilon_{mnp}*f^{n}_\beta*f^{p}_\gamma\nn\\
& - 2\lp\alpha_1+\frac{2}{3}\rp\omega_{\parallel}^{m\alpha} - 2\alpha_2\omega_{\perp}^{m\alpha} = 0\nn\\
%
\pup{r}j_{rm}^\alpha: & -2a_1\omega^\alpha_{\parallel m} - 2a_3\varepsilon^{\alpha\beta\gamma}\epsilon_{mnp}\omega^{n}_{\parallel\beta}*f^{p}_\gamma + 2\lp\alpha_1-\frac{1}{3}\rp *f_m^\alpha\nn\\
& + \varepsilon^{\alpha\beta\gamma}\epsilon_{mnp}\bigg[ 2\beta_1\omega^{n}_{\perp\beta}\omega^{p}_{\perp\gamma} + 6\beta_1\omega_{\parallel\beta}^{n}\omega_{\parallel\gamma}^{p} + \lp 4\beta_2 + \frac{1}{3}\rp\omega_{\perp\beta}^{n}\omega_{\parallel\gamma}^{p}\bigg] = 0\nn\\
%
\puo{r}j_{rm}^\alpha: & -2a_1\omega^\alpha_{\perp m} - 2a_3\varepsilon^{\alpha\beta\gamma}\epsilon_{mnp}\omega^{n}_{\perp\beta}*f^{p}_\gamma +2\alpha_2*f_m^\alpha\nn\\
& + \varepsilon^{\alpha\beta\gamma}\epsilon_{mnp}\bigg[ 4\beta_1\omega_{\perp\beta}^{n}\omega_{\parallel\gamma}^{p} - \lp -2\beta_2 + \frac{1}{3}\rp\omega_{\parallel\beta}^{n}\omega_{\parallel\gamma}^{p} + 6\beta_2\omega_{\perp\beta}^{n}\omega_{\perp\gamma}^{p}\bigg] = 0
%
\end{align}
If we read off the coefficients to each order in $\omega_\parallel$, $\omega_\perp$ and $*f$ it is easy to see that the equations are inconsistent unless we have some relation between the fields entering them. 
We let $*f = *f(\omega_\parallel,\omega_\perp)$ and expand it to second order in these fields
\begin{align}
*f_m^\alpha = b_1\omega^\alpha_{\parallel m} + b_2\omega^\alpha_{\perp m} +\epsilon_{mnp}\varepsilon^{\alpha\beta\gamma}\lp b_3\omega^{n}_{\perp\beta}\omega^{p}_{\perp\gamma} + b_4\omega^{n}_{\perp\beta}\omega^{p}_{\parallel\gamma} + b_5\omega^{n}_{\parallel\beta}\omega^{p}_{\parallel\gamma} \rp
\end{align}
where $b_i$ are expansion coefficients with a possible $p$-dependence.
The expanded duality equations now becomes
\begin{align}
*k_m^\alpha: & 2\lp a_2b_1 - \alpha_1 -\frac{2}{3} \rp\omega^\alpha_{\parallel m} + 2\lp a_2b_2 - \alpha_2\rp\omega^\alpha_{\perp m}\nn\\
& +\epsilon_{mnp}\varepsilon^{\alpha\beta\gamma}\Big[ \lp 2a_2b_3 + 3a_4b_2^2 + a_3\rp\omega^{n}_{\perp\beta}\omega^{p}_{\perp\gamma}\nn\\
& + \lp 2a_2b_4 + 6a_4b_1b_2\rp\omega^{n}_{\perp\beta}\omega^{p}_{\parallel\gamma} + \lp 2a_2b_5 + 3a_4b_1^2 + a_3\rp\omega^{n}_{\parallel\beta}\omega^{p}_{\parallel\gamma} \Big] = 0\nn\\
%
\pup{r}j_{rm}^\alpha: & 2\lp -a_1 + \alpha_1b_1 - \frac{1}{3}b_1\rp\omega^\alpha_{\parallel m} + 2\lp\alpha_1-\frac{1}{3}\rp b_2\omega^\alpha_{\perp m} \nn\\
& +\epsilon_{mnp}\varepsilon^{\alpha\beta\gamma}\Big[ \lp 2\lp\alpha_1-\frac{1}{3}\rp b_3 + 2\beta_1\rp\omega^{n}_{\perp\beta}\omega^{p}_{\perp\gamma} \nn\\
& + \lp -2a_3b_2 + 2\lp\alpha_1-\frac{1}{3}\rp b_4 + 4\beta_2 + \frac{1}{3}\rp\omega^{n}_{\perp\beta}\omega^{p}_{\parallel\gamma}\nn\\
& + \lp -2a_3b_1 + 2\lp\alpha_1-\frac{1}{3}\rp b_5 + 6\beta_1\rp\omega^{n}_{\parallel\beta}\omega^{p}_{\parallel\gamma} \Big] = 0\nn\\
%
\intertext{}
\puo{r}&j_{rm}^\alpha: 0 = 2\alpha_2b_1\omega^\alpha_{\parallel m} + 2\lp -a_1 + \alpha_2b_2\rp\omega^\alpha_{\perp m} \\
& +\epsilon_{mnp}\varepsilon^{\alpha\beta\gamma}\Big[ \lp - 2a_3b_2 + 2\alpha_2b_3 + 6\beta_2\rp\omega^{n}_{\perp\beta}\omega^{p}_{\perp\gamma}\nn\\
& + \lp - 2a_3b_1 + 2\alpha_2b_4 + 4\beta_1\rp\omega^{n}_{\perp\beta}\omega^{p}_{\parallel\gamma}  + \lp 2\alpha_2b_5 + 2\beta_2 - \frac{1}{3}\rp\omega^{n}_{\parallel\beta}\omega^{p}_{\parallel\gamma}\Big].\nn
\end{align}
For a general $\omega$, the two projections $\omega_\parallel$ and $\omega_\perp$ must be treated as independent variables and to second order we thus have 5 independent combinations of $\omega$. 
Reading off the factors multiplying each of the coefficients in the equations above and setting them to zero still yields contradictious values on the parameters. 
We are therefore also forced to assume there to be a relation between $\omega_\parallel$ and $\omega_\perp$ (or that one of them is zero but such an analysis will not be the general case).

The fact that the equations are contradictious could also point in the direction that they are wrong.
One reason to explain this is although trying to do things as general as possible, the analysis hasn't been general enough and something important has been excluded. 
Another reason could be that the starting action \eqnref{dynamics_final_action}, which was an ansatz, is not on the right form.
Moreover, nothing tells us that there should be an explicit relation between $*f$ and $\omega$, it could be more complicated.
Although these possibilities, the relations between the 1-form fields are nothing strange but rather what we would expect. 
As was seen in the previous chapter we expect the theory to have $3$ scalar degrees of freedom instead of the $9$ entering the potentials to $\omega^{rm}$ and $f_m$.

%\subsection{General series expansion with $\pdg{r}F^{rm} = 0$}
%Impose the constraint that the projection of $F$ in one specific direction $\pdg{{}}$ vanishes, i.e.
%\begin{align}
%\pdg{r}F^{rm} = \lp\tilde\gamma_1\pdp{r}+\tilde\gamma_2\pdo{r}\rp F^{rm} = 0
%\end{align} 
%implying the closed forms of integration 
%\begin{align}
%\gamma^m &= |p|\pdg{r}\omega^{rm}\nn\\
%\delta_{rm} &= |p|\lp\tilde\delta_1\pdp{r}+\tilde\delta_2\pdo{r}\rp\epsilon_{mnp}\pdg{s}\omega^{sn}\we\pdg{t}\omega^{tp} 
%\end{align}
%giving, once again, the series expansion of the duality equations to second order
%\begin{align}
%*k_m^\alpha: & \lp \tilde\gamma_1 + 2a_2b_1 - 2\alpha_1 -\frac{4}{3} \rp\omega^\alpha_{\parallel m} + \lp \tilde\gamma_2 + 2a_2b_2 - 2\alpha_2\rp\omega^\alpha_{\perp m}\nn\\
%& +\epsilon_{mnp}\varepsilon^{\alpha\beta\gamma}\Big[ \lp 2a_2b_3 + 3a_4b_2^2 + a_3\rp\omega^{n}_{\perp\beta}\omega^{p}_{\perp\gamma} + \lp 2a_2b_4 + 6a_4b_1b_2\rp\omega^{n}_{\perp\beta}\omega^{p}_{\parallel\gamma}\nn\\
%& + \lp 2a_2b_5 + 3a_4b_1^2 + a_3\rp\omega^{n}_{\parallel\beta}\omega^{p}_{\parallel\gamma} \Big] = 0\nn\\
%%
%\pup{r}j_{rm}^\alpha: & 2\lp -a_1 + \alpha_1b_1 - \frac{1}{3}b_1\rp\omega^\alpha_{\parallel m} + 2\lp\alpha_1-\frac{1}{3}\rp b_2\omega^\alpha_{\perp m} \nn\\
%& +\epsilon_{mnp}\varepsilon^{\alpha\beta\gamma}\Big[ \lp \tilde\delta_1\tilde\gamma_2^2 - \frac{1}{2}\tilde\gamma_2 + 2\lp\alpha_1-\frac{1}{3}\rp b_3 + 2\beta_1\rp\omega^{n}_{\perp\beta}\omega^{p}_{\perp\gamma} \nn\\
%& + \lp 2\tilde\delta_1\tilde\gamma_1\tilde\gamma_2 - \frac{1}{2}\tilde\gamma_1 -2a_3b_2 + 2\lp\alpha_1-\frac{1}{3}\rp b_4 + 4\beta_2 + \frac{1}{3}\rp\omega^{n}_{\perp\beta}\omega^{p}_{\parallel\gamma}\nn\\
%& + \lp \tilde\delta_1\tilde\gamma_1^2 -2a_3b_1 + 2\lp\alpha_1-\frac{1}{3}\rp b_5 + 6\beta_1\rp\omega^{n}_{\parallel\beta}\omega^{p}_{\parallel\gamma} \Big] = 0\nn\\
%%
%\puo{r}j_{rm}^\alpha: & 2\alpha_2b_1\omega^\alpha_{\parallel m} + 2\lp -a_1 + \alpha_2b_2\rp\omega^\alpha_{\perp m} \nn\\
%& +\epsilon_{mnp}\varepsilon^{\alpha\beta\gamma}\Big[ \lp \tilde\delta_2\tilde\gamma_2^2 - 2a_3b_2 + 2\alpha_2b_3 + 6\beta_2\rp\omega^{n}_{\perp\beta}\omega^{p}_{\perp\gamma}\nn\\
%& + \lp 2\tilde\delta_2\tilde\gamma_1\tilde\gamma_2 + \frac{1}{2}\tilde\gamma_2 - 2a_3b_1 + 2\alpha_2b_4 + 4\beta_1\rp\omega^{n}_{\perp\beta}\omega^{p}_{\parallel\gamma}\nn\\
%& + \lp \tilde\delta_2\tilde\gamma_1^2 + \frac{1}{2}\tilde\gamma_1 + 2\alpha_2b_5 + 2\beta_2 - \frac{1}{3}\rp\omega^{n}_{\parallel\beta}\omega^{p}_{\parallel\gamma} \Big] = 0
%%
%\end{align}
%which can be solved by
%% Equations of the series expansion
%\begin{align}
%0 =&\tilde\gamma_1 + 2a_2b_1 - 2\alpha_1 -\frac{4}{3}\nn\\
%0 =&\tilde\gamma_2 + 2a_2b_2 - 2\alpha_2\nn\\
%0 =& 2a_2b_3 + 3a_4b_2^2 + a_3\nn\\
%0 =& 2a_2b_4 + 6a_4b_1b_2\nn\\
%0 =& 2a_2b_5 + 3a_4b_1^2 + a_3\nn\\
%0 =& -a_1 + \alpha_1b_1 - \frac{1}{3}b_1\nn\\
%0 =& \lp\alpha_1-\frac{1}{3}\rp b_2\nn\\
%0 =& \tilde\delta_1\tilde\gamma_2^2 - \frac{1}{2}\tilde\gamma_2 + 2\lp\alpha_1-\frac{1}{3}\rp b_3 + 2\beta_1\nn\\
%0 =& 2\tilde\delta_1\tilde\gamma_1\tilde\gamma_2 - \frac{1}{2}\tilde\gamma_1 -2a_3b_2 + 2\lp\alpha_1-\frac{1}{3}\rp b_4 + 4\beta_2 + \frac{1}{3}\nn\\
%0 =& \tilde\delta_1\tilde\gamma_1^2 -2a_3b_1 + 2\lp\alpha_1-\frac{1}{3}\rp b_5 + 6\beta_1\nn\\
%0 =& 2\alpha_2b_1\nn\\
%0 =& \lp -a_1 + \alpha_2b_2\rp\nn\\
%0 =& \tilde\delta_2\tilde\gamma_2^2 - 2a_3b_2 + 2\alpha_2b_3 + 6\beta_2\nn\\
%0 =& 2\tilde\delta_2\tilde\gamma_1\tilde\gamma_2 + \frac{1}{2}\tilde\gamma_2 - 2a_3b_1 + 2\alpha_2b_4 + 4\beta_1\nn\\
%0 =& \tilde\delta_2\tilde\gamma_1^2 + \frac{1}{2}\tilde\gamma_1 + 2\alpha_2b_5 + 2\beta_2 - \frac{1}{3}
%\end{align}
%\newpage
%\begin{align}
%0 =&\tilde\gamma_1 + 2a_2b_1 - 2\alpha_1 -\frac{4}{3}\nn\\
%0 =&\tilde\gamma_2 + 2a_2b_2 - 2\alpha_2\nn\\
%0 =& a_4(b_2^2 -b_1^2)\nn\\
%0 =& a_4b_1b_2\nn\\
%a_3 =& -3a_4b_1^2\nn\\
%0 =& -\alpha_2b_2 + \alpha_1b_1 - \frac{1}{3}b_1\nn\\
%0 =& \lp\alpha_1-\frac{1}{3}\rp b_2\nn\\
%2\beta_1 =& -\frac{1}{3}\tilde\delta_1\tilde\gamma_1^2\nn\\
%0 =& 2\alpha_2b_1\nn\\
%a_1 =& \alpha_2b_2\nn\\
%2\beta_2 =& -\frac{1}{3}\tilde\delta_2\tilde\gamma_2^2\nn\\
%0 =& \tilde\delta_1\tilde\gamma_2^2 - \frac{1}{2}\tilde\gamma_2 -\frac{1}{3}\tilde\delta_1\tilde\gamma_1^2\nn\\
%0 =& 2\tilde\delta_1\tilde\gamma_1\tilde\gamma_2 - \frac{1}{2}\tilde\gamma_1 -\frac{2}{3}\tilde\delta_2\tilde\gamma_2^2 + \frac{1}{3}\nn\\
%0 =& 2\tilde\delta_2\tilde\gamma_1\tilde\gamma_2 + \frac{1}{2}\tilde\gamma_2 -\frac{2}{3}\tilde\delta_1\tilde\gamma_1^2\nn\\
%0 =& \tilde\delta_2\tilde\gamma_1^2 + \frac{1}{2}\tilde\gamma_1 -\frac{1}{3}\tilde\delta_2\tilde\gamma_2^2 - \frac{1}{3}\nn\\
%\tilde\delta_2(\tilde\gamma_2^2 - \tilde\gamma_1^2) =& 2\tilde\delta_1\tilde\gamma_1\tilde\gamma_2 \nn\\
%\tilde\delta_1(\tilde\gamma_2^2 -\tilde\gamma_1^2) =& - 2\tilde\delta_2\tilde\gamma_1\tilde\gamma_2 \nn\\
%\tilde\delta_2/\tilde\delta_1 = -\tilde\delta_1/\tilde\delta_2
%\end{align}
%Solutions:\\
%$b_3=b_4=b_5=0$, $\gamma_2 = 0$, $\gamma_1 = 2/3$, $\delta_1 = 0$, $\delta_2 = 0$, $\beta_1 = 0$, $\beta_2 = 0$, $b_1 = b_2 =0$, $\alpha_1=-1/3$, $\alpha_2=0$, $a_1 =0$, $a_3 = 0$, $a_2 = a_4 = ?$\nn\\
%$b_3=b_4=b_5=0$, $\gamma_2 = 0$, $\gamma_1 = 2/3$, $\delta_1 = 0$, $\delta_2 = 0$, $\beta_1 = 0$, $\beta_2 = 0$, $a_4 = 0$, $a_3 = 0$, $b_2=0$, $a_1=0$, $\alpha_2=0$, $\alpha_1=1/3$, $a_2 = \frac{2}{3b_1}$, $b_1=?$\nn\\
%Kan inte �terskapa pappersl�sningen eftersom denna kr�ver $\omega_\perp = 0$ (Antingen genom att s�tta den till 0 direkt eller genom att kr�va attt ansatzen inte ska inneh�lla n�gra laddningar $w=0$)
%
%$\gamma_2\ne 0$, $\gamma_1\ne 0$, $\delta_1\ne 0$
%
%\begin{align}
%0 =&a_2b_1 - \alpha_1 -\frac{1}{3}\nn\\
%0 =&a_2b_2 - \alpha_2\nn\\
%0 =& -\alpha_2b_2 + \alpha_1b_1 - \frac{1}{3}b_1\nn\\
%0 =& \lp\alpha_1-\frac{1}{3}\rp b_2\nn\\
%0 =& 2\alpha_2b_1\nn\\
%a_1 =& \alpha_2b_2\nn\\
%\end{align}
%
%\begin{align}
%\Phi = a_1 v^2 + a_2 u^2 + a_3uv\nn\\ 
%\partial_v\Phi = 2 a_1 v + a_3 u = 2/3v\nn\\ 
%\partial_u\Phi = 2 a_2 u + a_3 v = -2/3u\nn\\ 
%\partial_v\Phi = 2 a_1 v + a_3 b_1v = 2/3v\nn\\ 
%\partial_u\Phi = 2 a_2 b_1v + a_3 v = -2/3b_1v\nn\\ 
%\end{align}
%
%\newpage
%\subsubsection{case $\alpha_1=1/3$, $\alpha_2=0$}
%\begin{align}
%\tilde\gamma_1 =& 2(1 - a_2b_1)\nn\\
%\tilde\gamma_2 =& - 2a_2b_2 \nn\\
%0 =& 2a_2b_3 + 3a_4b_2^2 + a_3\nn\\
%0 =& a_2b_4 + 3a_4b_1b_2\nn\\
%0 =& 2a_2b_5 + 3a_4b_1^2 + a_3\nn\\
%0 =& 4\tilde\delta_1a_2^2b_2^2 + a_2b_2 + 2\beta_1\nn\\
%0 =& -8\tilde\delta_1(1 - a_2b_1)a_2b_2 - 1 + a_2b_1 -2a_3b_2 + 4\beta_2 + \frac{1}{3}\nn\\
%0 =& 2\tilde\delta_1(1 - a_2b_1)^2 -a_3b_1 + 3\beta_1\nn\\
%0 =& -\tilde\delta_2a_2b_2 - a_3b_2 + 3\beta_2\nn\\
%0 =& -8\tilde\delta_2(1 - a_2b_1) - a_2b_2 - 2a_3b_1 + 4\beta_1\nn\\
%0 =& 4\tilde\delta_2(1 - a_2b_1)^2 + 1 - a_2b_1 + 2\beta_2 - \frac{1}{3}
%\end{align}
%$a_1=0$
%\newpage
%\subsubsection{case $b_2=0$, $\alpha_2=0$}
%\begin{align}
%0 =&\tilde\gamma_1 + 2a_2b_1 - 2\alpha_1 -\frac{4}{3}\nn\\
%0 =&\tilde\gamma_2 + 2a_2b_2 - 2\alpha_2\nn\\
%0 =& 2a_2b_3 + 3a_4b_2^2 + a_3\nn\\
%0 =& 2a_2b_4 + 6a_4b_1b_2\nn\\
%0 =& 2a_2b_5 + 3a_4b_1^2 + a_3\nn\\
%0 =& -a_1 + \alpha_1b_1 - \frac{1}{3}b_1\nn\\
%0 =& \lp\alpha_1-\frac{1}{3}\rp b_2\nn\\
%0 =& \tilde\delta_1\tilde\gamma_2^2 - \frac{1}{2}\tilde\gamma_2 + 2\lp\alpha_1-\frac{1}{3}\rp b_3 + 2\beta_1\nn\\
%0 =& 2\tilde\delta_1\tilde\gamma_1\tilde\gamma_2 - \frac{1}{2}\tilde\gamma_1 -2a_3b_2 + 2\lp\alpha_1-\frac{1}{3}\rp b_4 + 4\beta_2 + \frac{1}{3}\nn\\
%0 =& \tilde\delta_1\tilde\gamma_1^2 -2a_3b_1 + 2\lp\alpha_1-\frac{1}{3}\rp b_5 + 6\beta_1\nn\\
%0 =& 2\alpha_2b_1\nn\\
%0 =& \lp -a_1 + \alpha_2b_2\rp\nn\\
%0 =& \tilde\delta_2\tilde\gamma_2^2 - 2a_3b_2 + 2\alpha_2b_3 + 6\beta_2\nn\\
%0 =& 2\tilde\delta_2\tilde\gamma_1\tilde\gamma_2 + \frac{1}{2}\tilde\gamma_2 - 2a_3b_1 + 2\alpha_2b_4 + 4\beta_1\nn\\
%0 =& \tilde\delta_2\tilde\gamma_1^2 + \frac{1}{2}\tilde\gamma_1 + 2\alpha_2b_5 + 2\beta_2 - \frac{1}{3}
%\end{align}
%
%\newpage
%\subsubsection{case $\alpha_1=1/3$, $b_1=0$}
%\begin{align}
%0 =&\tilde\gamma_2 + 2a_2b_2 - 2\alpha_2\nn\\
%0 =& 2a_2b_3 + 3a_4b_2^2 -2a_2b_5\nn\\
%0 =& 2a_2b_4\nn\\
%a_3 =& -2a_2b_5\nn\\
%0 =& \tilde\delta_1\tilde\gamma_2^2 - \frac{1}{2}\tilde\gamma_2 + 2\beta_1\nn\\
%0 =& 4\tilde\delta_1\tilde\gamma_2 +4a_2b_5b_2 + 4\beta_2 - \frac{2}{3}\nn\\
%0 =& 2\tilde\delta_1 + 3\beta_1\nn\\
%0 =& \alpha_2b_2\nn\\
%0 =& \tilde\delta_2\tilde\gamma_2^2 +4a_2b_5b_2 + 2\alpha_2b_3 + 6\beta_2\nn\\
%0 =& 4\tilde\delta_2\tilde\gamma_2 + \frac{1}{2}\tilde\gamma_2 + 2\alpha_2b_4 + 4\beta_1\nn\\
%0 =& 4\tilde\delta_2 + 2\alpha_2b_5 + 2\beta_2 + \frac{2}{3}
%\end{align}
%$a_1=0, \tilde\gamma_1 = 2$
%\subsubsection{subcase $\alpha_2=0$}
%\begin{align}
%\tilde\gamma_2 =&- 2a_2b_2\nn\\
%0 =& 2a_2b_3 + 3a_4b_2^2 -2a_2b_5\nn\\
%a_3 =& -2a_2b_5\nn\\
%0 =& 4\tilde\delta_1a_2^2b_2^2 + a_2b_2 -\frac{4}{3}\tilde\delta_1\nn\\
%\tilde\delta_1  =& -\frac{a_2b_2}{4( a_2^2b_2^2 -\frac{1}{3})} \nn\\
%a_2b_2b_5 =& - \frac{1}{4}\nn\\
%\beta_1 =& -\frac{2}{3}\tilde\delta_1\nn\\
%0 =& a_2^4b_2^4 + 2a_2^2b_2^2 +1\nn\\
%0 =& 8(\tilde\delta_1a_2b_2 + \frac{3}{8})a_2b_2 -a_2b_2 -\frac{8}{3}\tilde\delta_1\nn\\
%\beta_2 =& -2\tilde\delta_2 - \frac{1}{3}\nn\\
%\tilde\delta_2 =& -\tilde\delta_1a_2b_2 - \frac{3}{8}  
%\end{align}
%$a_2\ne 0$, $b_4=0$, $b_2\ne 0$, $b_5\ne 0$
%
%\subsubsection{subcase $b_2=0$}
%\begin{align}
%0 =&\tilde\gamma_2 - 2\alpha_2\nn\\
%0 =& 2a_2b_3 + a_3\nn\\
%0 =& 2a_2b_4\nn\\
%0 =& 2a_2b_5 + a_3\nn\\
%0 =& \tilde\delta_1\tilde\gamma_2^2 - \frac{1}{2}\tilde\gamma_2 + 2\beta_1\nn\\
%0 =& 4\tilde\delta_1\tilde\gamma_2 + 4\beta_2 - \frac{2}{3}\nn\\
%3\beta_1 =& -2\tilde\delta_1\nn\\
%0 =& \tilde\delta_2\tilde\gamma_2^2 + 2\alpha_2b_3 + 6\beta_2\nn\\
%0 =& 4\tilde\delta_2\tilde\gamma_2 + \frac{1}{2}\tilde\gamma_2 + 2\alpha_2b_4 + 4\beta_1\nn\\
%0 =& 4\tilde\delta_2 + 2\alpha_2b_5 + 2\beta_2 + \frac{2}{3}
%\end{align}
%
%\newpage
%\subsubsection{case $b_2=0$, $b_1=0$}
%\begin{align}
%\tilde\gamma_1 =& 2\lp\alpha_1 + \frac{2}{3}\rp\nn\\
%\tilde\gamma_2 =& 2\alpha_2\nn\\
%a_3 =& -2a_2b_3\nn\\
%0 =& a_2b_4\nn\\
%0 =& a_2(b_5 - b_3)\nn\\
%6\beta_1 =& -12\tilde\delta_1\alpha_2^2 + 3\alpha_2 - 6\alpha_1b_3 + b_3\nn\\
%0 =& 24\tilde\delta_1\alpha_1\alpha_2 + 16\tilde\delta_1\alpha_2 - 3\alpha_1 - 1 + 6\alpha_1b_4-b_4 -8\tilde\delta_2\tilde\alpha_2^2 - 4\alpha_2b_3\nn\\
%0 =& 4\tilde\delta_1\lp 3\alpha_1 + 2\rp^2 + 18\alpha_1b_5 - 6b_5 -108\tilde\delta_1\alpha_2^2 + 27\alpha_2 - 54\alpha_1b_3 + 9b_3\nn\\
%3\beta_2 =& -2\tilde\delta_2\tilde\alpha_2^2 - \alpha_2b_3\nn\\
%0 =& 24\tilde\delta_2\alpha_1\alpha_2 + 16\tilde\delta_2\alpha_2 + 3\alpha_2 + 6\alpha_2b_4 -24\tilde\delta_1\alpha_2^2 + 6\alpha_2 - 12\alpha_1b_3 + 2b_3\nn\\
%0 =& 4\tilde\delta_2\lp 3\alpha_1 + 2\rp^2 + 9\alpha_1 + 3 + 18\alpha_2b_5 -12\tilde\delta_2\tilde\alpha_2^2 - 6\alpha_2b_3
%\end{align}
%$a_1=0$
%
%
%\newpage
%If we assume the relation between $\omega_\perp$ and $\omega_\parallel$ to be linear
%\begin{align}
%\omega_\perp^m = c\omega_\parallel^m
%\end{align}
%we  get the series expansion of the duality equations to second order (note that this condition let us set $b_2 = b_3 = b_4 = 0$, if $\omega_\parallel\ne 0$)
%\begin{align}
%*k_m^\alpha: & 2\lp a_2b_1 - \alpha_1 -\frac{2}{3} - c\alpha_2\rp\omega^\alpha_{\parallel m} \nn\\
%& +\epsilon_{mnp}\varepsilon^{\alpha\beta\gamma}\Big[ c^2\lp 2a_2b_5 + a_3\rp  + 3a_4b_1^2 + a_3 \Big]\omega^{n}_{\parallel\beta}\omega^{p}_{\parallel\gamma} = 0\nn\\
%%
%\pup{r}j_{rm}^\alpha: & 2\lp -a_1 + \alpha_1b_1 - \frac{1}{3}b_1 \rp\omega^\alpha_{\parallel m} \nn\\
%& +\epsilon_{mnp}\varepsilon^{\alpha\beta\gamma}\Big[ -2a_3b_1 + 6\beta_1 + c\lp + 4\beta_2 + \frac{1}{3}\rp + c^2\lp 2\lp\alpha_1-\frac{1}{3}\rp b_5 + 2\beta_1\rp\Big]\omega^{n}_{\parallel\beta}\omega^{p}_{\parallel\gamma} = 0\nn\\
%%
%\puo{r}j_{rm}^\alpha: & 2\lp\alpha_2b_1 - ca_1\rp\omega^\alpha_{\parallel m} \nn\\
%& +\epsilon_{mnp}\varepsilon^{\alpha\beta\gamma}\Big[ 2\beta_2 - \frac{1}{3} + c\lp - 2a_3b_1 + 4\beta_1\rp + c^2\lp  + 2\alpha_2b_5 + 6\beta_2\rp \Big]\omega^{n}_{\parallel\beta}\omega^{p}_{\parallel\gamma} = 0
%%
%\end{align}
%For the case $\omega_\parallel = 0$ ($b_1=b_4=b_5 = 0$) we instead get
%\begin{align}
%*k_m^\alpha: & 2\lp a_2b_2 - \alpha_2\rp\omega^\alpha_{\perp m}\nn\\
%& +\epsilon_{mnp}\varepsilon^{\alpha\beta\gamma}\Big[ \lp 2a_2b_3 + 3a_4b_2^2 + a_3\rp\omega^{n}_{\perp\beta}\omega^{p}_{\perp\gamma}\Big] = 0\nn\\
%%
%\pup{r}j_{rm}^\alpha: & 2\lp\alpha_1-\frac{1}{3}\rp b_2\omega^\alpha_{\perp m} \nn\\
%& +\epsilon_{mnp}\varepsilon^{\alpha\beta\gamma}\Big[ \lp 2\lp\alpha_1-\frac{1}{3}\rp b_3 + 2\beta_1\rp\omega^{n}_{\perp\beta}\omega^{p}_{\perp\gamma}\Big] = 0\nn\\
%%
%\puo{r}j_{rm}^\alpha: & 2\lp -a_1 + \alpha_2b_2\rp\omega^\alpha_{\perp m} \nn\\
%& +\epsilon_{mnp}\varepsilon^{\alpha\beta\gamma}\Big[ \lp - 2a_3b_2 + 2\alpha_2b_3 + 6\beta_2\rp\omega^{n}_{\perp\beta}\omega^{p}_{\perp\gamma}\Big] = 0
%%
%\end{align}
%This relation will force $F=0$ since the action with an external derivative on the relation between $\omega_\parallel$ and $\omega_\perp$ yields
%\begin{align}
%d\omega_\perp^m - cd\omega_\parallel^m = -\lp\pdo{r} - c\pdp{r}\rp F^r = 0
%\end{align}
%
%\subsubsection{case $\omega_\parallel = 0$, $b_2=1$, $b_3=0$}
%\begin{align}
%*k_m^\alpha: & 2\lp a_2 - \alpha_2\rp\omega^\alpha_{\perp m}\nn\\
%& +\epsilon_{mnp}\varepsilon^{\alpha\beta\gamma}\Big[ \lp 3a_4 + a_3\rp\omega^{n}_{\perp\beta}\omega^{p}_{\perp\gamma}\Big] = 0\nn\\
%%
%\pup{r}j_{rm}^\alpha: & 2\lp\alpha_1-\frac{1}{3}\rp \omega^\alpha_{\perp m} \nn\\
%& +\epsilon_{mnp}\varepsilon^{\alpha\beta\gamma}\Big[ \lp 2\beta_1\rp\omega^{n}_{\perp\beta}\omega^{p}_{\perp\gamma}\Big] = 0\nn\\
%%
%\puo{r}j_{rm}^\alpha: & 2\lp -a_1 + \alpha_2\rp\omega^\alpha_{\perp m} \nn\\
%& +\epsilon_{mnp}\varepsilon^{\alpha\beta\gamma}\Big[ \lp - 2a_3 + 6\beta_2\rp\omega^{n}_{\perp\beta}\omega^{p}_{\perp\gamma}\Big] = 0
%%
%\end{align}
%The conditions are
%\begin{align}
%\beta_1=0\nn\\
%\alpha_1=\frac{1}{3}\nn\\
%a_1 = \alpha_2\nn\\
%a_2 = \alpha_2\nn\\
%\end{align}

\subsection{The equations of the $F = 0$ ($\omega_\perp = 0$) case}
We have seen that we must have relations between all the fields $\omega_\parallel$, $\omega_\perp$ and $*f$ for the equations to be consistent.
In \cite{artikeln} the parameters like $\alpha_1$ and $\beta_2$ were introduced to gain consistency in the special case $\omega_\perp = 0$, but as we shall see in this section this is not necessary. The equations can still be solved with the use the closed forms $\Gamma$ and $\Delta$.
Thus we reintroduce the closed forms $\Gamma$ and $\Delta$ and deintroduce the parameters $\alpha$, $\beta$ and $\gamma$, i.e. putting them to zero. 
One way to get a useful integration form $\Gamma$ is to put a restriction on the background fields, e.g. letting the projection of $F^{rm}$ disappear in some direction $\pdg{{}}$, i.e. 
\begin{align}
\pdg{r}F^{rm} = \lp\tilde\Gamma_1\pdp{r}+\tilde\Gamma_2\pdo{r}\rp F^{rm} = 0.
\end{align}
As was shown in \cite{artikeln} such an assumption does not lead to a U-duality invariant formulation due to the $p$-dependence of the constraint and we should therefore rather use the stronger condition $F^{rm} = 0$.
In such a background we can introduce the closed forms 
\begin{align}
\Gamma^m &= |p|\pdg{r}\omega^{rm}\mbox{ and }\nn\\
\Delta_{rm} &= |p|\epsilon_{mnp}\lp\tilde\Delta_1\pdp{r}+\tilde\Delta_2\pdo{r}\rp\Gamma^n\we\Gamma^p
\end{align}
to get the integrated equations of motion
\begin{align}
*k^m =& -\bigg[\lp\frac{4}{3} - \tilde\Gamma_1\rp \omega_\parallel^m - \tilde\Gamma_2\omega_\perp^m\bigg]\sqrt{1+\Phi}\\
j_{rm} =& \bigg\{ 
-\frac{2}{3}*f_m \hat p_{\parallel r} + \epsilon_{mnp}*\lp\omega^{sn} \wedge \omega^{tp}\rp\Big[\lp \frac{1}{3} - \frac{\tilde\Gamma_1}{2}\rp \epsilon_{rs}\pdp{t} - \frac{\tilde\Gamma_2}{2}\epsilon_{rs}\pdo{t}\nn\\
& + \lp\tilde\Delta_1\pdp{r}+\tilde\Delta_2\pdo{r}\rp\lp\tilde\Gamma_1^2\pdp{s}\pdp{t} + 2\tilde\Gamma_1\tilde\Gamma_2\pdp{s}\pdo{t} + \tilde\Gamma_2^2\pdo{s}\pdo{t}\rp\Big]
\bigg\}\sqrt{1+\Phi}.\nn
\end{align}
We will now make the variable substitutions 
\begin{align}
\eqnlab{solution_8d_vartrans}
u^m & =*f^m\nn\\ 
v^m & =\omega_{\parallel}^m = \pdp{r}\omega^{rm}\nn\\
w^m & =\omega_{\perp}^m = \pdo{r}\omega^{rm}.
\end{align}
Using this variable transformation we note that all solutions $\Phi(u,v,w)$ with non-vanishing $v$ or $w$-dependence  must also have a $p$-dependence, which is not allowed, since at the level of deriving the equations of motion we must treat $p$ as $p^r=\lambda*h^r$.
The solution to this problem is of course that all $\omega$:s enters the action with the $SL(2,\rr)$-index contracted (using $\W_{rs}$ or $\epsilon_{rs}$) with another $\omega$ and thus all $v$ and $w$ must enter $\Phi$ (using the Pythagorean theorem \eqnref{solution_pythgoras}) either squared symmetric $v_\alpha^mv_\beta^n + w_\alpha^mw_\beta^n$ or mixed antisymmetric $v_\alpha^mw_\beta^n-w_\alpha^mv_\beta^n$, which is $p$-independent.
The variations in the new variables becomes
\begin{align}
*k^m &= *\frac{\partial \Phi}{\partial f_m} = *\lp\frac{\partial \Phi}{\partial u_{\alpha n}}\frac{\partial u_{\alpha n}}{\partial f_m}\rp = \half d\xi^\delta\varepsilon_{\delta\gamma\beta}\lp \frac{\partial \Phi}{\partial u_m^{\alpha}} \frac{1}{2}2\varepsilon^{\alpha\beta\gamma}\rp = -\frac{\partial \Phi}{\partial u_m}\nn\\
j_{rm} &= \frac{\partial \Phi}{\partial \omega^{rm}} = \frac{\partial \Phi}{\partial v^n_\alpha}\frac{\partial v^n_\alpha}{\partial \omega^{rm}} + \frac{\partial \Phi}{\partial w^n_\alpha}\frac{\partial w^n_\alpha}{\partial \omega^{rm}} = \frac{\partial \Phi}{\partial v^m}\pdp{r} + \frac{\partial \Phi}{\partial w^m}\pdo{r}.
\end{align}

Taking out the parallel and orthogonal projections of the latter duality equation we get
\begin{align}
\frac{\partial\Phi}{\partial u_m} =& \bigg[\lp\frac{4}{3} - \tilde\Gamma_1\rp v^m - \tilde\Gamma_2w^m\bigg]\sqrt{1+\Phi}\nn\\
\frac{\partial\Phi}{\partial v^m} =& \bigg\{-\frac{2}{3}u_m + \epsilon_{mnp}\bigg[\lp\tilde\Delta_1\tilde\Gamma_1^2 \rp*\lp v^n \wedge v^p\rp + \lp\tilde\Delta_1\tilde\Gamma_2^2 - \frac{\tilde\Gamma_2}{2}\rp*\lp w^n \wedge w^p\rp\nn\\
&+\lp 2\tilde\Delta_1\tilde\Gamma_1\tilde\Gamma_2 + \frac{1}{3} - \frac{\tilde\Gamma_1}{2}\rp*\lp w^n \wedge v^p\rp \bigg]\bigg\}\sqrt{1+\Phi}\nn\\
\frac{\partial\Phi}{\partial w^m} =& \epsilon_{mnp}\bigg[\lp\tilde\Delta_2\tilde\Gamma_1^2 + \frac{\tilde\Gamma_1}{2}-\frac{1}{3}\rp*\lp v^n \wedge v^p\rp + \lp\tilde\Delta_2\tilde\Gamma_2^2 \rp*\lp w^n \wedge w^p\rp\nn\\
&+\lp 2\tilde\Delta_2\tilde\Gamma_1\tilde\Gamma_2 - \frac{\tilde\Gamma_2}{2}\rp*\lp w^n \wedge v^p\rp\bigg]\sqrt{1+\Phi}.
\end{align}
There exists one set of parameters $\tilde\Gamma_2 = 0, \tilde\Gamma_1 = \frac{2}{3}, \tilde\Delta_1 = 0, \tilde\Delta_2 = 0$ to remove all the $*\omega\we\omega$ terms\footnote{It is not obvious at all that the $*\omega\we\omega$ terms should be removed. The only reason we do this is that it makes the equations easier to solve, because $\Phi$ need only contain even order terms when the right hand side is proportional to the field strengths and since it removes one variable $w$.}.   
The remaining duality equations are
\begin{align}
\eqnlab{solution_8d_duality_paper}
\frac{\partial\Phi}{\partial u_m} &= \frac{2}{3} v^{m}\sqrt{1+\Phi}\nn\\
\frac{\partial\Phi}{\partial v^m} & = -\frac{2}{3}u_m \sqrt{1+\Phi}\nn\\
\frac{\partial\Phi}{\partial w^m} & = 0
\end{align}
which, if $w=0$, are the same equations (up to a variable rescaling) obtained in \cite{artikeln} with $\alpha_1 = -1/6$, $\alpha_2 = 0$, $\beta_1 = 0$ and $\beta_2 = 1/6$ and $F=0$ which was derived from the condition that $\hat p$ and $\omega$ are aligned, i.e. $\omega_\perp = 0$.
Actually, this condition is not trivially given by the above equations. Indeed $w=0$ implies $\frac{\partial\Phi}{\partial w}=0$ when $w$ enters $\Phi$ quadratically as it must, but the contrary is not true (c.f. the trivial solutions in section \secref{solution_result}). 

In conclusion we note that the equations in \cite{artikeln} was derived under the conditions $\omega_\perp = 0$, implying $F_\perp^m = 0$, implying $F^{rm} = 0$, while ours were derived under the somewhat weaker condition $F^{rm} = 0$.
Another difference is that we have not been forced to include the $\alpha$ and $\beta$ parameters. 
Introducing $\alpha_1$ now would just mean changing the multiplying factor $2/3$ to $(-2\alpha_1+2/3)$ in the first equation and $-2/3$ to $(-2\alpha_1-2/3)$ in the second equation. Unless we find a constraint on $\alpha_1$ we can thus choose the 2 factors multiplying each of the duality equations arbitrary with the help of a field rescaling. 
With the condition $\omega_\perp=0$ the introduction of $\beta_2$ would add such an $\omega\we\omega$ term we removed with our choice of $\Gamma$ and $\Delta$ and would thus not serve any purpose. 
These were the parameters considered in \cite{artikeln}. The introduction of $\beta_1$ works more or less like the $\beta_2$ parameter and the introduction of $\alpha_2$ would give the constraint $f\propto \omega_\parallel\we\omega_\parallel$, which is too restrictive because it put a constraint on $H_m$ from the Bianchi identity of $f_m$. 
So, for the $\omega_\perp = 0$ case the introduction of constant $\alpha$ and $\beta$ parameters does not contribute anything at all.

We will try to solve these equations later on in \secref{csolution_8d_const_old}.

\subsection{The equations of the general parameter free case}
%The dualities of $v\we v$ and $v\we v\we v$:\\
%Let $B_{\beta\alpha} = 2v^n_\alpha\we v^p_\beta$ and $C_{\gamma\beta\alpha} = 6v^m_\alpha\we v^n_\beta\we v^p_\gamma$, which gives
%\begin{align}
%*B_{(2)} &= \frac{1}{2}d\xi^\gamma\varepsilon_{\gamma\alpha\beta}B^{\beta\alpha} = d\xi^\gamma\varepsilon_{\gamma\alpha\beta}v^{\alpha n}\we v^{\beta p} = *(v^n\we v^p)\\ 
%*C_{(3)} &= \frac{1}{6}\varepsilon_{\alpha\beta\gamma}C^{\gamma\beta\alpha} = \varepsilon_{\alpha\beta\gamma}v^{\alpha m}\we v^{\beta n}\we v^{\gamma p} = *(v^m\we v^n\we v^p)
%\end{align} 
We now consider the variable substitution
\begin{align}
u_m & =*f_m\nn\\ 
v^m & =\qdp{r}\omega^{rm}\nn\\
w^m & =\qdo{r}\omega^{rm},
\end{align}
where $\qdp{r} = q_1\pdp{r} + q_2\pdo{r}$ and $\qdo{r} = q_2\pdp{r} - q_1\pdo{r}$ with normalization $q_1^2+q_2^2=1$.
In the general case, without $\alpha$, $\beta$ and $\gamma$ terms, the duality equations then becomes
\begin{align}
\eqnlab{solution_8d_duality_general}
\frac{\partial\Phi}{\partial u_m} &= \frac{4}{3}\lp q_1 v^m + q_2 w^m\rp\sqrt{1+\Phi}\\
\frac{\partial\Phi}{\partial v^m} & = \frac{1}{3}\bigg\{-2q_1 u_m + \epsilon_{mnp}\Big[ q_1*\lp w^n \wedge v^p\rp + q_2*\lp w^n \wedge w^p\rp\Big] \bigg\}\sqrt{1+\Phi}\nn\\
\frac{\partial\Phi}{\partial w^m} & = \frac{1}{3}\bigg\{-2q_2 u_m - \epsilon_{mnp}\Big[ q_2*\lp w^n \wedge v^p\rp + q_1*\lp v^n \wedge v^p\rp\Big] \bigg\}\sqrt{1+\Phi}\nn
\end{align}
These equations will be exposed for a solve attempt in section \secref{csolution_8d_general}.


%\subsubsection{Bengts bad term 2}
%It would be nice to make a variable substitution such that the dependence of the new variables on $\omega$ is either it's parallel or orthogonal projection on $p$, so that the $r$ and $m$ indices in \eqnref{solution_8d_jrm_orig} decouple, letting us project the equation on $\hat p$ and $\hat p_\perp$ to get conditions on the parameters $\alpha$ and $\beta$. 
%We try the following substitution according to the combinations of $p$, $\omega$ and $f$ entering the equations 
%\begin{align}
%u_m &= *f_m + c_{(st)}\epsilon_{mnp}*(\omega^{sn}\we\omega^{tp})\nn\\
%v^m &= \qdp{r}\omega^{rm}\nn\\
%w^m &= \qdo{r}\omega^{rm}\nn\\
%\end{align}
%where 
%\begin{align}
%c_{(st)} = c_1\qdp{s}\qdp{t} + c_2\qdo{s}\qdo{t} + c_3\qdp{{(s}}\qdo{{t)}}
%\end{align}
%and $c_i$ are constants to be determined later.
%The variations of $\Phi(u(\omega,f),v(\omega))$ becomes
%\begin{align}
%*k^m &= -\frac{\partial \Phi}{\partial u_m}\nn\\
%j_{rm} &= \frac{\partial \Phi}{\partial \omega^{rm}} = \frac{\partial \Phi}{\partial u_{\alpha n}}\frac{\partial u_{\alpha n}}{\partial \omega^{rm}} + \frac{\partial \Phi}{\partial v^n_\alpha}\frac{\partial v^n_\alpha}{\partial \omega^{rm}} + \frac{\partial \Phi}{\partial w^n_\alpha}\frac{\partial w^n_\alpha}{\partial \omega^{rm}}\nn\\
%& = \frac{\partial \Phi}{\partial u_{n\alpha}}\frac{\partial}{\partial \omega^{rm}}\lp c_{(st)}\epsilon_{nn'p}\varepsilon_{\alpha\beta\gamma}\omega^{\beta sn'}\omega^{\gamma tp}\rp + \frac{\partial \Phi}{\partial v^m}\qdp{r} + \frac{\partial \Phi}{\partial w^m}\qdo{r}\nn\\
%& = -2c_{(rt)}\epsilon_{mnp}\varepsilon_{\alpha\beta\gamma}\frac{\partial \Phi}{\partial u_{n\alpha}}\omega^{\gamma tp}d\xi^\beta + \frac{\partial \Phi}{\partial v^m}\qdp{r} + \frac{\partial \Phi}{\partial w^m}\qdo{r}
%\end{align}
%and the duality relations are
%\begin{align}
%\frac{\partial \Phi}{\partial u_m} = \frac{4}{3}\lp q_1v^m + q_2w^m\rp\sqrt{1+\Phi},
%\eqnlab{solution_8d_km_trans}
%\end{align}
%\begin{align}
%\frac{\partial\Phi}{\partial v^m} 
%& = \frac{1}{3}\bigg\{-2q_1u_m + \epsilon_{mnp}\Big[ 2q_1c_1*(v^n\we v^p) + 2q_1c_2*(w^n\we w^p) + 2q_1c_3*(v^n\we w^p) + q_1*\lp w^n \wedge v^p\rp + q_2*\lp w^n \wedge w^p\rp\Big] \bigg\}\sqrt{1+\Phi}\nn\\
%& -\frac{8}{3}\epsilon_{mnp}*\lp\lp q_1v^{n} + q_2w^{n}\rp\we \lp c_1v^{p} + \half c_3w^p \rp\rp\sqrt{1+\Phi}\nn\\
%& = \frac{1}{3}\bigg\{-2q_1u_m + \epsilon_{mnp}\Big[ 2c_1q_1\lp 1  -4\rp*(v^n\we v^p) \nn\\
%& + \lp 2q_1c_2 + q_2 - 4q_2c_3\rp*(w^n\we w^p) + \lp 2q_1c_3 + q_1 - 4q_1c_3 - 8q_2c_1\rp*\lp v^n \wedge w^p\rp  
%\Big]\bigg\}\sqrt{1+\Phi}\nn\\
%\end{align}
%and
%\begin{align}
%\frac{\partial\Phi}{\partial w^m} 
%& = \frac{1}{3}\bigg\{-2q_2u_m + \epsilon_{mnp}\Big[ 2q_2c_1 *(v^n\we v^p) + 2q_2c_2 *(w^n\we w^p) + 2q_2c_3 *(v^n\we w^p) - q_2*\lp w^n \wedge v^p\rp - q_1*\lp v^n \wedge v^p\rp\Big] \bigg\}\sqrt{1+\Phi}\nn\\
%& -\frac{8}{3}\epsilon_{mnp}*\lp\lp q_1v^{n} + q_2w^{n}\rp\we \lp c_2w^{p}+\half c_3v^{p}\rp\rp\sqrt{1+\Phi}\nn\\
%& = \frac{1}{3}\bigg\{-2q_2u_m + \epsilon_{mnp}\Big[ \lp 2q_2c_1  - q_1 - 4c_3q_1\rp*(v^n\we v^p)\nn\\
%& + 2q_2c_2\lp 1 - 4 \rp*(w^n\we w^p) + \lp 2q_2c_3 -q_2 - 8q_1c_2 - 4q_2c_3\rp*\lp v^n \wedge w^p\rp
%\Big] \bigg\}\sqrt{1+\Phi}\nn\\
%\end{align}
%where we have used $\frac{\partial \Phi}{\partial u_m}$ from \eqnref{solution_8d_km_trans} to get
%\begin{align}
%c_{(rt)}\epsilon_{mnp}\varepsilon_{\alpha\beta\gamma}\frac{\partial \Phi}{\partial u_{n\alpha}}\omega^{\gamma tp}d\xi^\beta = -\frac{4}{3}c_{(rt)}\epsilon_{mnp}*\lp\lp q_1 v^{n} + q_2 w^{n}\rp\we\omega^{tp}\rp\sqrt{1+\Phi}
%\end{align}









% -----------------------------------------------------------------------------------------------------------------------------------
%\paragraph{Expansion of the duality equations (REMOVE!)}
%To third order, let
%\begin{align}
%\Phi = a_1\tr u^2 + a_2\lp \tr v^2 + \tr w^2\rp + \epsilon_{mnp}\varepsilon^{\alpha\beta\gamma}\lp a_3u^m_\alpha u^n_\beta u^p_\gamma + a_4u^m_\alpha\lp v^n_\beta v^p_\gamma + w^n_\beta w^p_\gamma\rp\rp
%\end{align}
%gives 
%
%\begin{align}
%2a_1 u^m + \epsilon_{mnp}\varepsilon^{\alpha\beta\gamma}\lp 3a_3u^n_\beta u^p_\gamma + a_4\lp v^n_\beta v^p_\gamma + w^n_\beta w^p_\gamma\rp\rp &= \frac{4}{3}v^m\nn\\
%2a_2 v_m + \epsilon_{mnp}\varepsilon^{\alpha\beta\gamma} 2a_4u^n_\beta v^p_\gamma &= -\frac{2}{3}u_m + \frac{1}{3}\epsilon_{mnp}*\lp w^n \wedge v^p\rp\nn\\
%2a_2 w_m + \epsilon_{mnp}\varepsilon^{\alpha\beta\gamma} 2a_4u^n_\beta w^p_\gamma & = -\frac{1}{3}\epsilon_{mnp}*\lp v^n \wedge v^p\rp
%\end{align}
%To second order let
%\begin{align}
%u_m &= b_1v_m + b_2\epsilon_{mnp}\varepsilon^{\alpha\beta\gamma}v^n_\beta v^p_\gamma\nn\\
%w_m &= c_1v_m + c_2\epsilon_{mnp}\varepsilon^{\alpha\beta\gamma}v^n_\beta v^p_\gamma\nn\\
%\end{align}
%giving
%\begin{align}
%\lp 2a_1 b_1 - \frac{4}{3}\rp v_m + \epsilon_{mnp}\varepsilon^{\alpha\beta\gamma}\lp 2a_1b_2 + 3a_3b_1^2 + a_4 + a_4c_1^2\rp v^n_\beta v^p_\gamma &= 0\nn\\
%\lp\frac{2}{3}b_1 + 2a_2\rp v_m + \epsilon_{mnp}\varepsilon^{\alpha\beta\gamma}\lp \frac{2}{3}b_2 + 2a_4b_1 - \frac{1}{3}c_1\rp v^n_\beta v^p_\gamma &= 0\nn\\
%2a_2c_1v_m + \epsilon_{mnp}\varepsilon^{\alpha\beta\gamma}\lp 2a_2c_2 + 2a_4b_1c_1 + \frac{1}{3}\rp v^n_\beta v^p_\gamma & = 0
%\end{align}
%i.e.\\ 
%First order: $c_1 = 0$,  $a_1 = \frac{2}{3}\frac{1}{b_1}$ and $a_2 = -\frac{1}{3}b_1$, $b_1\ne 0$\\
%Second order: $c_2 = \frac{1}{2b_1}$, $a_3 = -\frac{1}{3}\frac{b_2}{b_1^3}$ and $a_4 = -\frac{1}{3}\frac{b_2}{b_1}$ ($b_2$ can be 0)
%
%(v och w som funktioner av u ist'llet:)
%\begin{align}
%2a_1 u^m + \epsilon_{mnp}\varepsilon^{\alpha\beta\gamma}\lp 3a_3u^n_\beta u^p_\gamma + a_4\lp b_1^2u^n_\beta u^p_\gamma + c_1^2u^n_\beta u^p_\gamma\rp\rp &= \frac{4}{3}b_1u^m + \frac{4}{3}b_2\epsilon_{mnp}\varepsilon^{\alpha\beta\gamma}u^n_\beta u^p_\gamma\nn\\
%2a_2 b_1u_m + 2a_2b_2\epsilon_{mnp}\varepsilon^{\alpha\beta\gamma}u^n_\beta u^p_\gamma + \epsilon_{mnp}\varepsilon^{\alpha\beta\gamma} 2a_4b_1u^n_\beta u^p_\gamma &= -\frac{2}{3}u_m\nn\\
%2a_2 c_1u_m + 2a_2c_2\epsilon_{mnp}\varepsilon^{\alpha\beta\gamma}u^n_\beta u^p_\gamma + \epsilon_{mnp}\varepsilon^{\alpha\beta\gamma} 2a_4c_1u^n_\beta u^p_\gamma & = 0
%\end{align}
%i.e.\\ 
%First order: $c_1 = 0$,  $a_1 = \frac{2}{3}b_1$ and $a_2 = -\frac{1}{3b_1}$, $b_1\ne 0$\\
%Second order: $c_2 = 0$, $a_3 = \frac{1}{3}b_2$ and $a_4 = \frac{1}{3}\frac{b_2}{b_1^2}$ ($b_2$ can be 0)
%
%To fourth order, let
%\begin{align}
%\Phi &= a_1\tr u^2 + a_2\lp \tr v^2 + \tr w^2\rp + a_3\star(uuu) + a_4\lp\star(uvv)+\star(uww)\rp\nn\\
%&+a_5\lp\lp\tr\lp uv\rp\rp^2+\lp\tr\lp uw\rp\rp^2\rp + a_6\lp\tr u^2\rp^2 + a_7\tr u^2\tr\lp v^2+w^2\rp\nn\\
%&+a_8\lp\tr\lp v^2 + w^2\rp\rp^2 + a_9\tr u^4 + a_{10}\tr \lp u^2\lp v^2+w^2\rp\rp + a_{11}\tr\lp\lp v^2+w^2\rp^2\rp
%\end{align}
%with variations 
%\begin{align}
%\frac{\partial\Phi}{\partial u} &= 2a_1u + 3a_3\star(uu) + a_4\lp\star(vv)+\star(ww)\rp+2a_5\lp v\tr\lp uv\rp+w\tr\lp uw\rp\rp\nn\\
%& + 4a_6u\tr u^2 + 2a_7u\tr\lp v^2+w^2\rp + 4a_9u^3 + 2a_{10}u\lp v^2+w^2\rp
%\end{align}
%\begin{align}
%\frac{\partial\Phi}{\partial v} &= 2a_2v + 2a_4\star(uv) + 2a_5u\tr\lp uv\rp + 2a_7v\tr u^2\nn\\
%&+4a_8v\lp\tr\lp v^2 + w^2\rp\rp + 2a_{10}vu^2 + 4a_{11}v\lp v^2+w^2\rp
%\end{align}
%\begin{align}
%\frac{\partial\Phi}{\partial w} &= 2a_2w + 2a_4\star(uw) + 2a_5u\tr\lp uw\rp + 2a_7w\tr u^2\nn\\
%&+4a_8w\lp\tr\lp v^2 + w^2\rp\rp + 2a_{10}wu^2 + 4a_{11}w\lp v^2+w^2\rp
%\end{align}
%\begin{align}
%\sqrt{1+\Phi} = 1+\half\Phi = 1 + \half a_1\tr u^2 + \half a_2\lp \tr v^2 + \tr w^2\rp + \half a_3\star(uuu) + \half a_4\lp\star(uvv)+\star(uww)\rp
%\end{align}
%gives (third order equations) 
%\begin{align}
%\frac{\partial\Phi}{\partial u_m} &= v\Bigg[\frac{4}{3} + \frac{2}{3} a_1\tr u^2 + \frac{2}{3} a_2\lp \tr v^2 + \tr w^2\rp \Bigg]\nn\\
%\frac{\partial\Phi}{\partial v^m} & = u_m\Bigg[-\frac{2}{3} -\frac{1}{3}a_1\tr u^2 -\frac{1}{3}a_2\lp \tr v^2 + \tr w^2\rp \Bigg] + \frac{1}{3}\star(vw)\nn\\
%\frac{\partial\Phi}{\partial w^m} & = -\frac{1}{3}\star(vv)
%\end{align}
%
%To third order let
%\begin{align}
%u &= b_1v + b_2\star(vv) + b_3T_2v + b_4v^3\nn\\
%w &= c_1v + c_2\star(vv) + c_3T_2v + c_4v^3\nn\\
%u^2 &= b_1^2v^2 + \frac{2}{3}b_1b_2\star(vvv)\nn\\  
%w^2 &= c_1^2v^2 + \frac{2}{3}c_1c_2\star(vvv)\nn\\  
%uw &= b_1c_1v^2 + \frac{1}{3}(b_1c_2+c_1b_2)\star(vvv)  \nn\\
%uw^2 &= b_1c_1^2v^3\nn\\
%\star(uu) &= -4b_1b_2T_2v + b_1^2S_2 +4b_1b_2v^3\nn\\ 
%\star(uv) &= -2b_2T2v + b_1S2 + 2b_2v^3\nn\\
%\star(vw) &= -2c_2T2v + c_1S2 + 2c_2v^3\nn\\
%\star(uw) &= -2b_1c_2T_2v - 2b_2c_1T_2v + b_1c_1S_2 + 2b_1c_2v^3 + 2b_2c_1v^3 \nn\\
%\star(ww) & = -4c_1c_2T_2v + c_1^2S_2 + 4c_1c_2v^3\nn\\
%\end{align}
%The variations becomes (to third order)
%\begin{align}
%\frac{\partial\Phi}{\partial u} &= \lp 2a_1b_3 - 12a_3b_1b_2 - 4a_4c_1c_2 + 2a_5b_1(1+c_1^2) + 4a_6b_1^3 + 2a_7b_1(1+c_1^2)\rp T_2v\nn\\
%& + \lp 2a_1b_4 + 12a_3b_1b_2 + 4a_4c_1c_2 + 4a_9b_1^3 + 2a_{10}b_1(1+c_1^2) \rp v^3 \nn\\
%\end{align}
%\begin{align}
%\frac{\partial\Phi}{\partial v} &= \lp -4a_4b_2 + 2a_5b_1^2 + 2a_7b_1^2 + 4a_8(1+c_1^2)\rp T2v + \lp 4a_4b_2 + 2a_{10}b_1^2 + 4a_{11}(1+c_1^2)\rp v^3\nn\\
%\end{align}
%\begin{align}
%\frac{\partial\Phi}{\partial w} &= \lp 2a_2c_3 - 4a_4b_1c_2 - 4a_4b_2c_1 + 2a_5b_1^2c_1 + 2a_7b_1^2c_1 + 4a_8c_1(1+c_1^2)\rp T_2v\nn\\
%& + \lp 2a_2c_4 + 4a_4b_1c_2 + 4a_4b_2c_1 + 2a_{10}c_1b_1^2 + 4a_{11}c_1(1+c_1^2)\rp v^3\nn\\
%\end{align}
%\begin{align}
%\sqrt{1+\Phi} = 1 + \half a_1b_1T_2 + \half a_2(1+c_1^2)T_2
%\end{align}
%The equations becomes
%\begin{align}
%0 &= a_1b_3 - 6a_3b_1b_2 - 2a_4c_1c_2 + a_5b_1(1+c_1^2) + 2a_6b_1^3 + a_7b_1(1+c_1^2) - \frac{1}{3}a_1b_1 - \frac{1}{3}a_2(1+c_1^2)\nn\\
%0 &= a_1b_4 + 6a_3b_1b_2 + 2a_4c_1c_2 + 2a_9b_1^3 + a_{10}b_1(1+c_1^2)\nn\\
%0 &= -4a_4b_2 + 2a_5b_1^2 + 2a_7b_1^2 + 4a_8(1+c_1^2) + \frac{2}{3}b_3 + \frac{1}{3}a_1b_1^3 + \frac{1}{3}a_2b_1(1+c_1^2) + \frac{2}{3}c_2\nn\\
%0 &= 4a_4b_2 + 2a_{10}b_1^2 + 4a_{11}(1+c_1^2) + \frac{2}{3}b_4 - \frac{2}{3}c_2\nn\\
%0 &= 2a_2c_3 - 4a_4b_1c_2 - 4a_4b_2c_1 + 2a_5b_1^2c_1 + 2a_7b_1^2c_1 + 4a_8c_1(1+c_1^2)\nn\\
%0 &= 2a_2c_4 + 4a_4b_1c_2 + 4a_4b_2c_1 + 2a_{10}c_1b_1^2 + 4a_{11}c_1(1+c_1^2)\nn\\
%\end{align}
%The linear relation (to fourth order)
%\begin{align}
%\frac{\partial\Phi}{\partial u}u &= 2a_1u^2 + 3a_3u\star(uu) + a_4u\lp\star(vv)+\star(ww)\rp+2a_5\lp uv\tr\lp uv\rp+uw\tr\lp uw\rp\rp\nn\\
%& + 4a_6u^2\tr u^2 + 2a_7u^2\tr\lp v^2+w^2\rp + 4a_9u^4 + 2a_{10}u^2\lp v^2+w^2\rp\nn\\
%&= \lp 8a_1b_2^2 + 12a_3b_1^2b_2 + 4a_4b_2(1+c_1^2)\rp T_4\nn\\
%& + \lp 4a_1b_1b_3 + 8a_1b_2^2 + 12a_3b_1^2b_2 - 4a_4b_1c_1c_2 + 4a_4b_2(1+c_1^2) + 2a_5b_1^2(1+c_1^2) + 4a_6b_1^4 + 2a_7b_1^2(1+c_1^2)\rp T_2v^2 \nn\\
%& + \lp 4a_1b_1b_4 - 8a_1b_2^2 + 4a_4b_1c_1c_2 - 4a_4b_2(1+c_1^2) + 4a_9b_1^4 + 2a_{10}b_1^2(1+c^2)\rp v^4 \nn\\
%\end{align}
%\begin{align}
%\frac{\partial\Phi}{\partial v}v &= 2a_2v^2 + 2a_4v\star(uv) + 2a_5uv\tr\lp uv\rp + 2a_7v^2\tr u^2\nn\\
%&+4a_8v^2\lp\tr\lp v^2 + w^2\rp\rp + 2a_{10}v^2u^2 + 4a_{11}v^2\lp v^2+w^2\rp\nn\\
%&= 2a_2v^2 + 2a_4v\star(uv) + 2a_5uv\tr\lp uv\rp + 2a_7v^2\tr u^2\nn\\
%&+4a_8v^2\lp\tr\lp v^2 + w^2\rp\rp + 2a_{10}v^2u^2 + 4a_{11}v^2\lp v^2+w^2\rp\nn\\
%\end{align}
%\begin{align}
%\frac{\partial\Phi}{\partial w} &= 2a_2w^2 + 2a_4w\star(uw) + 2a_5uw\tr\lp uw\rp + 2a_7w^2\tr u^2\nn\\
%&+4a_8w^2\lp\tr\lp v^2 + w^2\rp\rp + 2a_{10}w^2u^2 + 4a_{11}w^2\lp v^2+w^2\rp\nn\\
%&= 2a_2w^2 + 2a_4w\star(uw) + 2a_5uw\tr\lp uw\rp + 2a_7w^2\tr u^2\nn\\
%&+4a_8w^2\lp\tr\lp v^2 + w^2\rp\rp + 2a_{10}w^2u^2 + 4a_{11}w^2\lp v^2+w^2\rp\nn\\
%\end{align}


% -----------------------------------------------------------------------------------------------------------------------------------
%\subsubsection{General case, $\gamma=\delta=0$}
%The equations to solve
%\begin{align}
%*k^m = \bigg[ - 2\lp\alpha_1+\frac{2}{3}\rp\omega_\parallel^m - 2\alpha_2\omega_\perp^m\bigg]\sqrt{1+\Phi}
%\end{align}
%\begin{align}
%j_{rm} & = \bigg\{ 
%2\lp\alpha_1-\frac{1}{3}\rp *f_m \hat p_{\parallel r} + 2\alpha_2*f_m \hat p_{\perp r} \nn\\
%& + \epsilon_{mnp}*\lp\omega^{sn} \wedge \omega^{tp}\rp\bigg[ 2\beta_1\W_{st}\hat p_{\parallel r} + 4\beta_1\W_{sr}\hat p_{\parallel t}\nn\\
%&\hspace{2cm} + \lp -2\beta_2 + \frac{1}{3}\rp \epsilon_{rs}\hat p_{\parallel t} + 6\beta_2\W_{sr}\hat p_{\perp t}\bigg]\bigg\}\sqrt{1+\Phi}
%\end{align}
%Letting
%\begin{align}
%u^m &= ?\nn\\
%v^m &= 2\lp\alpha_1+\frac{2}{3}\rp\omega_\parallel^m + 2\alpha_2\omega_\perp^m\nn\\
%w^m &= -2\alpha_2\omega_\parallel^m + 2\lp\alpha_1+\frac{2}{3}\rp\omega_\perp^m
%\end{align}
%gives
%\begin{align}
%*k^m = v^m\sqrt{1+\Phi}
%\end{align}
%\begin{align}
%j_{rm} & = \frac{\partial \Phi}{\partial \omega^{rm}} = \frac{\partial \Phi}{\partial u_{\alpha n}}\frac{\partial u_{\alpha n}}{\partial \omega^{rm}} + \frac{\partial \Phi}{\partial v^n_\alpha}\frac{\partial v^n_\alpha}{\partial \omega^{rm}} + \frac{\partial \Phi}{\partial w^n_\alpha}\frac{\partial w^n_\alpha}{\partial \omega^{rm}}\nn\\ 
%& = v^n\sqrt{1+\Phi}\frac{\partial u_{\alpha n}}{\partial \omega^{rm}} + \frac{\partial \Phi}{\partial v^n_\alpha}2\lbp \lp\alpha_1+\frac{2}{3}\rp\pdp{r} + \alpha_2\pdo{r}\rbp + \frac{\partial \Phi}{\partial w^n_\alpha}2\lbp -\alpha_2\pdp{r} + \lp\alpha_1+\frac{2}{3}\rp\pdo{r}\rbp\nn\\ 
%& = \bigg\{ 
%2\lp\alpha_1-\frac{1}{3}\rp *f_m \hat p_{\parallel r} + 2\alpha_2*f_m \hat p_{\perp r} \nn\\
%& + \epsilon_{mnp}*\lp\omega^{sn} \wedge \omega^{tp}\rp\bigg[ 2\beta_1\W_{st}\hat p_{\parallel r} + 4\beta_1\W_{sr}\hat p_{\parallel t}\nn\\
%&\hspace{2cm} + \lp -2\beta_2 + \frac{1}{3}\rp \epsilon_{rs}\hat p_{\parallel t} + 6\beta_2\W_{sr}\hat p_{\perp t}\bigg]\bigg\}\sqrt{1+\Phi}
%\end{align}
%The projection in the $\lp\alpha_1+\frac{2}{3}\rp\pdp{{}} + \alpha_2\pdo{{}}$ direction 
%\begin{align}
%\frac{\partial \Phi}{\partial v^n_\alpha} & = \lbp\lp\alpha_1+\frac{2}{3}\rp^2 + \alpha_2^2\rbp^{-1}\bigg\{ 
%- \half v^n\frac{\partial u_{\alpha n}}{\partial \omega^{rm}} + \lp\alpha_1-\frac{1}{3}\rp *f_m \hat p_{\parallel r} + \alpha_2*f_m \hat p_{\perp r} \nn\\
%& + \epsilon_{mnp}*\lp\omega^{sn} \wedge \omega^{tp}\rp\bigg[ \beta_1\W_{st}\hat p_{\parallel r} + 2\beta_1\W_{sr}\hat p_{\parallel t}\nn\\
%&\hspace{2cm} + \lp -\beta_2 + \frac{1}{6}\rp \epsilon_{rs}\hat p_{\parallel t} + 3\beta_2\W_{sr}\hat p_{\perp t}\bigg]\bigg\}\sqrt{1+\Phi}\lbp\lp\alpha_1+\frac{2}{3}\rp\pup{r} + \alpha_2\puo{r}\rbp\nn\\
%& = \lbp\lp\alpha_1+\frac{2}{3}\rp^2 + \alpha_2^2\rbp^{-1}\bigg\{ 
%- \half v^n\frac{\partial u_{\alpha n}}{\partial \omega^{rm}}\lbp\lp\alpha_1+\frac{2}{3}\rp\pup{r} + \alpha_2\puo{r}\rbp + \lp\alpha_1-\frac{1}{3}\rp\lp\alpha_1+\frac{2}{3}\rp *f_m + \alpha_2^2*f_m \nn\\
%& + \epsilon_{mnp}*\lp\omega^{sn} \wedge \omega^{tp}\rp\bigg[ \beta_1\lp\alpha_1+\frac{2}{3}\rp\W_{st} + 2\beta_1\hat p_{\parallel t}\lbp\lp\alpha_1+\frac{2}{3}\rp\pdp{s} + \alpha_2\pdo{s}\rbp\nn\\
%&\hspace{2cm} + \lp -\beta_2 + \frac{1}{6}\rp \hat p_{\parallel t}\lbp\lp\alpha_1+\frac{2}{3}\rp\pdo{s} - \alpha_2\pdp{s}\rbp + 3\beta_2\hat p_{\perp t}\lbp\lp\alpha_1+\frac{2}{3}\rp\pdp{s} + \alpha_2\pdo{s}\rbp\bigg]\bigg\}\sqrt{1+\Phi}\nn\\
%\end{align}
%The projection in the $-\alpha_2\pdp{{}} + \lp\alpha_1+\frac{2}{3}\rp\pdo{{}}$ direction 
%\begin{align}
%\frac{\partial \Phi}{\partial w^n_\alpha} & = \lbp\lp\alpha_1+\frac{2}{3}\rp^2 + \alpha_2^2\rbp^{-1}\bigg\{ 
%- \half v^n\frac{\partial u_{\alpha n}}{\partial \omega^{rm}} + \lp\alpha_1-\frac{1}{3}\rp *f_m \hat p_{\parallel r} + \alpha_2*f_m \hat p_{\perp r} \nn\\
%& + \epsilon_{mnp}*\lp\omega^{sn} \wedge \omega^{tp}\rp\bigg[ \beta_1\W_{st}\hat p_{\parallel r} + 2\beta_1\W_{sr}\hat p_{\parallel t}\nn\\
%&\hspace{2cm} + \lp -\beta_2 + \frac{1}{6}\rp \epsilon_{rs}\hat p_{\parallel t} + 3\beta_2\W_{sr}\hat p_{\perp t}\bigg]\bigg\}\sqrt{1+\Phi}\lbp -\alpha_2\pup{r} + \lp\alpha_1+\frac{2}{3}\rp\puo{r}\rbp\nn\\
%& = \lbp\lp\alpha_1+\frac{2}{3}\rp^2 + \alpha_2^2\rbp^{-1}\bigg\{ 
%- \half v^n\frac{\partial u_{\alpha n}}{\partial \omega^{rm}}\lbp -\alpha_2\pup{r} + \lp\alpha_1+\frac{2}{3}\rp\puo{r}\rbp - \alpha_2\lp\alpha_1-\frac{1}{3}\rp *f_m + \alpha_2\lp\alpha_1+\frac{2}{3}\rp*f_m \nn\\
%& + \epsilon_{mnp}*\lp\omega^{sn} \wedge \omega^{tp}\rp\bigg[ - \beta_1\alpha_2\W_{st} + 2\beta_1\hat p_{\parallel t}\lbp -\alpha_2\pdp{s} + \lp\alpha_1+\frac{2}{3}\rp\pdo{s}\rbp\nn\\
%&\hspace{2cm} + \lp -\beta_2 + \frac{1}{6}\rp \hat p_{\parallel t}\lbp -\alpha_2\pdo{s} - \lp\alpha_1+\frac{2}{3}\rp\pdp{s}\rbp + 3\beta_2\hat p_{\perp t}\lbp -\alpha_2\pdp{s} + \lp\alpha_1+\frac{2}{3}\rp\pdo{s}\rbp\bigg]\bigg\}\sqrt{1+\Phi}\nn\\ 
%\end{align}

% -----------------------------------------------------------------------------------------------------------------------------------
%\subsubsection{General variable substitution}
%The equations to solve
%\begin{align}
%*k^m = \bigg[ - 2\lp\alpha_1+\frac{2}{3}\rp\omega_\parallel^m - 2\alpha_2\omega_\perp^m\bigg]\sqrt{1+\Phi}
%\end{align}
%\begin{align}
%j_{rm} & = \bigg\{ 
%2\lp\alpha_1-\frac{1}{3}\rp *f_m \hat p_{\parallel r} + 2\alpha_2*f_m \hat p_{\perp r} \nn\\
%& + \epsilon_{mnp}*\lp\omega^{sn} \wedge \omega^{tp}\rp\bigg[ 2\beta_1\W_{st}\hat p_{\parallel r} + 4\beta_1\W_{sr}\hat p_{\parallel t}\nn\\
%&\hspace{2cm} + \lp -2\beta_2 + \frac{1}{3}\rp \epsilon_{rs}\hat p_{\parallel t} + 6\beta_2\W_{sr}\hat p_{\perp t}\bigg]\bigg\}\sqrt{1+\Phi}
%\end{align}
%We have the 1-forms $\omega_\parallel^m, \omega_\perp^m, *f^m, \epsilon_{mnp}*(\omega_\parallel^n\we\omega_\parallel^p), \epsilon_{mnp}*(\omega_\parallel^n\we\omega_\perp^p), \epsilon_{mnp}*(\omega_\perp^n\we\omega_\perp^p)$.
%Let
%\begin{align}
%u^m &= d_{11}*f^m + \lbp d_{12}\pdp{r} + d_{13}\pdo{r}\rbp\omega^{rm} + \epsilon_{mnp}*(\omega^{sn}\we\omega^{tp})\lbp d_{14}\pdp{s}\pdp{t} + d_{15}\pdp{s}\pdo{t} + d_{16}\pdo{s}\pdo{t}\rbp\nn\\
%v^m &= d_{21}*f^m + \lbp d_{22}\pdp{r} + d_{23}\pdo{r}\rbp\omega^{rm} + \epsilon_{mnp}*(\omega^{sn}\we\omega^{tp})\lbp d_{24}\pdp{s}\pdp{t} + d_{25}\pdp{s}\pdo{t} + d_{26}\pdo{s}\pdo{t}\rbp\nn\\
%w^m &= d_{31}*f^m + \lbp d_{32}\pdp{r} + d_{33}\pdo{r}\rbp\omega^{rm} + \epsilon_{mnp}*(\omega^{sn}\we\omega^{tp})\lbp d_{34}\pdp{s}\pdp{t} + d_{35}\pdp{s}\pdo{t} + d_{36}\pdo{s}\pdo{t}\rbp\nn\\
%\end{align}
%The variations becomes
%\begin{align}
%*k^m &= *\frac{\partial \Phi}{\partial f_m} = - \frac{\partial \Phi}{\partial *f_m} = -d_{11}\frac{\partial \Phi}{\partial u_m} - d_{21}\frac{\partial \Phi}{\partial v_m} - d_{31}\frac{\partial \Phi}{\partial w_m}\nn\\
%j_{rm} &= \frac{\partial \Phi}{\partial \omega^{rm}} = \frac{\partial \Phi}{\partial u_{\alpha n}}\frac{\partial u_{\alpha n}}{\partial \omega^{rm}} + \frac{\partial \Phi}{\partial v^n_\alpha}\frac{\partial v^n_\alpha}{\partial \omega^{rm}} + \frac{\partial \Phi}{\partial v^n_\alpha}\frac{\partial w^n_\alpha}{\partial \omega^{rm}}\nn\\
%& = \frac{\partial \Phi}{\partial u^m}\lbp d_{12}\pdp{r} + d_{13}\pdo{r}\rbp + \frac{\partial \Phi}{\partial v^m}\lbp d_{22}\pdp{r} + d_{23}\pdo{r}\rbp + \frac{\partial \Phi}{\partial w^m}\lbp d_{32}\pdp{r} + d_{33}\pdo{r}\rbp\nn\\
%& -2\epsilon_{mnp}\varepsilon_{\alpha\beta\gamma}\frac{\partial \Phi}{\partial u_{n\alpha}}\omega^{\gamma tp}d\xi^\beta\lbp d_{14}\pdp{r}\pdp{t} + d_{15}\pdp{r}\pdo{t} + d_{16}\pdo{r}\pdo{t}\rbp\nn\\
%& -2\epsilon_{mnp}\varepsilon_{\alpha\beta\gamma}\frac{\partial \Phi}{\partial v_{n\alpha}}\omega^{\gamma tp}d\xi^\beta\lbp d_{24}\pdp{r}\pdp{t} + d_{25}\pdp{r}\pdo{t} + d_{26}\pdo{r}\pdo{t}\rbp\nn\\
%& -2\epsilon_{mnp}\varepsilon_{\alpha\beta\gamma}\frac{\partial \Phi}{\partial w_{n\alpha}}\omega^{\gamma tp}d\xi^\beta\lbp d_{34}\pdp{r}\pdp{t} + d_{35}\pdp{r}\pdo{t} + d_{36}\pdo{r}\pdo{t}\rbp
%\end{align}
%
%We choose $v^m$ to correspond to the vielbein (c.f. \cite{artikeln}) (......... Lite mer text om det funkar! .........)
%\begin{align}
%G_{\alpha\beta}=g_{\alpha\beta} + \tilde\omega^m_\alpha\tilde\omega_{m\beta}
%\end{align}
%$v$ will be the 1-form coming from the other side, i.e. $u$ and $w$ are functions of $v$ and we create the action
%\begin{align}
%S = \int d^3\xi\lbp f(v)\rbp + S_{s.c.}
%\end{align}
%The Bianchi identities of the defined forms are
%\begin{align}
%du^m &= d_{11}d*f^m + \lbp d_{12}\pdp{r} + d_{13}\pdo{r}\rbp d\omega^{rm} + \epsilon_{mnp}d*(\omega^{sn}\we\omega^{tp})\lbp d_{14}\pdp{s}\pdp{t} + d_{15}\pdp{s}\pdo{t} + d_{16}\pdo{s}\pdo{t}\rbp\nn\\
%v^m &= d_{21}*f^m + \lbp d_{22}\pdp{r} + d_{23}\pdo{r}\rbp\omega^{rm} + \epsilon_{mnp}*(\omega^{sn}\we\omega^{tp})\lbp d_{24}\pdp{s}\pdp{t} + d_{25}\pdp{s}\pdo{t} + d_{26}\pdo{s}\pdo{t}\rbp\nn\\
%w^m &= d_{31}*f^m + \lbp d_{32}\pdp{r} + d_{33}\pdo{r}\rbp\omega^{rm} + \epsilon_{mnp}*(\omega^{sn}\we\omega^{tp})\lbp d_{34}\pdp{s}\pdp{t} + d_{35}\pdp{s}\pdo{t} + d_{36}\pdo{s}\pdo{t}\rbp\nn\\
%\end{align}
%
%At this level we should only have one 1-form with known Bianchi identity
%such that the equations of motion coming from the variation gives the Bianchi identities
%\begin{align}
%\qdp{r}d\omega^{rm} &= -\qdp{r}F^{rm}\nn\\
%df_m &= -H_m + \half\epsilon_{mnp}\epsilon_{st}F^{sn}\we\omega^{tp}
%\end{align}
%
%Try
%\begin{align}
%u^m &= d_{11}*f^m + \epsilon_{mnp}*(\omega^{sn}\we\omega^{tp})\lbp d_{14}\pdp{s}\pdp{t} + d_{15}\pdp{s}\pdo{t} + d_{16}\pdo{s}\pdo{t}\rbp\nn\\
%\end{align}
%Let $H_m=0$ (comes from other side).
%We want
%\begin{align}
%d*u_m &= -d\left[d_{11}f_m + \epsilon_{mnp}\omega^{sn}\we\omega^{tp}\lbp d_{14}\pdp{s}\pdp{t} + d_{15}\pdp{s}\pdo{t} + d_{16}\pdo{s}\pdo{t}\rbp\right]\nn\\
%& = \epsilon_{mnp}F^{sn}\we\omega^{tp}\left[ -\half d_{11}\epsilon_{st} + 2\lbp d_{14}\pdp{s}\pdp{t} + d_{15}\pdp{s}\pdo{t} + d_{16}\pdo{s}\pdo{t}\rbp \right]
%\end{align}
%If we don't know $d\tilde\omega^m = -\tilde F^m$, we cannot get such terms from the other side (because we vary with respect to $\phi$ and don;t know what $d\tilde\omega^m$ is), we are forced to use $d_{15}=0$ and $d_{14} = d_{16} = \frac{1}{4}$.

% -----------------------------------------------------------------------------------------------------------------------------------
%\subsubsection{Bengts bad term}
%It would be nice to make a variable substitution such that the dependence of the new variables on $\omega$ is either it's parallel or orthogonal projection on $p$, so that the $r$ and $m$ indices in \eqnref{solution_8d_jrm_orig} decouple, letting us project the equation on $\hat p$ and $\hat p_\perp$ to get conditions on the parameters $\alpha$ and $\beta$. 
%We try the following substitution according to the combinations of $p$, $\omega$ and $f$ entering the equations 
%\begin{align}
%u_m &= b*f_m + c_{(st)}\epsilon_{mnp}*(\omega^{sn}\we\omega^{tp})\nn\\
%v^m &= a\lbp\lp\alpha_1+\frac{2}{3}\rp\omega_\parallel^m + \alpha_2\omega_\perp^m\rbp = \qdp{r}\omega^{rm}\nn\\
%%\eqnlab{solution_8d_vartrans}
%\end{align}
%where $a$, $b$ and $c_{(st)}$ are constants constructed using combinations of $p_r$ and $\epsilon_{rs}$ and $\qdp{{}}\cdot\qdp{{}} = 1$, giving $a=\lbp\lp\alpha_1+\frac{2}{3}\rp^2 + \alpha_2^2\rbp^{-1/2}$.
%The variations of $\Phi(u(\omega,f),v(\omega))$ becomes
%\begin{align}
%*k^m &= *\frac{\partial \Phi}{\partial f_m} = *\lp\frac{\partial \Phi}{\partial u_{\alpha n}}\frac{\partial u_{\alpha n}}{\partial f_m}\rp = \half d\xi^\delta\varepsilon_{\delta\gamma\beta}\lp \frac{\partial \Phi}{\partial u_m^{\alpha}} \frac{b}{2}2\varepsilon^{\alpha\beta\gamma}\rp = -b\frac{\partial \Phi}{\partial u_m}\nn\\
%j_{rm} &= \frac{\partial \Phi}{\partial \omega^{rm}} = \frac{\partial \Phi}{\partial u_{\alpha n}}\frac{\partial u_{\alpha n}}{\partial \omega^{rm}} + \frac{\partial \Phi}{\partial v^n_\alpha}\frac{\partial v^n_\alpha}{\partial \omega^{rm}}\nn\\
%& = \frac{\partial \Phi}{\partial u_{n\alpha}}\frac{\partial}{\partial \omega^{rm}}\lp c_{(st)}\epsilon_{nn'p}\varepsilon_{\alpha\beta\gamma}\omega^{\beta sn'}\omega^{\gamma tp}\rp + \frac{\partial \Phi}{\partial v^m}\qdp{r}\nn\\
%& = -2c_{(rt)}\epsilon_{mnp}\varepsilon_{\alpha\beta\gamma}\frac{\partial \Phi}{\partial u_{n\alpha}}\omega^{\gamma tp}d\xi^\beta + \frac{\partial \Phi}{\partial v^m}\qdp{r}
%\end{align}
%and thus the duality relations are
%\begin{align}
%\frac{\partial \Phi}{\partial u_m} = bv^m\sqrt{1+\Phi}
%\eqnlab{solution_8d_km_trans}
%\end{align}
%and
%\begin{align}
%\qdp{r}&\frac{\partial \Phi}{\partial v^m} 
%= \bigg\{
%\frac{2}{b}\left[\lp\alpha_1-\frac{1}{3}\rp\pdp{r} + \alpha_2\pdo{r}\right]\lp u_m - c_{(st)}\epsilon_{mnp}*(\omega^{sn}\we\omega^{tp})\rp\nn\\
%& + \epsilon_{mnp}*\lp\omega^{sn} \wedge \omega^{tp}\rp\bigg[ 2\beta_1\W_{st}\hat p_{\parallel r} + 4\beta_1\W_{sr}\hat p_{\parallel t} + \lp -2\beta_2 + \frac{1}{3}\rp \epsilon_{rs}\pdp{t} + 6\beta_2\pdo{t}\W_{sr}\bigg]\nn\\
%& + \frac{2}{b}c_{(rt)}\epsilon_{mnp}\varepsilon_{\alpha\beta\gamma}\qdp{s}\omega^{\alpha sn}\omega^{\gamma tp}d\xi^{\beta} 
%\bigg\}\sqrt{1+\Phi}\nn\\
%&= \bigg\{
%\frac{2}{b}\left[\lp\alpha_1-\frac{1}{3}\rp\pdp{r} + \alpha_2\pdo{r}\right] u_m
%+ \epsilon_{mnp}*(\omega^{sn}\we\omega^{tp})\bigg[- \frac{2}{b}\left[\lp\alpha_1-\frac{1}{3}\rp\pdp{r} + \alpha_2\pdo{r}\right] c_{(st)}\nn\\
%& - 2b c_{(rt)}\qdp{s} + 2\beta_1\W_{st}\hat p_{\parallel r} + 4\beta_1\W_{sr}\hat p_{\parallel t} + \lp - 2\beta_2 + \frac{1}{3}\rp \epsilon_{rs}\pdp{t} + 6\beta_2 \pdo{t}\W_{sr}   
%\bigg]\bigg\}\sqrt{1+\Phi}
%\end{align}
%where we have used $\frac{\partial \Phi}{\partial u_m}$ from \eqnref{solution_8d_km_trans}.
%
%We now decompose the equation $e_r$ as 
%\begin{align}
%e_r = (e\cdot\qdp{{}})\qdp{{}} + (e\cdot\qdo{{}})\qdo{{}},   
%\end{align}  
%where $\qdo{{}}$ is defined from $\qdp{{}}\cdot\qdo{{}} = 0$ and $\qdo{{}}\cdot\qdo{{}}=1$ as 
%%\begin{align}
%%\qdp{{}} = \lbp\lp\alpha_1+\frac{2}{3}\rp^2 + \alpha_2^2\rbp^{-1/2}\lbp\lp\alpha_1+\frac{2}{3}\rp\pdp{{}} + \alpha_2\pdo{{}}\rbp
%%\end{align}
%%\begin{align}
%%\qdo{{}} = \lbp\lp\alpha_1+\frac{2}{3}\rp^2+\alpha_2^2\rbp^{-1/2}\lbp\alpha_2\pdp{{}} -\lp\alpha_1+\frac{2}{3}\rp\pdo{{}}\rbp
%%\end{align}
%\begin{align}
%\qdo{{}} = a\lbp\alpha_2\pdp{{}} -\lp\alpha_1+\frac{2}{3}\rp\pdo{{}}\rbp.
%\end{align}
%Multiplying the equation $e_r$ with $\qdo{{}}$ gives it's projection along the $\qdo{{}}$ direction 
%\begin{align}
%0 &= \bigg\{
%-\frac{2}{b}a\alpha_2u_m
%+ \epsilon_{mnp}*(\omega^{sn}\we\omega^{tp})\bigg[ \frac{2}{b}a\alpha_2c_{(st)}\nn\\
%& - 2b \quo{r}c_{(rt)}\qdp{s} + 2\beta_1a\alpha_2\W_{st} + 4\beta_1\qdo{s}\hat p_{\parallel t} - \lp - 2\beta_2 + \frac{1}{3}\rp \qdp{s}\pdp{t} + 6\beta_2 \pdo{t}\qdo{s}   
%\bigg]\bigg\}\sqrt{1+\Phi}\nn\\
%&= \bigg\{
%-\frac{2}{b}a\alpha_2u_m
%+ \epsilon_{mnp}*(\omega^{sn}\we\omega^{tp})\bigg[ \frac{2}{b}a\alpha_2c_{(st)}\nn\\
%& - 2ab\lbp\alpha_2\pup{r} -\lp\alpha_1+\frac{2}{3}\rp\puo{r}\rbp c_{(rt)}a\lbp\lp\alpha_1+\frac{2}{3}\rp\pdp{s} + \alpha_2\pdo{s}\rbp + 2\beta_1a\alpha_2\W_{st} + 4\beta_1a\lbp\alpha_2\pdp{s} -\lp\alpha_1+\frac{2}{3}\rp\pdo{s}\rbp\hat p_{\parallel t}\nn\\
%& - \lp - 2\beta_2 + \frac{1}{3}\rp a\lbp\lp\alpha_1+\frac{2}{3}\rp\pdp{s} + \alpha_2\pdo{s}\rbp\pdp{t} + 6\beta_2 \pdo{t}a\lbp\alpha_2\pdp{s} -\lp\alpha_1+\frac{2}{3}\rp\pdo{s}\rbp   
%\bigg]\bigg\}\sqrt{1+\Phi}
%\end{align}
%When $w=0$ we must have $\alpha_2 = 0$.
%For this equation to hold the factor multiplying $*(\omega\we\omega)$ must be zero and gives conditions for $c$ and $\beta$. The only symmetric combinations we can come up with for $c_{rs}$ are $p^u\epsilon_{u(r}p_{s)}$, $\W_{rs}$ and $p_rp_s$. The only term that cancels the $\epsilon$ is the first so we let
%\begin{align}
%c_{(rs)}=c_1\pdp{r}\pdp{s} + c_2\pdo{(r}\pdp{s)} + c_3\pdo{r}\pdo{s}
%\end{align}
%and get
%\begin{align}
%0 &= 2ab \lp\alpha_1+\frac{2}{3}\rp\lbp \half c_2\pdp{t} + c_3\pdo{t} \rbp\pdp{s} - 4\beta_1\pdo{s}\pdp{t}\nn\\
%& - \lp - 2\beta_2 + \frac{1}{3}\rp\pdp{s}\pdp{t} - 6\beta_2 \pdo{s}\pdo{t}   
%\end{align}
%$\qdp{{}}\ne 0$ implies $\alpha_1\ne-\frac{2}{3}$ and thus $\beta_2 = 0$,
%\begin{align}
%c_2 &= \frac{1}{3ab}\lp\alpha_1+\frac{2}{3}\rp^{-1} = \frac{1}{3b}\mbox{ and}\nn\\
%c_3 &= 2\beta_1\lp\alpha_1+\frac{2}{3}\rp^{-1}    
%\end{align}
%
%Multiplying the equation $e_r$ with $\qdp{{}}$ gives it's projection along the $\qdp{{}}$ direction 
%\begin{align}
%\frac{\partial \Phi}{\partial v^m} 
%&= \bigg\{
%\frac{2a}{b}\left[\lp\alpha_1-\frac{1}{3}\rp\lp\alpha_1+\frac{2}{3}\rp + \alpha_2^2\right] u_m
%+ \epsilon_{mnp}*(\omega^{sn}\we\omega^{tp})\bigg[- \frac{2a}{b}\left[\lp\alpha_1-\frac{1}{3}\rp\lp\alpha_1+\frac{2}{3}\rp + \alpha_2^2\right]c_{(st)}\nn\\
%& - 2b\qup{r}c_{(rt)}\qdp{s} + 2\beta_1\W_{st} + 4\beta_1\qdp{s}\hat p_{\parallel t} + \lp - 2\beta_2 + \frac{1}{3}\rp \qdo{s}\pdp{t} + 6\beta_2 \pdo{t}\qdp{s}   
%\bigg]\bigg\}\sqrt{1+\Phi}
%\end{align}
%which with the conditions found above becomes
%\begin{align}
%%\qdp{{}} &= a\lp\alpha_1+\frac{2}{3}\rp\pdp{{}}\nn\\
%%c_{(rs)} & =\lbp c_1\pdp{r}\pdp{s} + \frac{b}{3a}\lp\alpha_1+\frac{2}{3}\rp^{-1}\pdo{(r}\pdp{s)} + 2\beta_1\lp\alpha_1+\frac{2}{3}\rp^{-1}\pdo{r}\pdo{s}\rbp\nn\\
%\frac{\partial \Phi}{\partial v^m} 
%&= \bigg\{
%\frac{2a}{b}\lp\alpha_1-\frac{1}{3}\rp\lp\alpha_1+\frac{2}{3}\rp u_m\nn\\
%& + \epsilon_{mnp}*(\omega^{sn}\we\omega^{tp})\bigg[- \frac{2a}{b}\lp\alpha_1-\frac{1}{3}\rp\lp\alpha_1+\frac{2}{3}\rp \lbp c_1\pdp{s}\pdp{t} + \frac{b}{3a}\lp\alpha_1+\frac{2}{3}\rp^{-1}\pdo{(s}\pdp{t)} + 2\beta_1\lp\alpha_1+\frac{2}{3}\rp^{-1}\pdo{s}\pdo{t}\rbp\nn\\
%& - 2ba^2\lp\alpha_1+\frac{2}{3}\rp^2\lbp c_1\pdp{t} + \frac{b}{6a}\lp\alpha_1+\frac{2}{3}\rp^{-1}\pdo{t}\rbp\pdp{s} + 2\beta_1\W_{st} + 4\beta_1a\lp\alpha_1+\frac{2}{3}\rp\pdp{s}\hat p_{\parallel t} - \frac{1}{3}a\lp\alpha_1+\frac{2}{3}\rp\pdo{s}\pdp{t}   
%\bigg]\bigg\}\sqrt{1+\Phi}\nn\\
%& = \bigg\{
%\frac{2a}{b}\lp\alpha_1-\frac{1}{3}\rp\lp\alpha_1+\frac{2}{3}\rp u_m\nn\\
%& + \epsilon_{mnp}*(\omega^{sn}\we\omega^{tp})\bigg[\pdp{s}\pdp{t}\left[- \frac{2a}{b}\lp\alpha_1-\frac{1}{3}\rp\lp\alpha_1+\frac{2}{3}\rp c_1 - 2ba^2\lp\alpha_1+\frac{2}{3}\rp^2c_1 + 2\beta_1 + 4\beta_1a\lp\alpha_1+\frac{2}{3}\rp\right]\nn\\
%& + \pdp{(s}\pdo{t)}\left[- \frac{2}{3}\lp\alpha_1-\frac{1}{3}\rp - \frac{ab^2}{3}\lp\alpha_1+\frac{2}{3}\rp - \frac{1}{3}a\lp\alpha_1+\frac{2}{3}\rp\right]\nn\\
%& + \pdo{s}\pdo{t}\left[- \frac{2a}{b}2\beta_1\lp\alpha_1-\frac{1}{3}\rp + 2\beta_1\right]
%\bigg]\bigg\}\sqrt{1+\Phi}
%\end{align}
%We have assumed the orthogonal projection $w^m$ doesn't enter the right hand side, so $\Phi$ independent on $w^m$. This means the factors multiplying $\pdp{(s}\pdo{t)}$ and $\pdo{s}\pdo{t}$ must be $0$. 
%We get (using $a$ as it is defined)
%\begin{align}
%0 &= \alpha_1-\frac{1}{3} + \alpha_1+\frac{2}{3}\nn\\
%0 &= \beta_1\lbp\frac{2}{b}\lp\alpha_1-\frac{1}{3}\rp - 1\rbp 
%\end{align}
%giving $\alpha_1=-\frac{1}{6}$ and $b=-1$.
%
%The factor multiplying $\pdp{s}\pdp{t}$
%\begin{align}
%c_1 + 6\beta_1
%\end{align}
%which can be chosen to be 0, we let $c_1=\beta_1 = 0$.
%
%%The paper equations with the given parameter values becomes
%\begin{align}
%j_{rm} & = \bigg\{ 
%2\lp\alpha_1-\frac{1}{3}\rp *f_m \hat p_{\parallel r} + \epsilon_{mnp}*\lp\omega^{sn} \wedge \omega^{tp}\rp\bigg[ \lp -2\beta_2 + \frac{1}{3}\rp \epsilon_{rs}\pdp{t} + 6\beta_2\W_{sr}\pdo{t}\bigg]\bigg\}\sqrt{1+\Phi}\nn\\
%& = \bigg\{-*f_m \pdp{r} + \epsilon_{mnp}*\lp\omega^{sn} \wedge \omega^{tp}\rp\bigg[ \frac{1}{3}\epsilon_{rs}\pdp{t} \bigg]\bigg\}\sqrt{1+\Phi}
%\end{align}
%We have
%\begin{align}
%*f_m &= bu_m + c\pdo{(s}\pdp{t)}\epsilon_{mnp}*(\omega^{sn}\we\omega^{tp})\nn\\
%\end{align}
%and
%\begin{align}
%j_{rm} &= -2c\pdo{(r}\pdp{t)}\epsilon_{mnp}\varepsilon_{\alpha\beta\gamma}\frac{\partial \Phi}{\partial u_{n\alpha}}\omega^{\gamma tp}d\xi^\beta + \frac{\partial \Phi}{\partial v^m}\pdp{r}\nn\\
%& = -2c\pdo{(r}\pdp{t)}\epsilon_{mnp}\varepsilon_{\alpha\beta\gamma}\lp\pdp{s}\omega^{\alpha sn}\rp\omega^{\gamma tp}d\xi^\beta\sqrt{1+\Phi} + \frac{\partial \Phi}{\partial v^m}\pdp{r}\nn\\
%& = 2c\pdo{(r}\pdp{t)}\pdp{s}\epsilon_{mnp}*\lp\omega^{sn}\we\omega^{tp}\rp\sqrt{1+\Phi} + \frac{\partial \Phi}{\partial v^m}\pdp{r}
%\end{align}
%giving
%\begin{align}
%\frac{\partial \Phi}{\partial v^m}\pdp{r} &=\bigg\{ 
%-*f_m \pdp{r} + \epsilon_{mnp}*\lp\omega^{sn} \wedge \omega^{tp}\rp\bigg[ \frac{1}{3}\epsilon_{rs}\pdp{t} - 2c\pdo{(r}\pdp{t)}\pdp{s}\bigg]\bigg\}\sqrt{1+\Phi}\nn\\ 
%& = -bu_m\pdp{r} + \epsilon_{mnp}*\lp\omega^{sn} \wedge \omega^{tp}\rp\bigg[- c\pdo{(s}\pdp{t)}\pdp{r} + \frac{1}{3}\epsilon_{rs}\pdp{t} - 2c\pdo{(r}\pdp{t)}\pdp{s}\bigg]\bigg\}\sqrt{1+\Phi} 
%\end{align}
%(Fel! Ska nog vara en extra faktor 2 i $c_2$, g[r iaf att l;sa vilket inneb'r att det som gjorts i pappret kan vara specialvallet av denna l;sningen med $\omega_\perp = 0$, s[ inga laddningar i $\Phi$). D.v.s. denna l;sningen som kommer fr[n att man plockar bort $w^m$ fr[n ekvationerna har vara den l;sningen utan laddningar och pappersl;sningen beh;ver inte vara ett specialfall av den allm'nna l;sningen (eftersom den 'r ett specialfall av denna l;sningen!).

%\begin{align}
%- c\pdo{(s}\pdp{t)} + \frac{1}{3}\pdo{s}\pdp{t} - c\pdo{t}\pdp{s} = 0\\
%\frac{1}{3}\pdp{s}\pdp{t} - c\pdp{t}\pdp{s} = 0\\
%\end{align}
%
%\begin{align}
%a|p|\frac{\partial \Phi}{\partial v^m} 
%&= \bigg\{
%\frac{2}{b}\lp\alpha_1-\frac{1}{3}\rp u_m
%+ \epsilon_{mnp}*(\omega^{sn}\we\omega^{tp})\bigg[- \frac{2}{b}\lp\alpha_1-\frac{1}{3}\rp \lp c_1\pdp{s}\pdp{t} -\frac{b}{6|p|^2}\lp\alpha+\frac{2}{3}\rp^{-1}\pdo{(s}\pdp{t)}\rp\nn\\
%& - \frac{2}{b} \lp c_1\pdp{t} -\frac{b}{12|p|^2}\lp\alpha+\frac{2}{3}\rp^{-1}\pdo{t}\rp\lp\alpha+\frac{2}{3}\rp\pdp{s} + \frac{1}{3}\pdo{s}\pdp{t}    
%\bigg]\bigg\}\sqrt{1+\Phi}
%\end{align}
%
%\begin{align}
%a|p|\frac{\partial \Phi}{\partial v^m} 
%&= \bigg\{
%\frac{2}{b}\lp\alpha-\frac{1}{3}\rp u_m 
%+ \epsilon_{mnp}*(\omega^{sn}\we\omega^{tp})\bigg[\nn\\
%& \frac{1}{3}\bigg\{ \lp\alpha-\frac{1}{3}\rp\lp\alpha+\frac{2}{3}\rp^{-1} + 2  
%\bigg\}p^u\epsilon_{us}p_{t} \bigg]\bigg\}\sqrt{1+\Phi}
%\end{align}
%This $SL(2,\rr)$ scalar equation $(e\cdot\hat p_\parallel)$ is the same for the two values of $r$ in $\hat p_\parallel$ meaning the starting $2\cdot 3 + 3 = 9$ equations in the $SL(2,\rr)$ and $SL(3,\rr)$ indices has come down to $1\cdot 3+3=6$ equations. 
%According to (.........  ..........) there should only be three scalar fields in the theory so we are left with what appears to be 6 equations of motion for 3 scalars.
%To completely remove the $p$ dependence in the duality relations we first choose $a=1/|p|$ meaning $v^m = \omega_\parallel^m$, i.e. $v^m$ is the projection of $\omega^m$ along the constant $\hat p_\parallel$ direction.
%Since there could be no $p_sp_t = |p|^2\hat p_\parallel\otimes \hat p_\parallel$ term in $c_{(rs)}$ we cannot rewrite both the $\omega$:s in terms of $v$ and thus we have to choose $\alpha = -1/3$ so the $*(\omega\we\omega)$ term becomes zero. 
%We also choose $b = 1$.
%The duality relations thus becomes
%\begin{align}
%\frac{\partial \Phi}{\partial u_m} &= \frac{2}{3} v^m\sqrt{1+\Phi}\nn\\
%\frac{\partial \Phi}{\partial v^m} &= -\frac{4}{3} u_m \sqrt{1+\Phi}
%\eqnlab{solution_equations_8d_final}
%\end{align}
%which is our final form of the duality relations for the $d=8$ $D2$ case with $\alpha$ and $\beta$ constant. We will try and solve these in \secref{csolution_8d_const}.
%
%To see whether these equations are reasonable or not we series expand $\Phi$ to third order
%\begin{align}
%\Phi &= a_0 u_\alpha^mu^\alpha_m + a_1 u_\alpha^mv^\alpha_m + a_2 v_\alpha^mv^\alpha_m\nn\\
%& + \epsilon_{mnp}\varepsilon^{\alpha\beta\gamma}\lp a_3u_\alpha^mu_\beta^nu_\gamma^p + a_4u_\alpha^mu_\beta^nv_\gamma^p + a_5u_\alpha^mv_\beta^nv_\gamma^p + a_6v_\alpha^mv_\beta^nv_\gamma^p \rp.
%\end{align}
%Since there should only be 3 scalars in the theory and we have 6 scalars in the 1-forms $u$ and $v$, there must be a relation $u = u(v)$ such that as to get the indices right 
%\begin{align}
%u_\alpha^m = b_1v_\alpha^m + b_2\varepsilon_{\alpha\beta\gamma}\epsilon^{mnp} v^\beta_nv^\gamma_p + \Ordo(v^3). 
%\end{align}
%The series expansion of the duality equations to second order in $u$ and $v$ thus becomes
%\begin{align}
%2 a_0 u^\alpha_m + a_1 v^\alpha_m + \epsilon_{mnp}\varepsilon^{\alpha\beta\gamma}\lp 3 a_3u_\beta^nu_\gamma^p + 2a_4u_\beta^nv_\gamma^p + a_5v_\beta^nv_\gamma^p \rp &= \frac{2}{3} v_m^\alpha\nn\\
% a_1 v^\alpha_m + a_2 v^\alpha_m + \epsilon_{mnp}\varepsilon^{\alpha\beta\gamma}\lp a_4u_\beta^nu_\gamma^p + 2a_5u_\beta^nv_\gamma^p + 3a_6v_\beta^nv_\gamma^p \rp &= -\frac{4}{3} u_m^\alpha
%\end{align}
%and using $u = u(v)$ to second order in $v$
%\begin{align}
%0 &= 2a_0b_1v^\alpha_m + 2a_0b_2\varepsilon^{\alpha\beta\gamma}\epsilon_{mnp} v_\beta^nv_\gamma^p + a_1v^\alpha_m + \epsilon_{mnp}\varepsilon^{\alpha\beta\gamma}\lp 3 a_3b_1^2v^\beta_nv^\gamma_p + 2a_4b_1v^\beta_nv_\gamma^p + a_5v_\beta^nv_\gamma^p \rp - \frac{2}{3} v_m^\alpha \nn\\
%0 &= a_1 v^\alpha_m + a_2 v^\alpha_m + \epsilon_{mnp}\varepsilon^{\alpha\beta\gamma}\lp a_4b_1^2v^\beta_nv^\gamma_p + 2a_5b_1v^\beta_nv_\gamma^p + 3a_6v_\beta^nv_\gamma^p \rp + \frac{4}{3}b_1v^\alpha_m + \frac{4}{3}b_2\varepsilon^{\alpha\beta\gamma}\epsilon_{mnp} v_\beta^nv_\gamma^p
%\end{align}
%from which we can read of
%\begin{align}
%0 &= 2a_0b_1 + a_1 - \frac{2}{3}\nn\\
%0 &= 2a_0b_2 + 3a_3b_1^2 + 2a_4b_1 + a_5
%\end{align}
%from the first equation and
%\begin{align}
%0 &= a_1 + a_2 + \frac{4}{3}b_1\nn\\
%0 &= a_4b_1^2 + 2a_5b_1 + 3a_6 + \frac{4}{3}b_2
%\end{align}
%from the second equation.
%Demanding the equations to be equal and thus contain the same information we (......... Ja, vad�? Det kommer ju alltid vara mycket fler coefficienter �n villkor, vilket g�r att vi inte kan s�ga n�got om h�gre ordningars expansioner .........)
%

\section{Duality equations of the $d=9$ $D1$, const $\alpha$ case}
We restate the equations of motion found earlier with $\alpha$ constant
\begin{align}
&1 + \Phi - *f_m*f_n\M^{mn}=0\\
&d \Big{[}\lambda \M^{mn}*f_n\Big{]}=0\\
&d\Big{[}\lambda *j_{1m} - 2 \lambda \alpha_{m'm} \omega^2 *f_{n'} \M^{m'n'}\Big{]}=0\\
&d\Big{[}\lambda*j_2 - 2\alpha_{mn}\omega^{1n} *f_{n'}\M^{mn'}\Big{]} - 2\lambda \epsilon_{mn} F^{1n}*f_{n'}\M^{mn'}=0,
\end{align}
As in the $d=8$ $D2$ case we can identify the charges $p^m=\lambda \M^{mn}*f_n$ and together with the Bianchi identity $d\omega^{1m}=-F^{1m}$ we can rewrite the last two equations as
\begin{align}
&d\Big{[}\frac{|p|}{\sqrt{1+\Phi}}*j_{1m} - 2 \alpha_{m'm} \alpha \omega^2 p^{m'}\Big{]}=0\nn\\
&d\Big{[}\frac{|p|}{\sqrt{1+\Phi}}*j_2 + 2\lp -\alpha_{mn} + \epsilon_{mn}\rp\omega^{1n} p^m \Big{]}=0
\end{align}
We will let $\alpha_{mn} = \alpha_1\M_{mn} + \alpha_2\epsilon_{mn}$.
Integration gives 
\begin{align}
&*j_{1m} = \lbp 2\alpha_1\hat p_{\parallel m}\omega^2 + 2\alpha_2\hat p_{\perp m}\omega^2 + \frac{\Gamma_{1m}}{|p|} \rbp\sqrt{1+\Phi}\nn\\
&*j_2 = \lbp 2\lp\alpha_2 -1\rp\omega_\perp^{1} + 2\alpha_1\omega_\parallel^{1} + \frac{\Gamma_{2}}{|p|} \rbp\sqrt{1+\Phi}
\end{align}
where $\Gamma$ are 1-forms such that $d\Gamma=0$. We will use $\Gamma = 0$. 
\\
Similar to before, we make the variable substitution
\begin{align}
u &= *\omega^2\nn\\
v &= \omega^1_\parallel\nn\\
w &= \omega^1_\perp
\end{align}
giving the variations
\begin{align}
j_{1m} &= \frac{\partial \Phi}{\partial \omega^{1m}} = \frac{\partial \Phi}{\partial v_\alpha}\frac{\partial v_\alpha}{\partial \omega^{1m}} + \frac{\partial \Phi}{\partial w_\alpha}\frac{\partial w_\alpha}{\partial \omega^{1m}}= \frac{\partial \Phi}{\partial v}p_{\parallel m} + \frac{\partial \Phi}{\partial w}p_{\perp m}\nn\\
*j_2 &= *\frac{\partial \Phi}{\partial \omega^2} = *\lp\frac{\partial \Phi}{\partial u_{\alpha}}\frac{\partial u_{\alpha}}{\partial \omega^2}\rp = d\xi^\gamma\varepsilon_{\gamma\beta}\lp \frac{\partial \Phi}{\partial u^{\alpha}} \varepsilon^{\alpha\beta}\rp = \frac{\partial \Phi}{\partial u}
\end{align}
which in turn gives the duality relations
\begin{align}
\frac{\partial \Phi}{\partial u} &= 2\alpha_1 v + 2\lp\alpha_2 -1\rp w \sqrt{1+\Phi}\\
\frac{\partial \Phi}{\partial v} &= - 2 \alpha_1 u\sqrt{1+\Phi}\\
\frac{\partial \Phi}{\partial w} &= - 2 \alpha_2 u\sqrt{1+\Phi}
\end{align}
There should only be one scalar degree of freedom in the theory, but we have 3 equations for scalar field strengths, so the field strengths should be related by some duality relations.  
We will discuss these equations some more in section \secref{csolution_9d_const}.


%\paragraph{General case, series expansion, no parameters}
%The equations we are about to solve are
%\begin{align}
%*j_2 &= -2\omega_\perp^{1}\sqrt{1+\Phi}\nn\\
%j_{1m} &= 0
%\end{align}
%
%Use the shorthand notation
%\begin{align}
%u & =*\omega^2,\mbox{ and }u^2 = *\omega^2_\alpha*\omega^{2\alpha}\nn\\
%v & = \pdp{m}\omega^{1m}\nn\\
%w & = \pdo{m}\omega^{1m}\nn\\
%\end{align} 
%to get the equations
%\begin{align}
%\frac{\partial\Phi}{\partial u} &= -2w\sqrt{1+\Phi}\nn\\
%\frac{\partial\Phi}{\partial v} &= 0\nn\\
%\frac{\partial\Phi}{\partial w} &= 0\nn\\
%\end{align}
%
%To second order, let 
%\begin{align}
%\Phi = a_1u^2 + a_2\lp v^2+w^2\rp
%\end{align}
%with variations
%\begin{align}
%\frac{\partial\Phi}{\partial u}= 2a_1u\nn\\
%\frac{\partial\Phi}{\partial v} = 2a_2v\nn\\
%\frac{\partial\Phi}{\partial w} = 2a_2w
%\end{align}
%The equations to second order
%\begin{align}
%2a_1u &= -2w\nn\\
%2a_2v &= 0\nn\\
%2a_2w &= 0
%\end{align}
%Let
%\begin{align}
%u = b_1 v\nn\\
%w = c_1 v
%\end{align} 
%gives
%\begin{align}
%a_1b_1v &= -c_1v\nn\\
%a_2v &= 0\nn\\
%a_2c_1v &= 0
%\end{align}
%with solutions\\
%$a_2 = 0, c_1 = -a_1b_1$\\
%
%Next we examine the order 4 expansion of $\Phi$
%\begin{align}
%\Phi = a_1u^2 + a_3\lp u^2\rp^2 + a_4u^2\lp v^2+w^2\rp + a_5\lp(uv)^2+(uw)^2\rp + a_6\lp v^2+w^2\rp^2 
%\end{align}
%with variations
%\begin{align}
%\frac{\partial\Phi}{\partial u} &= 2a_1u + 4a_3uu^2 + 2a_4u\lp v^2+w^2\rp + 2a_5\lp v(uv)+w(uw)\rp\nn\\
%\frac{\partial\Phi}{\partial v} &= 2a_4u^2v + 2a_5u(uv) + 4a_6v\lp v^2+w^2\rp\nn\\
%\frac{\partial\Phi}{\partial w} &= 2a_4u^2w + 2a_5u(uw) + 4a_6w\lp v^2+w^2\rp\nn\\
%\end{align}
%We insert this into the duality equations, using $\sqrt{1+\Phi} = 1 + \frac{\Phi}{2} + \Ordo{(u^4,v^4)}$
%\begin{align}
%2a_1u + 4a_3uu^2 + 2a_4u\lp v^2+w^2\rp + 2a_5\lp v(uv)+w(uw)\rp = -w(2+a_1u^2)\nn\\
%2a_4u^2v + 2a_5u(uv) + 4a_6v\lp v^2+w^2\rp = 0\nn\\
%2a_4u^2w + 2a_5u(uw) + 4a_6w\lp v^2+w^2\rp = 0\nn\\
%\end{align}
%We use (no dualities) 
%\begin{align}
%u &= b_1v + b_2vv^2\nn\\
%w &= -a_1b_1v + c_2vv^2\nn\\
%\end{align}
%to get the equations
%\begin{align}
%2a_1b_2vv^2 + 4a_3b_1^3vv^2 + 2a_4b_1v\lp 1+a_1^2b_1^2\rp v^2 + 2a_5\lp b_1+a_1^2b_1^3\rp v(vv) = 2c_2vv^2+a_1b_1^3a_1vv^2\nn\\
%2a_4b_1^2v^2v + 2a_5b_1^2vv^2 + 4a_6\lp 1+a_1^2b_1^2\rp vv^2 = 0\nn\\
%2a_4b_1^2c_1v^2v + 2a_5b_1^2c_1vv^2 + 4a_6c_1\lp 1+a_1^2b_1^2\rp vv^2 = 0\nn\\
%\end{align}
%
%A general term $\phi_{ij}$ of order $2(i+j)+4k$ in the ansatz will be on the form
%\begin{align}
%\phi_{ijk} = a_{ijk}u^{2i}(v^2+w^2)^{j}((uv)^{2}+(uw)^{2})^k  
%\end{align}
%i.e. all terms in $\Phi$ of order $2n$ are given by
%\begin{align}
%\Phi^{(n)} = \sum_{k=0}^{[n/2]}\sum_{j=0}^{n-2k}\sum_{i=0}^{n-2k-j} a_{ijk}u^{2i}(v+w)^{2j}((uv)^{2}+(uw)^{2})^k
%\end{align}
%with variations
%\begin{align}
%\frac{\partial\Phi^{(n)}}{\partial u} &= \sum_{k=0}^{[n/2]}\sum_{j=0}^{n-2k}\sum_{i=0}^{n-2k-j} a_{ijk} \lbp 2iuu^{2(i-1)}(v^2+w^2)^{j}((uv)^{2}+(uw)^{2})^k + 2k(v(uv)+w(uw)) u^{2i}(v^2+w^2)^{j}((uv)^{2}+(uw)^{2})^{k-1}\rbp\nn\\
%\frac{\partial\Phi^{(n)}}{\partial v} &= \sum_{k=0}^{[n/2]}\sum_{j=0}^{n-2k}\sum_{i=0}^{n-2k-j} a_{ijk} \lbp 2jvu^{2i}(v^2+w^2)^{j-1}((uv)^{2}+(uw)^{2})^k + 2ku(uv)u^{2i}(v^2+w^2)^{j}((uv)^{2}+(uw)^{2})^{k-1}\rbp\nn\\
%\frac{\partial\Phi^{(n)}}{\partial w} &= \sum_{k=0}^{[n/2]}\sum_{j=0}^{n-2k}\sum_{i=0}^{n-2k-j} a_{ijk} \lbp 2jwu^{2i}(v^2+w^2)^{j-1}((uv)^{2}+(uw)^{2})^k + 2ku(uw)u^{2i}(v^2+w^2)^{j}((uv)^{2}+(uw)^{2})^{k-1}\rbp\nn\\
%\end{align}

%\paragraph{General case, series expansion}
%Flytta kanske till efter computer solutions
%
%The equations we are about to solve are
%\begin{align}
%*j_2 &= \lbp 2\lp\alpha_2 -1\rp\omega_\perp^{1} + 2\alpha_1\omega_\parallel^{1} \rbp\sqrt{1+\Phi}\nn\\
%j_{1m} &= \lbp 2\alpha_1\hat p_{\parallel m}*\omega^2 + 2\alpha_2\hat p_{\perp m}*\omega^2 \rbp\sqrt{1+\Phi}
%\end{align}
%
%When $\alpha$ is constant the right hand side of the equations is independent of variations of $\epsilon_{mn}*(\omega^{1m}\we\omega^{1n})$ and similar terms, so we construct $\Phi$ from contractions of $\omega^{1m}$ and $\omega^{2}$ only.
%Use the shorthand notation
%\begin{align}
%u & =*\omega^2,\mbox{ and }u^2 = *\omega^2_\alpha*\omega^{2\alpha}\nn\\
%v^m & = \omega^{1m},\mbox{ }v^2 = \omega^{1m}_\alpha\omega^{1\beta}_m,\mbox{ and }\tr v^2 = \omega^{1m}_\alpha\omega^{1\alpha}_m,
%\end{align} 
%i.e. we consider $v^2$ as $2\times 2$-matrices.
%To second order, let 
%\begin{align}
%\Phi = a_1u^2 + a_2\tr v^2
%\end{align}
%with variations
%\begin{align}
%*\frac{\partial\Phi}{\partial\omega^2} &= \frac{\partial\Phi}{\partial*\omega^2} = \frac{\partial\Phi}{\partial u}= 2a_1u\nn\\
%\frac{\partial\Phi}{\partial\omega^{1m}_\alpha} &= \frac{\partial\Phi}{\partial v^m_\alpha} = 2a_2v_m^\alpha
%\end{align}
%The equations to second order
%\begin{align}
%2a_1u &= 2\lp\alpha_2 -1\rp v_\perp + 2\alpha_1 v_\parallel\nn\\
%2a_2v_m &= 2\alpha_1\hat p_{\parallel m}u + 2\alpha_2\hat p_{\perp m}u
%\end{align}
%Let $u = b_1 v_\parallel + b_2 v_\perp + b_3*v_\parallel + b_4*v_\perp$
%gives
%\begin{align}
%2a_1\lp b_1v_\parallel + b_2v_\perp + b_3*v_\parallel + b_4*v_\perp\rp &= \lbp 2\lp\alpha_2 -1\rp v_\perp + 2\alpha_1v_\parallel \rbp\nn\\
%a_2v_\parallel &= \alpha_1\lp b_1v_\parallel + b_2v_\perp + b_3*v_\parallel + b_4*v_\perp\rp\nn\\
%a_2v_\perp &= \alpha_2\lp b_1v_\parallel + b_2v_\perp + b_3*v_\parallel + b_4*v_\perp\rp
%\end{align}
%giving the equations
%\begin{align}
%a_1b_1 = \alpha_1\nn\\
%a_1b_2 = \alpha_2 -1\nn\\
%a_1b_3 = 0\nn\\
%a_1b_4 = 0\nn\\
%a_2 = \alpha_1 b_1\nn\\
%0 = \alpha_1b_4\nn\\
%\alpha_1b_2 = 0\nn\\
%\alpha_1b_3 = 0\nn\\
%a_2 = \alpha_2b_2\nn\\
%0 = \alpha_2b_3\nn\\
%\alpha_2b_1 = 0\nn\\
%\alpha_2b_4 = 0
%\end{align}
%with solutions\\
%$a_1 \ne 0, \alpha_1=0, a_2 = 0, \alpha_2 = 1, b_1= b_2=b_3=b_4=0$\\
%$a_1 \ne 0, \alpha_1 = 0, b_3 = 0, b_4 = 0, a_2=0, b_1 = 0, a_1b_2 = \alpha_2-1$\\
%The first relation ($\alpha_2 = 1$) removes the $u$ dependence (The $a_1$ terms is still there since the relations is $u = 0 + \Ordo{(v^3)}$). 
%We therefore use the second solution set, giving the equation system
%\begin{align}
%\frac{\partial\Phi}{\partial u} &= 2\lp\alpha_2 -1\rp v_\perp\sqrt{1+\Phi}\nn\\
%\frac{\partial\Phi}{\partial v^m} &= 2\alpha_2\hat p_{\perp m}u \sqrt{1+\Phi}
%\end{align}
%with 
%\begin{align}
%\Phi &= \frac{\alpha_2-1}{b_2}u^2+ \Ordo{\lp v^4,u^4\rp}\nn\\
%u &= b_2v_\perp + \Ordo{\lp v^3\rp}
%\end{align}
%Next we examine the order 4 expansion of $\Phi$
%\begin{align}
%\Phi = \frac{\alpha_2-1}{b_2}u^2 + a_3u^4 + a_4u^2\tr v^2 + a_5(uv)^2 + a_6\lp\tr v^2\rp^2 + a_7\tr v^4 
%\end{align}
%with variations
%\begin{align}
%\frac{\partial\Phi}{\partial u} &= 2\frac{\alpha_2-1}{b_2}u + 4a_3u^3 + 2a_4u\tr v^2 + 2a_5v^mu_\alpha v^\alpha_m\nn\\
%\frac{\partial\Phi}{\partial v^m} &= 2a_4u^2v_m + 2a_5uu_\alpha v^\alpha_m + 4a_6v\tr v^2 + 4a_7v_nv^n_\alpha v^\alpha_m
%\end{align}
%We insert this into the duality equations, using $\sqrt{1+\Phi} = 1 + \frac{\Phi}{2} + \Ordo{(u^4,v^4)}$
%\begin{align}
%2\frac{\alpha_2-1}{b_2}u + 4a_3u^3 + 2a_4u\tr v^2 + 2a_5v^mu_\alpha v^\alpha_m &= 2\lp\alpha_2 -1\rp v_\perp \lp 1 + \frac{\alpha_2-1}{2b_2}u^2 \rp\nn\\
%2a_4u^2v_m + 2a_5uu_\alpha v^\alpha_m + 4a_6v_m\tr v^2 + 4a_7v_nv^n_\alpha v^\alpha_m &= 2\alpha_2\hat p_{\perp m}u \lp 1 + \frac{\alpha_2-1}{2b_2}u^2 \rp
%\end{align}
%We use (no dualities) 
%\begin{align}
%u &= b_2v_\perp + b_5v_\perp^3 + b_6v_\perp v^\alpha_\parallel v_{\perp\alpha} + b_7v_\perp v_\parallel^2 + b_8v_\parallel v_\perp^2 + b_9v_\parallel v^\alpha_\parallel v_{\perp\alpha} + b_{10}v_\parallel^3
%\end{align}
%to get the equations
%\begin{align}
%2\frac{\alpha_2-1}{b_2}\lp b_5v_\perp^3 + b_6v_\perp v^\alpha_\parallel v_{\perp\alpha} + b_7v_\perp v_\parallel^2 + b_8v_\parallel v_\perp^2 + b_9v_\parallel v^\alpha_\parallel v_{\perp\alpha} + b_{10}v_\parallel^3\rp\nn\\
%+ 4a_3b_2^3v_\perp^3 + 2a_4b_2v_\perp\lp v_\perp^2 + v_\parallel^2\rp + 2a_5b_2v_{\perp\alpha} \lp v^\alpha_\perp v_\perp + v^\alpha_\parallel v_\parallel\rp = \lp\alpha_2 -1\rp^2 b_2v_\perp^3 \nn\\
%2a_4b_2^2v_\perp^2v_\parallel + 2a_5b_2^2v_\perp v_{\perp\alpha}v_\parallel^{\alpha} + 4a_6v_\parallel\lp v_\perp^2 + v_\parallel^2\rp + 4a_7\lp v_\perp v_{\perp\alpha} + v_\parallel v_{\parallel\alpha} \rp v^\alpha_\parallel = 0\nn\\
%2a_4b_2^2v_\perp^2v_\perp + 2a_5b_2^2v_\perp v_{\perp\alpha}v_\perp^{\alpha} + 4a_6v_\perp\lp v_\perp^2 + v_\parallel^2\rp + 4a_7\lp v_\perp v_{\perp\alpha} + v_\parallel v_{\parallel\alpha} \rp v^\alpha_\perp = \alpha_2\lp\alpha_2-1\rp b_2^2v_\perp^3
%\end{align}
%giving
%\begin{align}
%2\frac{\alpha_2-1}{b_2}b_5 + 4a_3b_2 + 2a_4b_2 + 2a_5b_2 = \lp\alpha_2 -1\rp^2 b_2\nn\\
%2\frac{\alpha_2-1}{b_2}b_6 = 0\nn\\
%2\frac{\alpha_2-1}{b_2}b_7 + 2a_4b_2 = 0\nn\\
%2\frac{\alpha_2-1}{b_2}b_8 = 0\nn\\
%2\frac{\alpha_2-1}{b_2}b_9 + 2a_5b_2 = 0\nn\\
%2\frac{\alpha_2-1}{b_2}b_{10} = 0\nn\\
%0 = 0\nn\\
%a_5b_2^2 + 2a_7 = 0\nn\\
%0 = 0\nn\\
%2a_4b_2^2 + 4a_6 = 0\nn\\
%0 = 0\nn\\
%a_6 + a_7 = 0\nn\\
%2a_4b_2^2 + 2a_5b_2^2 + 4a_6 + 4a_7 = \alpha_2\lp\alpha_2-1\rp b_2^2\nn\\ 
%0 = 0\nn\\
%4a_6 = 0\nn\\
%0 = 0\nn\\
%4a_7 = 0\nn\\
%0 =0\nn\\
%\end{align}
%with solution
%$a_ 4 = a_5 = a_6 = a_7 = 0$
%
%\begin{align}
%\frac{\alpha_2-1}{b_2}b_5 + 4a_3b_2 = \lp\alpha_2 -1\rp^2 b_2\nn\\
%\frac{\alpha_2-1}{b_2}b_6 = 0\nn\\
%\frac{\alpha_2-1}{b_2}b_7 = 0\nn\\
%\frac{\alpha_2-1}{b_2}b_8 = 0\nn\\
%\frac{\alpha_2-1}{b_2}b_9 = 0\nn\\
%\frac{\alpha_2-1}{b_2}b_{10} = 0\nn\\
%0 = \alpha_2\lp\alpha_2-1\rp b_2^2\nn\\ 
%\end{align}
%
%\paragraph{General case, series expansion after substitution}
%The equations we are about to solve are
%\begin{align}
%*j_2 &= \lbp 2\lp\alpha_2 -1\rp\omega_\perp^{1} + 2\alpha_1\omega_\parallel^{1} \rbp\sqrt{1+\Phi}\nn\\
%j_{1m} &= \lbp 2\alpha_1\hat p_{\parallel m}*\omega^2 + 2\alpha_2\hat p_{\perp m}*\omega^2 \rbp\sqrt{1+\Phi}
%\end{align}
%
%Letting
%\begin{align}
%u &= *\omega^2\nn\\
%v &= 2\lp\alpha_2 -1\rp\omega_\perp^{1} + 2\alpha_1\omega_\parallel^{1}\nn\\
%w &= -2\alpha_1\omega_\perp^{1} + 2\lp\alpha_2 -1\rp\omega_\parallel^{1}\nn\\
%\end{align}
%gives
%\begin{align}
%\frac{\partial\Phi}{\partial u} &= v\sqrt{1+\Phi}\nn\\
%\frac{\partial\Phi}{\partial v} &\lbp \lp\alpha_2 -1\rp\pdo{m} + \alpha_1\pdp{m}\rbp + \frac{\partial\Phi}{\partial v}\lbp -\alpha_1\pdo{m} + \lp\alpha_2 -1\rp\pdp{m}\rbp\nn\\
%& = \lbp \alpha_1\hat p_{\parallel m}u + \alpha_2\hat p_{\perp m}u \rbp\sqrt{1+\Phi}
%\end{align}
%where the projection of the last equation gives
%\begin{align}
%\frac{\partial\Phi}{\partial v} & = \lbp \alpha_1^2 + \alpha_2^2 - \alpha_2 \rbp\lbp \lp\alpha_2 -1\rp^2 + \alpha_1^2\rbp^{-1} u\sqrt{1+\Phi}\nn\\
%\frac{\partial\Phi}{\partial w} & = -\alpha_1\lbp \lp\alpha_2 -1\rp^2 + \alpha_1^2\rbp^{-1} u\sqrt{1+\Phi}
%\end{align}
%Series expand
%\begin{align}
%\Phi(u,v,w) = a_1u^2 + a_2v^2 + a_3w^2 + a_4uv + a_5uw + a_6vw 
%\end{align} 
%with variations
%\begin{align}
%\frac{\partial\Phi}{\partial u} &= 2a_1u + a_4v + a_5w\nn\\ 
%\frac{\partial\Phi}{\partial v} &= 2a_2v + a_4u + a_6w\nn\\
%\frac{\partial\Phi}{\partial w} &= 2a_3w + a_5u + a_6v
%\end{align} 
%Try $a_2=a_3$ and $a_4 = a_5 = a_6 = 0$ gives
%\begin{align}
%2a_1u &= v\nn\\
%2a_2v & = \lbp \alpha_1^2 + \alpha_2^2 - \alpha_2 \rbp\lbp \lp\alpha_2 -1\rp^2 + \alpha_1^2\rbp^{-1} u\nn\\
%2a_2w & = -\alpha_1\lbp \lp\alpha_2 -1\rp^2 + \alpha_1^2\rbp^{-1} u
%\end{align}
%Let $u = b_1v + b_2w$ gives
%\begin{align}
%2a_1\lp b_1v + b_2w\rp &= v\nn\\
%2a_2v & = \lbp \alpha_1^2 + \alpha_2^2 - \alpha_2 \rbp\lbp \lp\alpha_2 -1\rp^2 + \alpha_1^2\rbp^{-1} \lp b_1v + b_2w\rp\nn\\
%2a_2w & = -\alpha_1\lbp \lp\alpha_2 -1\rp^2 + \alpha_1^2\rbp^{-1} \lp b_1v + b_2w\rp
%\end{align}
%giving the equations
%\begin{align}
%a_1b_2 = 0\nn\\
%2a_1b_1 = 1\nn\\
%\lbp \alpha_1^2 + \alpha_2^2 - \alpha_2 \rbp\lbp \lp\alpha_2 -1\rp^2 + \alpha_1^2\rbp^{-1}b_2 = 0\nn\\
%2a_2 = \lbp \alpha_1^2 + \alpha_2^2 - \alpha_2 \rbp\lbp \lp\alpha_2 -1\rp^2 + \alpha_1^2\rbp^{-1}b_1\nn\\
%-\alpha_1\lbp \lp\alpha_2 -1\rp^2 + \alpha_1^2\rbp^{-1}b_1 = 0\nn\\
%2a_2 = -\alpha_1\lbp \lp\alpha_2 -1\rp^2 + \alpha_1^2\rbp^{-1}b_2
%\end{align}
%$b_2 = 0, a_2 = 0, \alpha_1 = 0, \alpha_2 = 0$
%\begin{align}
%2a_1b_1 = 1\nn\\
%\end{align}


%\paragraph{General case, series expansion after substitution}
%We consider the same equations as in the previous section for $\alpha$ a nonconstant function of $\omega^{1m}$ and $\omega^2$.
%The new equations are
%\begin{align}
%&d\Big{[}\lambda *j_{1m} + 2 \lambda {\{} \epsilon_{mm'} \alpha \omega^2 - \epsilon_{m'n}*{\partial \alpha \over \partial \omega^{1m}}*(\omega^{1n} \wedge \omega^2){\}}*f_{n'} \M^{m'n'}\Big{]}=0\nn\\
%&d\Big{[}\lambda*j_2 - 2\lambda \epsilon_{mn}{\{}\lp\alpha -1\rp\omega^{1n} + *{\partial \alpha \over \partial \omega^2 }*(\omega^{1n} \wedge \omega^2){\}}*f_{n'}\M^{mn'}\Big{]}=0,
%\end{align}
%or after integration and with $p$ (without integration forms)
%\begin{align}
%*j_{1m} &= 2 {\{} \alpha \hat p_{\perp m} \omega^2 + *{\partial \alpha \over \partial \omega^{1m}}*(\omega^{1}_\perp \wedge \omega^2){\}}\sqrt{1+\Phi}\nn\\
%*j_2 &= 2{\{}\lp\alpha -1\rp\omega^{1}_\perp + *{\partial \alpha \over \partial \omega^2 }*(\omega^{1}_\perp \wedge \omega^2){\}}\sqrt{1+\Phi}.
%\end{align}
%Hum, g�r n�t smart nu d�...

%To get a nicer expression without hodge dualities and $p$ we define
%\begin{align}
%v^m &= \omega^{1m}\nn\\
%u^m &= q^m*\omega^2,\hspace{1cm}\Rightarrow\omega^2=\frac{q_m}{q^2}*u^m 
%\end{align}
%where $q^m$ is introduced to get an $SL(2,\rr)$-index on $u$ and is constructed by some combination of the constant charges $p^m$ to remove them from the equations.
%\begin{align}
%\frac{\partial\Phi}{\partial v^m} &= - 2\epsilon_{mn}{\{}\alpha \frac{q_p}{q^2}u^p  + \frac{\partial \alpha}{\partial v^{n'}}\frac{q_p}{q^2} v_\alpha^{n'}u^{\alpha p}) + *\gamma{\}}\frac{p^{n}}{|p|}\sqrt{1+\Phi}\\
%\frac{\partial\Phi}{\partial u^m}q_m &= 2\epsilon_{mn}{\{}\lp\alpha-1\rp v^{n} + \frac{\partial\alpha}{\partial u^p}q_p\frac{q_{n'}}{q^2} v_\alpha^{n}u^{\alpha {n'}}) + \delta^n{\}}\frac{p^m}{|p|}\sqrt{1+\Phi}
%\end{align}
%thus we see that we should choose
%\begin{align}
%q_m =  
%\end{align}
%to get
%\begin{align}
%\frac{\partial\Phi}{\partial v^m} &= - 2\epsilon_{mn}{\{}\alpha \frac{q_p}{q^2}u^p  + \frac{\partial \alpha}{\partial v^{n'}}\frac{q_p}{q^2} v_\alpha^{n'}u^{\alpha p}) + *\gamma{\}}\frac{p^{n}}{|p|}\sqrt{1+\Phi}\\
%\frac{\partial\Phi}{\partial u^m}q_m &= 2\epsilon_{mn}{\{}\lp\alpha-1\rp v^{n} + \frac{\partial\alpha}{\partial u^p}q_p\frac{q_{n'}}{q^2} v_\alpha^{n}u^{\alpha {n'}}) + \delta^n{\}}\frac{p^m}{|p|}\sqrt{1+\Phi}
%\end{align}

%The theory should only contain 2 scalars (.......... ????? ........).
%There are 2 scalars in $\omega^{1m}$ and one scalar in $\omega^2$. We must therefore relate $\omega^2$ to $\omega^{1m}$ somehow.  


%Dualize and multiply the first relation by $*\omega^{2\alpha}$
%\begin{align}
%*\omega^{2\alpha}j_{1\alpha m} &= - 2 *\omega^{2\alpha}{\{}\alpha *\omega_\alpha^2 + {\partial \alpha \over \partial \omega^{1\alpha n'}}*(\omega^{1n'} \wedge \omega^2) + *\gamma{\}}\epsilon_{mn}*f_{m'} \M^{m'n}
%\end{align}
%which means we can express $\epsilon_{mn}*f_{m'} \M^{m'n}$ explicitly as a function of $\omega^2$ and $\omega^{1m}$ and inserting it in the second relation we get 
%\begin{align}
%*j_{2\alpha } *\omega^{2\alpha}{\{}\alpha *\omega_\alpha^2 + {\partial \alpha \over \partial \omega^{1\alpha n'}}*(\omega^{1n'} \wedge \omega^2) + *\gamma_{\alpha m}{\}} &= - {\{}\lp\alpha-1\rp \omega_\alpha ^{1n} + *{\partial \alpha \over \partial \omega^{2\alpha}}*(\omega^{1n} \wedge \omega^2) + \delta^n{\}}*\omega^{2\alpha}j_{1\alpha m} 
%\end{align}

%We want to express $\omega^2_\alpha$ as a function of $\omega^{1m}_\alpha$. Since $\omega^2_\alpha$ does not have an $SL(2,\rr)$ index it must be contracted somehow in the construction.  

%\chapter{Computer solutions}
\chlab{csolutions}
\section{Computer solution of the aligned $d=8$ $D2$, const $\alpha\mbox{, }\beta$ case}
\seclab{csolution_8d_const_old}
We will here solve the derived duality equations \eqnref{solution_8d_km_orig} and \eqnref{solution_8d_jrm_orig} with $\omega^{rm}$ pointing in the $\hat p_\parallel$ direction implying $\omega_\perp^m = 0$ and further with the background field strength constraint $F^{rm} = 0$, which was studied in \cite{artikeln}.
In addition to the case studied in \cite{artikeln}, where the equations were solved for a sixth order ansatz $\Phi$ with $\alpha=\alpha_1=-\frac{1}{6}$, we will solve the equations using higher order ansatzes and general values of $\alpha$. We also choose the closed forms $\Gamma=\Delta=0$ and the parameters $\alpha_2=\beta_1=0$ and $\beta=\beta_2=\frac{1}{6}$ to get the equations on the desired form.
The equations we are about to solve, after the variable substitution \eqnref{solution_8d_vartrans}, are on the form 
\begin{align}
\eqnlab{csolution_equations_8D_w0_alpha}
\frac{\partial\Phi}{\partial u} &= \tilde\alpha_1\sqrt{1+\Phi}v\nn\\
\frac{\partial\Phi}{\partial v} &= \tilde\alpha_2\sqrt{1+\Phi}u,
\end{align}
where $\tilde\alpha_1$ and $\tilde\alpha_2$ are constant scalars. In particular $\tilde\alpha_1=-\tilde\alpha_2=2/3$ corresponds to equations \eqnref{solution_8d_duality_paper} and $\tilde\alpha_1=2\alpha_1+4/3$ and $\tilde\alpha_2=2\alpha_1-2/3$ corresponds to the equations in the special case of this section.
Choosing $\alpha_1=-1/6$ and making a field redefinition $u\rightarrow-u$ yields
\begin{align}
\eqnlab{solution_equations_8D_aligned}
\frac{\partial\Phi}{\partial u} &= -\sqrt{1+\Phi}v\nn\\
\frac{\partial\Phi}{\partial v} &= \sqrt{1+\Phi}u,
\end{align}
The change of sign on $u$ is made to get the same equations as in \cite{artikeln}, eq. (3.18). This sign will be most misfortunate when we choose a relation between $u$ and $v$ later in subsection \ssecref{csolution_uvrelation}.

%\begin{align}
%S&[\Phi[\omega(\phi),f(\phi,a)],h(\phi,a,b,\omega(\phi),f(\phi,a)),\lambda]\nn\\
%& = \int d^3\xi\sqrt{-g}\lambda\left[1 + \Phi[\omega(\phi),f(\phi,a)] - *h^r\W_{rs}*h^s\right]
%\end{align}

\subsection{The procedure to solve the duality equations}
We will solve the equations mathematically, interpreting $u_\alpha^m$ and $v_\alpha^m$ as $3\times 3$-matrices, ignoring the two different types of indices.
If we find solutions that cannot be modified to have the correct index structure, meaning one of the symmetries is broken, we will have to deal with that then.  
Since $\Phi(u,v)$ is a scalar function in the matrices u and v, we need to contract the u and v matrix indices in different ways to form scalar expressions.   
As we do not know anything about how the solution will look like, we will try to make an as general scalar ansatz in u and v as possible, i.e. the ansatz will be a polynomial of all different terms we can contract from the u and v matrices.
The equations we want to solve are nonlinear matrix differential equations which we can series expand using our ansatz $\Phi$. 
To reduce the number of different terms and collect dependent terms we introduce the Cayley-Hamilton matrix theory, which will let us reduce high powers of a matrix into lower powers and some invariants of the same matrix.

\subsubsection{Cayley-Hamilton theory}
Consider an $n\times n$ matrix $M$, with $n$ different eigenvalues $m_1,\dots ,m_n$ and eigenvectors ${\bf{v}}_1\dots{\bf{v}}_n$ and calculate its characteristic polynomial
\begin{equation}
\eqnlab{solution_char_pol}
0 = \det(\lambda\id-M) = \prod_{i=1}^n\lp\lambda-m_i\rp 
= \lambda^n - V_1\lambda^{n-1} - \cdots - V_m
\end{equation}
Now form the product $\prod_{i=1}^n\lp M-m_i\id\rp$, where all factors commute and act with it on each of the eigenvectors
\begin{equation}
\prod_{i=1}^n\lp M-m_i\id\rp {\bf{v}}_j = \mzero
\end{equation}
Since this is true for all the eigenvectors, spanning a basis, we must have
\begin{equation}
\eqnlab{solution_char_matrix_pol}
M^n - V_1M^{n-1} - \cdots - V_m\id = \prod_{i=1}^n\lp M-m_i\id\rp = \mzero,
\end{equation}
i.e. M satisfies its own characteristic equation. This can be proven to be true also for matrices with repeated eigenvalues (using Jordan forms). 
The coefficients $V_i$ are independent of n and can be computed iteratively.
We get the invariants:
\begin{center}
\begin{tabular}{ll}
n=1: & $V_1 = \tr M$\cr
n=2: & $V_2 = \half\tr M^2 - \half\lp\tr M\rp^2$\cr
n=3: & $V_3 = \frac{1}{3}\tr M^3 - \half\tr M^2\tr M + \frac{1}{6}\lp\tr M\rp^3$\cr
n=4: & $V_4 = - \frac{1}{4}\tr M^4 - \frac{1}{3}\tr M^3\tr M - \frac{1}{8}\tr M^2\tr M^2$\cr
 & $\phantom{V_4 =} + \frac{1}{4}\tr M^2\lp\tr M\rp^2 - \frac{1}{24}\lp\tr M\rp^4$\cr
\end{tabular}
\end{center}
The nice thing is that we can now use \eqnref{solution_char_matrix_pol} to rewrite any matrix powers greater than $n-1$, in terms of lower powers of the matrix.
For instance when $n = 2$: $M^2 = V_1M + V_2$ and so on for higher values of n. 

\subsubsection{Ansatz}
\sseclab{solution_ansatz}
In this case u and v are 3$\times$3 matrices so we make a polynomial ansatz for $\Phi$ as
\begin{align}
\eqnlab{solution_phi_ansatz}
\Phi=\sum_{i_1=0}^{\infty}\dots\sum_{i_n=0}^{\infty}a_{i_1\dots i_n}\phi_1^{i_1}\dots\phi_n^{i_n}
\end{align}
where the polynomial variables $\phi_{i}$ are all n possible independent contractions in u and v.
We can use the relation
\begin{align}
\frac{\partial}{\partial v}\lp\tr\lp uv^3\rp\rp &= uv^2+vuv+v^2u = \frac{\partial}{\partial v}\lp\tr\lp u\lp V_3 + V_2v + V_1 v^2 \rp\rp\rp \nn\\
&= \tr\lp uv^2\rp - V_2\tr u - V_1\tr\lp uv\rp + v\lp\tr\lp uv\rp - V_1\tr u\rp\nn\\
& + V_2u +v^2\tr u + V_1\lp uv + vu\rp    
\end{align}
to find a relation (multiply by u and take trace)
\begin{align}
\eqnlab{csolution_tracecommuterelation}
2\tr(u^2v^2)+\tr(uvuv)&=
2\tr u\tr\lp uv^2\rp + 2V_1\tr(u^2v) - 2V_1\tr u\tr\lp uv\rp\nn\\
&+ V_2\lp\tr u^2- \lp\tr u\rp^2\rp + \lp\tr(uv)\rp^2
\end{align}
between $\tr(u^2v^2)$ and $\tr(uvuv)$ in traces of order 3 in u and v, so we don't need to include $\tr(uvuv)$ in our ansatz.
In a similar fashion we see that we can skip all other terms like $\tr(u^2vuv)$, $\tr(uv^2uv)$ and so on.
Thus the polynomial variables we will use in our ansatz are
\begin{equation}
\eqnlab{solution_polvars}
\phi=\{\tr(u^2v^2),\tr v^3,\tr u^3,\tr(u^2v),\tr(uv^2),\tr v^2,\tr u^2,\tr(uv),\tr v,\tr u\}.
\end{equation} 
The total maximal power in these variables is unknown and can, for all that we know, be infinite.
%Hence we start at low orders and use the duality relations to find all conditions up to orders that wont change when we add further terms of higher order to the ansatz. When all conditions of a certain order has been considered we increase the order of the ansatz and repeat the procedure.
\begin{table}[h]
\tablab{solution_ansatz_orders}
\begin{center}
\begin{tabular}{|l|cccccccc|}
\hline
Ansatz order & 0 & 1 & 2 & 3 & 4 & 5 & 6 & 7 \cr\hline
No. of terms & 1 & 3 & 9 & 23 & 52 & 108 & 214 & 398 \cr
\hline\hline
Ansatz order & 8 & 9 & 10 & 11 & 12 & 13 & 14 & 15\cr\hline
No. of terms & 712 & 1228 & 2050 & 3326 & 5271 & 8162 & 12391 & 18477\cr
\hline
\end{tabular}
\end{center}
\caption{The length of the ansatzes for different orders. The growth is close to a doubling for each increment of the order.}
\end{table}
As we can see in \Tabref{solution_ansatz_orders} the number of terms in the ansatz increases rapidly with the order of the ansatz and for technical reasons we must confine ourselves to ansatzes of quite low orders\footnote{Already at ansatz order 10 the number of coefficients becomes really large. Solving nonlinear equations in over 2000 variables is something you usually won't do before breakfast.}.
Just as an illustration we give some terms in the order 4 ansatz
\begin{align}
\Phi &= a_1 + a_2\tr u + a_3\lp\tr u\rp^2 + a_4\lp\tr u\rp^3 + a_5\lp\tr u\rp^4 + a_6\tr v + a_7\tr v\tr u\nn\\
& + \dots + a_{49}\tr v^3 + a_{50}\tr v^3\tr u + a_{51}\tr v^3\tr v + a_{52}\tr(u^2v^2)
\end{align}

\subsection{Transformation of the equations}
\seclab{solution_equation_transformation}
Since the equations are nonlinear in the ansatz $\Phi$, due to the $\sqrt{1+\Phi}$, we want to transform them into linear equations which are faster to solve. 
The most obvious way to do this is to perform a variable transformation $X=\sqrt{1+\Phi}$, such that
\begin{align}
\eqnlab{solution_duality_transformed_linear}
\frac{\partial\Phi}{\partial u} &= \frac{\partial X^2}{\partial u} = 2X\frac{\partial X}{\partial u} = -\sqrt{1+\Phi}v = -Xv &&\Rightarrow \frac{\partial X}{\partial u} = -\frac{v}{2}\nn\\
\frac{\partial\Phi}{\partial v} &= \frac{\partial X^2}{\partial v} = 2X\frac{\partial X}{\partial v} = \sqrt{1+\Phi}u = Xu &&\Rightarrow \frac{\partial X}{\partial v} = \frac{u}{2}
\end{align}
These equations looks a lot easier to solve, but a polynomial ansatz for $X$ isn't at all equivalent to a polynomial ansatz for $\Phi$, since $\Phi = X^2 - 1$ roughly speaking is the square of the polynomial ansatz X.
So it looks like we should solve the equations 2 times, one with $\Phi$ a polynomial ansatz using $g(\Phi)=\sqrt{1+\Phi}$ on the right hand side and one with $X$ a polynomial ansatz using $g(X)=1/2$ on the right hand side. This is not the entire story though. 
If we do another variable transformation $X=f(Y)$ we get the equations
\begin{align}
\eqnlab{solution_duality_transformed_general}
\frac{\partial X}{\partial u} &= \frac{\partial f}{\partial Y}\frac{\partial Y}{\partial u} = -\frac{v}{2} &&\Rightarrow \frac{\partial Y}{\partial u} = - \frac{v}{2} \lp\frac{\partial f}{\partial Y}\rp^{-1} = -g(Y) v\nn\\
\frac{\partial X}{\partial v} &= \frac{\partial f}{\partial Y}\frac{\partial Y}{\partial v} = \frac{u}{2} &&\Rightarrow \frac{\partial Y}{\partial v} = \frac{u}{2} \lp\frac{\partial f}{\partial Y}\rp^{-1} = g(Y) u 
\end{align}
Thus meaning we should solve the equations with a polynomial expansion for $Y$ for each possible function $g(Y)$ on the right hand side.
In other words, there is nothing "holy" about the $\sqrt{1+\Phi}$ factors entering the right hand side of the duality relations and we can in fact solve the equations with arbitrary functions $g$ multiplying u and v on the right hand side and transform these to get a solution $\Phi=X^2-1=f(Y)^2-1$ to the equations on the original form.
So if we expand $Y$ rather than $\Phi$ as a polynomial ansatz, we can express $\Phi$ as a function of a polynomial in $\phi_i$ rather than just as a plain polynomial in $\phi_i$.
Note that one can always add an integration constant to f to get rid of possible constant terms in $\Phi$.
As an easy illustration consider $g(Y)=\sqrt{1+Y}$ giving $f(Y)=\sqrt{1+Y} + C$, where C is a constant, which gives the starting equations
\begin{align}
\frac{\partial Y}{\partial u} &= - \frac{v}{2} \lp\frac{\partial f}{\partial Y}\rp^{-1} = -\sqrt{1+Y}v\nn\\
\frac{\partial Y}{\partial v} &= \frac{u}{2} \lp\frac{\partial f}{\partial Y}\rp^{-1} = \sqrt{1+Y}u 
\end{align}
and the solution is $\Phi=(\sqrt{1+Y} + C)^2-1 = Y +2C\sqrt{1+Y} + C^2$.
So, if we find a solution $Y = \Phi_i^{(0)} = \Phi_i$ to the original duality equations, we can insert it in this solution to get a new solution $\Phi_i^{(1)}= \Phi_i^{(0)} +2C^{(0)}\sqrt{1+\Phi_i^{(0)}} + \lp C^{(0)}\rp^2$, which is trivially checked to solve the starting duality equations.    
Since $\Phi_i^{(1)}$ is a solution we can iteratively construct an infinite amount of new solutions $\Phi_i^{(n)} = \Phi_i^{(n-1)} +2C^{(n-1)}\sqrt{1+\Phi_i^{(n-1)}} + \lp C^{(n-1)}\rp^2$.
If we have the condition to not have constant terms in our solution $\Phi$ and allow $\Phi$ to be of infinite order, we are forced to choose $C^{(n)}=-2$ (or $C^{(n)}=0$), giving the solutions $\Phi_i^{(n)}= \Phi_i^{(n-1)} - 4\sqrt{1+\Phi_i^{(n-1)}} + 4$ (or $\Phi_i^{(n)}= \Phi_i^{(n-1)}$). 

At a first sight it might look like an impossible project to solve the equations with an arbitrary function $g$ at the right hand side, but it turns out that the overlapping equations independent on $g$ really constrain most of the ansatz. 
The overlapping equations we will use is first to put one of the equations linear in the ansatz $Y$ by just combining the two equations into
\begin{equation}
\eqnlab{solution_duality_linear}
\frac{\partial Y}{\partial v}v + \frac{\partial Y}{\partial u}u = 0.
\end{equation}
We are then left with one of the nonlinear equations
\begin{align}
\eqnlab{solution_duality_transformed_general_dux}
\frac{\partial Y}{\partial u} = -g(Y) v
\end{align}
which we find, using the Cayley-Hamilton theory and the assumption that the relation $u=u(v)$ is known, really is three matrix equations (multiplying each of $\id$, $v$ and $v^2$) out of which only one (the one multiplying $v$) is $g$ dependent and thus nonlinear. 

\subsection{The relation between u and v}
\sseclab{csolution_uvrelation}
As mentioned before, the theory should only contain $3$ scalar degrees of freedom and as we saw in the general expansion, there must be relations between the triplets $u$, $v$ and $w$. Here, where $w=0$, we must have one relation between $u$ and $v$ to get $3$ scalar dofs, but we do not know what it should look like.
The most simple guess we make is that $u$ is proportional to $v$, $u=cv$, for which the equations \eqnref{csolution_equations_8D_w0_alpha} are highly underdetermined and it is easy (from e.g. expanding the original equations or the substitution $X=\sqrt{1+\Phi}$ to first order in $u^tu, u^tv$ and $v^tv$) to find several solutions to the duality equations, e.g.
\begin{align}
\Phi_1 =& -\frac{1}{2c}\lp a +\half\rp \tr u^2 + a\tr(uv) - \frac{c}{2} \lp a - \frac{1}{2}\tilde\alpha_2\rp \tr v^2\mbox{, }\tilde\alpha_2 = 1\nn\\
\Phi_2 =& \lp-\frac{1}{2c}\lp a +\half\rp \tr u^2 + a\tr(uv) - \frac{c}{2} \lp a - \frac{1}{2}\tilde\alpha_2\rp \tr v^2\rp^2\nn\\
& -\frac{1}{c}\lp a +\half\rp \tr u^2 + 2a\tr(uv) - c \lp a - \frac{1}{2}\tilde\alpha_2\rp \tr v^2
\end{align}
where $a$ are different arbitrary scalars in the two solutions and we have put $\tilde\alpha_1=-1$.
We note that $\tilde\alpha_2$ is determined in the first solution $\Phi_1$ but undetermined in the second solution $\Phi_2$, i.e. finding the value of $\tilde\alpha_2$ compatible with one solution doesn't rule out other solutions with different values on it.   

Out of the 15 expansion parameters, when using $u=v$ and expanding $\Phi$ to second order in the squared $u$ and $v$ matrices, 11 are left undetermined by the constraining duality equations.
We expect many of these solutions to disappear when $w$ and the dual terms on the right hand side are introduced.
Instead we will try to get the relation between $u$ and $v$ by considering another relation between 1-form triplets we expect to be valid.
As was shown in \cite{artikeln}, we can use the $M_2$-brane compactified on $T_3$ to argue that there should be 2 different related 1-form triplets, one coming from the pullback of the internal vielbein and one being its conjugate variable.    
We have 
\begin{align}
S = \int d^3\xi\sqrt{-\det G},
\end{align}
where $G_{\alpha\beta} = g_{\alpha\beta} + e_\alpha^me_{m\beta}$ and the 1-form $\omega_\circ^m = e_\alpha^md\xi^\alpha$ is the pullback to the world volume of the internal vielbein $\hat e^m$ and is taken as the first 1-form triplet.
The second 1-form is given by the conjugate
\begin{align}
\frac{\partial \mathcal L}{\partial e_\alpha^m} &= \frac{1}{2}\sqrt{-\det G}G^{\alpha'\beta}\frac{\partial G_{\alpha'\beta}}{\partial w_\alpha^m} = \sqrt{-\det G} G^{\alpha\beta}e_{m\beta}   
\end{align}
We will assume that $\omega_\circ^m$ is some projection of $\omega^{rm}$ in a certain direction $-\qdp{r}(p)$ \footnote{The sign is chosen to make correspondence to $\qdp{}$ in \eqnref{solution_8d_duality_general}.}, i.e. $\omega_\circ^m = -\qdp{r}\omega^{rm} = -q_1(p)\omega_\parallel^m - q_2(p)\omega_\perp^m$.
If we note that $de^m = F^{1m}$ and that the pullback is natural, then $q_1$ and $q_2$ are uniquely determined for a given constant vector $p$ by $F^{1m} = -\qdp{r}d\omega^{rm} = \qdp{r}F^{rm}$, from which we also find that $q_1^2+q_2^2=1$ is always true.  
In particular, for p=(1,0) we have $F_\parallel^m = F^{1m}$ and $F_\perp^m = F^{2m}$, giving $q_1=1$ and $q_2=0$ and thus $\omega_\circ^m = -v^m$ \footnote{We use this condition as if $F\ne 0$. If $F=\omega_\perp=0$ we could actually choose the coefficient multiplying $v$ arbitrary as far as we can see.} which we will use from now on until we solve the more general equations \eqnref{csolution_equations_8D_w0_alpha} in \ssecref{csolution_paper_general_alpha}.
If we assume that $*f$ equals the second 1-form triplet we find
\begin{align}
u = -\sqrt{-\det G}G^{-1}v,
\end{align}
which we will use to solve the duality equations \eqnref{solution_equations_8D_aligned}.
\footnote{Note that these equations was derived using a field redefinition $u\rightarrow-u$, so we should actually use $u = +\sqrt{\det G}G^{-1}v$ if we would not want to make correspondence to \cite{artikeln}. This sign will make the difference between needing to introduce the parameters $\alpha_1$ and $\alpha_2$ or not, as we will see when we solve the equations for general values on $\alpha_1$ and $\alpha_2$ in {\it{Solution for general values of $\alpha$}} in \ssecref{csolution_paper_general_alpha}}

Now, considering $v_\alpha^m$ as $3\times 3$-matrices and ignoring the index types and signature, we get the following Cayley-Hamilton invariants
\begin{align}
V_1 &= \tr v\nn\\
V_2 &= \half\lp\tr v^2-\lp\tr v\rp^2\rp\nn\\
V_3 &= \frac{1}{3}\tr v^3 - \half\tr v\tr v^2 + \frac{1}{6}\lp\tr v\rp^3.
\end{align}
We will express the metric in a flat $8$-dimensional basis and a curved $3$-dimensional one, giving the pullbacked metric 
\begin{align}
G = \id + v^2  
\end{align}
with determinant
\begin{align}
\det G=(1 + V_2)^2 + (V_1-V_3)^2 = \lp 1 + v_1^2\rp\lp 1 + v_2^2\rp\lp 1 + v_3^2\rp
\end{align}
and inverse
\begin{align}
G^{-1}=\frac{1}{\det G}\bigg\{&\lp 1+V_1^2+2V_2+V_2^2-V_1V_3\rp\id \nn\\
& + \lp V_3+V_1V_2\rp v + \lp -1-V_2\rp v^2\bigg\}.
\end{align}
It is now easy to calculate
\begin{align}
\eqnlab{solution_u8}
u &= -\sqrt{G}\lp G^{-1}v\rp\\
& = -\frac{1}{\sqrt{G}}\lbp\lp-V_3-V_2V_3\rp\id + \lp 1+V_1^2+V_2-V_1V_3\rp v + \lp V_3-V_1\rp v^2\rbp\nn.  
\end{align}

\subsection{Implementation}
\seclab{solution_implementation}
We choose to solve the equations by writing a program in Object Pascal Delphi, where we have better control on memory and performance\footnote{Our program runs thousands of times faster and is hundreds of times more memory efficient than an analogue amateur implementation in Mathematica.} compared to e.g. Mathematica.   
To reduce the number of expansion variables we will always insert the duality relation \eqnref{solution_u8} into the polynomial variables \eqnref{solution_polvars} before representing them in the computer, meaning we will only need to implement polynomials in the variables $\{v,V_1,V_2,V_3\}$.
A good way to represent the polynomials is to just store the coefficients to all possible combinations of these 4 variables next to each other in the memory as shown in \Figref{solution_memory}. 
The factors will either be two integers of size 32, 64 or 128 bits each\footnote{The reason we need this good precision is the binomial coefficients from the expansion of $1/\sqrt{\det G}$ in u. The total of $2\cdot 32$, $2\cdot 64$ and $2\cdot 128$ bits turns out to be sufficient precision to expand the polynomial variables in \eqnref{solution_polvars} up to orders 26, 47 and 89 in v (Order 89 corresponds to more than 340000 expansion terms). Due to multiplications and additions the allowed orders will be somewhat lowered when the ansatz is created from the polynomial variables.}\footnote{32 and 64 bit integers are part of the Delphi language, but we need to implement the 128 bits integers by hand in assembler, which is presented in appendix \chref{int128}.}, representing the numerator and denominator of a number coefficient\footnote{In this problem all denominators are powers of 2 at all times so a better representation of the coefficients would have been an integer factor times $2^{\mbox{Int32}}$, which would be less memory consuming, allow expansions to somewhat higher orders and be much faster to reduce.}, or a reference to a coefficient structure if the coefficients are constituted of variables or polynomials of variables.  

What is good with this representation is that we can create one map, for all polynomials, which tells us which element corresponds to which order in the polynomial variables.
We can then perform all power-independent operations like addition of 2 polynomials fast by termwise addition of coefficients not caring about the power they multiply, since it is always the same. More complex operations like multiplication of 2 polynomials can be done using the power maps, not doing the multiplication if the total power becomes greater than the calculation order.
Since most polynomials we will use are not sparse there is no need to worry about the memory waste from representing all coefficients of value zero. It also turns out that the variable coefficient structures are the real memory thieves.  

It takes some time and memory (for deallocation information) to allocate memory areas from the system and since we need to store hundreds of millions small objects for the variable polynomials, it is fruitful to create some kind of memory manager which, when needed, automatically retrieves new large memory areas from the system and tells us were there is free space to store new objects.

\setlength{\unitlength}{1.0mm}
\begin{center}
\begin{figure}[h]
\begin{picture}(125,42)(-2,0)
\path(0,0)(125,0)
\path(0,5)(125,5)
\path(0,0)(0,5)
\path(125,0)(125,5)
\path(30,0)(30,5)
\path(47.5,0)(47.5,5)
\path(77.5,0)(77.5,5)
\path(95,0)(95,5)
\multiput(37,2.5)(2,0){3}{\circle*{0.5}}
\multiput(84.5,2.5)(2,0){3}{\circle*{0.5}}
\put(15,2.5){\makebox(0,0){\scriptsize{$v^0$}}}
\put(62.5,2.5){\makebox(0,0){\scriptsize{$v^i$}}}
\put(110,2.5){\makebox(0,0){\scriptsize{$v^{\min(n-1,N)}$}}}

\path(47.5,5)(0,8)
\path(77.5,5)(125,8)
\path(0,8)(125,8)
\path(0,13)(125,13)
\path(0,8)(0,13)
\path(125,8)(125,13)
\path(30,8)(30,13)
\path(47.5,8)(47.5,13)
\path(77.5,8)(77.5,13)
\path(95,8)(95,13)
\multiput(37,10.5)(2,0){3}{\circle*{0.5}}
\multiput(84.5,10.5)(2,0){3}{\circle*{0.5}}
\put(15,10.5){\makebox(0,0){\scriptsize{$V_3^0$}}}
\put(62.5,10.5){\makebox(0,0){\scriptsize{$V_3^j$}}}
\put(110,10.5){\makebox(0,0){\scriptsize{$V_3^{[(N-i)/3]}$}}}

\path(47.5,13)(0,16)
\path(77.5,13)(125,16)
\path(0,16)(125,16)
\path(0,21)(125,21)
\path(0,16)(0,21)
\path(125,16)(125,21)
\path(30,16)(30,21)
\path(47.5,16)(47.5,21)
\path(77.5,16)(77.5,21)
\path(95,16)(95,21)
\multiput(37,18.5)(2,0){3}{\circle*{0.5}}
\multiput(84.5,18.5)(2,0){3}{\circle*{0.5}}
\put(15,18.5){\makebox(0,0){\scriptsize{$V_2^0$}}}
\put(62.5,18.5){\makebox(0,0){\scriptsize{$V_2^k$}}}
\put(110,18.5){\makebox(0,0){\scriptsize{$V_2^{[(N-3j-i)/2]}$}}}

\path(47.5,21)(0,24)
\path(77.5,21)(125,24)
\path(0,24)(125,24)
\path(0,29)(125,29)
\path(0,24)(0,29)
\path(125,24)(125,29)
\path(30,24)(30,29)
\path(47.5,24)(47.5,29)
\path(77.5,24)(77.5,29)
\path(95,24)(95,29)
\multiput(37,26.5)(2,0){3}{\circle*{0.5}}
\multiput(84.5,26.5)(2,0){3}{\circle*{0.5}}
\put(15,26.5){\makebox(0,0){\scriptsize{$V_1^0$}}}
\put(62.5,26.5){\makebox(0,0){\scriptsize{$V_1^l$}}}
\put(110,26.5){\makebox(0,0){\scriptsize{$V_1^{N-2k-3j-i}$}}}

\path(47.5,29)(0,32)
\path(77.5,29)(125,32)
\path(0,32)(125,32)
\path(0,37)(125,37)
\path(0,32)(0,37)
\path(62.5,32)(62.5,37)
\path(125,32)(125,37)
\put(31.25,34.5){\makebox(0,0){\scriptsize{$32|64|128$ bits of numerator data}}}
\put(93.75,34.5){\makebox(0,0){\scriptsize{$32|64|128$ bits of denominator data}}}

\path(0,37)(125,37)
\path(0,42)(125,42)
\path(0,37)(0,42)
\path(125,37)(125,42)
\put(62.5,39.5){\makebox(0,0){\scriptsize{Pointer to variable coefficient, or}}}
\end{picture}
\caption{Illustration of the memory structure of an entire polynomial of order $N$. The polynomial coefficients are stored next to each other in the memory. Which variables the coefficient multiplies is decided by in which variable intervals it belongs, e.g. the coefficient in the figure multiplies $v^iV_3^jV_2^kV_1^l$. [ ] denotes integer part.}
\figlab{solution_memory}
\end{figure}
\end{center}


To implement the problem we first calculate all occurring combinations of $u$ and $v$ and their derivatives with the duality relation inserted.
This is straightforward, if we create $u$ we can multiply it with itself and $v$ and take traces to create all polynomial variables needed.
We use $u$ given in \eqnref{solution_u8} and simply Taylor expand the determinant factor as
\begin{align}
\frac{1}{\sqrt{\det G}} &= \frac{1}{\sqrt{\lp 1+V_2\rp^2+\lp V_1-V_3\rp^2}}\\
&= \sum_{i=0}^\infty \binom{-1/2}{i}\lp 2V_2 + V_2^2 + V_1^2 - 2V_1V_3 + V_3^2\rp^i 
\end{align}
which is easily calculated recursively up to the decided order ($i$ runs from $0$ to $[N/2]$) using our polynomial representations.
Of course the binomial factors are precalculated in a table, it seems unnecessary to calculate 40 digit factorials each time the program is run.
We also implement the Cayley-Hamilton reduction of matrix powers mentioned above whenever the power in $v$ exceeds 2.
The ansatz and its derivatives are created by recursively summing over all variable combinations up to the chosen order $N$.
If we let $\phi_i(v,V_3,V_2,V_1)$ be the i:th polynomial variable in \eqnref{solution_polvars} of order $n_i$ with the duality relation inserted, the ansatz is simply calculated as (remember we are solving the transformed functions using the ansatz $Y$)
\begin{align}
Y=\sum_{i_1=0}^{[N/n_1]}\phi_1^{i_1}\cdot\lp\sum_{i_2=0}^{[(N-n_1i_1)/n_2]}\phi_2^{i_2}\cdot\lp\dots\lp\sum_{i_{10}=0}^{[(N-n_1i_1-\dots-n_9i_9)/n_{10}]}\phi_{10}^{i_{10}}a_j\rp\dots\rp\rp
\end{align}
where $j$ is increased for each term added and where we need $N$ temporary polynomials $p_i$ to store intermediate results to reduce the number of total multiplications.
The derivatives w.r.t. u can be calculated parallel to the ansatz by storing previously calculated derivatives in polynomials $d_i$ so that at $i$:th level we have
\begin{equation}
d_i = \frac{\partial\phi_i}{\partial u}p_{i-1}+d_{i-1}\phi_i.
\end{equation} 
The derivatives w.r.t. $v$ is of course calculated similarly as the ones w.r.t. $u$ and we have thus created the ansatz and its derivatives with the duality relation inserted from the start.

First we note that the right hand side of \eqnref{solution_duality_transformed_general_dux} is proportional to $v$, meaning we can solve $\frac{\partial Y}{\partial u} = 0$ with the duality relation inserted for the parts of $\frac{\partial Y}{\partial u}$ that are multiplying $\id V_3^jV_2^kV_1^l$ and $v^2V_3^jV_2^kV_1^l$ for all nonnegative integer values of $j$, $k$ and $l$. 
This turns out to be very powerful conditions giving many different short equations that are easy to solve w.r.t. the ansatz variables $a_i$ and thus a suitable starting point. 
Next we insert the ansatz in the linear equations \eqnref{solution_duality_linear} and read off the coefficients $c_{ijkl}(a_1,a_2,\dots)$ from what is multiplying each order of polynomial variables $v^iV_3^jV_2^kV_1^l$ in the expanded equations and solve $c_{ijkl}=0$ to determine more ansatz variables $a_i$. 
These equations are linear and thus fairly easy to solve and usually a majority of the coefficients can be determined by the 2 sets of equations solved so far.  

Now is a good time to save all solutions so far and try to find solutions to \eqnref{solution_duality_transformed_general_dux} using different functions $g(Y)$.
We start by choosing $g=1/2$ corresponding to the linear equations in \eqnref{solution_duality_transformed_linear}, which can be solved in the same fashion. 
Even if we don't find any nontrivial solution on this form, all the trivial solutions (see \secref{solution_result}) will be found and we can use them to remove one coefficient per trivial solution (see {\it Trivial solutions} in section \ssecref{solution_trivial}) in the previously saved solution.

Next, considering the saved solution with the coefficients corresponding to trivial solutions removed, there is usually only a few coefficients left and by setting $g(Y)=\sqrt{Y}$ corresponding to the starting equations, we can try to solve them by inserting the ansatz in one of the squared\footnote{This avoids the series expansion of $\sqrt{Y}$, which would (probably) give equations of too high order in the ansatz variables $a_i$ to solve. Now we get equations to order 2 in $a_i$ and possible solutions with the wrong sign can easily be removed by verification of the solution in the equation we really want to solve.} nonlinear relations \eqnref{solution_equations_8D_aligned}.
The second order equations in the remaining $N_r$ variables will be on the form
\begin{align}
\eqnlab{solution_second_order}
\sum_{i=0}^{N_r}\sum_{j=0}^i c^{(p)}_{ij}a\bd{i}a\bd{j}=0
\end{align}
where we have introduced $a_0=1$ to get a compact expression and $c^{(p)}_{ij}$ are the different coefficients for the $p$:th equation, i.e. we are left to solve general form second order equations in $N_r$ variables.  
Just trying to solve these second order equations will be practically impossible when $N_r$ isn't very small, since inserting the solutions of second order equations containing square roots of the coefficients quickly increases complexity of the problem. 
Instead we try to transform the equations into a system of linear equations in $N_r(N_r+3)/2$ variables, i.e. the equations becomes  
\begin{align}
\sum_{k=0}^{N_r(N_r+3)/2} c^{(p)}_{k}b\bd{k}=0
\end{align}
where $b_k=a_ia_j$ are all the combinations of degree 2 in $a$ and $c^{(p)}_k$ are the corresponding coefficients $c^{(p)}_{ij}$ from \eqnref{solution_second_order}.
Solving these equations corresponds to an elimination of unwanted nonlinear terms in the starting equations \eqnref{solution_second_order}.    
This seems to be a successful method only if the number of variables left is not too high (and could thus as well be solved using Mathematica on their original form).

Note that when we solve the equations for an ansatz $Y^{(N)}$ of order $N$, we expand the equations to arbitrary high orders.
If we want to solve the equations for an ansatz $Y^{(N+1)}$ of order $N+1$, which contain all terms in $Y^{(N)}$ plus terms of order $N+1$, the only information we can use from the previous solution $Y^{(N)}$ is conditions from the equations expanded to order $N-1$, all higher order equations will be affected by terms coming from (possibly differentiated) $N+1$ order terms in $Y^{(N+1)}$.    
Since the number of ansatz coefficients grows faster than the number of overlapping equations it is more or less meaningless to save the information gained when solving the equations for $Y^{(N)}$ and we can neither use solutions of lower order ansatzes to say anything about the number of and appearances of solutions of higher order ansatzes. 


\subsection{Result}
\seclab{solution_result}
We will distinguish between 2 different kinds of solutions.
First we have trivial solutions $\Psi_i(u,v)$ such that
\begin{equation}
\Psi_i(-\sqrt{G}G^{-1}v,v) = \frac{\partial\Psi_i}{\partial u}(-\sqrt{G}G^{-1}v,v) = \frac{\partial\Psi_i}{\partial v}(-\sqrt{G}G^{-1}v,v) = 0,  
\end{equation}
which solves the equations for all functions $g(Y)$ with $g(0)=0$.
Thus some coefficients in the ansatz will be undetermined when using the relation $u=-\sqrt{G} G^{-1}v$ and we can add different function combinations of $\Psi_i$ without changing the solution.
We are not so interested in such solutions because we expect them to be fixed when introducing $w$ and we thus remove them from the ansatz when detected. 
  
Secondly we have solutions $\Phi_i$ which are not 0 when inserting the duality relation but nevertheless solves the equations \eqnref{solution_equations_8D_aligned}.
Since the equations are nonlinear we cannot add such solutions to each other and we don't know how many (if any) different solutions we can expect to find.
In other words the solution should be on the form
\begin{align}
\Phi = \Phi_i + \sum_j g_j(u,v)\cdot\lp \psi_j(\Psi_1,\Psi_2,\dots) - \psi_j(0,0,\dots)\rp
\end{align}
where $g_j$ are general finite scalar functions and $\psi_j(\Psi_1,\Psi_2\dots)$ are scalar functions of the $\Psi$:s such that $\frac{\partial \psi_j}{\partial \Psi_i}(0,0,\dots)$ is finite. 

\subsubsection{Trivial solutions}
\sseclab{solution_trivial}
For low ansatz orders the found unique, i.e. not constructed from a previous trivial solution multiplied by a polynomial, trivial $\Psi$ solutions are
\begin{align}
\mbox{ order 4: }&\Psi_1 = -\lp \tr u\rp^2\lp \tr v\rp^2 + \tr u^2\lp \tr v\rp^2 + 4 \tr u \tr v \tr (uv) - 2 \lp \tr (uv)\rp^2\nn\\
&- 4 \tr v \tr (u^2v) + \lp \tr u\rp^2 \tr v^2 - \tr u^2 \tr v^2 - 4 \tr u \tr (uv^2) + 6 \tr (u^2v^2)\nn\\ 
%
\mbox{ order 5: }&\Psi_2 = -\tr u^2\lp \tr v\rp^3 + 2\tr u\lp \tr v\rp^2\tr (uv) - 4\tr v\lp \tr (uv)\rp^2\nn\\
&+ \lp \tr v\rp^2\tr (u^2v) - \lp \tr u\rp^2\tr v\tr v^2 + 4\tr u^2\tr v\tr v^2 - 3\tr (u^2v)\tr v^2\nn\\
&- 2\tr u\tr v\tr (uv^2) + 6\tr (uv)\tr (uv^2) + \lp \tr u\rp^2\tr v^3 - 3\tr u^2\tr v^3\nn\\
%
&\Psi_3 = -\tr u \tr u^2\lp \tr v\rp^2 + \tr u^3\lp \tr v\rp^2 + 2\lp\tr u\rp^2\tr v\tr(uv)\nn\\
& - 4\tr u\lp \tr(uv)\rp^2 - 2\tr u\tr v\tr(u^2v) + 6\tr(uv)\tr(u^2v) - \lp\tr u\rp^3\tr v^2\nn\\
& + 4\tr u\tr u^2\tr v^2 - 3\tr u^3\tr v^2 + \lp\tr u\rp^2\tr(uv^2) - 3\tr u^2\tr(uv^2)\nn\\
%
\mbox{ order 6: }&\Psi_{4-8} \nn\\
\mbox{ order 7: }&\Psi_{9-15}\mbox{, and so on.}
\end{align}
The number of trivial solutions grows pretty quick since each unique trivial solution of order n should also be included multiplying a polynomial of order $N-n$, where $N$ is the total ansatz order, e.g. with ansatz order 12 the number of trivial solutions becomes about 3130 (this number might vary a little due to overlap of terms coming from different trivial solutions). 

Note that the $\Psi$ solutions can be used to set some ansatz coefficients to 0 before doing any calculations at all. This is because one term in the $\Psi$ solution effectively can be rewritten as minus the other terms and the term is thus not needed. The only information lost in the solution is the removed $\Psi$ solution which we already know and can add manually at the end to possibly simplify other solutions. 
E.g. in the order 8 ansatz we can set (read the factors from table \tabref{solution_ansatz_orders})
\begin{align}
\underbrace{1\cdot 52}_{\Psi_1} + \underbrace{2\cdot 23}_{\Psi_{2,3}} + \underbrace{5\cdot 9}_{\Psi_{4-8}} + \underbrace{7\cdot 3}_{\Psi_{9-15}} = 164 
\end{align}
different coefficients to 0, knowing the previously calculated $\Psi$ solutions up to order 7.
Since a higher ansatz covers all solutions of a lower ansatz we are only interested in solving the equations for one ansatz order which is as high as possible.
We can find the trivial solutions easy by simply solving the equations $\Phi=\frac{\partial\Phi}{\partial u}=\frac{\partial\Phi}{\partial v}=0$, which are pretty simple linear equations.
All found trivial solutions now let us put a pretty big number of ansatz coefficients to 0 (compare the example of order 8 where 23\% of the coefficients can be set to 0) and finally we can put all power in one effort to solve the linear-nonlinear equation pair.
We have the opportunity to arbitrarily choose which term of the trivial solution to remove by setting its ansatz coefficient to zero.
A good choice seems to be to remove the ansatz term which, when varied w.r.t. $u$ and with $u(v)$ inserted to a certain order, consist of the highest term count (as a polynomial in $\{v,V_1,V_2,V_3\}$). 
This should reduce the number of terms coming from $\frac{\partial\Phi}{\partial u}$ when constructing the duality equations.
Note that we have to be very careful not setting all the coefficients in one trivial solution to zero, since that would mean removing the corresponding terms completely from possible nontrivial solutions.  

\subsubsection{Nontrivial solutions}
We now want to find nontrivial solutions to the transformed duality equations \eqnref{solution_duality_transformed_general} by first solving all linear conditions as mentioned above and then try to construct a solution by choosing a combination of the remaining ansatz coefficients and g(Y) such that \eqnref{solution_duality_transformed_general} is fulfilled with the duality relation inserted.
For ansatz orders less than 6 no coefficients are left to solve with the nonlinear condition.
For ansatz order 6 we have, with the duality relation inserted
\begin{align}
Y & = a_1 + \frac{a_2}{\det G} \lp V_1^4 + 4V_1^2V_2 + 8V_1V_3 - 8V_3^2 - 2V_1^2V_3^2 - 4V_2V_3^2 + V_3^4 \rp\nn\\
\frac{\partial Y}{\partial u} & = -\frac{2a_2}{\sqrt{\det G}} \lp 2 + V_1^2 + 2V_2 - V_3^2\rp v   
\end{align}
and we wish to solve 
\begin{align}
g(Y) = \frac{2a_2}{\sqrt{\det G}} \lp 2 + V_1^2 + 2V_2 - V_3^2\rp. 
\end{align}
We see that $(a_1 = b, a_2 = 0)$, where $b$ is a constant such that $g(b)=0$ is a solution. Since $a_1$ corresponds to the constant term in the ansatz this solution simply corresponds to constant $\Phi=-1$, which is not a valid solution because the $\sqrt{1+\Phi}$ factor originally occurred as a denominator and since we expect actions on the form \eqnref{dynamics_final_action} not to have a constant term other than the existing $1$.
We also see that $(a_1 = 1, a_2 = 1/4)$ solves $g(Y) = \sqrt{Y}$, which corresponds to the starting equations, giving 
\begin{align}
\eqnlab{csolution_paper_solution}
\Phi=\frac{1}{2}\left[\tr v^2 - \tr u^2 + \frac{1}{2}\lp\tr\lp uv\rp\rp^2 - \tr\lp u^2v^2\rp - V_3^2\right]
\end{align}
where
\begin{align}
V_3^2 &= \frac{1}{36}\lp\tr v\rp^6 - \frac{1}{6}\lp\tr v\rp^4\tr v^2 + \frac{1}{4}\lp\tr v\rp^2\lp\tr v^2\rp^2\nn\\
&+ \frac{1}{9}\lp\tr v\rp^3\tr v^3 - \frac{1}{3}\tr v\tr v^2\tr v^3 + \frac{1}{9}\lp\tr v^3\rp^2 
\end{align}

In the same fashion we have calculated ansatz orders 8, 10 and 12, each with an extra term and the result for order 12 is
\begin{align}
\eqnlab{csolution_order12}
&Y = a_1 + \frac{a_2}{\det G} \big( V_1^4 + 4V_1^2V_2 + 8V_1V_3 - 8V_3^2 - 2V_1^2V_3^2 - 4V_2V_3^2 + V_3^4 \big)\nn\\
&+\frac{a_3}{\lp\det G\rp^2}\big(
(V_1 - V_3)(V_1^3 + 4V_1V_2 + 8V_3 + V_1^2V_3 + 4V_2V_3 - V_1V_3^2 - V_3^3)\nn\\
&\cdot(16 + 16V_1^2 + V_1^4 + 32V_2 + 4V_1^2V_2 + 16V_2^2 - 24V_1V_3 + 8V_3^2 - 2V_1^2V_3^2 - 4V_2V_3^2 + V_3^4)
\big)\nn\\
&+\frac{a_4}{\lp\det G\rp^{5/2}}\big(
(V_1 - V_3)(2 + V_1^2 + 2V_2 - V_3^2)(8V_3 + (V_1 + V_3)(V_1^2 + 4V_2 - V_3^2))
\big)\nn\\
&+\frac{a_5}{\lp\det G\rp^3}\big( 
(V_1 - V_3)(8V_3 + (V_1 + V_3)(V_1^2 + 4V_2 - V_3^2))(14V_1^8 - 34(1 + V_2)^4 - 38V_1^5V_3\nn\\
& - 4(1 + V_2)^2(279 + 131V_2)V_3^2 +(-55 + V_2(634 + 355V_2))V_3^4 - (93 + 112V_2)V_3^6 + 14V_3^8\nn\\
& + V_1^6(131 + 112V_2 - 56V_3^2) + 4V_1^3V_3(296 - 38V_2 + 19V_3^2) +2V_1V_3(592(1 + V_2)^2\nn\\
& + 4(186 + 19V_2)V_3^2 - 19V_3^4) + V_1^4(97 + 786V_2 + 355V_2^2 - (355 + 336V_2)V_3^2 + 84V_3^4) \nn\\
& +V_1^2(4(1 + V_2)^2(-17 + 131V_2) - 2(1357 + 355V_2(2 + V_2))V_3^2 + (317 + 336V_2)V_3^4 - 56V_3^6))
\big)\nn\\
&\frac{\partial Y}{\partial u} = -\frac{2a_2}{\sqrt{\det G}}v \big( 2 + V_1^2 + 2V_2 - V_3^2\big) \nn\\   
&-\frac{4a_3}{\lp\det G\rp^{3/2}}v \big( (2 + V_1^2 + 2V_2 - V_3^2)(8 + 8V_1^2 + V_1^4 + 16V_2 + 4V_1^2V_2\nn\\
&\hspace{2.5cm} + 8V_2^2 - 8V_1V_3 - 2V_1^2V_3^2 - 4V_2V_3^2 + V_3^4)\big) \nn\\
&+\frac{a_4}{\lp\det G\rp^{2}}v \big((V_1 - V_3)(8V_3 + (V_1 + V_3)(V_1^2 + 4V_2 - V_3^2))\nn\\
&\hspace{2.5cm} \cdot(5V_1^4 + 16(1 + V_2)^2 + 8V_1V_3 - 4(6 + 5V_2)V_3^2 + 5V_3^4 + 2V_1^2(8 + 10V_2 - 5V_3^2))\big)\nn\\
&-\frac{4a_5}{\lp\det G\rp^{5/2}}v \big(
(2 + V_1^2 + 2V_2 - V_3^2)(21V_1^8 - 17(1 + V_2)^4 - 2(1 + V_2)^2(541 + 262V_2)V_3^2\nn\\
& + 74V_1^5V_3 + (410 + V_2(1082 + 467V_2))V_3^4 -(205 + 168V_2)V_3^6 + 21V_3^8 +2V_1V_3(558(1 + V_2)^2\nn\\
&  + V_1^6(131 + 168V_2 - 84V_3^2) - 4V_1^3V_3(-279 - 74V_2 + 37V_3^2) - 2(-131 + 74V_2)V_3^2 + 37V_3^4)\nn\\ 
& +V_1^2(2(1 + V_2)^2(-17 + 262V_2) - 2(1082 + 467V_2(2 + V_2))V_3^2 + (541 + 504V_2)V_3^4 - 84V_3^6))\nn\\
& + V_1^4(114 + 786V_2 + 467V_2^2 - (467 + 504V_2)V_3^2 + 126V_3^4)
\big).
\end{align}
We have not been able to choose $g(Y)$ and the coefficients $a_i$ to find any more solutions.
In conclusion, the only found solution so far is \eqnref{csolution_paper_solution}, also found in \cite{artikeln}.
Although the order 12 equations \eqnref{csolution_order12} look rough to solve with other solutions, we should remember that moving to higher orders will most likely introduce even more terms to adjust, so it is still an open problem to prove the uniqueness of the solution with the used relation between $u$ and $v$.
Of course the form (especially the square roots) of the equations \eqnref{csolution_order12} is highly correlated to the choice $u(v)$ and other such relations would most likely give other solutions.  
If we demand $\Phi$ to be uniquely determined from the equations of motion, we have failed, because of all the trivial solutions.
The reason these appear is probably that the relation $u(v)$ is wrong or that the general case $w\ne 0$ will be more restrictive.
According to the procedure in \secref{solution_equation_transformation} we could also construct new solutions from the found one if we do not restrict $\Phi$ to a finite polynomial.

%The paper solution
%\begin{align}
%\Phi = &+ \frac{1}{2}\tr v^2 - \frac{1}{2}\tr u^2 + \frac{1}{4}\lp\tr\lp uv\rp\rp^2 - \frac{1}{2}\tr\lp u^2v^2\rp\nn\\
%& - \frac{1}{12}\lp\tr v^2\rp^3 + \frac{1}{4}\tr v^2\tr v^4 - \frac{1}{6}\tr v^6\nn\\ 
%& + a\bigg[ 
%\lp\tr\lp uv\rp\rp^2\lp\tr u^2 - 2\tr v^2\rp
%+ 2\tr\lp u^2v^2\rp\lp\tr v^2-\tr u^2\rp\nn\\
%&
%+ 2\tr\lp uv\rp\lp 3\tr\lp uv^3\rp - \tr\lp u^3v\rp\rp 
%+ 3\tr\lp v^2u^4\rp 
%- 6\tr\lp u^2v^4\rp\nn\\
%& + \lp\tr v^2\rp^3
%- 4\tr v^2\tr v^4 
%+ 3\tr v^6
%\bigg]
%\end{align}
%where $a$ is a constant, multiplying an order 6 function $\Psi(u,v)$

\subsubsection{Solution for general values on $\tilde\alpha_1$ and $\tilde\alpha_2$}
\sseclab{csolution_paper_general_alpha}
To solve the equations \eqnref{csolution_equations_8D_w0_alpha} for general values of $\tilde\alpha_1$ and $\tilde\alpha_2$ we assume the relation between $u$ and $v$ to be proportional to the conjugate 1-form as before. 
To reduce the complexity of the now more nonlinear equations we use the invariants for $v^2$ (see header of section \secref{csolution_8d_general}) and a smaller ansatz (all terms of even orders and quadratic in $v$, see subsection \ssecref{solution_general_ansatz}). 
When expanding the equations to sixth order, there is a unique relation between the proportionality factor, $\tilde\alpha_1$ and $\tilde\alpha_2$. 
If we use $\omega_\circ$ in the direction $-\qdp{r}$ we get $v=-q_1\omega_\circ = q_1\omega_\parallel$ for the $w=0$ case.
Rather than changing the relation between $u$ and $v$, which get complex, we redefine $v\rightarrow q_1 v$ in the duality equations, so to leave them unchanged we also have to redefine $\tilde\alpha_1\rightarrow \tilde\alpha_1/q_1$ and $\tilde\alpha_2\rightarrow \tilde\alpha_2/q_1$. 
If we let 
\begin{align}
\eqnlab{csolution_dualit_general_alpha}
u(v) = \frac{2}{\tilde\alpha_2-\tilde\alpha_1}\sqrt{\det G}G^{-1}v
\end{align}
we find the sixth order nontrivial solution\footnote{There are now 2 trivial solutions, in contrary to one before, out of which one is independent of $\tilde\alpha$.} to the duality equations to be
\begin{align}
\eqnlab{csolution_phi_general_alpha}
\Phi &= \det G - 1 + \frac{\tilde\alpha_1^2}{4}\lp\tr\lp uv\rp\rp^2 + \frac{\tilde\alpha_1}{4}\lp\tilde\alpha_2-\tilde\alpha_1\rp\tr\lp u^2\lp \id+v^2\rp\rp\nn\\
&+ \frac{\tilde\alpha_1}{\tilde\alpha_2-\tilde\alpha_1}\lp V_2 - 2V_4 - 3\frac{1}{36}\star(vvv)^2\rp  
% These terms are without the u-transformation
%&+ a_1\bigg[-\frac{1}{2}\star(uvv)^2 + \frac{1}{2}\star(uuv)\star(vvv) + \lp\tr\lp uv\rp\rp^2\tr v^2\nn\\
%& - \tr u^2\lp\tr v^2\rp^2 + \tr v^2\tr\lp u^2v^2\rp - 2\tr\lp uv\rp\tr\lp uv^3\rp + \tr u^2\tr v^4\bigg]\nn\\
%&+ a_2\bigg[-\frac{1}{2}\star(uuv)^2 - \lp\tr u^2\rp^2\tr v^2 + \tr u^4\tr v^2\nn\\
%&+ b^2\lp \star(uuv)\star(vvv) + 2\lp\tr\lp uv\rp\rp^2\tr v^2 - 2\tr v^2\tr\lp u^2v^2\rp\rp\nn\\
%&+ b^4\lp - \frac{1}{2}\star(vvv)^2 - \lp\tr v^2\rp^3 + \tr v^2\tr v^4\rp\bigg]
.
\end{align}
To easily see that this really is a solution we note that
\begin{align}
\left.\tr\lp u^2\lp \id+v^2\rp\rp\right|_{u=u(v)} &= \frac{2}{\tilde\alpha_2-\tilde\alpha_1}\sqrt{G}\left.\tr\lp uv\rp\right|_{u=u(v)}\\
& = \lp\frac{2}{\tilde\alpha_2-\tilde\alpha_1}\rp^2\lp V_2 - 2V_4 - 3\frac{1}{36}\star(vvv)^2\rp\nn 
\end{align}
makes $1+\Phi$ an even square, so
\begin{align}
\sqrt{1+\Phi}\left.\right|_{u=u(v)} = \sqrt{G} + \frac{\tilde\alpha_1}{2}\left.\tr\lp uv\rp\right|_{u=u(v)}
\end{align}
and the variations
\begin{align}
\frac{\partial\Phi}{\partial u} &=\frac{\tilde\alpha_1^2}{2}\tr\lp uv\rp v + \frac{\tilde\alpha_1}{2}\lp\tilde\alpha_2-\tilde\alpha_1\rp u\lp \id+v^2\rp\nn\\
& = \tilde\alpha_1\lbp \frac{\tilde\alpha_1}{2}\tr\lp uv\rp + \sqrt{G}\rbp v \nn\\
\frac{\partial\Phi}{\partial v} &=2\det G G^{-1}v + \frac{\tilde\alpha_1^2}{2}\tr\lp uv\rp u + \frac{\tilde\alpha_1}{2}\lp\tilde\alpha_2-\tilde\alpha_1\rp u^2v \nn\\
& + \frac{2\tilde\alpha_1}{\tilde\alpha_2-\tilde\alpha_1}\lp \id - 2v^2 + 2V_2 + 3v^4 - 3V_4 - 3V_2v^2\rp v \nn\\
&= \lbp (\tilde\alpha_2-\tilde\alpha_1)\sqrt{G} + \frac{\tilde\alpha_1^2}{2}\tr\lp uv\rp +\frac{\alpha_1}{\sqrt{G}}\lp\id + 2V_2 - 3V_4 + 4V_6\rp\rbp u \nn\\
&= \lbp (\tilde\alpha_2-\tilde\alpha_1)\sqrt{G} + \frac{\tilde\alpha_1^2}{2}\tr\lp uv\rp +\alpha_1\sqrt{G} + \alpha_1\frac{\tilde\alpha_2-\tilde\alpha_1}{2}\tr\lp uv\rp \rbp u\nn\\
&= \tilde\alpha_2\lbp \sqrt{G} + \frac{1}{2}\tr\lp uv\rp \rbp u
\end{align}
where we have used $G=\id+v^2$ and $v = \frac{\tilde\alpha_2-\tilde\alpha_1}{2}\frac{1}{\sqrt{-\det G}}Gu$, obviously solves the duality equations.
We end this section by stating the solution to the equations on its final form without introduction of any parameters other than the relation between $\omega_\parallel$ and $\omega_\circ$ and maybe $\beta_2$.
The equations are
\begin{align}
\frac{\partial\Phi}{\partial u} &= \frac{4}{3}q_1\sqrt{1+\Phi}v\nn\\
\frac{\partial\Phi}{\partial v} &= -\frac{2}{3}q_1\sqrt{1+\Phi}u,
\end{align}
where $v=q_1\omega_\parallel$, $u=*f$, $w=\omega_\perp = 0$ and
\begin{align}
\eqnlab{csolution_phi_alpha_zero}
\Phi &= \frac{1}{3}\lbp \frac{4}{3}\lp\tr\lp uv\rp\rp^2 - 2\tr\lp u^2\lp \id+v^2\rp\rp + V_2 - V_4 + \frac{1}{12}\star(vvv)^2\rbp
\end{align}
together with
\begin{align}
u(v) = -q_1\sqrt{-\det G}G^{-1}v.
\end{align}
Remember that we still use $\beta_2=\frac{1}{6}$ to remove the $*(v\we v)$ term. Some attempts to interpret these equations with $\beta_2=0$ will be found in section \secref{solution_result2}. 

\section{Computer solution of the $d=8$ $D2$ general case}
\seclab{csolution_8d_general}
%In this section we will try and solve the equations of motion in the general case.

In the previous section we solved the equations, treating $u$ and $v$ as matrices, not caring about the actual index structure.
For the duality equations of the general parameter free case \eqnref{solution_8d_duality_general}, we need to include dualities of the field strengths and we thus have to be more careful about the index structure.
We will treat $\lp v^2\rp\od{\alpha\alpha'} = v\od{\alpha}\ou{m}v\od{m\alpha'}$ as matrices in the world volume indices, with invariants
\begin{align}
V_2 &=\tr v^2\nn\\
V_4 &=\half\lp\tr v^4 - \lp\tr v^2\rp^2\rp\nn\\
V_6 &=\frac{1}{3}\tr v^6 - \frac{1}{2}\tr v^2\tr v^4 + \frac{1}{6}\lp\tr v^2\rp^3
\end{align}
and the Cayley-Hamilton relation
\begin{align}
\eqnlab{csolution_cayley6}
v^6 = V_6 + V_4v^2 + V_2v^4
\end{align}

\subsection{Introduction of dual field strengths}
We want to alter the formalism so we can include 1-forms of the type $\epsilon_{mnp}*\lp v^n\we v^p\rp$, which appears in the duality equations \eqnref{solution_8d_duality_general}.
We introduce the following notation
\begin{align}
\star(uv)_m^\alpha &= \epsilon_{mnp}\lp *(u^n\we v^p)\rp^\alpha = \epsilon_{mnp}\varepsilon^{\alpha\beta\gamma} u^n_\beta v^p_\gamma\\
\star(uvw) &= \epsilon_{mnp}*(u^m\we v^n\we w^p) = \epsilon_{mnp}\varepsilon^{\alpha\beta\gamma}u^m_\alpha v^n_\beta w^p_\gamma,
\end{align}
where the positional order of the tensors $u$, $v$ and $w$ doesn't matter.
In particular we have $\star(vvv)=\frac{1}{3!}\det v_\alpha^m$.
\paragraph{Star squared}
We will frequently use the multiplication of different dualities of $v$, so we start by calculating them straightforwardly as
\begin{align}
\eqnlab{csolution_starsquared1}
\star(vv)_m^\alpha\star(vv)^{m'}_{\alpha'} &= \epsilon_{mnp}\varepsilon^{\alpha\beta\gamma} v^n_\beta v^p_\gamma \epsilon^{m'n'p'}\varepsilon_{\alpha'\beta'\gamma'} v_{n'}^{\beta'} v_{p'}^{\gamma'}\nn\\ 
&=-36\delta_{[mnp]}^{m'n'p'}\delta^{\alpha\beta\gamma}_{[\alpha'\beta'\gamma']} v^n_\beta v^p_\gamma v_{n'}^{\beta'} v_{p'}^{\gamma'}\nn\\
&=4\Big(\delta^{\alpha}_{\alpha'}\delta_{m}^{m'}V_4 + \delta^{\alpha}_{\alpha'} V_2\lp v^2\rp_m^{m'} - \delta^{\alpha}_{\alpha'} \lp v^4\rp_m^{m'}\nn\\
&-\delta_{m}^{m'}\lp v^4\rp^\alpha_{\alpha'} - \lp v^2\rp^{m'}_m \lp v^2\rp^\alpha_{\alpha'} - V_2 v^{m'}_{\alpha'}v_{m}^{\alpha}\nn\\
&+\lp v^3\rp^{m'}_{\alpha'} v_{m}^{\alpha} + \delta_{m}^{m'}V_2 \lp v^2\rp^\alpha_{\alpha'} + \lp v^3\rp_{m}^{\alpha}v^{m'}_{\alpha'}\Big)\\
%
\star(vv)_m^\alpha\star(vv)^{m}_{\alpha'} &= \delta^m_{m'}\star(vv)_m^\alpha\star(vv)^{m'}_{\alpha'}\nn\\
& = 4\lp V_4\delta^{\alpha}_{\alpha'} + V_2\lp v^2\rp_{\alpha'}^\alpha - \lp v^4\rp_{\alpha'}^\alpha\rp,\\
%
\eqnlab{csolution_star_squared3}
\star(vv)_m^\alpha\star(vvv) &= v_{m'}^{\alpha'}\star(vv)_m^\alpha\star(vv)^{m'}_{\alpha'}\nn\\
& = 12\lp V_4v_{m}^{\alpha} + V_2\lp v^3\rp_{m}^{\alpha} - \lp v^5\rp_{m}^{\alpha}\rp,\\
%
\star(vvv)\star(vvv) &= v^m_\alpha\star(vv)_m^\alpha\star(vvv)\nn\\
& = 12\lp V_2V_4 + V_2\tr v^4 - \tr v^6\rp = -36V_6.
\end{align}
We now let $\VS{3} = \star(vvv) = 6\sqrt{-V_6}$, which we will use instead of $V_6$ when expanding the equations, i.e. we expand the equations in the independent variables $V_2, \VS{3}$ and $V_4$. 

\paragraph{Reductions}
We want to be able to reduce $\star$ acting on $v$ of different powers.
We start by trying to reduce tensors on the forms $v^m_\alpha$, $\lp v^3\rp^m_\alpha$, $\lp v^5\rp^m_\alpha$, $\star(vv)^m_\alpha$, $\star(vv)^m_\beta \lp v^2\rp^\beta_\alpha$ and $\star(vv)^m_\beta \lp v^4\rp^\beta_\alpha$ in terms of each other and the invariants $V_2$, $\VS{3}$ and $V_4$.
Examine the expression
\begin{align}
\star&(vvv)\lp v^x\rp^\alpha_m = \epsilon_{m'n'p'}\M_{mn}\varepsilon^{\alpha'\beta'\gamma'}g^{\alpha\beta}v^{m'}_{\alpha'} v^{n'}_{\beta'} v^{p'}_{\gamma'}\lp v^x\rp_\beta^n\nn\\
&= \lp \epsilon_{mn'p'}\M_{m'n} + \epsilon_{m'mp'}\M_{nn'} + \epsilon_{m'n'm}\M_{np'}\rp\varepsilon^{\alpha'\beta'\gamma'}g^{\alpha\beta}v^{m'}_{\alpha'} v^{n'}_{\beta'} v^{p'}_{\gamma'}\lp v^x\rp_\beta^n\nn\\
&= \epsilon_{mnp}\varepsilon^{\alpha'\beta\gamma} v^{n}_{\beta} v^{p}_{\gamma}\lp v^{x+1}\rp^{\alpha}_{\alpha'} - \epsilon_{mnp}\varepsilon^{\alpha'\beta\gamma}v^{n}_{\alpha'} v^{p}_{\gamma}\lp v^{x+1}\rp^{\alpha}_\beta + \epsilon_{mnp}\varepsilon^{\alpha'\beta\gamma}v^{n}_{\alpha'} v^{p}_{\beta}\lp v^{x+1}\rp^{\alpha}_\gamma\nn\\
&= 3\lp v^{x+1}\star(vv)\rp^\alpha_m,
\end{align}
where x is an odd positive integer, giving
\begin{align}
\eqnlab{csolution_S2v2_reduction}
\star(vv)^m_\beta \lp v^2\rp^\beta_\alpha & = \frac{1}{3}\star(vvv)v_\alpha^m\\
\star(vv)^m_\beta \lp v^4\rp^\beta_\alpha & = \frac{1}{3}\star(vvv)\lp v^3\rp_\alpha^m
\end{align}
In a similar way we can reduce
\begin{align}
\star(vv)_\beta^m \lp v\rp_m^\alpha & = \frac{1}{3}\star(vvv)\delta^\alpha_\beta\\
\star(vv)_\beta^m \lp v^3\rp_m^\alpha & = \frac{1}{3}\star(vvv)\lp v^2\rp^\alpha_\beta
\end{align}
By using \eqnref{csolution_star_squared3} we can reduce $v^5$ in terms of $v$, $\star(vv)$ and $v^3$ and the 3 invariants as
\begin{align}
\lp v^5\rp^\alpha_m
&= -\frac{1}{12}\VS{3}\star(vv)^\alpha_m + V_4v^\alpha_m + V_2\lp v^3\rp^\alpha_m,
\end{align}
which becomes the ordinary Cayley-Hamilton relation \eqnref{csolution_cayley6} if multiplied by $v$ (use \eqnref{csolution_S2v2_reduction}).
We can thus express the equations using $v$, $\star(vv)$ and $v^3$ and the 3 invariants. 
To be able to insert the duality relations we will need to reduce terms of the types $\star(uvw)$ and $\star(uv)^\alpha_m$, coming from the variations of $\star(uvw)$, where $u$, $v$ and $w$ are tensors of one of the forms $v^m_\alpha$, $\lp v^3\rp^m_\alpha$ and $\star(vv)^m_\alpha$.
We have
\begin{align}
\star((uxy)&v)^\alpha_m = \epsilon_{mnp}\M_{qq'}\varepsilon^{\alpha\beta\gamma}g^{\delta\epsilon} u^q_\beta v^p_\gamma x_\epsilon^{q'}y_\delta^n\nn\\
&= \lp \epsilon_{qnp}\M_{mq'} + \epsilon_{mqp}\M_{q'n} + \epsilon_{mnq}\M_{q'p} \rp\varepsilon^{\alpha\beta\gamma}g^{\delta\epsilon} u^q_\beta v^p_\gamma x_\epsilon^{q'}y_\delta^n\nn\\
&= -\epsilon_{qnp}\varepsilon^{\alpha\beta\gamma} u^n_\beta v^p_\gamma x^\delta_my_\delta^q + \epsilon_{mnp}\varepsilon^{\alpha\beta\gamma} u^n_\beta v^p_\gamma x^\delta_{q}y_\delta^{q} - \epsilon_{mnp}\varepsilon^{\alpha\beta\gamma} u^n_\beta v^{q}_\gamma x^\delta_{q}y_\delta^p \nn\\
&= -x_m^\beta y_\beta^n\star(uv)^\alpha_n + \star(uv)^\alpha_m\tr(xy) - \star(u(vxy))^\alpha_m
\end{align} 
i.e.
\begin{align}
\eqnlab{csolution_star2reduce}
\star((uxy)v)^\alpha_m + \star(u(vxy))^\alpha_m + x_m^\beta y_\beta^n\star(uv)^\alpha_n = \star(uv)^\alpha_m\tr(xy)
\end{align} 
and also
\begin{align}
\eqnlab{csolution_star3reduce}
\star((uxy)vw) + \star(u(vxy)w) + \star(uv(wxy)) &= \star(uvw)\tr(xy).
\end{align} 
We can use these relations to recursively reduce $\star$ acting on different powers of $v$ into the $V_2$, $\VS{3}$ and $V_4$ invariants and the 3 tensor forms mentioned above.

As a start, consider
\begin{align}
\star(v^xv^y)^\alpha_m + \star(v^{x+y-1}v)^\alpha_m + \lp v^{y-1}\rp_m^n\star(v^xv)^\alpha_n = \star(v^xv)^\alpha_m\tr v^{y-1}\\
\star(v^xv)^\alpha_m + \star(vv^x)^\alpha_m + \lp v^{x-1}\rp_m^n\star(vv)^\alpha_n = \star(vv)^\alpha_m\tr v^{x-1}
\end{align}
where x and y are odd positive integers, giving
\begin{align}
\eqnlab{csolution_S2reduction1}
\star(v^xv)^\alpha_m &= \frac{1}{2}\lp \star(vv)^\alpha_m\tr v^{x-1} - \lp v^{x-1}\rp_m^n\star(vv)^\alpha_n \rp\\
\star(v^xv^y)^\alpha_m &= \frac{1}{2}\star(vv)^\alpha_m\tr v^{x-1}\tr v^{y-1} \nn\\
& - \frac{1}{2}\star(vv)^\alpha_m\tr v^{x+y-2} + \lp v^{x+y-2}\rp_m^n\star(vv)^\alpha_n\nn\\
& -\frac{1}{2}\lp v^{y-1}\rp_m^n\star(vv)^\alpha_n\tr v^{x-1}- \frac{1}{2}\lp v^{x-1}\rp_m^n\star(vv)^\alpha_n\tr v^{y-1}
\end{align}
which is our first reduction relation.
Multiplying the latter equation with $\lp v^z\rp_\alpha^m$, where $z$ is an odd positive integer, gives the next reduction relation
\begin{align}
\eqnlab{csolution_S3reduction1}
\star(v^xv^yv^z) &= \frac{1}{6}\Big( \tr v^{x-1}\tr v^{y-1}\tr v^{z-1}  + 2\tr v^{x+y+z-3} - \tr v^{x-1}\tr v^{y+z-2}\nn\\
& - \tr v^{y-1}\tr v^{x+z-2} - \tr v^{z-1}\tr v^{x+y-2}\Big)\star(vvv).   
\end{align}
We can derive similar relations with $v^x\star(vv)$ inside the $\star$
\begin{align}
\eqnlab{csolution_S2reduction2}
\star((v^x\star(vv))v^y)^\alpha_m &= \epsilon_{mnp}\varepsilon^{\alpha\beta\gamma}\lp v^x\rp^n_{m'}\epsilon^{m'n'p'}\varepsilon_{\beta\beta'\gamma'}v^{\beta'}_{n'}v^{\gamma'}_{p'}\lp v^y\rp^p_\gamma\nn\\
&= 12\delta_{[\beta'\gamma']}^{\alpha\gamma}\lp v^x\rp^n_{[m}v^{\beta'}_{n}v^{\gamma'}_{p]}\lp v^y\rp^p_\gamma\nn\\
&= 12\lp v^x\rp^n_{[m}v^{\alpha}_{n}v^{\gamma}_{p]}\lp v^y\rp^p_\gamma\nn\\
&= 2\tr v^{y+1}\lp v^{x+1}\rp^\alpha_{m} + 2\tr v^x\lp v^{y+2}\rp^\alpha_m + 2\tr v^{x+y+1}v^{\alpha}_{m}\nn\\
& - 2\tr v^x\tr v^{y+1}v^{\alpha}_{m} - 4\lp v^{x+y+2}\rp^\alpha_m
\end{align}
where $x$ is an even nonnegative integer and $y$ is a positive odd integer and
\begin{align}
\eqnlab{csolution_S2reduction3}
\star((v^x&\star(vv))(v^y\star(vv)))^\alpha_m %&= \epsilon_{mnp}\varepsilon^{\alpha\beta\gamma}\lp v^x\rp^n_{m'}\epsilon^{m'n'p'}\varepsilon_{\beta\beta'\gamma'}v^{\beta'}_{n'}v^{\gamma'}_{p'}\lp v^y\rp^p_\gamma\nn\\
%&= 12\delta_{[\beta'\gamma']}^{\alpha\gamma}\lp v^x\rp^n_{[m}v^{\beta'}_{n}v^{\gamma'}_{p]}\lp v^y\star(vv)\rp^p_\gamma\nn\\
%&= 12\lp v^x\rp^n_{[m}v^{\alpha}_{n}v^{\gamma}_{p]}\lp v^y\star(vv)\rp^p_\gamma\nn\\
%&= 2\star(v^{y+1}vv) \lp v^{x+1}\rp^\alpha_{m} + 2\tr v^x \lp v^{y+2}\star(vv)\rp^\alpha_m\nn\\
%&+ 2\star(v^{x+y+1}vv) v^{\alpha}_{m} - 2\tr v^x \star(v^{y+1}vv) v^{\alpha}_{m} - 4\lp v^{x+y+2}\star(vv)\rp^\alpha_m\nn\\
%&= \frac{2}{3}\tr v^{y}\star(vvv)\lp v^{x+1}\rp^\alpha_{m} + 2\tr v^x \lp v^{y+2}\star(vv)\rp^\alpha_m\nn\\
%&+ \frac{2}{3}\tr v^{x+y}\star(vvv)v^{\alpha}_{m} - \frac{2}{3}\tr v^x\tr v^{y}\star(vvv)v^{\alpha}_{m} - 4\lp v^{x+y+2}\star(vv)\rp^\alpha_m\nn\\
= \frac{2}{3}\Big( \tr v^{y}\lp v^{x+1}\rp^\alpha_{m} + \tr v^x \lp v^{y+1}\rp^\alpha_m\nn\\
&+ \tr v^{x+y}v^{\alpha}_{m} - \tr v^x\tr v^{y}v^{\alpha}_{m} - 2\lp v^{x+y+1}\rp^\alpha_m\Big)\star(vvv)
\end{align}
where both $x$ and $y$ are nonnegative even integers.
Multiplying \eqnref{csolution_S2reduction2} with $\lp v^z\rp_\alpha^m$ and \eqnref{csolution_S2reduction2} with both $\lp v^z\rp_\alpha^m$ and $(v^x\star(vv))_\alpha^m$ gives the remaining 3 needed reduction relations
\begin{align}
\eqnlab{csolution_S3reduction2}
\star((v^x\star(vv))&v^yv^z) %= \epsilon_{mnp}\varepsilon^{\alpha\beta\gamma}\lp v^x\rp^m_{m'}\epsilon^{m'n'p'}\varepsilon_{\alpha\beta'\gamma'}v^{\beta'}_{n'}v^{\gamma'}_{p'}\lp v^y\rp^n_\beta\lp v^z\rp^p_\gamma\nn\\
%&= -12\lp v^x\rp^m_{[m}v^{\beta}_{n}v^{\gamma}_{p]}\lp v^y\rp^n_{[\beta}\lp v^z\rp^p_{\gamma]}\nn\\
= 2\tr v^{x+y+1}\tr v^{z+1} + 2\tr v^{x+z+1}\tr v^{y+1} + 2\tr v^x\tr v^{y+z+2}\nn\\
& -2\tr v^x \tr v^{y+1}\tr v^{z+1} - 4\tr v^{x+y+z+2} 
\end{align}
where $x$ is an even nonnegative integer and $y$ and $z$ are odd positive integers,
\begin{align}
\eqnlab{csolution_S3reduction3}
\star((v^x\star(vv))&\star((v^y\star(vv))v^z) 
= \frac{2}{3}\Big( \tr v^{z+1}\tr v^{x+y} + \tr v^x\tr v^{y+z+1} \nn\\
& + \tr v^{y}\tr v^{x+z+1} - \tr v^x\tr v^{y}\tr v^{z+1} - 2\tr v^{x+z+y+1}\Big)\star(vvv)
\end{align}
where $x$ and $y$ are even nonnegative integers and $y$ is odd and positive integer and finally 
\begin{align}
\eqnlab{csolution_S3reduction4}
\star((v^x&\star(vv))(v^y\star(vv))(v^z\star(vv))) 
= \frac{2}{9}\Big( \tr v^x \tr v^{y+z} + \tr v^{y}\tr v^{x+z}\nn\\
&+ \tr v^{z}\tr v^{x+y} - \tr v^x\tr v^{y}\tr v^{z} - 2\tr v^{x+y+z}\Big)\lp \star(vvv)\rp^2
\end{align}
where all of $x,y$ and $z$ are nonnegative even integers.
In conclusion we can use \eqnref{csolution_S2reduction1}, \eqnref{csolution_S2reduction2}, \eqnref{csolution_S2reduction3}, \eqnref{csolution_S3reduction1}, \eqnref{csolution_S3reduction2}, \eqnref{csolution_S3reduction3} and \eqnref{csolution_S3reduction4} together with the usual Cayley-Hamilton to reduce all contractions of $\star$ in terms of $v$, $v^3$, $*(v\wedge v)$, $V_2$, $\VS3$ and $V_4$.

\subsection{Ansatz (2)}
\sseclab{solution_general_ansatz}
It is now time to create an ansatz in the same fashion as the one in subsection \ssecref{solution_ansatz}. 
Because of the relation $u=u(v)$ (or expected but not derived relations like \eqnref{csolution_tracecommuterelation}) we will treat $u$ and $v$ as commuting variables inside the traces.
We note that we, except from $u^2$ and $v^2$, also can consider $uv$ as a matrix, obeying the Cayley-Hamilton equations and the reduction of $\star^2$, so all traces with more than a total of 4 $u$:s and $v$:s can be reduced.  
%S�, anv�nd ej:
%\begin{align}
%\tr\lp u^4v^4\rp &= \tr w^4 = \tr (W_3w+W_2w^2+W_1w^3)\nn\\ 
%&= \frac{1}{3}\tr w^3 - \half\tr w\tr w^2 + \frac{1}{6}\lp\tr w\rp^3\tr w+\half\lp\tr w^2-\lp\tr w\rp^2\rp\tr w^2+\tr w^3\tr w\nn\\
%&= \lp\frac{1}{3}\tr w^3 - \half\tr w\tr w^2 + \frac{1}{6}\lp\tr w\rp^3\rp\tr w+\half\lp\tr w^2-\lp\tr w\rp^2\rp\tr w^2+\tr w^3\tr w\nn\\
%& = \frac{4}{3}\tr w^3\tr w - \tr w^2\lp \tr w\rp + \half\lp\tr w^2\rp^2 + \frac{1}{6}\lp\tr w\rp^4
%\end{align} 
Furthermore we can reduce $\star$ acting on different combinations of $u$ and $v$, similar to what we did in the previous section.
%Reduction of $\star$ on different powers:
%\begin{align}
%\star((uxy)vw) &+ \star(u(vxy)w) + \star(uv(wxy)) = \star(uvw)\tr(xy)\nn\\
%\star(\star(xy)zw) &= \epsilon_{mnp}\varepsilon^{\alpha\beta\gamma} z^n_\beta w^p_\gamma \epsilon^{mn'p'}\varepsilon_{\alpha'\beta'\gamma'} x_{n'}^{\beta'} y_{p'}^{\gamma'}\nn\\
%&= -12\delta_{[\alpha'}^{\alpha} z^n_\beta w^p_{\gamma]} x_{[n}^{\beta} y_{p]}^{\gamma}\nn\\
%&= 2\delta_{\alpha'}^{\alpha}\tr\lp xyzw\rp + \tr\lp yw\rp xz + \tr\lp xz\rp yw + \tr\lp yz\rp xw + \tr\lp xw\rp yz\nn\\
%&-\delta_{\alpha'}^{\alpha} \tr\lp xz\rp\tr\lp yw\rp -\delta_{\alpha'}^{\alpha} \tr\lp xw\rp\tr\lp yz\rp - 4xyzw\nn\\
%\tr\lp\star(\star(xy)zw)\rp &= \delta_{\alpha}^{\alpha'}\star(\star(xy)zw)\nn\\
%&= 2\tr\lp xyzw\rp - \tr\lp xz\rp\tr\lp yw\rp - \tr\lp xw\rp\tr\lp yz\rp\nn\\
%\end{align} 
E.g. the new needed order 4 reductions are
\begin{align}
\star(\star(vv)uu) &= \star(\star(uu)vv) = 2\tr\lp u^2v^2\rp - 2\lp\tr\lp uv\rp\rp^2\nn\\
\star(\star(uv)uv) &= -\lp\tr\lp uv\rp\rp^2 - \tr u^2\tr v^2 + 2\tr\lp u^2v^2\rp
\end{align} 
%the order 5 reductions are
%\begin{align}
%\star(u^3vv) &= \star(uvv)\tr u^2 - \star(uuv)\tr\lp uv\rp + \frac{1}{3}\star(uuu)\tr v^2\nn\\
%\star((u^2v)uv) &= \half\star(uuv)\tr\lp uv\rp - \frac{1}{6}\star(uuu)\tr v^2\nn\\
%\star((uv^2)uu) &= \frac{1}{3}\star(uuu)\tr v^2\nn\\
%\star((v^2u)uu) &= \frac{1}{3}\star(vvv)\tr\lp uv\rp\nn\\
%\star((v^3)uv) &= \half\star(uvv)\tr v^2 - \frac{1}{6}\star(vvv)\tr\lp uv\rp
%\end{align}
We thus use the following set of polynomial variables  
\begin{align}
\phi=\{&\tr u^2, \tr v^2, \tr\lp uv\rp, \tr u^4, \tr\lp u^3v\rp, \tr\lp u^2v^2\rp, \tr\lp uv^3\rp, \tr v^4,\nn\\
& \star(uuu), \star(uuv), \star(uvv), \star(vvv)\}.
\end{align}
to build our ansatz. 
Note that some combinations of these are still reducible, e.g. the sixth order relation  
%\paragraph{Reductions containing $(\star)^2$}
%\begin{align}
%\star&(uvw)\star(xyz) = -36u_{[m}^{[\alpha} v_n^\beta w_{p]}^{\gamma]} x^m_\alpha y^n_\beta z^p_\gamma \nn\\
%%& = -6u_{[m}^{\alpha} v_n^\beta w_{p]}^{\gamma} x^m_\alpha y^n_\beta z^p_\gamma \nn\\
%%&-6u_{[m}^{\beta} v_n^\gamma w_{p]}^{\alpha} x^m_\alpha y^n_\beta z^p_\gamma \nn\\
%%&-6u_{[m}^{\gamma} v_n^\alpha w_{p]}^{\beta} x^m_\alpha y^n_\beta z^p_\gamma \nn\\
%%&+6u_{[m}^{\beta} v_n^\alpha w_{p]}^{\gamma} x^m_\alpha y^n_\beta z^p_\gamma \nn\\
%%&+6u_{[m}^{\alpha} v_n^\gamma w_{p]}^{\beta} x^m_\alpha y^n_\beta z^p_\gamma \nn\\
%%&+6u_{[m}^{\gamma} v_n^\beta w_{p]}^{\alpha} x^m_\alpha y^n_\beta z^p_\gamma \nn\\
%& = - \tr\lp ux\rp\tr\lp vy\rp\tr(wz) - \tr\lp ux\rp\tr\lp vz\rp\tr(wy) - \tr\lp uy\rp\tr\lp vx\rp\tr\lp wz\rp\nn\\
%& - \tr\lp uy\rp\tr\lp vz\rp\tr\lp wx\rp - \tr\lp uz\rp\tr\lp vx\rp\tr\lp wy\rp  - \tr\lp uz\rp\tr\lp vy\rp\tr\lp wx\rp \nn\\
%&-12\tr\lp uvwxyz\rp \nn + 2\tr\lp uvxy\rp \tr\lp wz\rp + 2\tr\lp vwyz\rp\tr\lp ux\rp +2\tr\lp uwxz\rp\tr\lp vy\rp\nn\\
%&+2\tr\lp uvxz\rp \tr\lp wy\rp +2\tr\lp vwyx\rp\tr\lp uz\rp +2\tr\lp uwyz\rp\tr\lp vx\rp\nn\\
%&+2\tr\lp uvyz\rp \tr\lp wx\rp +2\tr\lp vwxz\rp\tr\lp uy\rp +2\tr\lp uwxy\rp\tr\lp vz\rp
%\end{align} 
%\begin{align}
%\star(uuv)\star(vvv)
%& = - 6\tr v^2\lp\tr\lp uv\rp\rp^2 - 12\tr\lp u^2v^4\rp\nn\\
%& + 12\tr\lp uv^3\rp \tr\lp uv\rp + 6\tr\lp u^2v^2\rp\tr v^2\nn
%\end{align}
%\begin{align}
%\star(uuv)\star(uuv) 
%& = - 4\tr u^2\lp\tr\lp uv\rp\rp^2 - 2\tr v^2\lp\tr u^2\rp^2 - 12\tr\lp u^4v^2\rp\nn\\
%& + 8\tr\lp uuvv\rp \tr\lp uu\rp + 8\tr\lp vuuu\rp\tr\lp uv\rp +2\tr\lp uuuu\rp\tr\lp vv\rp\nn
%\end{align} 
%\begin{align}
%\star(uvv)\star(uvv) 
%& = -2\tr\lp uu\rp\lp\tr\lp v^2\rp\rp^2 - 4\lp\tr\lp uv\rp\rp^2\tr v^2\nn\\
%&-12\tr\lp u^2v^4\rp \nn + 8\tr\lp u^2v^2\rp \tr v^2 + 8\tr\lp uv^3\rp\tr\lp uv\rp + 2\tr\lp vvvv\rp\tr\lp uu\rp \nn
%\end{align} 
%\begin{align}
%\star&(uuu)\star(vvu) = - 6\lp\tr\lp uv\rp\rp^2\tr u^2 -12\tr\lp u^4v^2\rp + 6\tr\lp u^2v^2\rp \tr\lp u^2\rp + 12\tr\lp u^3v\rp\tr\lp uv\rp \nn
%\end{align} 
%
%Sann �ven med b�de $v$ och $w$:
\begin{align}
&-\half\star(vuu)^2 + \half\star(uuu)\star(vvu) + \tr u^2\lp\tr\lp uv\rp\rp^2 - 2\tr\lp uv\rp\tr\lp u^3v\rp\nn\\
& - \lp\tr u^2\rp^2\tr v^2 + \tr u^4\tr v^2 + \tr u^2\tr\lp u^2v^2\rp = 0
\end{align}
%Ej sann med w (eftersom 4 st $w$ i per term $\Rightarrow$ korstermer):
%\begin{align}
%&-\half\star(vvu)^2 + \half\star(vuu)\star(vvv) + \lp\tr vu\rp^2\tr vv - \tr uu\lp\tr vv\rp^2\nn\\
%& + \tr vv\tr vvuu - 2\tr vu\tr vvvu + \tr uu\tr vvvv = 0
%\end{align}
can be used to express one of the terms with help of the others.
Solving $\Phi = 0$ for a general (random) expansion $u=u(v)$ helps us find all dependent ansatz terms which we remove.  
Of course we could have limited the ansatz to terms with at most one $\star$ per term, but that would force us to introduce traces over all seven sixth order polynomials as ansatz variables.

The last condition on our ansatz is that terms with an odd number of $v$:s are discarded.
This is because we build the ansatz from contractions of the form $\omega^{rm}\omega_r^n$, which also makes the ansatz in terms of $v$ and $w$ independent of $p$. 
When $w\ne 0$ it should be added symmetrically to each combination of 2 multiplying $v$:s in the ansatz.

Note that an ansatz built from these polynomial variables in this way is not the most general one, but when varied it produces the terms entering the right hand side of the duality equations.
Other polynomial variables that could be added are e.g. terms including antisymmetric $SL(2,\rr)$ contractions $\epsilon_{st}\omega^{sn}_\alpha\omega^{tp}_\beta = v_\alpha^nw_\beta^p - w_\alpha^nv_\beta^p$. 

%\paragraph{Linear equations}
%\begin{align}
%\tr\lp u\frac{\partial\Phi}{\partial u}\rp &= \frac{4}{3}\sqrt{1+\Phi}\tr\lp uv\rp\nn\\
%\tr\lp v\frac{\partial\Phi}{\partial u}\rp &= \frac{4}{3}\sqrt{1+\Phi}\tr v^2\nn\\
%\tr\lp w\frac{\partial\Phi}{\partial u}\rp &= \frac{4}{3}\sqrt{1+\Phi}\tr\lp vw\rp\nn\\
%\tr\lp u\frac{\partial\Phi}{\partial v}\rp & = \bigg\{-\frac{2}{3}\tr u^2 + \frac{1}{3}\star(uvw)\bigg\}\sqrt{1+\Phi}\nn\\
%\tr\lp v\frac{\partial\Phi}{\partial v}\rp & = \bigg\{-\frac{2}{3}\tr\lp uv\rp + \frac{1}{3}\star(vvw)\bigg\}\sqrt{1+\Phi}\nn\\
%\tr\lp w\frac{\partial\Phi}{\partial v}\rp & = \bigg\{-\frac{2}{3}\tr\lp uw\rp + \frac{1}{3}\star(vww)\bigg\}\sqrt{1+\Phi}\nn\\
%\tr\lp u\frac{\partial\Phi}{\partial w}\rp & = -\frac{1}{3}\star(uvv)\sqrt{1+\Phi}\nn\\
%\tr\lp v\frac{\partial\Phi}{\partial w}\rp & = -\frac{1}{3}\star(vvv)\sqrt{1+\Phi}\nn\\
%\tr\lp w\frac{\partial\Phi}{\partial w}\rp & = -\frac{1}{3}\star(vvw)\sqrt{1+\Phi}\nn\\
%\intertext{}
%\tr\lp \star(uu)\frac{\partial\Phi}{\partial u}\rp &= \frac{4}{3}\sqrt{1+\Phi}\star(uuv)\nn\\
%\tr\lp \star(uv)\frac{\partial\Phi}{\partial u}\rp &= \frac{4}{3}\sqrt{1+\Phi}\star(uvv)\nn\\
%\tr\lp \star(uw)\frac{\partial\Phi}{\partial u}\rp &= \frac{4}{3}\sqrt{1+\Phi}\star(uvw)\nn\\
%\tr\lp \star(vv)\frac{\partial\Phi}{\partial u}\rp &= \frac{4}{3}\sqrt{1+\Phi}\star(vvv)\nn\\
%\tr\lp \star(vw)\frac{\partial\Phi}{\partial u}\rp &= \frac{4}{3}\sqrt{1+\Phi}\star(vvw)\nn\\
%\tr\lp \star(ww)\frac{\partial\Phi}{\partial u}\rp &= \frac{4}{3}\sqrt{1+\Phi}\star(vww)\nn\\
%\tr\lp \star(uu)\frac{\partial\Phi}{\partial v}\rp & = \bigg\{-\frac{2}{3}\star(uuu) + \frac{1}{3}\lbp 2\tr\lp u^2vw\rp - 2\tr\lp uv\rp\tr\lp uw\rp\rbp\bigg\}\sqrt{1+\Phi}\nn\\
%\tr\lp \star(uv)\frac{\partial\Phi}{\partial v}\rp & = \bigg\{-\frac{2}{3}\star(uuv) + \frac{1}{3}\lbp 2\tr\lp uv^2w\rp - \tr\lp uv\rp\tr\lp vw\rp - \tr\lp uw\rp\tr v^2\rbp\bigg\}\sqrt{1+\Phi}\nn\\
%\tr\lp \star(uw)\frac{\partial\Phi}{\partial v}\rp & = \bigg\{-\frac{2}{3}\star(uuw) + \frac{1}{3}\lbp 2\tr\lp uvw^2\rp - \tr\lp uv\rp\tr w^2 - \tr\lp uw\rp\tr\lp vw\rp\rbp\bigg\}\sqrt{1+\Phi}\nn\\
%\tr\lp \star(vv)\frac{\partial\Phi}{\partial v}\rp & = \bigg\{-\frac{2}{3}\star(uvv) + \frac{1}{3}\lbp 2\tr\lp v^3w\rp - 2\tr v^2\tr\lp vw\rp\rbp\bigg\}\sqrt{1+\Phi}\nn\\
%\tr\lp \star(vw)\frac{\partial\Phi}{\partial v}\rp & = \bigg\{-\frac{2}{3}\star(uvw) + \frac{1}{3}\lbp 2\tr\lp v^2w^2\rp - \tr v^2\tr w^2 - \lp\tr\lp vw\rp\rp^2\rbp\bigg\}\sqrt{1+\Phi}\nn\\
%\tr\lp \star(ww)\frac{\partial\Phi}{\partial v}\rp & = \bigg\{-\frac{2}{3}\star(uww) + \frac{1}{3}\lbp 2\tr\lp vw^3\rp - 2\tr\lp vw\rp\tr w^2\rbp\bigg\}\sqrt{1+\Phi}\nn\\
%\tr\lp \star(uu)\frac{\partial\Phi}{\partial w}\rp & = -\frac{1}{3}\sqrt{1+\Phi}\lbp 2\tr\lp u^2v^2\rp - 2\lp\tr\lp uv\rp\rp^2\rbp\nn\\
%\tr\lp \star(uv)\frac{\partial\Phi}{\partial w}\rp & = -\frac{1}{3}\sqrt{1+\Phi}\lbp 2\tr\lp uv^3\rp - 2\tr\lp uv\rp\tr v^2\rbp\nn\\
%\tr\lp \star(uw)\frac{\partial\Phi}{\partial w}\rp & = -\frac{1}{3}\sqrt{1+\Phi}\lbp 2\tr\lp uv^2w\rp - 2\tr\lp uv\rp\tr\lp vw\rp\rbp\nn\\
%\tr\lp \star(vv)\frac{\partial\Phi}{\partial w}\rp & = -\frac{1}{3}\sqrt{1+\Phi}\lbp 2\tr v^4 - 2\lp\tr v^2\rp^2 \rbp\nn\\
%\tr\lp \star(vw)\frac{\partial\Phi}{\partial w}\rp & = -\frac{1}{3}\sqrt{1+\Phi}\lbp 2\tr\lp v^3w\rp - 2\tr v^2\tr\lp vw\rp\rbp\nn\\
%\tr\lp \star(ww)\frac{\partial\Phi}{\partial w}\rp & = -\frac{1}{3}\sqrt{1+\Phi}\lbp 2\tr\lp v^2w^2\rp - 2\lp\tr\lp vw\rp\rp^2\rbp\nn\\
%\intertext{}
%\tr\lp \frac{\partial\Phi}{\partial u}\frac{\partial\Phi}{\partial u}\rp &= \frac{16}{9}\tr v^2(1+\Phi)\nn\\
%\tr\lp \frac{\partial\Phi}{\partial v}\frac{\partial\Phi}{\partial u}\rp &= \bigg\{-\frac{8}{9}\tr\lp uv\rp + \frac{4}{9}\star(vvw)\bigg\}(1+\Phi)\nn\\
%\tr\lp \frac{\partial\Phi}{\partial w}\frac{\partial\Phi}{\partial u}\rp &= -\frac{4}{9}\star(vvv)(1+\Phi)\nn\\
%\tr\lp \frac{\partial\Phi}{\partial v}\frac{\partial\Phi}{\partial v}\rp & = \bigg\{\frac{4}{9}\tr u^2 - \frac{4}{9}\star(uvw) + \frac{1}{9}\lbp 2\tr\lp v^2w^2\rp - \tr v^2\tr w^2 - \lp\tr\lp vw\rp\rp^2\rbp\bigg\}(1+\Phi)\nn\\
%\tr\lp \frac{\partial\Phi}{\partial w}\frac{\partial\Phi}{\partial v}\rp & = \bigg\{\frac{2}{9}\star(vvu) - \frac{1}{9}\lbp 2\tr\lp v^3w\rp + 2\tr v^2\tr\lp vw\rp\rbp\bigg\}(1+\Phi)\nn\\
%\tr\lp \frac{\partial\Phi}{\partial w}\frac{\partial\Phi}{\partial w}\rp & = \frac{1}{9}\lbp 2\tr v^4 - 2\lp\tr v^2\rp^2 \rbp(1+\Phi)\nn
%\end{align}
%
%
%Giving the linear equations
%\begin{align}
%\tr\lp v\frac{\partial\Phi}{\partial u}\rp\star(vvv) + 4\tr\lp v\frac{\partial\Phi}{\partial w}\rp\tr v^2 = 0
%\end{align}
%
%\begin{align}
%\tr\lp u\frac{\partial\Phi}{\partial u}\rp + 2\tr\lp v\frac{\partial\Phi}{\partial v}\rp + 2\tr\lp w\frac{\partial\Phi}{\partial w}\rp &= 0 
%\end{align}
%A term $\phi_i$ in $\Phi$ of order $n$ consist of $n_u$ $u$:s, $n_v$ $v$:s and $n_w$ $w$:s and $n = n_u + n_v + n_w$. Multiplying it's derivative gives back $\phi_i$ multiplied by the number of that variable in $\phi_i$, i.e. we have   
%\begin{align}
%0 &= \sum_i\lbp \tr\lp u\frac{\partial\phi_i}{\partial u}\rp + 2\tr\lp v\frac{\partial\phi_i}{\partial v}\rp + 2\tr\lp w\frac{\partial\phi_i}{\partial w}\rp\rbp\nn\\
%& = \sum_i\lbp n_u\phi_i + 2n_v\phi_i + 2n_w\phi_i \rbp\nn\\ 
%& = \sum_i\lbp n_u + 2n_v + 2n_w \rbp\phi_i 
%\end{align}
%This must be true to each order in the ansatz $\Phi$. 
%
%Why we must take trace:
%\begin{align}
%\star(vw)_m^\alpha v^m_{\alpha'} &= \epsilon_{mnp}\varepsilon^{\alpha\beta\gamma} v^n_\beta w^p_\gamma v^m_{\alpha'}\nn\\
%\star(vv)_m^\alpha w^m_{\alpha'} &= \epsilon_{mnp}\varepsilon^{\alpha\beta\gamma} v^n_\beta v^p_\gamma w^m_{\alpha'}\nn\\
%\star(vvw)\M_{mn} &= \epsilon_{m'n'p'}\M_{mn}\varepsilon^{\alpha\beta\gamma}v^{m'}_{\alpha} v^{n'}_{\beta} w^{p'}_{\gamma}\nn\\
%&= \lp \epsilon_{mn'p'}\M_{m'n} + \epsilon_{m'mp'}\M_{nn'} + \epsilon_{m'n'm}\M_{np'}\rp\varepsilon^{\alpha\beta\gamma}v^{m'}_{\alpha} v^{n'}_{\beta} w^{p'}_{\gamma}\nn\\
%&= 2\epsilon_{mn'p'}\varepsilon^{\alpha\beta\gamma}v_{\alpha n} v^{n'}_{\beta} w^{p'}_{\gamma} - \epsilon_{mn'p'}\varepsilon^{\alpha\beta\gamma}v^{p'}_{\alpha} v^{n'}_{\beta} w_{\gamma n}\nn\\
%&= 2*(vw)_m^\alpha v_{\alpha n} + *(vv)_m^\alpha w_{\alpha n}\nn\\
%\end{align}
%giving 
%\begin{align}
%\frac{\partial\Phi}{\partial u}u + 2\frac{\partial\Phi}{\partial v}v + a\frac{\partial\Phi}{\partial w}w 
%= \frac{2}{3}\Bigg[\star(vw)v - \star(vv)w \Bigg]\sqrt{1+\Phi}\ne 0\nn\\
%\end{align}
%
%\paragraph{Compare to the solved case}
%\begin{align}
%0 &= \sum_i\lbp n_u + n_v\rbp\phi_i 
%\end{align}
%
%\begin{align}
%\Phi = &+ \frac{1}{2}\tr v^2 - \frac{1}{2}\tr u^2 + \frac{1}{4}\lp\tr\lp uv\rp\rp^2 - \frac{1}{2}\tr\lp u^2v^2\rp - \frac{1}{2}V_3^2\nn\\ 
%\end{align}
%
%Order 2: Ok!
%
%Order 4: Ok!
%
%Order 6:
%\begin{align}
%\tr u^2 &= T_1^6 + 6T_1^4T_2 + 8T_1^2T_2^2 + 8T_1^3T_3 + 16T_1T_2T_3 + 9T_3^2\\  
%\lp\tr\lp uv\rp\rp^2 &= -(T_1^2 + 2T_2)(T_1^4 + 4T_1^2T_2 + 8T_1T_3)\\
%& = -T_1^6 - 6T_1^4T_2 - 8T_1^3T_3 - 8T_1^2T_2^2 - 16T_1T_2T_3\\
%\tr\lp u^2v^2\rp &= -T_1^6 - 6T_1^4T_2 - 8T_1^2T_2^2 - 8T_1^3T_3 - 16T_1T_2T_3 - 6T_3^2\\
%\end{align}
%Ok!
%
%\paragraph{The linear relation}-\\
%\begin{align}
%0 &= \sum_i\lbp n_u + 2n_v + 2n_w \rbp\phi_i 
%\end{align}
% 
% 
%\begin{align}
%u &= b_1v + b_2\star(vv) + b_3v\tr v^2 + b_4v^3 \nn\\
%w &= c_1v + c_2\star(vv) + c_3v\tr v^2 + c_4v^3 \nn\\
%\end{align}
%\paragraph{Order 2}
%\begin{align}
%\Phi = a_1\tr u^2 + a_2\lp\tr v^2+\tr w^2\rp
%\end{align}
%gives
%\begin{align}
%0 = 2a_1b_1^2 + 4a_2 + 4a_2c_1^2
%\end{align}
%Ok!
%\paragraph{Order 3}
%\begin{align}
%\Phi = a_1\tr u^2 + a_2\tr w^2 + a_3\star(uuu) + a_4\star\lp u\lp vv+ww\rp\rp
%\end{align}
%gives
%\begin{align}
%0 &= 4a_1b_1b_2 + 8a_2c_1c_2 + 3a_3b_1^3 + 5a_4b_1 + 5a_4b_1c_1^2\\
%\end{align}
%Ok!
%\paragraph{Order 4}
%\begin{align}
%\Phi &= a_1\tr u^2 + a_2\tr w^2 + a_3\star u^3 + a_4\star\lp u\lp v^2+w^2\rp\rp\nn\\
%& + a_5\lp\tr u^2\rp^2 + a_6\lp \lp\tr uv\rp^2 + \lp\tr uw\rp^2\rp + a_7\tr\lp v^2+w^2\rp\tr uu\nn\\
%& + a_8\lp\tr\lp v^2+w^2\rp\rp^2 + a_9\tr u^4 + a_{10}\tr\lp\lp v^2+w^2\rp u^2\rp + a_{11}\tr\lp v^2+w^2\rp^2
%\end{align}
%gives
%
%K;r allm'n expansion av u i programmeringen ist'llet...
%
%First order: $c_1 = 0$,  $a_1 = \frac{2}{3}\frac{1}{b_1}$ and $a_2 = -\frac{1}{3}b_1$, $b_1\ne 0$\\
%Second order: $c_2 = \frac{1}{2b_1}$, $a_3 = -\frac{1}{3}\frac{b_2}{b_1^3}$ and $a_4 = -\frac{1}{3}\frac{b_2}{b_1}$ ($b_2$ can be 0)
%
%Fler linj�ra ekvationer?
%Use the notation $\star(vuw) = \epsilon_{mnp}\varepsilon^{\alpha\beta\gamma}v^m_\alpha v^{n}_\beta v^{p}_\gamma$, where the order of the factors $u$, $v$ and $w$ doesn't matter (A hodge star up to the $\epsilon_{mnp}$ and possibly a factor).
%We get the following scalar equations (kanske ocks[ multiplicera med $*(v\we v)$ o.s.v.)
%\begin{align}
%\tr\lp\frac{\partial \Phi}{\partial u}u\rp &= \bigg[d_{11}\tr\lp uv\rp + d_{12}\tr\lp uw\rp\bigg]\sqrt{1+\Phi} = x_1\sqrt{1+\Phi}\nn\\
%\tr\lp\frac{\partial \Phi}{\partial u}v\rp &= \bigg[d_{11}\tr v^2 + d_{12}\tr\lp vw\rp\bigg]\sqrt{1+\Phi} = x_2\sqrt{1+\Phi}\nn\\
%\tr\lp\frac{\partial \Phi}{\partial u}w\rp &= \bigg[d_{11}\tr\lp vw\rp + d_{12}\tr w^2\bigg]\sqrt{1+\Phi} = x_3\sqrt{1+\Phi}\nn\\
%\tr\lp\frac{\partial \Phi}{\partial v}u\rp &= \bigg\{d_{21}\tr u^2 + \star\left[d_{22}uv^2 + d_{23}uvw + d_{24}uw^2\right]\bigg\}\sqrt{1+\Phi} = x_4\sqrt{1+\Phi}\nn\\
%\tr\lp\frac{\partial \Phi}{\partial v}v\rp &= \bigg\{d_{21}\tr\lp uv\rp + \star\left[d_{22}v^3 + d_{23}v^2w + d_{24}vw^2\right]\bigg\}\sqrt{1+\Phi} = x_5\sqrt{1+\Phi}\nn\\
%\tr\lp\frac{\partial \Phi}{\partial v}w\rp &= \bigg\{d_{21}\tr\lp uw\rp + \star\left[d_{22}v^2w + d_{23}vw^2 + d_{24}w^3\right]\bigg\}\sqrt{1+\Phi} = x_6\sqrt{1+\Phi}\nn\\
%\tr\lp\frac{\partial \Phi}{\partial w}u\rp &= \bigg\{d_{31}\tr u^2 + \star\left[d_{32}uv^2 + d_{33}uvw + d_{34}uw^2\right]\bigg\}\sqrt{1+\Phi} = x_7\sqrt{1+\Phi}\nn\\
%\tr\lp\frac{\partial \Phi}{\partial w}v\rp &= \bigg\{d_{31}\tr\lp uv\rp + \star\left[d_{32}v^3 + d_{33}v^2w + d_{34}vw^2\right]\bigg\}\sqrt{1+\Phi} = x_8\sqrt{1+\Phi}\nn\\
%\tr\lp\frac{\partial \Phi}{\partial w}w\rp &= \bigg\{d_{31}\tr\lp uw\rp + \star\left[d_{32}v^2w + d_{33}vw^2 + d_{34}w^3\right]\bigg\}\sqrt{1+\Phi} = x_9\sqrt{1+\Phi}
%\end{align}

\subsection{Result (2)}
\seclab{solution_result2}
So far we have found free parameters of different types in the duality equations coming from five sources: the closed forms $\Gamma$ and $\Delta$, from additional gauge invariant terms ($\alpha$, $\beta$ and $\gamma$), from a redefinition of the background potential affecting the gauge transformation and thus the world volume field strength, using $u\rightarrow u+c\star(\omega\omega)$ and last the relation between the pullbacked vielbein $\omega_\circ$, $v$ and $w$ as $\omega_\circ^m=-\qdp{r}\omega^{rm}$.  
Because of all this arbitrariness it only makes sense to study two cases, either the completely parameterless case or the case with parameters for all terms of the types mentioned above.
We begin with the parameterless one.

\subsubsection{The equations without parameters}
If we let all parameters become $0$ and $\omega_\circ=-v$, i.e. the $p=(1,0)$ case, we find the general duality equations \eqnref{solution_8d_duality_general} to become
\begin{align}
\eqnlab{csolution_general_duality}
\frac{\partial\Phi}{\partial u_m} &= \frac{4}{3}v^m\sqrt{1+\Phi}\nn\\
\frac{\partial\Phi}{\partial v^m} & = \bigg\{-\frac{2}{3}u_m + \frac{1}{3}\star(wv)\bigg\}\sqrt{1+\Phi}\nn\\
\frac{\partial\Phi}{\partial w^m} & = -\frac{1}{3}\star(vv)\sqrt{1+\Phi}.
\end{align}
We do not gain much by adding these equations to make them linear in $\Phi$ since the expansion parameters in $u$ and $v$ will come nonlinear anyway, so we simply try to solve the equations as they are, series expanding the $\sqrt{1+\Phi}$ factors.
If we let
\begin{align}
u &= b_1v + b_2\star(vv) + b_3V_2v + b_4v^3 + b_5\VS{3}v + b_6V_2\star(vv) + \cdots \mbox{ and}\nn\\
w &= c_1v + c_2\star(vv) + c_3V_2v + c_4v^3 + c_5\VS{3}v + c_6V_2\star(vv) + \cdots
\end{align}
be general functions of $v$, the first and last equation will give exactly one equation per expansion coefficient determining it in terms of the ansatz parameters. 
Thus, the middle equation is left to determine all of the ansatz parameters of $\Phi$, which are too few equations and thus the equations are still underdetermined if we do not chop $\Phi$ off at some order as before.  
We find that there does not exist any solutions with $\Phi$ of order $4$ or less to these equations with explicit relations $u(v)$ and $w(v)$.
%$\Phi$ of order 2: General contradiction\\
%$\Phi$ of order 3: General contradiction\\
%$\Phi$ of order 4: General contradiction (some paths not taken but highly unlikely that any solutions exists, if a coefficient consist of rows after rows with squareroots and imaginary parts, they cannot turn out to give something simple, right?)\\
Since the equations get quite complicated at order $4$ we do not expect the same procedure to work for higher orders. 
%{\huge Note that the true equations are (if any derivatives) the ones expanded to one order lower than the ansatz (Not true for the linear relation, which can be calculated one order higher, since all derivatives becomes multiplied)}

\paragraph{Relation to the $F=0$ equations}
Since we made a big effort finding the solution \eqnref{csolution_phi_alpha_zero} of the equations \eqnref{csolution_equations_8D_w0_alpha} with $\tilde\alpha_1=-2\tilde\alpha_2=4/3$, we wonder if these equations could be a special case of the more general ones studied in this section.

If we assume $w=0$, the third equation gives that $\star(vv)=0$ with side effects (from the reduction relations) $\VS{3}=0$, $v^3=V_2v$ and $V_4=0$.
Expanding the equations with these conditions on the expansion variables makes them easier to solve than in the previous case, e.g.
\begin{align}
\Phi = -\frac{2}{3}\tr u^2 + \frac{1}{3}\tr v^2 - \frac{2}{9}\lp\tr \lp uv\rp\rp^2
\end{align}
with
\begin{align}
u(v) = -\sqrt{-\det G}G^{-1}v
\end{align}
is one solution. Although it makes correspondence with the solution found before (the coefficients of the first two terms are locked and the third coefficient is chosen to remove other terms), there are many coefficients in trivial solutions left to determine to get \eqnref{csolution_phi_alpha_zero} and thus the $w=0$ condition puts a too restrictive condition on $v$ to be considered an interpretation of the found solution.

Another interpretation is that $w\ne 0$ and instead $\star(vw)=0$, meaning the solution \eqnref{csolution_phi_alpha_zero} should be modified by introducing $w$ symmetric to each $v$ to make $\Phi$ charge independent.
The equations should thus be expanded with conditions on the expansion variables coming from $\star(vw)=0$.
Since we do not know anything about $w$ we do not know what these conditions should look like (everything we can know about $w$ comes from the equations which depends on the conditions).
We have not tried such an approach\footnote{It is quite much work per guess (and we would probably have to do many guesses) on the appearance of $w$ to get all side effects when letting $\star(vw)=0$ but it would be really rewarding to find a relation reproducing \eqnref{csolution_phi_alpha_zero}, because that would mean no need for the introduction of parameters and a big leap toward solving the general case when $\star(vw)\ne 0$.} but instead considered the generalization directly, i.e. we let
\begin{align}
u_m^\alpha = -\sqrt{G}G^{\alpha\beta}v_{m\beta} + {u_1}_m^\alpha
\end{align} 
where $u_1=0$ when $\star(vw)=0$ and $w$ is a general expansion in $v$.
Since the first term is $-\frac{\delta\sqrt{G}}{\delta v}$, we find that the variation of an action
\begin{align}
S=-\int d^3\xi\sqrt{\det G} + \int v^m\we B_m + S_{corr}(v)
\end{align}
w.r.t. the scalar potential of $v$ gives the equation of motion
\begin{align}
\eqnlab{csolution_eom_1form}
0 = d[*u-*u_1] - dB_m + \frac{\delta S_{corr}}{\delta \phi^m}
\end{align}
out of which we expect to gain the information of the Bianchi identities \eqnref{solution_8d_bianchi} for $u$ and $w$ and nothing more, i.e.
\begin{align}
\eqnlab{csolution_final_bianchi}
d*u_m &= -df_m = H_m - \frac{1}{2}\epsilon_{mnp}\epsilon_{st}F^{sn}\we\omega^{tp}\nn\\
& = H_m + \frac{1}{2}\epsilon_{mnp}\lp w^{n}\we F_\parallel^{p} - v^{n}\we F_\perp^{p}\rp\mbox{ and}\nn\\
dw^m &= -F_\perp^{m}
\end{align} 
should follow from this equation of motion.
Assuming these relations the equations of motion must thus vanquish, so inserting them and using the definition of $H_m$ in \eqnref{csolution_eom_1form} gives 
\begin{align}
d[*u_1] &= \frac{1}{2}\epsilon_{mnp}\lp w^{n}\we F_\parallel^{p} - v^{n}\we F_\perp^{p} + A_\parallel^{n}\we F_\perp^{p} - A_\perp^{n}\we F_\parallel^{p}\rp + \frac{\delta S_{corr}}{\delta \phi^m}
\end{align}
The $A$ terms must come from $S_{corr}$ and we thus make a guess, using possible 3-forms giving the correct types of terms and assuming no relations $w(v)$ to appear in $S_{corr}$.
%\begin{align}
%S_{corr} = \epsilon_{mnp}\lp v^m\we A_!^n\we A_?^p + v^m\we v^n\we A_@^{p} + yv^m\we v^n\we v^{p}\rp  
%\end{align}
%
%\begin{align}
%\delta \epsilon_{mnp}v^m\we v^n\we v^p = 3\epsilon_{mnp}v^{n}\we v^{p}\we d\delta \phi^{m} = -3\epsilon_{mnp}d[v^{n}\we v^{p}]\delta \phi^{m} = -6\epsilon_{mnp}v^{n}\we F_\parallel^{p}\delta \phi^{m}\nn\\
%\end{align}
%
%\begin{align}
%\delta \epsilon_{mnp}v^m\we v^n\we A_?^p = 2\epsilon_{mnp}v^{n}\we A_?^{p}\we d\delta \phi^{m} = -2\epsilon_{mnp}d[v^{n}\we A_?^{p}]\delta \phi^{m}\nn\\
% = 2\epsilon_{mnp}\lp -v^{n}\we F_?^{p} - A_?^{n}\we F_\parallel^{p}\rp \delta\phi^{m}
%\end{align}
%
%\begin{align}
%\delta \epsilon_{mnp}v^m\we A_!^n\we A_?^p = \epsilon_{mnp}A_!^{n}\we A_?^{p}\we d\delta \phi^{m} = -\epsilon_{mnp}d[A_!^{n}\we A_?^{p}]\delta \phi^{m}\nn\\
% = \epsilon_{mnp}\lp -A_!^{n}\we F_?^{p} - A_?^{n}\we F_!^{p}\rp \delta\phi^{m}
%\end{align}
%giving
%\begin{align}
%d[*u_1] &= \frac{1}{2}\epsilon_{mnp}\lp w^{n}\we F_\parallel^{p} - A_\perp^{n}\we F_\parallel^{p} - v^{n}\we F_\perp^{p} + A_\parallel^{n}\we F_\perp^{p}\rp\nn\\
%& +\epsilon_{mnp}\lp A_!^{n}\we F_?^{p} - F_!^{n}\we A_?^{p}\rp + \epsilon_{mnp}\lp 2v^{n}\we F_?^{p} - 2F_\parallel^{n}\we A_?^{p}\rp - 6y\epsilon_{mnp}v^{n}\we F_\parallel^{p}\nn\\
%&= \frac{1}{2}\epsilon_{mnp}\lp w^{n}\we F_\parallel^{p} - v^{n}\we F_\perp^{p}  + 4v^{n}\we F_@^{p} - A_\perp^{n}\we F_\parallel^{p} + A_\parallel^{n}\we F_\perp^{p} + 2A_!^{n}\we F_?^{p} + 2A_?^{n}\we F_!^{p} + 4A_@^{n}\we F_\parallel^{p} - 6yv^{n}\we F_\parallel^{p}\rp\nn\\
%\end{align}
%Want
%\begin{align}
%0 &= - A_\perp^{n}\we F_\parallel^{p} + A_\parallel^{n}\we F_\perp^{p} + 2(!_1A_\parallel^{n} + !_2A_\perp^{n})\we (?_1F_\parallel^{p} + ?_2F_\perp^{p}) + 2(?_1A_\parallel^{n}+?_2A_\perp^{n})\we (!_1F_\parallel^{p}+!_2F_\perp^{p}) + 4(@_1A_\parallel^{n}+@_2A_\perp^{n})\we F_\parallel^{p}\nn\\
%&= \lp 1 + 2!_1?_2 + 2!_2?_1\rp A_\parallel^{n}\we F_\perp^{p} + \lp -1 + 2!_2?_1 + 2!_1?_2 + 4@_2\rp A_\perp^{n}\we F_\parallel^{p} + \lp 4!_1?_1 + 4@_1\rp A_\parallel^{n}\we F_\parallel^{p} + \lp 4!_2?_2\rp A_\perp^{n}\we F_\perp^{p})\nn\\
%\end{align}
%so (use $!_1?_1 = x_1$, $!_2?_1 + !_1?_2 = x_2$, $!_2?_2 = x_3$)
%\begin{align}
%x_3 &= 0\nn\\
%@_1 &= -x_1\nn\\
%@_2 &= 1/2\nn\\
%x_2 &= -1/2
%\end{align}
We find that the variation of
\begin{align}
S_{corr} =& \epsilon_{mnp}\Big( v^m\we (-x A_\parallel^n\we A_\parallel^p + \frac{1}{2}A_\parallel^n\we A_\perp^p )\nn\\
& + v^m\we v^n\we\lp xA_\parallel^{p}-\frac{1}{2}A_\perp^p\rp - yv^m\we v^n\we v^p\Big)
\end{align}
removes all $A$-terms, leaving
\begin{align}
d[*u_1] &= \frac{1}{2}\epsilon_{mnp}\lp w^{n}\we F_\parallel^{p} - v^{n}\we F_\perp^{p}  + 4v^{n}\we (\frac{1}{2}F_\perp^{p} -xF_\parallel^{p}) + 12yv^n\we F_\parallel^p \rp \nn\\
%&= \frac{1}{2}\epsilon_{mnp}\lp w^{n}\we F_\parallel^{p} + v^{n}\we F_\perp^{p} - (4x+6y)v^{n}\we F_\parallel^{p} ) \rp\nn\\
&= \frac{1}{2}\epsilon_{mnp}d[-w^{n}\we v^{p} + 2(3y-x)v^{n}\we v^{p}] 
\end{align}
and thus
\begin{align}
u_1 = \frac{1}{2}\star(vw) + (x-3y)\star(vv)
\end{align}
for some parameters $x$ and $y$ seems to be a good choice for $u_1$.
%\begin{align}
%d[w^n\we v^p] &= -w^n\we F_\parallel^p - v^n\we F_\perp^p\nn\\
%d[v^n\we v^p] &= -2v^n\we F_\parallel^p 
%\end{align}
As a consistency check we see that the variation of the derived action together with the derived $u$ gives the equations of motion
\begin{align}
df_m &= -H_m + \half\epsilon_{mnp}\lbp v^n\we dw^p + 2v^n\we F_\perp^p - w^n\we F_\parallel^p \rbp\\
&= -H_m + \half\epsilon_{mnp}\epsilon_{st}F^{sn}\we \omega^{tp} + \half\epsilon_{mnp}\lbp v^n\we dw^p + v^n\we F_\perp^p\rbp\nn.
\end{align}
Acting on this relation with an external derivative gives
\begin{align}
0 &= -\half\epsilon_{mnp}F_\parallel^n\we\lbp dw^p + F_\perp^p\rbp,
\end{align}
forcing the relations $dw^m=-F_\perp$ and $df_m = -H_m + \half\epsilon_{mnp}\epsilon_{st}F^{sn}\we \omega^{tp}$, which equal the Bianchi identities \eqnref{csolution_final_bianchi} we wanted to gain from the variation.

Another indication that the action should be on this form is that using $u_1$ on the form
\begin{align}
u_1 = b_1\star(vw) + b_2\star(vv)
\end{align}
and expanding the equations \eqnref{csolution_general_duality} for a general relation $w(v)$, we find that when $b_2\ne 0$ we must have $b_1=1/2$ for a general ansatz order and when $b_2=0$ we must have either $b_1=1/2$ or $b_1=7/8$. 
It would be a pretty big coincidence if the two derivations of $b_1=1/2$ were unconnected.

Because we are considering a case where $u_1=0$ when $\star(vw)=0$ we let 
\begin{align}
\eqnlab{csolution_duality_u_final}
u = -\sqrt{G}G^{\alpha\beta}v + \half\star(vw)
\end{align} 
and try to find a solution to the equations \eqnref{csolution_general_duality} with a general expansion $w(v)$.
We do not find any solution of ansatz order less than 8 and unfortunately the equations get too complicated for higher ansatz orders. 
What we find for general expansion order though, is that
\begin{align}
\eqnlab{csolution_phi_final}
\Phi &= -\frac{2}{3}\tr u^2 + \frac{1}{3}\tr v^2 + \mbox{terms of order $\ge 4$ in $u$ and $v$}\nn\\
w &= -\half\star(vv) + \Ordo(v^4)
\end{align}
and we find that the solution \eqnref{csolution_phi_alpha_zero} has to be modified for orders $\ge 4$ in $u$ and $v$.

We have also expanded the equations using
\begin{align}
u = -\sqrt{G}G^{\alpha\beta}v + \half\star(vw) + b\star(vv)
\end{align} 
to ansatz order 6 but we still fail to find a solution.

As a concluding remark we notice that $u(v) = -\sqrt{-\det G}G^{-1}v$ does not solve the general parameter free equations \eqnref{csolution_general_duality} to any order in $\Phi$ with general $w(v)$.
% $u=-u0$, general $w$: General contradiction at order 7 
The relation must thus be modified either by finding a condition for which \eqnref{csolution_phi_alpha_zero} is a special case, giving new possible terms when removing the condition on the expansion variables or by introducing terms dependent on parameters (the term is of course removed for $\beta_2=1/6$ and present for $\beta_2=0$) very much like $\alpha$ enters \eqnref{csolution_dualit_general_alpha}.   
If the last option is the case, we would expect the equations to be solvable for general parameters $\alpha$ and $\beta$. 



%\subsubsection{Going from the solved case to the general one}
%For the solved case, the solution was
%\begin{align}
%\Phi &= \frac{1}{3}\lbp \frac{4}{3}\lp\tr\lp uv\rp\rp^2 - 2\tr\lp u^2\lp \id+v^2\rp\rp + V_2 - V_4 + \frac{1}{12}\star(vvv)^2\rbp
%\end{align}
%together with
%\begin{align}
%u(v) = -\sqrt{\det G}G^{-1}v.
%\end{align}
%
%\paragraph{Introduce $w$}
%Introduction of $w$ here gives
%\begin{align}
%V_2 &\rightarrow V_2 + W_2\nn\\
%V_4 &\rightarrow V_4 + W_4 + \tr\lp v^2w^2\rp - V_2W_2 \nn\\
%V_6 &\rightarrow V_6 + W_6 + \tr\lp v^4w^2 \rp + \tr\lp v^2w^4 \rp - V_2W_4 - W_2V_4 - V_2\tr\lp v^2w^2\rp - W_2\tr\lp v^2w^2\rp
%\end{align}
%
%\begin{align}
%\Phi &= \frac{1}{3}\bigg\{ \frac{4}{3}\lp\tr\lp uv\rp\rp^2 - 2\tr\lp u^2\lp \id+v^2\rp\rp\nn\\
%&+ V_2 + W_2 - V_4 - W_4 - \tr\lp v^2w^2\rp + V_2W_2\nn\\
%&-3V_6 -3W_6 -3\tr\lp v^4w^2 \rp -3\tr\lp v^2w^4 \rp +3V_2W_4 +3W_2V_4 +3V_2\tr\lp v^2w^2\rp +3W_2\tr\lp v^2w^2\rp \bigg\}
%\end{align}
%
%\begin{align}
%\frac{\partial\Phi}{\partial u} &=\frac{8}{9}\tr\lp uv\rp v + \frac{8}{9}\tr\lp uw\rp w - \frac{4}{3} u\lp \id+v^2+w^2\rp\nn\\
%\frac{\partial\Phi}{\partial v} &=\frac{8}{9}\tr\lp uv\rp u - \frac{4}{3} u^2v\nn\\
%& + \frac{2}{3}\bigg( \id + V_2 + W_2 + 3V_4 + 3W_4 + 3\tr\lp v^2w^2\rp - 3W_2V_2\nn\\
%& + \lp -1 + 3V_2 + 3W_2 \rp v^2 + \lp -1 + 3V_2 + 3W_2\rp w^2  - 3v^4 - 3w^4 - 6v^2w^2 \bigg) v \nn\\
%\frac{\partial\Phi}{\partial w} &=\frac{8}{9}\tr\lp uw\rp u - \frac{4}{3} u^2w\nn\\
%& + \frac{2}{3}\bigg( \id + V_2 + W_2 + 3V_4 + 3W_4 + 3\tr\lp v^2w^2\rp - 3W_2V_2\nn\\
%& + \lp -1 + 3V_2 + 3W_2 \rp v^2 + \lp -1 + 3V_2 + 3W_2\rp w^2  - 3v^4 - 3w^4 - 6v^2w^2 \bigg) w \nn\\
%\end{align}
%



\subsubsection{The very general equations}
Now we turn our attention to the case were all terms coming from different types of parameters are accounted. We make an ansatz for the equations
\begin{align}
\eqnlab{csolution_general_duality}
\frac{\partial\Phi}{\partial u_m} &= \lp d_1v^m + d_2w^m\rp\sqrt{1+\Phi}\nn\\
\frac{\partial\Phi}{\partial v^m} & = \bigg\{d_3u_m + d_4\star(vv) + d_5\star(vw) + d_6\star(ww)\bigg\}\sqrt{1+\Phi}\nn\\
\frac{\partial\Phi}{\partial w^m} & = \bigg\{d_7u_m + d_8\star(vv) + d_9\star(vw) + d_{10}\star(ww)\bigg\}\sqrt{1+\Phi}
\end{align}
where $d_i$ are parameters.
As a first test we let $u$ be proportional to the conjugate 1-form as before and $w$ be proportional to $v$. 
Once again we create an ansatz of order $6$ and we now find that  
\begin{align}
u(v) &= \frac{2 d_3}{d_3^2 + d_7^2 - d_1d_3 - d_2d_7}\sqrt{G}G^{-1}v\nn\\
w(v) &= \frac{d_7}{d_3}v
\end{align}
together with the conditions $d_3, d_7$ and $d_3^2 + d_7^2 - d_1d_3 - d_2d_7$ nonzero and
\begin{align}
d_4 = -\frac{d_5d_7}{d_3} -\frac{d_6d_7^2}{d_3^2} + \frac{d_3d_8}{d_7} + d_9 + \frac{d_7d_{10}}{d_3} 
\end{align}
solves the duality equations with quite a complicated $\Phi$. 
For instance, looking at the case $w(v)=v$, i.e. when $d_7=d_3$, the solution is
\begin{align}
\Phi &= \frac{d^2}{2}\lp\tr\lp uv\rp\rp^2 + d(d_3-d)\tr\lp u^2\lp \id + \half v^2\rp\rp\nn\\ 
&  + \half\frac{d_3}{d_3-d}V_2 - \frac{1}{4}\frac{d_3+d_1}{d_3-d_1}V_4 -\frac{1}{288}\frac{1}{d_3-d}\lp d_3 - 4d_3\bar d^2 + 2d + 4d\bar d^2  \rp\VS{3}^2\nn\\
& + \frac{1}{12}\lp d + d_3\rp\bar d\star(vvv)\tr\lp uv\rp + \frac{1}{2}\lp d_3 - d\rp\bar d\star(uvv)\nn\\
& - \frac{1}{3}\lp d_3 - d\rp^3\bar d\star(uuu) + \mbox{ symmetry }v\leftrightarrow w
\end{align}
where $d=\half\lp d_1+d_2\rp$, $\bar d = d_8+d_9+d_{10} = d_4+d_5+d_6$ and all $w$ terms should be added symmetrical between $v$ and $w$, so $v_\alpha^mv_\beta^n$ ought to be replaced by $v_\alpha^mv_\beta^n+w_\alpha^mw_\beta^n$, e.g. $V_4 \rightarrow V_4+W_4+\tr\lp v^2w^2\rp -V_2W_2$. 
It looks like we are still on track, since when letting $\bar d=0$ we once again obtain the previous found solution \eqnref{csolution_phi_general_alpha} if putting $w=v$ (effectively meaning letting $v\rightarrow \sqrt{2}v$ and removing all $w$), $d_3=\half\tilde\alpha_2$ and $d=\half\tilde\alpha_1$ as we should, using the symmetry of the equations.
$F^{rm}$ must still be $0$ since acting with an external derivative on the relation between $\omega_\parallel$ and $\omega_\perp$ yields
\begin{align}
d\omega_\perp^m - cd\omega_\parallel^m = -\lp\pdo{r} - c\pdp{r}\rp F^r = 0
\end{align}
This solution thus solves more or less all equations coming from the different types of parameters. 
%In particular it solves the equations with parameters $\Gamma$, $\Delta$, $q_1$ and $q_2$, which are the parameters expected when $F=0$.  
%
%\begin{align}
%\frac{\partial\Phi}{\partial u_m} &= \lp \lp\frac{4}{3}q_1-\tilde\Gamma_1\rp v^m + \lp\frac{4}{3}q_2-\tilde\Gamma_2\rp w^m\rp\sqrt{1+\Phi}\nn\\
%\frac{\partial\Phi}{\partial v^m} & = \frac{1}{3}\bigg\{-2q_1 u_m + \epsilon_{mnp}\Big[ 3\tilde\Delta_1\tilde\Gamma_1^2\star(vv) + \lp q_1 + 6\tilde\Delta_1\tilde\Gamma_1\tilde\Gamma_2 + 1 - 3\frac{\tilde\Gamma_1}{2}\rp\star(vw) + \lp q_2+3\tilde\Delta_1\tilde\Gamma_2^2 - 3\frac{\tilde\Gamma_2}{2}\rp\star(ww)\Big] \bigg\}\sqrt{1+\Phi}\nn\\
%\frac{\partial\Phi}{\partial w^m} & = \frac{1}{3}\bigg\{-2q_2 u_m - \epsilon_{mnp}\Big[ \lp q_1+3\tilde\Delta_2\tilde\Gamma_1^2 + 3\frac{\tilde\Gamma_1}{2}-1\rp\star(vv) + \lp q_2+6\tilde\Delta_2\tilde\Gamma_1\tilde\Gamma_2 - 3\frac{\tilde\Gamma_2}{2}\rp\star(vw) + 3\tilde\Delta_2\tilde\Gamma_2^2\star(ww) \Big] \bigg\}\sqrt{1+\Phi}\nn\\
%\end{align}
%
%Identify
%\begin{align}
%3d_1 &= 4q_1-3\tilde\Gamma_1\nn\\
%3d_2 &= 4q_2-3\tilde\Gamma_2\nn\\
%3d_3 &= -2q_1\nn\\
%3d_4 &= 3\tilde\Delta_1\tilde\Gamma_1^2\nn\\
%3d_5 &= q_1 + 6\tilde\Delta_1\tilde\Gamma_1\tilde\Gamma_2 + 1 - 3\frac{\tilde\Gamma_1}{2}\nn\\
%3d_6 &= q_2+3\tilde\Delta_1\tilde\Gamma_2^2 - 3\frac{\tilde\Gamma_2}{2}\nn\\
%3d_7 &= -2q_2\nn\\
%3d_8 &= q_1+3\tilde\Delta_2\tilde\Gamma_1^2 + 3\frac{\tilde\Gamma_1}{2}-1\nn\\
%3d_9 &= q_2+6\tilde\Delta_2\tilde\Gamma_1\tilde\Gamma_2 - 3\frac{\tilde\Gamma_2}{2}\nn\\
%3d_{10} &= 3\tilde\Delta_2\tilde\Gamma_2^2\nn\\
%\end{align}
%
%\begin{align}
%0 &= - q_1^2q_2^2d_4 - q_1q_2^3d_5 - q_2^4d_6 + q_1^3q_2d_8 + q_1^2q_2^2d_9 + q_1q_2^3d_{10}\nn\\ 
%&= - q_1^2q_2^2\tilde\Delta_1\tilde\Gamma_1^2 - q_1q_2^3d_5 - q_2^4d_6 + q_1^3q_2d_8 + q_1^2q_2^2d_9 + q_1q_2^3d_{10}\nn\\
%\end{align}
We would also like to try to solve these equations for some different relation $w(v)$ not imposing $F=0$. 
Letting $w(v)$ proportional to e.g. $\star(vv)$ or $v^3$ does not give any solution of ansatz order 6 and more complicated relations $w(v)$ tends to imply too complicated equations to solve using our method.

\section{Computer solution of the $d=9$ $D1$, $\alpha=\frac{1}{2}$ case}
\seclab{csolution_9d_const}
The equations found for the $D_1$-brane in $9$ dimensions
\begin{align}
\eqnlab{csolution_9d_general_duality}
\frac{\partial \Phi}{\partial u} &= 2\alpha_1 v + 2\lp\alpha_2 -1\rp w \sqrt{1+\Phi}\nn\\
\frac{\partial \Phi}{\partial v} &= - 2 \alpha_1 u\sqrt{1+\Phi}\nn\\
\frac{\partial \Phi}{\partial w} &= - 2 \alpha_2 u\sqrt{1+\Phi}
\end{align}
looks very similar to the $D_2$-brane in $8$ dimensions without the nonlinear terms, meaning they should be easier to solve.  
We have not tried to solve these equations, but we expect that creating an ansatz $\Phi$ using the polynomial variables
\begin{align}
\phi=\{& u^2, u\cdot v, v^2\}
\end{align}
with $v$ entering quadratically and symmetry $v\leftrightarrow w$ as usual, would make it if we found the correct duality relations between $u$, $v$ and $w$.  








%%\chapter{Comparison to the automorphic membrane}
\chlab{compare}


Antagligen
\begin{align}
u = u0 + b\star(vw)
\end{align}
tillsammans med termer
\begin{align}
c_1\star(vvF) + c_2\star(vw(v)F) + c_3\star(w(v)w(v)F)
\end{align}
fr[n andra sidan.


Consider the 1-form in the direction $\qdp{r} = q_1\pdp{r} + q_2\pdo{r}$, where $q_1^2 + q_2^2 = 1$.
Let $\tilde\phi^m = |q|\qdp{r}\tilde\phi^{rm}$ be the scalar potential to the 1-form $\tilde\omega^m=|q|\qdp{r}\omega^{rm} = |q|\lp q_1v^m + q_2w^m\rp$, i.e. $\tilde\phi$ corresponds to the 3 scalar degrees of freedom of the theory.
The metric becomes
\begin{align}
G_{\alpha\beta}=g_{\alpha\beta} + \tilde\omega^m_\alpha\tilde\omega_{m\beta} 
\end{align}
giving the conjugate variable 
\begin{align}
u_0 = \frac{\partial \mathcal{L}}{\partial \tilde\omega} = -\sqrt{-\det G}(G^{-1}\tilde\omega)
\end{align}
If we use 
\begin{align}
G_{\alpha\beta} = \id + |q|^2\lp q_1^2\lp v^2\rp_{\alpha\beta} + q_1q_2\lp vw\rp_{\alpha\beta} + q_2^2\lp w^2\rp_{\alpha\beta}\rp
\end{align}


The Bianchi identity for $f_m$ gives the equation of motion for $\tilde\phi^m$ 
\begin{align}
df_m & = - d\big[*u_m\big]\nn\\  
& = -H_m +\half\epsilon_{mnp}\epsilon_{st}F^{sn}\we\omega^{tp}
\end{align}

We also have $S_{(p,q)}$ such that the variation of $S_{(p,q)}$ with respect to $\tilde\phi^{m}$ gives its equations of motion.
We make the following attempt for $S_{(p,q)}$ and show that its variation indeed gives the equations of motion for $\tilde\phi$. 
\begin{align}
S_{(p,q)}^{(0)} = -\int d^3\xi|p|\sqrt{-\det G}
\end{align}
gives the equation of motion
\begin{align}
d\left[-*u_0\right] = 0
\end{align}
If we let $u = u_0 + u_1 + b_3*(vv) + b_4*(vw) + b_5*(ww)$ we get the equation of motion 
\begin{align}
d\left[-*u_0\right] &= d\left[*(-u + u_1 + b_3*(\tilde\omega\tilde\omega) + b_4*(\tilde\omega w(\tilde\omega)) + b_5*(w(\tilde\omega)w(\tilde\omega)))\right]\nn\\
&= d\left[f + *u_1 - b_3v \we v - b_4v \we w - b_5w \we w\right] = 0
\end{align}
giving
\begin{align}
df_m &=  - d[*u_{1m}] + \epsilon_{mnp}\lbp  2b_3v^n \we dv^p + b_4v^n \we dw^p  + b_4w^n \we dv^p + 2b_5w^n \we dw^p \rbp\nn\\
&= - d[*u_{1m}] + \epsilon_{mnp}\lbp -2b_3\omega_\parallel^n \we F_\parallel^p - b_4\omega_\parallel^n \we F_\perp^p  - b_4\omega_\perp^n \we F_\parallel^p - 2b_5\omega_\perp^n \we F_\perp^p \rbp
\end{align}


\begin{align}
\epsilon_{mnp}\epsilon_{st}\omega^{sn}\we F^{tp} &= \pup{r}\pdp{r}\epsilon_{mnp}\epsilon_{st}\omega^{sn}\we F^{tp}\nn\\
& = \epsilon_{mnp}( -\omega_\perp^{n}\we F_\parallel^{p} + \omega_\parallel^{n}\we F_\perp^{p})\nn\\
\end{align}

\begin{align}
\epsilon_{mnp}v^n \we dv^p &= -\epsilon_{mnp}\qdp{s}\qdp{t}\omega^{sn}\we F^{tp}\nn\\
&= -\epsilon_{mnp}(q_1^2\omega_\parallel^{n}\we F_\parallel^{p}+q_1q_2\omega_\parallel^{n}\we F_\perp^{p}+q_1q_2\omega_\perp^{n}\we F_\parallel^{p}+q_2^2\omega_\perp^{n}\we F_\perp^{p})
\end{align}
where $q_1^2+q_2^2 = 1$ and not enough to fulfill the Bianchi identities.
Let $w = -q_2\omega_\parallel + q_1\omega_\perp$, which gives
\begin{align}
\epsilon_{mnp}w^n \we dw^p &= -\epsilon_{mnp}(q_2^2\omega_\parallel^{n}\we F_\parallel^{p} - q_1q_2\omega_\parallel^{n}\we F_\perp^{p} - q_1q_2\omega_\perp^{n}\we F_\parallel^{p}+q_1^2\omega_\perp^{n}\we F_\perp^{p})
\end{align}
and
\begin{align}
\epsilon_{mnp}v^n \we dw^p &= -\epsilon_{mnp}(-q_1q_2\omega_\parallel^{n}\we F_\parallel^{p} + q_1^2\omega_\parallel^{n}\we F_\perp^{p} - q_2^2\omega_\perp^{n}\we F_\parallel^{p}+q_1q_2\omega_\perp^{n}\we F_\perp^{p})\nn\\
\epsilon_{mnp}w^n \we dv^p &= -\epsilon_{mnp}(-q_1q_2\omega_\parallel^{n}\we F_\parallel^{p} - q_2^2\omega_\parallel^{n}\we F_\perp^{p} + q_1^2\omega_\perp^{n}\we F_\parallel^{p}+q_1q_2\omega_\perp^{n}\we F_\perp^{p})\nn\\
\end{align}
giving
\begin{align}
\epsilon_{mnp}d[v^n \we w^p] &= \epsilon_{mnp}\lp v^n \we dw^p + w^n\we dv^p\rp\nn\\
&= -\epsilon_{mnp}(-2q_1q_2\omega_\parallel^{n}\we F_\parallel^{p} + \lp q_1^2-q_2^2\rp\lp\omega_\parallel^{n}\we F_\perp^{p} + \omega_\perp^{n}\we F_\parallel^{p}\rp+2q_1q_2\omega_\perp^{n}\we F_\perp^{p})
\end{align}
and
\begin{align}
\epsilon_{mnp}\lp v^n \we dw^p - w^n \we dv^p\rp = \epsilon_{mnp}\lbp -\omega_\parallel^{n}\we F_\perp^{p} + \omega_\perp^{n}\we F_\parallel^{p}\rbp = -\epsilon_{mnp}\epsilon_{st}\omega^{sn}\we F^{tp}
\end{align}

If $q_1=1, q_2=0$
\begin{align}
\epsilon_{mnp}v^n \we dv^p &= -\epsilon_{mnp}(\omega_\parallel^{n}\we F_\parallel^{p})\nn\\
\epsilon_{mnp}w^n \we dw^p &= -\epsilon_{mnp}(\omega_\perp^{n}\we F_\perp^{p})\nn\\
\epsilon_{mnp}d[v^n \we w^p] &= -\epsilon_{mnp}(\omega_\parallel^{n}\we F_\perp^{p} + \omega_\perp^{n}\we F_\parallel^{p})
\end{align}


Other possible terms in $S_{(p,q)}$
\begin{align}
S_{(p,q)}^{(1)} = \int\lp B_m\we \tilde\omega^m \rp
\end{align}
\begin{align}
\delta_{\tilde\phi^m} S_{(p,q)}^{(1)} &= \int\lp - dB_m \rp\delta\tilde\phi^m \nn\\
&= \int\lp - H_m + \half\epsilon_{mnp}\epsilon_{st}F^{sn}\we A^{tp} \rp\delta\tilde\phi^m 
\end{align}

\begin{align}
S_{(p,q)}^{(2)} = \int\frac{1}{|p|^2}\lp \epsilon_{mnp}\tilde\omega^m\we\tilde\omega^n\we\tilde\omega^p \rp
\end{align}
\begin{align}
\delta_{\tilde\phi^m} S_{(p,q)}^{(2)} &= \int\lp \frac{3}{|p|^2}\epsilon_{mnp}\tilde\omega^m\we\tilde\omega^n\we d\delta\tilde\phi^m \rp\nn\\
& = \int\lp -\frac{3}{|p|^2}\epsilon_{mnp}d\left[\tilde\omega^n\we\tilde\omega^p\right] \rp\delta\tilde\phi^m\nn\\
& = \int\lp \frac{6}{|p|^2}\epsilon_{mnp}\tilde\omega^n\we \tilde F^p \rp\delta\tilde\phi^m
\end{align}

\begin{align}
S_{(p,q)}^{(3)} = \int\lp \epsilon_{mnp}\W_{st}A^{sn}\we A^{tp}\we \tilde\omega^m \rp
\end{align}
\begin{align}
\delta_{\tilde\phi^m} S_{(p,q)}^{(3)} &= \int\lp \epsilon_{mnp}\W_{st}A^{sn}\we A^{tp}\we d\delta\tilde\phi^m \rp\nn\\
& = \int\lp -\epsilon_{mnp}d\left[\W_{st}A^{sn}\we A^{tp}\right] \rp\delta\tilde\phi^m\nn\\
& = \int\lp -2\epsilon_{mnp}\W_{st}A^{sn}\we F^{tp} \rp\delta\tilde\phi^m
\end{align}




N�t konstigt jag hittade i kapitel 5...
\begin{align}
S_1 &= \frac{\sqrt{\det\left[ZZ^T+\lp y^2+x_0^2\rp\id\right]}}{y^2+x_0^2}\nn\\
S_2 &= -\frac{x_0\det Z}{y\lp y^2+x_0^2\rp}
\end{align}
Thus
\begin{align}
S_1 + S_2 &= \frac{\sqrt{\det\left[ZZ^T+\lp y^2+x_0^2\rp\id\right]}}{y^2+x_0^2} -\frac{x_0\det Z}{y\lp y^2+x_0^2\rp}
\end{align}



%\chapter{Conclusions}
In this thesis we have made an attempt to derive the empiric covariant brane action \eqnref{dynamics_final_action} for some various cases in different dimensions. 
In particular we have tried to connect the 8-dimensional $D_2$ membrane action with manifest U-duality invariance to the action found using automorphic functions in \cite{pioline}. For this end we have followed the work done in \cite{artikeln} tightly.
Our main results are that the equations \eqnref{csolution_equations_8D_w0_alpha}, derived using the background field constraint $F=0$ considered in \cite{artikeln}, can be solved for general coefficients \eqnref{csolution_dualit_general_alpha} and \eqnref{csolution_phi_general_alpha} and that there might exist a consistent solution (of order higher than 7) which solves the equations \eqnref{csolution_general_duality} of a general background using \eqnref{csolution_duality_u_final} and \eqnref{csolution_phi_final} without the introduction of any additional parameters (the equations are derived and solved using $p=(1,0)$, so if we found relations and solution to these equations, we would expect them to be generalizable to general values on $p$).  

The implications of the first result would be that if we let $\alpha$ be a parameter of the theory, the simple identification $p^2 = y^2+x_0^2$ between our charges and the "extra variables" found in \cite{pioline} would be erroneous.  
Since this result comes from a solution of equations that are not representing the general case, we cannot draw too big conclusions though. It could of course also be that the values of all parameters will be locked when solving the equations for a general background, using the correct relations between $u$, $v$ and $w$ with the possibility of general $\alpha$ in the $F=0$ case.
Also, using $\kappa$-symmetry might lock the values of any parameters once and for all, but that is not obvious unless done.
Anyway, it is obvious that the significance of all free parameters is still unclear.
According to \eqnref{solution_8d_eom_a} and \eqnref{solution_8d_eom_phi} the inclusion of the gauge invariant field strength $\bar h$ (which consists of an odd number of $\omega$) in \eqnref{solution_8d_field_strengths} will effectively add odd terms in $v$ to \eqnref{solution_8d_eom_a}, even terms in $v$ to \eqnref{solution_8d_eom_phi} and terms of mixed order in $v$ to both equations, coming from the relations $u(v)$ and $w(v)$.   
We could try to solve these equations, using an expansion $\bar h$, but to be honest, that much arbitrariness is more than we can bear.

All solutions found have been of order 6 or lower, which should be interpreted as a result of shortage in our equation solving stamina, rather than a indication that the {\it{real}} solution is of this order.   
Knowing what the relations between $u$, $v$ and $w$ look like, we would expect much higher ansatz orders to be solvable.
Although we have derived the duality equations \eqnref{csolution_9d_general_duality} to solve for the $d=9$ $D_1$ case, we have mainly considered the $d=8$ membrane case.
Solving the equations of the $D_1$ case, which have some similarities to the $d=8$ membrane equations, could be one way to work around the problem and get a hint of what the relations between $u$, $v$ and $w$ should look like also in the membrane case.
Since we are using less polynomial variables for the ansatz in the $D_1$ case, we expect these equations to be somewhat simpler to solve (at least if the order of the ansatz is not greater than in the previous case).

One weakness of our approach in solving the equations is that we have assumed $u$ and $w$ to be explicit functions of $v$, with no more motivation than that $v$ is the most occurrent of the variables on the right hand side for the parameter free equations and thus gives the easiest equations to solve a priori.   
Because of the symmetry between $v$ and $w$ a better assumption might have been to use relations $v(u)$ and $w(u)$.  
Another improvement could be to include more types of polynomial variables in the ansatz but a bigger search space usually increases the complexity of the problem at hand.
We can of course neither give a guarantee that the program we have written to expand the duality equations are completely bug free. Even though it produces expansions which have been consistent in all checks made, there always exists more or less harmful bugs in computer programs (of course trying to expand all the equations by hand would produce way more errors and could probably neither be done in a lifetime).







%%%%%%%%%%%%%%%%%%%%%%%%%%%%%%%%%%%%%%%%%%%%
%
% Appendix
%
%%%%%%%%%%%%%%%%%%%%%%%%%%%%%%%%%%%%%%%%%%%%

%\appendix
%\pagestyle{plain}
%\chapter{Conventions and some basic formulae}
\chlab{conven}
Sometimes, we suspect that the authors of physical articles ignore to state their conventions on purpose, so that any weird minus signs or misprints in the text will be virtually impossible to check. 
Here we are brave enough to introduce the conventions we have used in this thesis. So if you find something weird in this thesis, it is a result of our own failure and not of some brilliant convention the world has never seen before.  
We also introduce some "good-to-know" formulas.

\section{Index conventions}

\begin{table}[h]
\begin{center}
\begin{tabular}{l l l l}
Space & Index & Signature & Range\\
11-dim curved & M,N,P,... &$\lp-+++\cdots+\rp$ & 1 ... $\hat D$\\
11-dim flat & A,B,C,...&$\lp-+++\cdots+\rp$ & 1 ... $\hat D$\\
Compactified curved & m,n,p,...&$\lp++++\cdots+\rp$ & 1 ... $D$\\
Compactified flat & a,b,c,...&$\lp++++\cdots+\rp$ & 1 ... $D$\\
Uncompactified curved & $\mu,\nu,\rho$,...&$\lp-+++\cdots+\rp$ & 1 ... $d$\\
Uncompactified flat & i,j,k,...&$\lp-+++\cdots+\rp$ & 1 ... $d$\\
World volume& $\alpha,\beta,\gamma$,... &$\lp-+++\cdots+\rp$ & 1 ... $p$\\
\end{tabular}
\caption{Index conventions and signature for different spaces}
\end{center}
\end{table}
Uncompactified tensors are denoted with a hat, like $\hat g_{MN}, \hat C_{PMN}$ etc.
When compactifying, we denote the dimension of the starting theory with $\hat D$, the number of compactified dimensions $D$ and the number of uncompactified dimensions $d$.\\

Throughout this thesis we set the Newton constant in 11 dimensions to $\kappa^2_{11}=\frac{1}{2}$.
\paragraph{Symmetrisation and antisymmetrisation} are denoted by
\begin{align}
A_{(M_1\cdots M_p)} &= \frac{1}{p!}\lp A_{M_1\cdots M_p} + \mbox{symmetric permutations} \rp,\nn\\
A_{[M_1\cdots M_p]} &= \frac{1}{p!}\lp A_{M_1\cdots M_p} + \mbox{antisymmetric permutations} \rp.
\end{align}
  

\section{Antisymmetric tensor and p-forms}
Define the Levi-Civita symbol with lower indices as $\epsilon_{A_1A_2\cdots A_n}$, antisymmetric in all indices and $\epsilon_{12\cdots n}=+1$.
Define further the Levi-Civita symbol with upper indices as $\epsilon^{A_1A_2\cdots A_n} = \lp-1\rp^s\epsilon_{A_1A_2\cdots A_n}$, where s is the number of timelike coordinates in the metric. 
In flat spacetime this symbol acts like a tensor.
In curved spacetime the Levi-Civita symbol transforms as
\begin{equation}
\epsilon^{M_1M_2\cdots M_n} \rightarrow \epsilon^{M'_1M'_2\cdots M'_n} 
\eqnlab{conven_epsilon_transform}
\end{equation}
under a general coordinate transformation $x\rightarrow x'$. 
Using that the determinant of some matrix A obeys the relation
\begin{equation}
\epsilon^{M_1M_2\cdots M_n}|A| = \epsilon^{N_1N_2\cdots N_n}A{_{N_1}}^{M_1}A{_{N_2}}^{M_2}\cdots A{_{N_n}}^{M_n},
\end{equation}
we find that (using A=$\partial x'^{M'_i}/\partial x^{M_j}$)
\begin{equation}
\epsilon^{M'_1M'_2\cdots M'_n} = \left|\frac{\partial x'}{\partial x}\right|^{-1}\frac{\partial x'^{M'_1}}{\partial x^{M_1}}\frac{\partial x'^{M'_2}}{\partial x^{M_2}}\cdots\frac{\partial x'^{M'_n}}{\partial x^{M_n}}\epsilon^{M_1M_2\cdots M_n}.\\
\eqnlab{conven_epsilon_prim}
\end{equation}
Compare this to the square root of the determinant of the metric $\sqrt{|g|}$ that transforms as (consider $\partial x^{N_i}/\partial x'^{M_j}$ as matrices) 
\begin{equation}
\sqrt{|g|}\rightarrow \sqrt{|g|'}=\sqrt{\left|\det\lp\frac{\partial x^{P}}{\partial x'^{M}}g_{PQ}\frac{\partial x^{Q}}{\partial x'^{N}}\rp\right|}=\left|\left|\frac{\partial x}{\partial x'}\right|\right|\sqrt{|g|}.
\eqnlab{conven_meric_transform}
\end{equation}
Combining equations \eqnref{conven_epsilon_prim} and \eqnref{conven_meric_transform} gives
\begin{equation}
\frac{1}{\sqrt{|g|}}\epsilon^{M_1\cdots M_n}\rightarrow \frac{1}{\sqrt{|g|'}}\epsilon'^{M_1\cdots M_n} = \frac{1}{\sqrt{|g|}}\frac{\partial x'^{M'_1}}{\partial x^{M_1}}\frac{\partial x'^{M'_2}}{\partial x^{M_2}}\cdots\frac{\partial x'^{M'_n}}{\partial x^{M_n}}\epsilon^{M_1M_2\cdots M_n}, 
\end{equation} 
which transforms as a four tensor. Thus one can form an antisymmetric covariant tensor in curved space $\varepsilon^{M_1M_2\cdots M_n}$ as 
\begin{equation}
\varepsilon^{M_1M_2\cdots M_n}=\frac{1}{\sqrt{|g|}}\epsilon^{M_1M_2\cdots M_n}.
\end{equation}
Define the antisymmetric tensor with lower indices as
\begin{equation}
\varepsilon_{M_1M_2\cdots M_n} = g_{M_1N_1}g_{M_2N_2}\cdots g_{M_nN_n} \varepsilon^{N_1N_2\cdots N_n} = \sqrt{|g|}\epsilon_{M_1M_2\cdots M_n},
\end{equation}
which transforms as a covariant tensor. 
\\
\paragraph{Contraction of two antisymmetric tensors}
\begin{equation}
\varepsilon^{M_1\cdots M_pN_{p+1}\cdots N_n}\varepsilon_{M_1\cdots M_pP_{p+1}\cdots P_n} = \lp-1\rp^s p!\lp n-p\rp!\delta_{[P_{p+1}\cdots P_n]}^{N_{p+1}\cdots N_n},
\end{equation}
where $\delta_{[P_{p+1}\cdots P_n]}^{N_{p+1}\cdots N_n}=\delta_{[P_{p+1}}^{N_{p+1}}\cdots \delta_{P_{n}]}^{N_{n}}$ is the generalized Kronecker delta with the nice property $\delta_{[M_{1}\cdots M_p]}^{N_{1}\cdots N_p}A^{M_{p}\cdots M_1} = A^{N_{p}\cdots N_1}$ for a $p$-form $A^{(p)}$.
\paragraph{In two dimensions}
By using contraction of antisymmetric tensors one can show the useful identities
\begin{equation}
\varepsilon_{MN}g^{NP}\varepsilon_{PQ} = (-1)^{s+1}g_{MQ}
\end{equation}
and
\begin{equation}
\varepsilon_{MN}g_{PQ} = \varepsilon_{MQ}g_{PN}+\varepsilon_{QN}g_{PM} = \varepsilon_{MP}g_{NQ}+\varepsilon_{PN}g_{MQ}.
\eqnlab{conven_2d_epsilon_rel}
\end{equation}

\paragraph{In three dimensions}
We get the corresponding relations to \eqnref{conven_2d_epsilon_rel}
\begin{align}
\varepsilon_{MNP}g_{S[Q}g_{R]T} &= \varepsilon_{STP} g_{M[Q}g_{R]N} - \varepsilon_{SNT}g_{P[Q}g_{R]M} + \varepsilon_{MST}g_{N[Q}g_{R]P}\nn\\
\varepsilon_{MNP}g_{QR} &= \varepsilon_{RNP}g_{MQ} + \varepsilon_{MRP}g_{QN} + \varepsilon_{MNR}g_{QP} 
\eqnlab{conven_3d_epsilon_rel}
\end{align}

% Derivation
%Try to find an equivalent to the 2-dimensional relations
%\begin{equation}
%\varepsilon_{MN}g^{NP}\varepsilon_{PQ} = (-1)^{s+1}g_{MQ}
%\end{equation}
%\begin{equation}
%\varepsilon_{MN}g_{PQ} = \varepsilon_{MQ}g_{PN}+\varepsilon_{QN}g_{PM} = \varepsilon_{MP}g_{NQ}+\varepsilon_{PN}g_{MQ}
%\end{equation}
%derived from
%\begin{align}
%\varepsilon_{MN}\varepsilon_{PQ} = (-1)^{s}2g_{M[P}g_{Q]N}\Rightarrow\nn\\
%\varepsilon_{RS}g^{SM}\varepsilon_{MN}\varepsilon_{PQ} = (-1)^{s+1}g_{RN}\varepsilon_{PQ} = (-1)^{s}2\varepsilon_{R[P}g_{Q]N}  
%\end{align}
%We have
%\begin{align}
%\varepsilon^{MNP}\varepsilon_{MNP} &= \lp-1\rp^s 6\nn\\
%\varepsilon^{MNP}\varepsilon_{MNP'} &= \lp-1\rp^s 2\delta_{P}^{P'}\nn\\
%\varepsilon^{MNP}\varepsilon_{MN'P'} &= \lp-1\rp^s 2\delta_{[NP]}^{N'P'}\nn\\
%\varepsilon^{MNP}\varepsilon_{M'N'P'} &= \lp-1\rp^s 6\delta_{[MNP]}^{M'N'P'}
%\end{align} 
%\begin{align}
%\varepsilon_{MNP}\varepsilon_{QRS} = \varepsilon_{MNP}\varepsilon^{PR'S'}g_{RR'}g_{SS'} = \lp-1\rp^s 6g_{Q[M}g_{N|R|}g_{P]S}\nn\\ 
%\varepsilon_{MNP}g^{PQ}\varepsilon_{QRS} = \varepsilon_{MNP}\varepsilon^{PR'S'}g_{RR'}g_{SS'} = \lp-1\rp^s 2g_{R[M}g_{N]S}\nn\\ 
%\varepsilon_{MNP}g^{NQ}g^{PR}\varepsilon_{QRS} = \varepsilon_{MNP}\varepsilon^{NPS'}g_{SS'} = \lp-1\rp^s 2g_{MS} 
%\end{align}
%\begin{align}
%\varepsilon_{MNP}g^{PQ}\varepsilon_{QRS}\varepsilon_{TUV} &= \lp-1\rp^s 6g^{PQ}\varepsilon_{QRS}g_{T[M}g_{N|U|}g_{P]V}\nn\\
%\varepsilon_{MNP}g^{NQ}g^{PR}\varepsilon_{QRS}\varepsilon_{TUV} &= \lp-1\rp^s 6g^{NQ}g^{PR}\varepsilon_{QRS}g_{T[M}g_{N|U|}g_{P]V}\nn\\
%&\Leftrightarrow\nn\\
%g_{R[M}g_{N]S}\varepsilon_{TUV} &= 3g^{PQ}\varepsilon_{QRS}g_{T[M}g_{N|U|}g_{P]V}\nn\\
%g_{MS}\varepsilon_{TUV} &= 3g^{NQ}g^{PR}\varepsilon_{QRS}g_{T[M}g_{N|U|}g_{P]V}\nn\\
%&\Leftrightarrow\nn\\
%g_{R[M}g_{N]S}\varepsilon_{TUV} &= \varepsilon_{VRS} g_{T[M}g_{N]U} + \varepsilon_{URS}g_{V[M}g_{N]T} + \varepsilon_{TRS}g_{U[M}g_{N]V}\nn\\
%g_{MS}\varepsilon_{TUV} &= -\half \varepsilon_{QRS}\lp \delta^Q_V\delta^R_U g_{TM} + \delta^Q_U\delta^R_T g_{MV} + \delta^Q_T\delta^R_V g_{MU} - \delta^Q_V\delta^R_T g_{MU} - \delta^Q_U\delta^R_V g_{TM} - \delta^Q_T\delta^R_U g_{MV}\rp\nn\\
%&= \varepsilon_{SUV}g_{TM} + \varepsilon_{TUS}g_{MV} + \varepsilon_{TSV}g_{MU}\nn\\
%\end{align} 


\paragraph{Forms}
We use the superspace convention of differential forms:
\begin{align}
A_{(p)} &= \frac{1}{p!}dx^{M_1}\wedge dx^{M_2}\cdots \wedge
dx^{M_{p-1}}\wedge dx^{M_p}A_{M_pM_{p-1}\cdots M_2M_1}\nn\\
& = \frac{1}{p!}e^{A_1}\wedge e^{A_2}\cdots \wedge e^{A_{p-1}}\wedge e^{A_p}A_{A_pA_{p-1}\cdots A_2A_1}
\end{align}
where $e^A = dx^Me{_M}^A$ and $A_{M_pM_{p-1}\cdots M_2M_1}$ is antisymmetric in all indices and the wedge products have the characteristic property $dx^M\we dx^N=-dx^N\we dx^M$. 
Whenever we use a form without form-index, its type is given according to table \tabref{conven_forms}. 
\begin{table}[h]
\tablab{conven_forms}
\begin{center}
\begin{tabular}{l|l l}
Type & p & notation\\\hline
Background gauge potentials & 1, 2, 3,... & A, B, C, D,...\\
Background field strengths & 2, 3, 4,... & F, H, G, I,...\\
World volume gauge potentials & 0, 1, 2, 3,... & $\phi$, a, b, c, d,...\\
World volume field strengths & 1, 2, 3, 4,... & $\omega$, f, h, g, i,...\\
\end{tabular}
\caption{P-form conventions.}
\end{center}
\end{table}
%
\paragraph{Exterior derivative} $d=dx^M\partial_M$ acting from right:
\begin{equation}
dA_{(p)} = \partial_{M_{p+1}} A_{(p)}\wedge dx^{M_{p+1}} = 
\frac{1}{p!}dx^{M_1}\wedge\cdots\wedge dx^{M_p}\wedge
dx^{M_{p+1}}\partial_{M_{p+1}} A_{M_p\cdots M_1}
\eqnlab{conven_extder}
\end{equation}
giving the wedge product derivation law
\begin{equation}
d\lp A_{(p)}\wedge B_{(q)}\rp=A_{(p)}\wedge dB_{(q)}+(-1)^qdA_{(p)}\wedge B_{(q)}.
\eqnlab{conven_extder_product}
\end{equation}
Note that, for well behaved functions, two derivatives commute so 
\begin{equation}
d^2A_{(p)} = \partial_{(M}\partial_{N)}A_{(p)}\we dx^{[N}\we dx^{M]} = 0.
\end{equation}
%
\paragraph{Hodge duality of a $p$-form}
%
Map from a $p$-form to an ($n-p$)-form defined as 
\begin{equation}
*A_{(p)} =
 \frac{1}{p!(n-p)!}dx^{M_{p+1}}\wedge\cdots\wedge dx^{M_n}\varepsilon_{M_{p+1}\cdots  M_{n}N_1\cdots N_p}A^{N_p\cdots N_1}, 
\eqnlab{conven_hodge_form}
\end{equation}
which in component form becomes
\begin{equation}
(*A)_{M_{n}\cdots M_{p+1}} = \frac{1}{p!}\varepsilon_{M_{p+1}\cdots M_{n}N_1\cdots N_p}A^{N_p\cdots N_1}  
\eqnlab{conven_hodge_comp}
\end{equation}
or looking only at the differentials
\begin{equation}
*\lp dx^{M_{1}}\wedge\cdots\wedge dx^{M_p}\rp = \frac{1}{(n-p)!}dx^{N_{p+1}}\wedge\cdots\wedge dx^{N_n}\varepsilon{_{N_{p+1}\cdots  N_{n}}}^{M_1\cdots M_p}.
\end{equation}
%
Using \eqnref{conven_hodge_comp} and swapping order of n with n-p antisymmetric indices, one sees that two hodge dualities performed after each other gives back the starting form with a possible additional minus sign
\begin{equation}
*\lp*A_{(p)}\rp = (-1)^{s+p(n-p)}A_{(p)},
\end{equation}
where $s=0$ for Riemannian space and $s=1$ for Minkowski space.
%
One can also rewrite an arbitrary $p$-form in terms of its Hodge duality by
\begin{align}
A_{(p)} &= (-1)^{s+p(n-p)}*(*A_{(p)})\\ 
& = \frac{(-1)^{s+p(n-p)}}{(n-p)!p!}dx^{M_{1}}\wedge\cdots\wedge dx^{M_p}\varepsilon_{M_1\cdots  M_{p}N_{p+1}\cdots N_{n}}\lp*A_{(p)}\rp^{N_{n}\cdots N_{p+1}}\nonumber
\end{align}
or in component form
\begin{equation}
A_{M_p\cdots M_1} = \frac{(-1)^{s+p(n-p)}}{(n-p)!}\varepsilon_{M_1\cdots M_{p}N_{p+1}\cdots N_n}(*A)^{N_{n}\cdots N_{p+1}}.
\eqnlab{conven_hodge_inverse_comp}
\end{equation}

\paragraph{Volume form}
The volume form is
\begin{align}
\sigma & = *1 = \frac{1}{n!}dx^{M_1}\we\cdots\we dx^{M_n}\varepsilon_{M_1\cdots M_n}\nonumber\\
& = \frac{(-1)^s}{n!}\sqrt{|g|}d^nx\varepsilon^{M_1\cdots M_n}\varepsilon_{M_1\cdots M_n} = +\sqrt{|g|}d^nx,
\eqnlab{conven_volume}
\end{align}
where we have used
\begin{equation}
dx^{M_1}\we\cdots\we dx^{M_n} = (-1)^sd^nx\epsilon^{M_1\cdots M_n} = (-1)^s\sqrt{|g|}d^nx\varepsilon^{M_1\cdots M_n}.
\eqnlab{conven_wedge_volume}
\end{equation}

\paragraph{Inner product}
If one has 2 forms $A$ and $B$ of equal length $p$ one can form an inner
product $<A,B>=\frac{1}{p!}A\cdot B$ from
\begin{align}
*A\wedge & B = \frac{\varepsilon_{M_{p+1}\cdots M_nN_1\cdots N_p}}{(n-p)!p!^2}A^{N_p\cdots N_1}B_{M_p\cdots M_1}dx^{M_{p+1}}\cdots dx^{M_n} dx^{M_{1}}\cdots dx^{M_p}\nonumber\\
& = \frac{(-1)^s}{(n-p)!p!^2}\sqrt{g}\epsilon_{M_{p+1}\cdots M_nN_1\cdots N_p}A^{N_p\cdots N_1}B_{M_p\cdots M_1}d^nx\epsilon^{M_{p+1}\cdots M_nM_1\cdots M_p}\nonumber\\
& = d^nx\sqrt{|g|}\frac{1}{(n-p)!p!^2}(n-p)!p!\delta_{[N_1\cdots N_p]}^{M_1\cdots M_p}A^{N_p\cdots N_1}B_{M_p\cdots M_1}\nonumber\\
& = \sigma\frac{1}{p!}A^{M_p\cdots M_1}B_{M_p\cdots M_1} = \sigma <A,B>,
\eqnlab{conven_inner}
\end{align}
where the inner product is $<A,B> = \frac{1}{p!}A^{M_p\cdots M_1}B_{M_p\cdots M_1}$.
Note that we never used that the hodge star was acting on the $A$ form. We could as well have acted on the $B$ form, so
\begin{equation}
\sigma<A,B> = *A\wedge B = A\wedge *B = p!\sigma<*A,*B>.
\end{equation}

\paragraph{Differentiation of forms}
The differentiation of the components of a $p$-form $A_{(p)}$ with respect to the components of another $p$-form of the same type is
\begin{equation}
\frac{\delta A_{M_p\cdots M_1}}{\delta A_{N_p\cdots N_1}} =\delta_{M_p}^{N_p}\cdots\delta_{M_2}^{N_2}\delta_{M_1}^{N_1} - \delta_{M_p}^{N_p}\cdots\delta_{M_1}^{N_1}\delta_{M_2}^{N_2} + \cdots  
= p!\delta_{[M_p\cdots M_1]}^{N_p\cdots N_1}.
\end{equation}
Now suppose a $p$-form $A_{(p)}$ constructed by 3 different forms $A^1_{(q)}$,
$A^2_{(r)}$ and $A^3_{(s)}$ of orders $q$, $r$ and $s$, with
$q+r+s=p$. I.e. $A_{(p)}=A^1_{(q)}\we A^2_{(r)}\we
A^3_{(s)}$.
%
Differentiate with respect to $A^2_{(r)}$ in component form
\begin{align}
\frac{\delta A_{M_p\cdots M_1}}{\delta A^2_{N_r\cdots N_1}} & = 
\frac{p!}{q!r!s!}A^1_{M_q\cdots M_1}\frac{\delta A^2_{M_{q+r}\cdots M_{q+1}}}{\delta
  A^2_{N_r\cdots N_1}} A^3_{M_{p}\cdots M_{q+r+1}}\nonumber\\
& =  \frac{p!}{q!r!s!}A^1_{M_q\cdots M_1}r!\delta_{[M_{q+r}M_{q+r-1}\cdots M_{q+1}]}^{N_rN_{r-1}\cdots\cdots\cdots\cdots\cdots N_1} A^3_{M_{p}\cdots M_{q+r+1}}.
\end{align}
Let $B_{(t)}=\frac{\delta A_{(p)}}{\delta A^2_{(r)}}$ be a $t$-form and multiply both sides from the right with $dx^{N_1}\wedge\cdots\wedge dx^{N_r}/r!$, so
\begin{align}
&B_{(t)}\wedge\frac{1}{r!}dx^{N_1}\wedge\cdots\wedge dx^{N_r} = 
\frac{1}{r!p!}\frac{\delta A_{M_p\cdots M_1}}{\delta A^2_{N_{r}\cdots N_{1}}}dx^{M_1}\we\cdots \we dx^{M_p}\nonumber\\
& = \frac{1}{q!r!s!}A^1_{M_q\cdots M_1}A^3_{M_{p}\cdots M_{q+r+1}}dx^{M_1}\cdots dx^{M_q}dx^{N_1}\cdots dx^{N_r}dx^{M_{q+r+1}}\cdots dx^{M_p}\nonumber\\
& = \frac{1}{r!}A^1_{(q)}\we dx^{N_1}\we\cdots\we dx^{N_r} \we A^3_{(s)}.\\
\end{align}
We can now identify $B_{(t)}$ as
\begin{equation}
B_{(t)} = B_{(p-r)} = \frac{\delta A_{(p)}}{\delta A^2_{(r)}}=(-1)^{rs}A^1_{(q)}\we A^3_{(s)},
\end{equation}
which we use as a definition of derivation of a form with respect to a form. The reason we put $dx^{N_1}\wedge\cdots\wedge dx^{N_r}/r!$ to the right of $B_{(t)}$ and not to the left, which would have given $B_{(t)}=(-1)^{qr}A^1_{(q)}\we A^3_{(s)}$, is that we put all variations of forms $\delta A^2_{(r)}$ to the right.   

To differentiate a scalar function $\Phi\lp A_{M_p\cdots M_1}\rp$, which depends only on the components of $A_{(p)}$, we just do a component differentiation
\begin{align}
B^{N_{p}\cdots N_{1}} = \frac{\delta \Phi\lp A_{M_p\cdots M_1}\rp}{\delta A_{N_{p}\cdots N_{1}}}  
\end{align}
and contract the free indices with $dx^{N_1}\wedge\cdots\wedge dx^{N_p}/p!$ to get a $p$-form 
\begin{equation}
B^{(p)} = \frac{\delta \Phi\lp A_{M_p\cdots M_1}\rp}{\delta A_{(p)}} = \frac{1}{p!}\frac{\delta \Phi\lp A_{M_p\cdots M_1}\rp}{\delta A_{N_{p}\cdots N_{1}}}dx^{N_1}\wedge\cdots\wedge dx^{N_p}.
\end{equation}

With the tools collected so far one easily obtains the differentiation with respect to a $p$-form $A_{(p)}$ of the hodge dual of a max-form $B_{(n)}$, such that $B_{(n)} = B^2_{(n-p)}\we A_{(p)}$ as
\begin{align}
*&\frac{\delta *B_{(n)}}{\delta A_{(p)}} = \frac{1}{p!(n-p)!}\varepsilon_{M_1\cdots M_n}B^{2M_n\cdots M_{p+1}}*\lp dx^{M_1}\we\cdots\we dx^{M_p}\rp\nonumber\\ 
& = \frac{(-1)^{s+p(n-p)}}{(n-p)!}B^2_{M_n\cdots M_{p+1}}dx^{M_{p+1}}\we\cdots\we dx^{M_{n}} = (-1)^{s+p(n-p)}B^2_{(n-p)}.  
\eqnlab{conven_hodge_variation}
\end{align}

\paragraph{Variation of an action with respect to forms}
Consider a scalar Lagrangian density $\Lagr\lp A_{(p)}\rp$ that is a function of some $p$-form $A_{(p)}$.
The variation of the action with respect to the components of $A_{(p)}$ is 
\begin{align}
\delta_{A} S&=\delta_{A}\left[\int d^nx\sqrt{|g|}\Lagr\lp A_{(p)}\rp\right] = \int d^nx\sqrt{|g|}\frac{\delta\Lagr\lp A_{(p)}\rp}{\delta A_{M_p\cdots M_1}}\delta A_{M_p\cdots M_1}.\nonumber\\
\end{align}
Consider 
\begin{equation}
B^{M_p\cdots M_1} = \frac{\delta\Lagr\lp A_{(p)}\rp}{\delta A_{M_p\cdots M_1}} 
\end{equation}
as the components of a $p$-form $B_{(p)}$ and use the definition of the inner product \eqnref{conven_inner} between $B_{(p)}$ and $\delta A_{(p)}$ to get 
\begin{align}
\delta_{A} S&= p!\int \sigma \left<\frac{\delta\Lagr\lp A_{(p)}\rp}{\delta A_{M_p\cdots M_1}},\delta A_{M_p\cdots M_1}\right>\nonumber\\
&=p! \int * \frac{\delta\Lagr\lp A_{(p)}\rp}{\delta A_{(p)}}\wedge\delta A_{(p)}.
\eqnlab{conven_deltaS}
\end{align}

\section{General relativity}
\paragraph{Covariant divergence}
of a general covariant vector $V^M$ is \cite{weinberg}  
\begin{align}
D_M V^M &= {1 \over \sqrt{|g|}} {\partial}_M\lp\sqrt{|g|}V^M\rp\nn\\
&\Rightarrow \int d^Dx\sqrt{|g|}D_M V^M = 0,\;\;\mbox{if $V^M=0$ at $\infty$}.
\eqnlab{conven_divergence}
\end{align}
\paragraph{Variation of $g=\det g_{MN}$ and pure gravity with respect to $g_{MN}$}
\begin{align}
\delta g = g g^{MN}\delta g_{MN},
\eqnlab{conven_gvar}
\end{align}
\begin{align}
\delta\lp\sqrt{g}R\rp=\lp R^{MN} - \half g^{MN}R\rp\delta g_{MN}. 
\eqnlab{conven_rvar}
\end{align}

\section{Matrix identities}
Consider a matrix $\M$ and calculate
\begin{equation}
0 = \partial\id = \partial\lp \M^{-1}\M\rp = \partial \M^{-1}\M + \M^{-1}\partial \M,
\eqnlab{conven_parinv}
\end{equation}
giving the identity
\begin{align}
\partial \M^{-1} &= - \M^{-1}\partial \M \M^{-1},\nn\\
\partial \M^{-1}\partial \M &= -\lp\M^{-1}\partial \M\rp^2.  
\eqnlab{conven_kinetic}
\end{align}

\paragraph{Determinant conditions}
% Anv�nder p 75 i Fields ist�llet
%Assume real matrix $\M$ of full rank, diagonalized by some diagonal matrix $D=T^{-1}\M T$ with eigenvalues $\lambda_i$. The determinant can then be written as 
%\begin{align}
%\det \M & = \det(TDT^{-1}) = \det D = \prod_i\lambda_i = \exp\left(\sum_i\ln\lambda_i\right) = \exp\left(\tr(\ln D)\right)\nonumber\\
%& = \exp\left(\tr(\ln T)-\tr(\ln T)+\tr(\ln D)\right) = \exp\left(\tr(\ln\M))\right)
%\eqnlab{conven_trlog}
%\end{align}
%\\
For an arbitrary matrix $\M$ we have 
\begin{equation}
\delta\ln\det(\M) = \tr\lp \M^{-1}\delta \M\rp  
\eqnlab{conven_trlog}
\end{equation}
from p. 106 in Weinberg\cite{weinberg}. Letting $\M=\exp(\N)$ gives
\begin{equation}
\delta\ln\det\exp(\N) = \tr\lp \exp(-\N)\delta \exp{\N}\rp = \tr\lp\delta \N\rp = \delta\tr\lp\N\rp,  
\end{equation}
giving
\begin{equation}
\det\M = \exp\lp\tr\lp\ln\M\rp\rp
\end{equation}
which will have many applications when dealing with determinants.

\paragraph{Determinant of block matrix}
Consider an $n\times n$-matrix and divide the rows and columns in two, forming a 4 piece block matrix.

\begin{equation}
\M = \toto{A}{B}{C}{D} = 
\setlength{\unitlength}{.4mm}
\left(\begin{array}{l}\cr\mbox{}\end{array}\right.
\begin{picture}(18,30)(8,12)% (size)(offset)
\put(0,0){
\path(0,12)(18,12)(18,30)(0,30)(0,12) % A (0,12) -> (18,30)
\path(0,0)(18,0)(18,10)(0,10)(0,0) % C (0,0) -> (18,10)
\path(20,12)(33,12)(33,30)(20,30)(20,12) % B
\path(20,0)(33,0)(33,10)(20,10)(20,0) % D
\put(9,21.5){\makebox(0,0){A}}
\put(9,5.5){\makebox(0,0){C}}
\put(27,21.5){\makebox(0,0){B}}
\put(27,5.5){\makebox(0,0){D}}
}
\end{picture}
\left.\begin{array}{l}\cr\mbox{}\end{array}\right)
\end{equation}
Note that we can decompose $\M$ as (proof by calculating backwards)
\begin{align}
\M = \toto{\id}{BD^{-1}}{0}{\id}\toto{A-BD^{-1}C}{0}{0}{D} \toto{\id}{D^{-1}C}{0}{\id}.
\end{align}
Take the determinant of $\M$ and expand along rows or columns consisting of only one 1 and zeroes to get two relations of the determinant
\begin{equation}
\det \M = \det\lp A-BD^{-1}C\rp \det D = \det A \det\lp D-CA^{-1}B\rp, 
\end{equation}
where the second relation comes from a similar decomposition of $\M$ into Lower-Diagonal-Upper-form.


%\chapter{Lengthy calculations}
\chlab{lengthy}
\section{Reduction and scaling of the Ricci scalar}
\seclab{ricci}
We will use the tangent space formalism to dimensionally reduce the Ricci scalar and transform the metric to the Einstein frame. 

\subsection{Locally flat geometry}
\sseclab{ricci_local_geom}
\paragraph{Locally flat frames}
To each point in a curved space with metric $g_{MN}$, we can assign a locally flat coordinate system with metric $\eta_{AB}$, tangent to the curved one. To transform between the two coordinate systems we use vielbeins $e{_M}^A$, defined to move between different metrics as
\begin{equation}
g_{MN} = e{_M}^A \eta_{AB} e{^B}{_N} 
\end{equation}
Since the n-dimensional covariant metric (without gauge conditions) has $n(n+1)/2$ degrees of freedom, there are $n^2-n(n+1)/2=n(n-1)/2$ undetermined components in the vielbeins.
These are related to the $n(n-1)/2$ degrees of freedom of a local SO($\hat D$) rotation, or a SO($\hat D$-1,1) local Lorentz transformation in our case where we have one timelike coordinate.  
This symmetry comes from the fact that the vielbeins form a scalar in the local indices $e{_M}^Ae{_A}{_N}$ and the metric is thus invariant under local Lorentz transformations
\begin{align}
e{_M}^A &\rightarrow e{_M}^B(\Lambda^{-1}){_B}^A\hspace{1cm} \nonumber\\ 
e{_A}^M &\rightarrow \Lambda{_A}^Be{_B}^M,\hspace{1cm} \Lambda = \Lambda(x)\in \mbox{SO}(\hat D-1,1)
\eqnlab{sugra_viel_rot}
\end{align}
This symmetry can be used to set $n(n-1)/2$ of the components in $e{_M}^A$ to whatever you want. 

\paragraph{Covariant derivative}
The exterior derivative acting on a tensor or form with m upper and n lower SO($\hat D$-1,1) Lorentz indices, transforms as
\begin{align}
dT\bb{A_1\cdots A_m}{C_1\cdots C_n} \rightarrow& d \lp\Lambda\od{C_1}\ou{D_1}\cdots \Lambda\od{C_n}\ou{D_n}\Lambda\ou{A_1}\od{B_1}\cdots \Lambda\ou{A_n}\od{B_n}T\bb{B_1\cdots B_m}{D_1\cdots D_n}\rp\nn\\ 
 =& \Lambda\cdots\Lambda dT + \Lambda\cdots\Lambda Td\Lambda\od{C_1}\ou{D_1}+\cdots+\Lambda\cdots\Lambda Td\Lambda\od{C_n}\ou{D_n}\nn\\
&+\Lambda\cdots\Lambda Td\Lambda\ou{A_1}\od{B_1}+\Lambda\cdots\Lambda Td\Lambda\ou{A_m}\od{B_m}
\end{align}
The $d\Lambda$ terms prevents this from being a Lorentz tensor. We thus introduce a covariant derivative
\begin{align}
DT\bb{A_1\cdots A_m}{C_1\cdots C_n} &= dT\bb{A_1\cdots A_m}{C_1\cdots C_n} + T\bb{B_1A_2\cdots A_m}{C_1\phantom{A_2}\cdots C_n}\omega\ou{A_1}\od{B_1} + \cdots + T\bb{A_1\cdots B_m}{C_1\cdots C_n}\omega\ou{A_m}\od{B_m}\nn\\
& + T\bb{A_1\phantom{C_2}\cdots A_m}{D_1C_2\cdots C_n}\omega\od{C_1}\ou{D_1} + \cdots + T\bb{A_1\cdots B_m}{C_1\cdots D_n}\omega\od{C_m}\ou{D_m} 
\end{align}
and demand $DT$ to transform as a Lorentz-tensor, i.e. $DT\rightarrow \Lambda\cdots\Lambda DT$, by assigning the correct transformation properties on the 1-form spin connection $\omega\ou{A}\od{B}$.
Since the transformation properties of $\omega$ are independent of n, we can look at the case m=1, n=0 with requirement $DT'^A \equiv \Lambda{^A}_B DT^B$, giving
\begin{align}
DT^A \rightarrow DT'^A &= dT'^A + T^B\omega'\ou{A}\od{B} = \Lambda\ou{A}\od{B}dT^{B} + T^{B}d\Lambda\ou{A}\od{B} + \Lambda\ou{B}\od{C}T^C\omega'\ou{A}\od{B} \nonumber\\
& \equiv \Lambda\ou{A}\od{B}DT^B = \Lambda\ou{A}\od{B}\lp dT^B + T^C\omega\ou{B}\od{C}\rp
\end{align}
after removing $T^C$ and multiplying by $\Lambda\od{B}\ou{C}$ we get
\begin{equation}
\omega'\ou{A}\od{B} = \Lambda\ou{A}\od{D}\omega\ou{D}\od{C}(\Lambda^{-1})\ou{C}\od{B} -  d\Lambda\ou{A}\od{C}(\Lambda^{-1})\ou{C}\od{B}
\end{equation}
which is the transformation $\omega$ must obey.
By calculating the covariant derivative in the coordinate basis, using the affine connection $\Gamma$ and comparing this to the covariant derivative in a mixed basis, using the spin connection $\omega$, we can find a relation between the two connections\cite{carroll}. 
The relation becomes
\begin{align}
\omega{{_M}^A}_B = e{_N}^A e{_B}^P\Gamma^N_{MP} - e{_B}^P\partial_M e{_P}^A
\eqnlab{ricci_spin_affine}
\end{align}
Using spin connections rather than the regular affine connections in the covariant derivative allows the descriptions of spinors in space time and it allows taking covariant derivatives of spinors (hence the name). Furthermore spin connection lets us describe the torsion and curvature as vector and (1,1)-tensor valued 2-forms as we shall see next.    
%
\paragraph{Torsion}
Now consider the 2-form $T^A = De^A$, which components can, by using \eqnref{ricci_spin_affine}, be identified as the torsion tensor
\begin{equation}
T{_{MN}}^Ae{_A}^P = T{_{MN}}^P = 2\Gamma^P_{[MN]},
\end{equation}
which becomes 0 for the standard Riemannian general relativity Christoffel connection $\Gamma^P_{(MN)}$. 
This constraint and the metricity condition $D_Mg_{NP}=0$ (also used in Riemannian geometry) uniquely defines $\omega$ by
\begin{equation}
T^A = De^A = de^A + e^B\we\omega\ou{A}\od{B} = de^A - 2\Omega^A = 0, 
\eqnlab{conven_torsion}
\end{equation}
which defines the 2-form $\Omega^A$ by
\begin{equation}
\Omega^A = \frac{1}{2!}e^B\we e^C\Omega\od{CB}\ou{A} = -\frac{1}{2}e^B\we\omega\ou{A}\od{B} = \frac{1}{2}e^B\we e^C\omega\od{CB}\ou{A}.
\eqnlab{conven_Omega}
\end{equation}
We can thus read of the components of $\Omega\ou{A}$ as
\begin{equation} 
\Omega\od{CBA} = \omega\od{[CB]A} = \frac{1}{2}\left(\omega_{CBA}-\omega_{BCA}\right)
\eqnlab{conven_Omega_comp}
\end{equation}
which gives
\begin{align}
\Omega{_{CBA}}+\Omega{_{ACB}}-\Omega{_{BAC}} = \frac{1}{2}&\left( \omega_{CBA}-\omega_{BCA} + \omega_{ACB}-\omega_{CAB}\right.\nonumber\\ 
&\left.- \omega_{BAC}+\omega_{ABC} \right)= \omega_{CBA},
\end{align}
where we have used the antisymmetry of the spin connection $\omega_{CBA}=-\omega_{CAB}$.
This antisymmetry is obvious if one considers the metricity condition of the Lorentz metric $D\eta_{AB}=0$, giving 
\begin{equation}
D\eta\od{AB} = d\eta\od{AB} + \eta\od{CA}\omega\od{B}\ou{C} + \eta\od{CB}\omega\od{A}\ou{C} = \omega\od{BA} + \omega\od{AB} = 0 
\end{equation}

\paragraph{Curvature}
Form the 2-form\footnote{Usually $\Theta$ is defined with a plus sign, but for the Bianchi identity $d\Theta$ to imply $D\Theta = 0$, we need the minus sign with our superspace convention of external derivatives acting from the right.}
\begin{align}
\Theta{^A}_{B} = d\omega{^A}_B-\omega{^A}_C\omega{^C}_B
\eqnlab{conven_theta_omegadef}
\end{align}
which transforms as (use \eqnref{conven_parinv} to cancel the terms)
\begin{align} 
\Theta\rightarrow&\Theta' =  d\omega'-\omega'\omega'\nn\\
& =  d\lp \Lambda\omega\Lambda^{-1} - d\Lambda\Lambda^{-1}\rp - \lp \Lambda\omega\Lambda^{-1} - d\Lambda\Lambda^{-1}\rp\lp \Lambda\omega\Lambda^{-1} - d\Lambda\Lambda^{-1}\rp\nn\\
& = \Lambda\omega d\Lambda^{-1} + \Lambda d\omega\Lambda^{-1} -  d\Lambda\omega\Lambda^{-1} - d\Lambda d \Lambda^{-1}\nn\\ 
& - \Lambda\omega\omega\Lambda^{-1} + \Lambda\omega\Lambda^{-1}d \Lambda\Lambda^{-1} + d\Lambda\omega\Lambda^{-1} - d\Lambda \Lambda^{-1}d\Lambda \Lambda^{-1}\nn\\
& = \Lambda\lp  d\omega - \omega\omega \rp\Lambda^{-1} = \Lambda \Theta \Lambda^{-1} 
\end{align} 
under Lorentz transformations i.e. $\Theta{^B}_A$ is a Lorentz tensor.
The antisymmetry of $\omega_{AB}$ makes $\Theta_{BA}$ antisymmetric in $A\leftrightarrow B$.
The definitions of $T^A$ \eqnref{conven_torsion} and $\Theta^B_A$ \eqnref{conven_theta_omegadef} are usually denoted as the Maurer Cartan structure equations (c.f. \Secref{maurer}).
Since $\Theta$ is a 2-form it can also be written on the form
\begin{align}
\Theta{^B}_A = \half  e^C\we e^D \Theta{{_{DC}}^B}_A = \half dx^P\we dx^Q e{_N}^Be{_A}^M R\od{QP}\ou{N}\od{M}
\eqnlab{conven_theta2}
\end{align} 
where $\Theta{{_{DC}}^B}_A$ is identified to be the Riemann tensor with flat indices (use \eqnref{ricci_spin_affine}). 

\subsection{Reduction of the Ricci scalar}
Here we will calculate the Ricci scalar after Kaluza-Klein compactification on $T^n$, for a general n.
%
From the Kaluza-Klein Ansatz made in \eqnref{reduct_kk_ansatz} we found that the vielbeins can be decomposed \eqnref{reduct_viel_red} as 
\begin{equation}
{\hat e}^A = (\hat e^a,\hat e^i) = (e^a,A^{1i} + e^i)
\end{equation}
and in our Kaluza-Klein analysis in \Secref{sugra_kk} we found that, for our considerations, we can set all fields to be independent on the compactified coordinates $x^m$.
This effectively means that all derivatives $\partial_m$ with respect to $x^m$ will be zero and we get the exterior derivative in $\hat D$ dimensions  
\begin{equation}
\hat d = d = dx^\mu\partial_\mu = e^a\partial_a 
\end{equation}
%
The torsion \eqnref{conven_torsion} of pure gravity is zero, which gives
\begin{equation}
\hat d \hat e^A = \hat e^B\we \hat e^C \Omh{CB}{A}
\eqnlab{ricci_torsion}
\end{equation}
Note that inverting $e^i = dx^me{_m}^i$ gives $dx^m = e^i e{_i}^m$ 
and calculate both left and right hand side of \eqnref{ricci_torsion} (use \eqnref{reduct_f1def} as definition of $F^1$)
\begin{align}
\mbox{LHS: }\hat d \hat e^A &= \partial_\mu\lp e^a,\hat e^i\rp\we dx^\mu = \lp de^a,\partial_\mu\lp A^{1i} + e^i\rp\we dx^\mu\rp\nonumber\\
& = \lp e^b\we e^c \Om{cb}{a},\partial\od{b}\lp A^{1i} + e^i\rp\we e^b\rp \nonumber\\ 
& = \lp \Omega^a,\partial\bd{b}\lp e\od{m}\ou{i}A\bb{1}{c}\bu{m}\rp e^c\we e^b + \partial_be^i\we e^b \rp\nonumber\\  
& = \lp \Omega^a,\frac{1}{2}e\od{m}\ou{i}F\bb{1}{cb}\bu{m}e^b\we e^c + \partial\bd{b}e\od{m}\ou{i}A\bb{1}{c}\bu{m} e^c\we e^b + \partial_be{_m}^idx^m\we e^b \rp \nonumber\\  
& = \lp \Omega^a,F^{1i} + A\ou{1m}de\od{m}\ou{i} + \partial_be{_m}^i e{_j}^m e^j\we e^b \rp\nn\\
& = \lp \Omega^a,F^{1i} + A\ou{1m}de\od{m}\ou{i} + \partial_be{_m}^i e{_j}^m \hat e^j\we e^b - \partial_be{_m}^i e{_j}^m A^{1j}\we e^b \rp\nn\\
& = \lp \Omega^a,F^{1i} + \partial_be{_m}^i e{_j}^m \hat e^j\we e^b\rp
\eqnlab{ricci_deA}
\\
\mbox{RHS: }\hat e^B\we&\hat e^C \Omega{_{CB}}^A = \left(e^b,\hat e^j\right)\we\left(e^c,\hat e^k\right)\Omega{_{CB}}^A\nonumber\\
%& = e^b\we e^c\Omh{cb}{A} + e^b\we\lp A^{1k} + e^k\rp\Omh{kb}{A} + \lp A^{1j} + e^j\rp\we e^c\Omh{cj}{A} + \lp A^{1j} + e^j\rp\we\lp A^{1k} + e^k\rp\Omh{kj}{A}\\
& = e^b\we e^c\Omh{cb}{A} + e^b\we\hat e^k\Omh{kb}{A} + \hat e^j\we e^c\Omh{cj}{A} + \hat e^j\we\hat e^k\Omh{kj}{A}\nonumber\\
& = e^b\we e^c\Omh{cb}{A} + 2e^b\we\hat e^k\Omh{kb}{A} + \hat e^j\we\hat e^k\Omh{kj}{A}
\end{align}
Comparing the two sides, starting with A=a
\begin{align}
e^b\we e^c \Om{cb}{a} = e^b\we e^c\Omh{cb}{a} + 2e^b\we\hat e^k\Omh{kb}{a} + \hat e^j\we\hat e^k\Omh{kj}{a}
\end{align}
and then with A=i
\begin{align}
\frac{1}{2}F\bb{1}{cb}\bu{i}e^b\we e^c & - \partial_be{_m}^i e{_k}^m e^b\we\hat e^k = e^b\we e^c\Omh{cb}{i} + 2e^b\we\hat e^k\Omh{kb}{i} + \hat e^j\we\hat e^k\Omh{kj}{i}
\end{align}
gives the following components of $\hat\Omega^A$
\begin{align}
&\Omh{cb}{a}=\Om{cb}{a},\hspace{0.5cm}
&&\Omh{cb}{i}=\frac{1}{2}F\bb{1}{cb}\bu{i},\hspace{0.5cm}\nonumber\\
&\Omh{kb}{a}=-\Omh{bk}{a}=0,\hspace{0.5cm}
&&\Omh{kb}{i}=-\Omh{bk}{i}=-\frac{1}{2}\partial_be{_m}^i e{_k}^m,\hspace{0.5cm}\nonumber\\
&\Omh{kj}{a}=0,
&&\Omh{kj}{i}=0
\end{align}
%
Thus the components of $\hat\omega$ becomes, using \eqnref{conven_Omega_comp}
\begin{align}
\hat\omega_{cba} &= \Omega_{cba} + \Omega_{acb} - \Omega_{bac} = \omega_{cba}\nonumber\\
\hat\omega_{cbi} &= - \hat\omega_{cib} = \half F^1_{cbi}+0+0 = \half F^1_{cbi}\nonumber\\ 
\hat\omega_{iba} &= 0+0-\half F^1_{bai} = -\half F^1_{bai}\nonumber\\ 
\hat\omega_{cji} &= \half \partial_ce_{mi}e{_j}^m - \half \partial_ce_{mj}e{_i}^m - 0 = \partial_ce_{m[i}e{_{j]}}^m\nonumber\\ 
\hat\omega_{jbi} &= - \hat\omega_{jib} = - \half \partial_be_{mi}e{_j}^m + 0 - \half \partial_be_{mj}e{_i}^m = - \partial_be_{m(j}e{_{i)}}^m \nonumber\\ 
\hat\omega_{kji} &= 0+0-0 = 0 
\end{align}
where we have used that flat indices can be raised and lowered inside the partial derivatives.
The components of $\hat\omega$ as a 1-form 
\begin{equation}
\hat\omega_{BA}=\hat e^C\hat\omega_{CBA} = \hat e^c\hat\omega_{cBA} + \hat e^i\hat\omega_{iBA} 
\end{equation}
becomes
\begin{align}
\hat\omega_{ba} &= \hat e^c\omega_{cba} - \half\hat e^i F^1_{bai} = \omega_{ba} - \half\hat e^i F^1_{bai}\nonumber\\ 
\hat\omega_{ja} &= -\hat\omega_{aj} = \half \hat e^c F^1_{acj} + \hat e^i\partial_a e_{m(j}e{_{i)}}^m\nonumber\\
\hat\omega_{kj} &= \hat e^c\partial_ce_{m[j}e{_{k]}}^m + 0 = \hat e^c\partial_ce_{m[j}e{_{k]}}^m.
\end{align}
%
From \eqnref{conven_theta2} we have
\begin{align}
\hat\Theta_{BA} &= \half\hat e^C\we\hat e^D\hat R_{DCBA}\nonumber\\ 
&= \half\hat e^c\we\hat e^d\hat R_{dcBA} + \hat e^c\we\hat e^i\hat R_{icBA} + \half\hat e^i\we\hat e^j\hat R_{jiBA}
\end{align}
where we have used the antisymmetry in the first two indices of the Riemann tensor $R_{DCBA} = -R_{CDBA}$ to put the cross terms together.
%
To calculate the Ricci scalar from the Riemann tensor
\begin{equation}
\hat R = \hat R{_{BA}}^{BA} = \hat R{_{ba}}^{ba} + 2\hat R{_{ja}}^{ja} + \hat R{_{ji}}^{ji}
\eqnlab{ricci_ricci}
\end{equation}
we need the $\hat R_{dcba}$, $\hat R_{lcja}$ and $R_{lkji}$ components of $\hat R_{DCBA}$.
These can be obtained from the definition of the 2-form $\hat\Theta_{BA}$, \eqnref{conven_theta_omegadef} (remember that we have already calculated $d\hat e^i$ in \eqnref{ricci_deA}) 
\begin{align}
&\hat\Theta_{ba} = d\hat\omega_{ba} - \hat\omega{_b}^c\we\hat\omega_{ca} - \hat\omega{_b}^i\we\hat\omega_{ia}=\nonumber\\
&= d\omega_{ba} - \half d\lp\hat e^i F^1_{bai}\rp - \lp\omega{_b}^c - \half\hat e^i F\bb{1}{b}\bu{c}\bd{i}\rp\we\lp\omega_{ca} - \half\hat e^j F^1_{caj}\rp\nonumber\\
&- \lp\half \hat e^c F\bb{1}{cb}\bu{i} - \hat e^{(k}\partial_b e{_n}^{i)}e{_{k}}^n\rp\we\lp\half \hat e^d F^1_{adi} + \hat e^j\partial_a e_{m(i}e{_{j)}}^m\rp\nonumber\\
&= d\omega_{ba} + \omega{_b}^c\omega_{ca} - \half d\hat e^i F^1_{bai} - \half \hat e^i d F^1_{bai} + \half\omega{_b}^c\we\hat e^j F^1_{caj} + \half\hat e^i F\bb{1}{b}\bu{c}\bd{i}\we\omega\bd{ca}\nonumber\\
&- \half\hat e^i F\bb{1}{b}\bu{c}\bd{i}\we\half\hat e^j F^1_{caj} - \half \hat e^c F\bb{1}{cb}\bu{i}\we\half \hat e^d F^1_{adi} + \hat e^{(k}\partial_b e{_n}^{i)}e{_{k}}^n\we\half \hat e^d F^1_{adi}\nonumber\\
&- \half \hat e^c F\bb{1}{cb}\bu{i}\we\hat e\bu{j}\partial\bd{a} e\bd{m(i}e\bd{j)}\bu{m} + \hat e^{(k}\partial_b e{_n}^{i)}e{_{k}}^n\we\hat e^j\partial_a e_{m(i}e{_{j)}}^m\nonumber\\
&=\Theta_{ba} - \half\lp \half F\bb{1}{dc}\bu{i}\hat e\bu{c}\we\hat e\bu{d} + \partial_c(e{_m}^i) e{_j}^m\hat e^j\we\hat e^c\rp F^1_{bai} - \frac{1}{4}\hat e\bu{i}\we\hat e\bu{j} F\bb{1}{b}\bu{c}\bd{i} F^1_{caj}\nonumber\\
&- \half \hat e^i \we \lp dF^1_{bai} + \omega\bd{b}\bu{c}F\bb{1}{cai} - F\bb{1}{bci}\omega\bu{c}\bd{a}\rp - \frac{1}{4} \hat e\bu{c}\we \hat e\bu{d} F\bb{1}{cb}\bu{i}F^1_{adi}\nonumber\\
&- \half \hat e\bu{d}\we\hat e\bu{(k}\partial\bd{b} e\bd{n}\bu{i)}e\bd{k}\bu{n} F^1_{adi} - \half \hat e\bu{c}\we\hat e\bu{j} F\bb{1}{cb}\bu{i}\partial\bd{a} e\bd{m(i}e\bd{j)}\bu{m} + 0\nonumber\\
%&=\frac{1}{4}e^c\we e^d\lbp 2R_{dcba} - F{^{1}_{dc}}^iF^1_{bai} + F^1{_{cb}}^iF^1_{adi} \rbp+\frac{1}{4} \hat e^i\we \hat e^j\lbp F^1{{_b}^c}_iF^1_{caj} \rbp\nn\\
%&+\half e^c\we \hat e^j\lbp \partial_ce{_m}^i e{_j}^mF^1_{bai} + D_cF^1_{baj} + \partial_b e{_m}^{i} F^1_{ac(i} e{_{j)}}^m + F^1{_{cb}}^i\partial_a e_{m(i}e{_{j)}}^m\rbp
\end{align}
For our purposes, we only need $\hat\Theta_{dcba}=\hat R_{dcba}$, which can be read off from what is multiplying $\half\hat e^c\we\hat e^d$ in $\hat\Theta_{ba}$, i.e.
\begin{equation}
\hat R_{dcba} = R_{dcba} - \half F\bb{1}{dc}\bu{i}F^1_{bai} - \half F\bb{1}{a[d}\bu{i}F^1_{c]bi}. 
\end{equation}
%
Next we attack
\begin{align}
\hat\Theta_{ja} &= d\hat\omega_{ja} - \hat\omega{_j}^b\we\hat\omega_{ba} - \hat\omega{_j}^i\we\hat\omega_{ia} \nonumber\\
& = d\lp\half \hat e^cF^1_{acj} + \hat e^i\partial_a e_{m(j}e{_{i)}}^m\rp\nn\\
& - \lp\half \hat e^cF{^{1b}}_{cj} + \hat e^i\partial^b e_{m(j}e{_{i)}}^m\rp\we\lp\omega_{ba} - \half\hat e^kF^1_{bak}\rp\nn\\
& - \lp \hat e^c\partial_c e_{m[k} e{_{j]}}^m\delta^{ik} \rp\we\lp \half \hat e^dF^1_{adi} + \hat e^l\partial_a e_{n(i}e{_{l)}}^n \rp\nn\\
\intertext{}
& = \half \partial\bd{d}\lp\hat e^cF^1_{acj}\rp\we\hat e^d + \lp\frac{1}{2}F\bb{1}{cb}\bu{i}\hat e^b\we\hat e^c + \partial_be{_n}^i e{_k}^n\hat e^k\we\hat e^b\rp\partial_a e_{m(j}e{_{i)}}^m\nn\\
& + \partial_c\lp\partial_a e_{m(j}e{_{i)}}^m\rp\hat e^i \we\hat e^c - \half F\bu{1b}\bd{cj}\hat e^c\we\omega\bd{ba} - \partial^b e_{m(j}e{_{i)}}^m\hat e^i\we\omega_{ba}\nn\\
& + \frac{1}{4}F\bu{1b}\bd{cj}F^1_{bak}\hat e\bu{c}\we\hat e\bu{k} + \half\partial^b e\bd{m(j}e\bd{i)}\bu{m}F^1_{bak}\hat e^i\we\hat e^k\nn\\
& - \half\partial\bd{c} e\bd{m[k} e\bd{j]}\bu{m}F\bb{1}{ad}\bu{k}\hat e^c\we \hat e^d - \partial_c e_{m[k} e{_{j]}}^m\delta^{ik} \partial_a e_{n(i}e{_{l)}}^n\hat e^c\we \hat e^l
\end{align}
%
Use $T_a = \partial_a e_{m(j}e{_{i)}}^m$ and note that 
\begin{align}
DT_a = dT_a + T_b\we\omega\od{a}\ou{b} = \partial_c&\lp\partial_a e_{m(j}e{_{i)}}^m\rp\hat e^c + \partial_b e_{m(j}e{_{i)}}^m\omega\od{a}\ou{b},
\end{align}
so we can rewrite term 3 and term 5 as
\begin{equation}
\partial_c\lp\partial_a e_{m(j}e{_{i)}}^m\rp\hat e^i \we\hat e^c - \partial^b e_{m(j}e{_{i)}}^m\hat e^i\we\omega_{ba} = D_c\lp\partial_a e_{m(j}e{_{i)}}^m\rp\hat e^i \we\hat e^c. 
\end{equation}
%
We only need $\hat\Theta_{lcja}=\hat R_{lcja}=-\hat R_{clja}$, which can be read off from what is multiplying $\hat e^c\we\hat e^l$ in $\hat\Theta_{ja}$ (no factor $\half$ for the crossterm), i.e.
\begin{align}
\hat R_{lcja} &= -e{_{l}}^n\partial_{c}e{_n}^i\partial_a e_{m(j}e{_{i)}}^m  
- D_{c}\lp\partial_{|a} e_{m|(l}e{_{j)}}^m\rp\nn\\
& + \frac{1}{4}F^1_{bal}F{^{1b}}_{cj}
- \partial_a e_{m(i}e{_{l)}}^m\partial_{c} e_{n[k} e{_{j]}}^n\delta^{ik}. 
\end{align}
%
Next
\begin{align}
&\hat\Theta_{ji} = d\hat\omega_{ji} - \hat\omega{_j}^b\we\hat\omega_{bi} - \hat\omega{_j}^k\we\hat\omega_{ki}=\nonumber\\
& = d\lp\hat e^c\partial_ce_{m[i}e{_{j]}}^m\rp
 - \hat e^c\partial_ce_{m[l}e{_{j]}}^m\delta^{kl}\we\hat e^d\partial_de_{m[i}e{_{k]}}^m\nn\\ 
& - \lp\half\hat e^cF{^{1b}}_{cj} + \hat e^k\partial^b e_{m(j}e{_{k)}}^m\rp\we\lp \half \hat e^cF^1_{cbi} - \hat e^l\partial_b e_{n(i}e{_{l)}}^n\rp.
\end{align}


Now we only need $\hat\Theta_{lkji}=\hat R_{lkji}$, which can be read off from what is multiplying $\half\hat e^k\we\hat e^l$ in $\hat\Theta_{ji}$, i.e.
\begin{align}
\hat R_{lkji} &= -2\partial_{b} e_{n(i}e{_{l)}}^n\partial^b e_{m(j}e{_{k)}}^m\nn\\
%& = -\half\lbp \partial_{b} e_{ni}e{_{l}}^n\partial^b e_{mj}e{_{k}}^m
%+\partial_{b} e_{ni}e{_{l}}^n\partial^b e_{mk}e{_{j}}^m\right.\nn\\
%& \hspace{.7cm}\left.+\partial_{b} e_{nl}e{_{i}}^n\partial^b e_{mj}e{_{k}}^m
%+\partial_{b} e_{nl}e{_{i}}^n\partial^b e_{mk}e{_{j}}^m\rbp\nn\\ 
\end{align}
where we use our heads to remember the antisymmetry in $l\leftrightarrow k$ and $j\leftrightarrow i$. 

And at last we can happily calculate the Ricci scalar as a contraction of the Riemann tensor
\begin{align}
\hat R &= \hat R{_{BA}}^{BA} = \hat R{_{ba}}^{ba} + 2\hat R{_{ja}}^{ja} + \hat R{_{ji}}^{ji}\nn\\
&= R - \half F\bb{1}{bai}F\bu{1bai} - \frac{1}{4}\lp F\bu{1a}\bd{b}\bu{i}F\bb{1}{a}\bu{b}\bd{i} - F\bu{1a}\bd{a}\bu{i}F\bb{1}{b}\bu{b}\bd{i} \rp\nn\\
& +2\bigg\{ 
-e{_{j}}^n\partial_{a}e{_n}^{(i}\partial^{|a|} e\od{m}\ou{j)}e\od{i}\ou{m}  
- D_{a}\lp\partial^{a} e_{|m|(j}e\od{k)}\ou{m}\delta^{jk}\rp\nn\\
& + \frac{1}{4} F\bb{1}{b}\bu{a}\bd{j}F\bu{1b}\bd{a}\bu{j} 
- \partial^a e_{m(i}e{_{j)}}^m\partial_{a} e\od{n}\ou{[i} e^{j]n}
\bigg\}\nn\\
&-2\partial_{b} e_{n(i'}e{_{[j)}}^n\partial^b e_{|m|(i]}e{_{j')}}^m\delta^{j[j'}\delta^{i']i}\nn\\
&= R + F\bb{1}{bai}F\bu{1bai}\lbp - \half - \frac{1}{4} + \half\rbp\nn\\
& -2e{_{j}}^n\partial_{a}e{_n}^{(i}\partial^{|a|} e\od{m}\ou{j)}e\od{i}\ou{m} - 2D_{a}\lp e\od{i}\ou{m}\partial^{a} e_{mi}\rp-0\nn\\
& - \lp \partial_{a} e_{n(j}e{_{j')}}^n\partial^a e_{m(i'}e{_{i)}}^m - \partial_{a} e_{n(j}e{_{i')}}^n\partial^a e_{m(j'}e{_{i)}}^m\rp\delta^{jj'}\delta^{i'i}\nn\\
&= R - \frac{1}{4} G_{mn}F\bb{1}{ba}\bu{m}F\bu{1ban} -\partial_{a}e{_n}^{(i}e\ou{j)n} e\od{i}\ou{m}\partial^{a} e\od{m}\od{j}\nn\\
&- 2D_{a}\lp e\od{i}\ou{m}\partial^{a} e\od{m}\ou{i}\rp
- \partial_{a} e\od{n}\ou{j}e\od{j}\ou{n}\partial^a e\od{m}\ou{i}e\od{i}\ou{m}
\eqnlab{ricci_reduced_scalar}
\end{align}

\subsection{The Ricci scalar in the Einstein frame}

%Vielbeinen 'r det som transformerar?\\
%Anv'nd inte affine connection?\\

Consider the Ricci scalar (use \eqnref{conven_theta_omegadef} to get $R\od{NM}\ou{B}\od{A}$)
\begin{align}
R(e,\omega) &= e{_B}^Ne^{AM}R\od{NM}\ou{B}\od{A} = \eta^{DB}\eta^{CA}\Theta_{DCBA} \nonumber \\
&= \eta^{DB}\eta^{CA}\lbp 2\partial_{[D}\omega_{C]BA} + 2\omega{_{[D|B|}}^E\omega_{C]EA} \rbp,
\end{align}
under a metric scaling
\begin{align}
e{_{M}}^A &= e^{-s\varphi}\tilde e{_{M}}^A,\hspace{2cm}e{_{A}}^M = e^{s\varphi}\tilde e{_{A}}^M\nn\\
g{_{MN}} &= e^{-2s\varphi}\tilde g_{MN},\hspace{1.75cm}g{^{MN}} = e^{2s\varphi}\tilde g^{MN}\nn\\
A_{(P)}^2 &= g^{M_1N_1}\dots g^{M_pN_p}A_{M_p\dots M_1}A_{N_p\dots N_1} = e^{2sp\varphi} \tilde A_{(P)}^2
\end{align}
where the indices of p-forms $A_{(p)}$ will be naturally downstairs curved and raised using the transformed metric $\tilde g^{MN}$. In particular we should be careful not moving around the indices on the derivatives too much.  
We can use \eqnref{ricci_spin_affine} to express the spin connection $\omega\od{M}\ou{A}\od{B}$ in terms of the Christoffel connection.
After some work we get R as

%The transformed affine connection becomes
%\begin{align}
%\Gamma^N_{MP} &= \half e^{2s\varphi}\tilde g^{NS}\lbp \partial_P\lp e^{-2s\varphi}\tilde g_{MS}\rp + \partial_M\lp e^{-2s\varphi}\tilde g_{PS}\rp - \partial_S\lp e^{-2s\varphi}\tilde g_{MP}\rp\rbp\nn\\
%& = \tilde\Gamma^N_{MP}+\frac{-2s}{2}\tilde g^{NS}\lbp \partial_P\varphi\tilde g_{MS} + \partial_M\varphi\tilde g_{PS} - \partial_S\varphi\tilde g_{MP}\rbp\nn\\
%& = \tilde\Gamma^N_{MP} - s\lbp \partial\bd{P}\varphi\delta\bb{N}{M} + \partial\bd{M}\varphi\delta\bb{N}{P} - \partial\od{S}\varphi\tilde g^{NS}\tilde g_{MP}\rbp
%\end{align}
%where $\tilde\Gamma^N_{MP}$ is the connection constructed using the transformed metric.
%Note that the vielbeins of the partial derivatives $\partial_M = e\od{M}\ou{A}\partial_A$ should not be transformed. 
%Thus the transformed spin connection becomes 
%\begin{align}
%\omega\od{M}\ou{A}\od{B} &= \tilde e{_N}^A\tilde e{_B}^P\lbp\tilde\Gamma^N_{MP} - s\lbp\partial\bd{P}\varphi\delta\bb{N}{M} + \partial\bd{M}\varphi\delta\bb{N}{P} - \partial\od{S}\varphi\tilde g^{NS}\tilde g_{MP}\rbp\rbp\nn\\
%& - e^{s\varphi}\tilde e{_{B}}^P\partial_M\lp e^{-s\varphi}\tilde e{_{P}}^A\rp\nn\\
%& = \tilde\omega\od{M}\ou{A}\od{B}-s\lbp\tilde e{_M}^A\tilde e{_B}^P\partial_P\varphi + \delta\bb{A}{B}\partial\bd{M}\varphi - \tilde e^{SA}\tilde e_{BM}\partial\od{S}\varphi \rbp + s\delta\bb{A}{B}\partial\bd{M}\varphi\nn\\
%& = \tilde\omega\od{M}\ou{A}\od{B}-2s\eta_{BC}\tilde e{_M}^{[A}\tilde e^{C]P}\partial_P\varphi
%\end{align}
%where $\omega$ of course is created with the transformed metric.
%\lbp \partial_{[D}\omega_{C]BA} + \omega{_{[D|B|}}^E\omega_{C]EA} \rbp

\begin{align}
%R &= e\od{[B}\ou{N}e\od{A]}\ou{M}\bigg\{ 2\partial\od{N}\lbp \tilde\omega\od{M}\ou{B}\ou{A}-2s \tilde e\od{M}\ou{B}\tilde e^{AP}\partial_P\varphi \rbp\nn\\ 
%& + 2\eta_{EF}\lbp\tilde\omega\od{N}\ou{B}\ou{F}-2s \tilde e\od{N}\ou{[B}\tilde e\ou{F]P} \partial\od{P}\varphi\rbp \lbp\tilde\omega\od{M}\ou{E}\ou{A}-2s \tilde e\od{M}\ou{[E}\tilde e\ou{A]P}\partial\od{P}\varphi\rbp\bigg\}\nn\\
%& = 2\tilde e\od{[B}\ou{N}\tilde e\od{A]}\ou{M}e^{2s\varphi}\bigg\{
% d omega
%\partial\od{N}\tilde\omega\od{M}\ou{B}\ou{A}
%-2s \partial\od{N}\tilde e\od{M}\ou{B}\tilde e^{AP}\partial_P\varphi
%-2s \tilde e\od{M}\ou{B}\partial\od{N}\tilde e^{AP}\partial_P\varphi
%-2s \tilde e\od{M}\ou{B}\tilde e^{AP}\partial\od{N}\partial_P\varphi
%\nn\\&
% omega*omega
%\tilde\omega\od{N}\ou{B}\od{E}\tilde\omega\od{M}\ou{E}\ou{A}
%-2s\tilde\omega\od{N}\ou{B}\od{E}\tilde e\od{M}\ou{[E}\tilde e\ou{A]P} \partial\od{P}\varphi
%-2s\tilde e\od{N}\ou{[B}\tilde e\ou{E]P}\partial\od{P}\varphi\tilde\omega\od{M}\od{E}\ou{A}
%+4s^2\eta_{EF}\tilde e\od{N}\ou{[B}\tilde e\ou{F]P}\partial\od{P}\varphi \tilde e\od{M}\ou{[E}\tilde e\ou{A]Q}\partial\od{Q}\varphi
%\bigg\}\nn\\
%
%
%& = e^{2s\varphi}\bigg\{ \tilde R 
% d omega
%-4s \tilde e\od{[B}\ou{N}\tilde e\od{A]}\ou{M}\partial\od{N}\tilde e\od{M}\ou{B}\tilde e^{AP}\partial_P\varphi
%-4s \tilde e\od{[B}\ou{N}\tilde e\od{A]}\ou{M}\tilde e\od{M}\ou{B}\partial\od{N}\tilde e^{AP}\partial_P\varphi
%-4s \tilde e\od{[B}\ou{N}\delta\bb{B}{A]}\tilde e^{AP}\partial\od{N}\partial_P\varphi
%\nn\\&
% omega*omega
%
%-4s\lbp
%\tilde e\bd{[B}\bu{N}\delta\bb{[E}{A]}\tilde e\bu{A]P}\partial\bd{P}\varphi\tilde\omega\bd{N}\bu{B}\bd{E}
%+\delta\bb{[B}{[B}\tilde e\bd{A]}\bu{|M|}\tilde e\bu{E]P}\partial\bd{P}\varphi\tilde\omega\bd{M}\bd{E}\bu{A}
%\rbp\nn\\&
%+8s^2\delta\bb{[B}{[B}\tilde e\bu{F]P}\partial\bd{P}\varphi \delta\bb{[E}{A]}\tilde e\bu{A]Q}\partial\bd{Q}\varphi \eta\bd{EF}
%\bigg\}\nn\\
%& = e^{2s\varphi}\bigg\{ \tilde R
%-2s^2\lp 1 - \delta\bb{B}{B}\rp\partial\bu{P}\varphi\partial\bd{P}\varphi
%-2s\lp 1 - \delta\bb{B}{B}\rp\partial\bu{P}\partial\bd{P}\varphi
%\nn\\&
%+s^2\lp \delta\bb{B}{B}-1-1+1-\delta\bb{B}{B}\delta\bb{A}{A}+\delta\bb{A}{B}\delta\bb{B}{A}-\delta\bb{A}{A}+1\rp \partial\bu{P}\varphi\partial\bd{P}\varphi
%\bigg\}\nn\\
%& = e^{2s\varphi}\bigg\{
%\tilde R
%+ 2s\lp d-1\rp\Box\varphi
%+ s^2\lp d-d^2+d-d- 2\lp 1 - d\rp\rp\lp\partial\varphi^2\rp
%\bigg\}\nn\\
&R = e^{2s\varphi}\bigg\{\tilde R + 2s\lp d-1\rp\Box\varphi + s^2(d-1)(d-2)\partial\varphi^2\bigg\}
\eqnlab{RicciE}
\end{align}
Note that by using \eqnref{conven_divergence} and that s is chosen to cancel the prefactor of $\tilde R= R_E$ in the Einstein frame, the second term disappears when integrating
\begin{equation}      
\int d^dx \sqrt{|g_E|} 2s(d-1)D_\mu D^\mu\varphi = 2s(d-1)\int d^dx \partial_\mu\lp\frac{1}{\sqrt{|g_E|}}D^\mu\varphi\rp = 0   
\end{equation}


%\begin{align}
%R\od{M}\ou{A} &= e\od{B}\ou{N}\bigg\{ 2\partial\od{[N}\lbp \tilde\omega\od{M]}\ou{B}\ou{A}-2s \tilde e\od{M]}\ou{[B}\tilde e^{A]P}\partial_P\varphi \rbp\nn\\ 
%& + 2\eta_{EF}\lbp\tilde\omega\od{[N}\ou{B}\ou{F}-2s \tilde e\od{[N}\ou{[B}\tilde e\ou{F]P} \partial\od{P}\varphi\rbp \lbp\tilde\omega\od{M]}\ou{E}\ou{A}-2s \tilde e\od{M]}\ou{[E}\tilde e\ou{A]P}\partial\od{P}\varphi\rbp\bigg\}\nn\\
%& = 2\tilde e\od{B}\ou{N}e^{2s\varphi}\bigg\{
%% d omega
%\partial\od{[N}\tilde\omega\od{M]}\ou{B}\ou{A}
%-2s \partial\od{[N}\tilde e\od{M]}\ou{[B}\tilde e^{A]P}\partial_P\varphi
%+2s \partial\od{[N}\tilde e^{[A|P|}\tilde e\od{M]}\ou{B]}\partial_P\varphi
%+2s \tilde e\od{[M}\ou{[B}\tilde e^{A]P}\partial\od{N]}\partial_P\varphi
%\nn\\&
%% omega*omega
%\tilde\omega\od{[N}\ou{B}\od{|E|}\tilde\omega\od{M]}\ou{E}\ou{A}
%-2s\tilde\omega\od{[N}\ou{B}\od{|E|}\tilde e\od{M]}\ou{[E}\tilde e\ou{A]P} \partial\od{P}\varphi
%-2s\tilde e\od{[N}\ou{[B}\tilde e\ou{E]P}\partial\od{|P|}\varphi\tilde\omega\od{M]}\od{E}\ou{A}
%+4s^2\eta_{EF}\tilde e\od{[N}\ou{[B}\tilde e\ou{F]P}\partial\od{|P|}\varphi \tilde e\od{M]}\ou{[E}\tilde e\ou{A]Q}\partial\od{Q}\varphi
%\bigg\}\nn\\
%%
%%
%& = e^{2s\varphi}\bigg\{ \tilde R 
%% d omega
%-4s \tilde e\od{B}\ou{N}\partial\od{[N}\tilde e\od{M]}\ou{[B}\tilde e^{A]P}\partial_P\varphi
%+4s \tilde e\od{B}\ou{N}\partial\od{[N}\tilde e^{[A|P|}\tilde e\od{M]}\ou{B]}\partial_P\varphi
%+4s \tilde e\od{B}\ou{N}\tilde e\od{[M}\ou{[B}\tilde e^{A]P}\partial\od{N]}\partial_P\varphi
%\nn\\&
%% omega*omega
%%
%-4s\tilde e\od{B}\ou{N}\tilde\omega\od{[N}\ou{B}\od{|E|}\tilde e\od{M]}\ou{[E}\tilde e\ou{A]P} \partial\od{P}\varphi
%-4s\tilde e\od{B}\ou{N}\tilde e\od{[N}\ou{[B}\tilde e\ou{E]P}\partial\od{|P|}\varphi\tilde\omega\od{M]}\od{E}\ou{A}
%+8s^2\tilde e\od{B}\ou{N}\eta_{EF}\tilde e\od{[N}\ou{[B}\tilde e\ou{F]P}\partial\od{|P|}\varphi \tilde e\od{M]}\ou{[E}\tilde e\ou{A]Q}\partial\od{Q}\varphi
%\bigg\}\nn\\
%& = e^{2s\varphi}\bigg\{ \tilde R 
%% d omega
%-4s \tilde e\od{B}\ou{N}\partial\od{[N}\tilde e\od{M]}\ou{[B}\tilde e^{A]P}\partial_P\varphi
%+4s \tilde e\od{B}\ou{N}\partial\od{[N}\tilde e^{[A|P|}\tilde e\od{M]}\ou{B]}\partial_P\varphi\nn\\&
%+s\big(
%\tilde e^{AP}\partial\od{M}\partial_P\varphi
%-\tilde e\od{M}\ou{A}\partial\ou{P}\partial_P\varphi
%-d\tilde e^{AP}\partial\od{M}\partial_P\varphi
%+\tilde e^{AP}\partial\od{M}\partial_P\varphi
%\big)
%\nn\\&
%% omega*omega
%%
%-4s\tilde e\od{B}\ou{N}\tilde\omega\od{[N}\ou{B}\od{|E|}\tilde e\od{M]}\ou{[E}\tilde e\ou{A]P} \partial\od{P}\varphi
%-4s\tilde e\od{B}\ou{N}\tilde e\od{[N}\ou{[B}\tilde e\ou{E]P}\partial\od{|P|}\varphi\tilde\omega\od{M]}\od{E}\ou{A}\nn\\&
%+s^2\big(
%d\tilde e\ou{AP}\partial\od{M}\varphi\partial\od{P}\varphi
%-d \tilde e\od{M}\ou{A}\partial\ou{P}\varphi\partial\od{P}\varphi
%-\tilde e\ou{AP}\partial\od{M}\varphi\partial\od{P}\varphi
%+\tilde e\od{M}\ou{A}\partial\ou{P}\varphi\partial\od{P}\varphi\nn\\&
%-\tilde e\ou{AP}\partial\od{M}\varphi\partial\od{P}\varphi
%+\tilde e\od{M}\ou{A}\partial\od{P}\varphi\partial\ou{P}\varphi
%+\tilde e\ou{AP}\partial\od{M}\varphi\partial\od{P}\varphi
%-\tilde e\ou{AP}\partial\od{M}\varphi\partial\od{P}\varphi 
%\big)
%\bigg\}\nn\\
%& = e^{2s\varphi}\bigg\{ \tilde R
%\bigg\}\nn\\
%\end{align}



%\begin{align}
%R &= 2e^{s\varphi}\partial\bu{[B}\lbp e^{s\varphi}\lbp\tilde\omega\bu{A]}\bd{BA}-2s \delta\bb{A]}{B}\partial\bd{A}\varphi \rbp\rbp\nn\\ 
%& + 2e^{2s\varphi}\lbp\tilde\omega\bu{[B}\bd{B}\bu{|E|}-2s\eta\bu{EF}\delta\bb{[B}{[B}\partial\bd{F]}\varphi\rbp \lbp\tilde\omega\bu{A]}\bd{EA}-2s\delta\bb{A]}{[E}\partial\bd{A]}\varphi\rbp\nn\\
%& = 2e^{2s\varphi}\bigg\{
%\half\tilde R
%% d omega
%+ s\partial\bu{[B}\varphi\tilde\omega\bu{A]}\bd{BA}
%- 2s^2\partial\bu{[B}\varphi\delta\bb{A]}{B}\partial\bd{A}\varphi
%- 2s\delta\bb{A}{[B}\partial\bu{B}\partial\bd{A]}\varphi
%\nn\\&
%% omega*omega
%-2s\tilde\omega\bu{[B}\bd{B}\bu{|E|}\delta\bb{A]}{[E}\partial\bd{A]}\varphi
%-2s\eta\bu{EF}\delta\bb{[B}{[B}\partial\bd{F]}\varphi\tilde\omega\bu{A]}\bd{EA}
%+4s^2\eta\bu{EF}\delta\bb{[B}{[B}\partial\bd{F]}\varphi\delta\bb{A]}{[E}\partial\bd{A]}\varphi
%\bigg\}\nn\\
%& = e^{2s\varphi}\bigg\{
%\tilde R
%+ 2s\partial\bu{[B}\varphi\tilde\omega\bu{A]}\bd{BA}
%- 4s\delta\bb{A}{[B}\partial\bu{B}\partial\bd{A]}\varphi
%\nn\\&
%+4s^2\lp 
%2\eta\bu{EF}\delta\bb{[B}{[B}\partial\bd{F]}\varphi\delta\bb{A]}{[E}\partial\bd{A]}\varphi
%- \partial\bu{[B}\varphi\delta\bb{A]}{B}\partial\bd{A}\varphi\rp
%\bigg\}\nn\\
%& = e^{2s\varphi}\bigg\{
%\tilde R
%+ 2s\partial\bu{[B}\varphi\tilde\omega\bu{A]}\bd{BA}
%- 2s\lp 1-\delta\bb{B}{B}\rp\partial\bu{A}\partial\bd{A}\varphi
%\nn\\&
%+s^2\lp \delta\bb{B}{B}-1-1+1-\delta\bb{B}{B}\delta\bb{A}{A}+\delta\bb{A}{B}\delta\bb{B}{A}-\delta\bb{A}{A}+1- 2\lp 1 - \delta\bb{B}{B}\rp\rp\partial\bu{A}\varphi\partial\bd{A}\varphi
%\bigg\}\nn\\
%& = e^{2s\varphi}\bigg\{
%\tilde R
%+ 2s\partial\bu{[B}\varphi\tilde\omega\bu{A]}\bd{BA}
%+ 2s\lp d-1\rp\partial\bu{A}\partial\bd{A}\varphi
%\nn\\&
%+s^2\lp d-d^2+ d - d -2 + 2d\rp\lp\partial\phi\rp^2
%\bigg\}\nn\\
%& = e^{2s\varphi}\bigg\{
%\tilde R
%+ 2s\partial\bu{[B}\varphi\tilde\omega\bu{A]}\bd{BA}
%+ 2s\lp d-1\rp\partial\bu{A}\partial\bd{A}\varphi
%+s^2(d-1)(d-2)\lp\partial\phi\rp^2
%\bigg\}
%\end{align}


%\begin{align}
%R &= 2\eta^{DB}\eta^{CA}\big[e^{s\varphi}\partial_{[D}\lbp e^{s\varphi}\lbp\tilde\omega_{C]BA}-s\lp \eta_{C]B}\partial_A\varphi - \eta_{C]A}\partial_B\varphi \rp\rbp\rbp\big]\nn\\ 
%& + 2e^{2s\varphi}\lbp\tilde\omega{_{[D}}^{DE}-s\lp \delta_{[D}^D\partial^E\varphi - \partial^{D}\varphi\delta_{[D}^E \rp\rbp \lbp\tilde\omega{_{C]E}}^C-s\lp \eta_{C]E}\partial^C\varphi - \delta_{C]}^C\partial_E\varphi \rp\rbp\nn\\
%& = 2e^{2s\varphi}\bigg\{
%\half\tilde R
%% d omega
%+ s\partial_{[D}\varphi\tilde\omega{_{C]}}^{DC}
%- s^2\partial_{[D}\varphi\delta_{C]}^D\partial^C\varphi
%+ s^2\partial_{[D}\varphi\delta_{C]}^C\partial^D\varphi
%\nn\\&
%-s\partial_{[D}\partial^C\varphi\delta_{C]}^D
%+s\partial_{[D}\partial^D\varphi\delta_{C]}^C
%% omega*omega
%-s\tilde\omega{_{[D}}^{DE}\eta_{C]E}\partial^C\varphi
%+s\tilde\omega{_{[D}}^{DE}\delta_{C]}^C\partial_E\varphi
%\nn\\&
%-s\delta_{[D}^D\partial^E\varphi\tilde\omega{_{C]E}}^C
%+s^2\delta_{[D}^D\partial^E\varphi\eta_{C]E}\partial^C\varphi
%-s^2\delta_{[D}^D\partial^E\varphi\delta_{C]}^C\partial_E\varphi
%\nn\\&
%+s\partial^{D}\varphi\delta_{[D}^E\tilde\omega{_{C]E}}^C
%-s^2\partial^{D}\varphi\delta_{[D}^E\eta_{C]E}\partial^C\varphi
%+s^2\partial^{D}\varphi\delta_{[D}^E\delta_{C]}^C\partial_E\varphi
%\bigg\}\nn\\
%%
%& = e^{2s\varphi}\bigg\{
%\tilde R
%+ 2s\lp
%% d omega
%\tilde\omega{_{[C}}^{CD}\partial_{D]}\varphi
%%+\partial_{[D}\partial^D\varphi\delta_{C]}^C
%+2\partial_{[D}\partial^D\varphi\delta_{C]}^C
%% omega*omega
%-\tilde\omega{_{[C}}^{CE}\eta_{D]E}\partial^D\varphi
%\right.\nn\\&\left.
%+\tilde\omega{_{[C}}^{CE}\delta_{D]}^D\partial_E\varphi
%-\tilde\omega{_{[C}}^{CE}\delta_{D]}^D\partial_E\varphi
%+\tilde\omega{_{[C}}^{CD}\partial_{D]}\varphi\rp\nn\\
%&+2s^2\lp
%% d omega
%+\partial_{[D}\varphi\partial^D\varphi\delta_{C]}^C
%+\partial_{[D}\varphi\partial^D\varphi\delta_{C]}^C
%% omega*omega
%-\partial_{[D}\varphi\partial^D\varphi\delta_{C]}^C
%-\delta_{[D}^D\delta_{C]}^C(\partial\varphi)^2
%-0
%+\partial_{[D}\varphi\partial^{D}\varphi\delta_{C]}^C\rp
%\bigg\}\nn\\
%& = e^{2s\varphi}\lbp 
%\tilde R
%+ 2s\lp \delta\bb{C}{C}-1\rp\Box\varphi
%-s^2\lp 2(\delta\bb{C}{C}-1) - (\delta\bb{D}{D}\delta\bb{C}{C}-\delta\bb{D}{C}\delta\bb{C}{D}) \rp(\partial\varphi)^2
%\rbp\nn\\
%& = e^{2s\varphi}\lbp 
%\tilde R
%+ 2s (d-1)\Box\varphi
%+s^2 (2d-2-d^2+d)(\partial\varphi)^2
%\rbp\nn\\
%& = e^{2s\varphi}\lbp \tilde R+2s(d-1)\Box\varphi -s^2(d-1)(d-2)(\partial \varphi)^2 \rbp
%\eqnlab{RicciE}
%\end{align}
%%
%Note that by using \Eqnref{conven_divergence} and that s is chosen to cancel the prefactor of $\tilde R= R_E$ in the Einstein frame, the second term disappears when integrating
%\begin{equation}      
%\int d^dx \sqrt{|g_E|} 2s(d-1)D_\mu D^\mu\varphi = 2s(d-1)\int d^dx \partial_\mu\lp\frac{1}{\sqrt{|g_E|}}D^\mu\varphi\rp = 0   
%\end{equation}





%\chapter{128 bit integer arithmetics}
\chlab{int128}
To solve the duality equations in chapter \chref{csolutions} we need to expand the equations to very high orders, implying very large coefficients due to the binomial factors in the expansion of $(\det G)^{-1/2}$. 
The current PC:s use a 32 bit word length ($\sim$ 9 decimal digits) and Delphi, the programming language we use, has support for 64 bit integers ($\sim$ 19 decimal digits) in its system.
To handle sufficiently large expansions with numbers larger than this we need to include 128 bit integer representations ($\sim$ 39 decimal digits) and some basic arithmetic operations on these. 
The easiest way to include such large integers would be to borrow the work of someone else, but to our astonishment we can't find such a work on the internet. Thus we have to write it ourselves and include it in this thesis so future solvers of grandiose duality equations can have the chance to find it.       

To implement 128 bit integers is one of the cases where it is actually motivated to write the code directly using x86 assembler.
To begin with we need the code to be fast since we will use the 128 bit arithmetic operations a lot.
An even more important motivation is that we need operations not accessible by ordinary code, e.g. shift over 32-bit boundaries, bit string scans, addition with carry bit and subtraction with borrow bit. Of course we could do this in a complicated way using standard operations in Delphi, but that would give unnecessarily big performance flaws and would probably be even harder to implement than an assembler analogue.    
To get started it is a good idea to check how the compiler translates the 64 bit integer operations to machine code and try to generalize to 128 bits (The Intel Architecture Manual\cite{intel} will be in handy).

\section{General assembler guidelines}
One problem when writing code in assembler is that there is always a big number of choices like which instructions to use and how to order them.
Each problem can be solved in so many ways and it is hard to know which one is the fastest.
We will try to follow the general guidelines in \cite{intelopt} to write fast integer code. The main points are
\begin{itemize}
\item Minimize branch count, i.e. avoid conditional jumps as far as possible. For the branches we cannot avoid, we should (if there is a choice) branch for the least probable condition result and fall through for the most probable result in an if statement and put the condition last in loops. This is because the branch predictor assumes, if there is no branch history for the current instruction, that a fall through will occur if the destination lies at a higher memory address and that a branch jump will occur when the destination is a lower memory address. 
\item Minimize memory accesses count, use the registers to keep variables as long as they are needed. Instructions involving only registers are in general faster executed than instructions involving memory accesses.
\item Pair pairable instructions so the processor can run them simultaneously in the two integer pipelines and do not pair other instructions, since that would be unnecessary work (Use data in \cite{intelopt} to see which instructions are pairable and how many $\mu$ops each instruction needs). Further do not use registers that has been changed the previous instruction, doing that prevents pairing. 
\item For fast cache memory access we should align our 32 byte coefficients to memory addresses divisible by 32 (the cache is read in blocks of 32 bytes), but that would increase either the memory size or the complexity of the data storage, so we store over cache boundaries with a slight (the data is still 4 byte aligned) performance loss.
\end{itemize}
An additional point should be to use smart algorithms, which of course can have greater effects than any possible optimizations using the points stated above.

\section{The Int128 type}
To represent the 128 bit integers we use the type Int128 which is simply defined as a structure of four succeeding 32 bit integers. Furthermore we define the type pInt128 as a pointer to an Int128. 
\begin{verbatim}
type
  Int128 = record
    data1,data2,data3,data4:Int32;
   end;
  pInt128 = ^Int128;
\end{verbatim}
We use the Little endian storage format, meaning that the least significant Int32 (= data1) is at the lowest memory address, the bytes within each Int32 are by machine default also Little endian\footnote{The bits within each byte is ordered least significant bit at highest address, but we will never see this when using machine instructions}. 
The integers should be signed so we use two's complement to store the integers in memory where the most significant bit in data4 is used to indicate sign, i.e. the positive numbers $0,1,\dots,2^{127}-1$ are represented by (in hexadecimal code) $\mbox{000}\dots\mbox{00},\mbox{000}\dots\mbox{01},\dots,\mbox{7FF}\dots\mbox{FF}$ and the negative numbers $-1,-2,\dots,-2^{127}$ becomes $\mbox{FFF}\dots\mbox{FF},\mbox{FFF}\dots\mbox{FE},\dots,\mbox{800}\dots\mbox{00}$. Note that this is not the order in which the integers are actually stored in memory.
We define the following 128 bit functions to handle all operations on Int128 needed: 
{\tiny
\begin{verbatim}
function IsZero128(x:pInt128):boolean;
function IsOne128(x:pInt128):boolean;
function IsNeg128(x:pInt128):boolean;
function IsPos128(x:pInt128):boolean;
function IsEqual128(x,y:pInt128):boolean;
procedure Neg128(x:pInt128);
procedure Abs128(x:pInt128);
procedure Add_128_128(x,y:pInt128);
procedure Add_128_128_128(res,x,y:pInt128);
procedure Mul_128_128_128(res,x,y:pInt128);
procedure ReduceFraction128(frac:pointer);
\end{verbatim}
}
\noindent
plus some conversion routines between Int128, Int32 and (text)strings.
The first five conditional functions are easy to implement and we illustrate them all with one example    
{\tiny
\begin{verbatim}
function IsPos128(x:pInt128):boolean;
// Result:=x > 0
// Params: eax = x; al = Result;
asm
 cmp [eax+12], 0     // Compare x.data4 - 0 to 0
 jnz @returnSign     // Return sign if x.data4 <> 0    
 cmp [eax+8],0       // Compare x.data3 - 0 to 0
 ja @returnTrue      // Return true if unsigned x.data3 > 0
 cmp [eax+4],0       // Compare x.data2 - 0 to 0
 ja @returnTrue      // Return true if unsigned x.data3 > 0
 cmp [eax],0         // Compare x.data1 - 0 to 0
 ja @returnTrue      // Return true if unsigned x.data3 > 0
                     // Return false on fallthrough
 mov al, 0           // Set false result
 ret                 // Return from function
@returnSign:
 setg al             // Set result = signed positive
 ret                 // Return from function
@returnTrue:
 mov al, 1           // Set true result
end;
\end{verbatim}
}
\noindent
which returns 1 in register al if the argument x is larger than 0 and returns 0 otherwise.
The negations in Neg128 and Abs128 and the additions in Add\_128\_128 (Add with result in first parameter) and Add\_128\_128\_128 are almost as easy. We illustrate them all with one example 
{\tiny
\begin{verbatim}
procedure Neg128(x:pInt128);
// var x:=-x;
// Params: eax = @x;
asm
 xor ecx, ecx        // ecx:=0
 neg [eax]           // x.data1:=- x.data1, sets flags
 sbb ecx, [eax+4]    // ecx:=0 - x.data2 - Borrow
 mov edx, 0          // edx:=0, preserves flags unlike xor
 mov [eax+4], ecx    // Write back the negation of x.data2
 sbb edx, [eax+8]    // edx:=0 - x.data3 - Borrow
 mov ecx, 0          // ecx:=0
 mov [eax+8], edx    // Write back the negation of x.data3
 sbb ecx, [eax+12]   // ecx:=0 - x.data4 - Borrow
{$IFOPT Q+}          // If overflow check is on
 jno @noOverflow
 call IntNegOverflow // Cannot negate $800000...000 =>error
@noOverflow:
{$ENDIF}
 mov [eax+12], ecx   // Write back the negation of x.data4
end;
\end{verbatim}
}
\noindent
that negates the value of x. The additions are done similarly with one "add" and 3 "adc" (add with carry) instructions.

Now there are only 2 functions left, but there is a reason they stand last. These functions will be more complex and we start with the multiplication function Mul\_128\_128\_128.
For the multiplication we finally, after 16 years of impatient wait, find an application for the elementary school second grade multiplication algorithm, which in the $2^{32}$ base between the 128 bit integers $x=x_4:x_3:x_2:x_1$ and $y=y_4:y_3:y_2:y_1$ becomes\\\\
\setlength{\unitlength}{1.0mm}
\begin{equation}
\begin{picture}(40,35)(0,0)% (size)(offset)
%\path(0,0)(40,0)(40,35)(0,35)(0,0) % Frame (remove)
\put(0,0){
\put(5,2.5){\makebox(0,0){$r_{7}$}}
\put(10,2.5){\makebox(0,0){$r_{6}$}}
\put(15,2.5){\makebox(0,0){$r_{5}$}}
\put(20,2.5){\makebox(0,0){$r_{4}$}}
\put(25,2.5){\makebox(0,0){$r_{3}$}}
\put(30,2.5){\makebox(0,0){$r_{2}$}}
\put(35,2.5){\makebox(0,0){$r_{1}$}}

\put(-2.5,5){\line(1,0){40}}

\put(0,7.5){\makebox(0,0){$+$}}
\put(5,7.5){\makebox(0,0){$z_{44}$}}
\put(10,7.5){\makebox(0,0){$z_{34}$}}
\put(15,7.5){\makebox(0,0){$z_{24}$}}
\put(20,7.5){\makebox(0,0){$z_{14}$}}

\put(10,12.5){\makebox(0,0){$z_{43}$}}
\put(15,12.5){\makebox(0,0){$z_{33}$}}
\put(20,12.5){\makebox(0,0){$z_{23}$}}
\put(25,12.5){\makebox(0,0){$z_{13}$}}

\put(15,17.5){\makebox(0,0){$z_{42}$}}
\put(20,17.5){\makebox(0,0){$z_{32}$}}
\put(25,17.5){\makebox(0,0){$z_{22}$}}
\put(30,17.5){\makebox(0,0){$z_{12}$}}

\put(20,22.5){\makebox(0,0){$z_{41}$}}
\put(25,22.5){\makebox(0,0){$z_{31}$}}
\put(30,22.5){\makebox(0,0){$z_{21}$}}
\put(35,22.5){\makebox(0,0){$z_{11}$}}

\put(5,27.5){\makebox(0,0){$c_{6}$}}
\put(10,27.5){\makebox(0,0){$c_{5}$}}
\put(15,27.5){\makebox(0,0){$c_{4}$}}
\put(20,27.5){\makebox(0,0){$c_{3}$}}
\put(25,27.5){\makebox(0,0){$c_{2}$}}
\put(30,27.5){\makebox(0,0){$c_{1}$}}

\put(2.5,30){\line(1,0){35}}

\put(15,32.5){\makebox(0,0){$\cdot$}}
\put(20,32.5){\makebox(0,0){$y_{4}$}}
\put(25,32.5){\makebox(0,0){$y_{3}$}}
\put(30,32.5){\makebox(0,0){$y_{2}$}}
\put(35,32.5){\makebox(0,0){$y_{1}$}}

\put(20,37.5){\makebox(0,0){$x_{4}$}}
\put(25,37.5){\makebox(0,0){$x_{3}$}}
\put(30,37.5){\makebox(0,0){$x_{2}$}}
\put(35,37.5){\makebox(0,0){$x_{1}$}}

\put(2,6){\dashbox{1}(15,23.5){}}


%\path(2.5,0)(17.5,0)(17.5,24)(2.5,24)(2.5,0) % Frame (remove)

}
\end{picture}
\eqnlab{int128_mulalg}
\end{equation}
where $z_{ij}=x_iy_j$, $c_i$ the carry from the addition on the previous column and $r_i$ is the $i$:th result component in the $2^{32}$ base given by addition of the column over it.
For the product to be a valid Int128 we must have $r_7=r_6=r_5=0$, i.e. all symbols in the dashed box must be zeroes (or hexadecimal FFFFFFFF:s, indicating the sign extension of a negative number).  
Furthermore the non sign information of the product (bit 0\dots 126) isn't allowed to change the value of the 127:th (sign) bit.
Checking this product for overflow (calculating and checking all the symbols in the dashed box plus checks for overflow to the 127:th bit) would almost take as long time as calculating the valid symbols. Therefore we, like Delphi in the 64 bit case\footnote{It would have saved us from much trouble if they had actually told us in the documentation rather than in the source code that the overflow check on most 64 bit integer operations wasn't supported yet.}, ignore the overflow bits, but unlike Delphi we at least make a sign consistency test, checking if the result has the correct sign knowing the operands signs. This should detect half the overflows of multiplications between completely random Int128 operands. 
In assembler the algorithm \eqnref{int128_mulalg} becomes 
%\makebox[\textwidth]{\hrulefill}
\begin{multicols}{2}
{\tiny
\begin{verbatim}
procedure MulPro_128_128_128(res,x,y:pInt128);
// dst:=x * y;
// Params: eax = @dst; edx = @x; ecx = @y;
asm
// Set up the stack frame (x and y might change if res=x|y)
 push esi
 push ebx
 push [ecx+12]
 push [ecx+8]
 push [ecx+4]
 push [ecx]
 push [edx+12]
 push [edx+8]
 push [edx+4]
 push [edx]

// y = (y4:y3,y2,y1) = [ESP+28]:[ESP+24]:[ESP+20]:[ESP+16]
// x = (x4:x3:x2:x1) = [ESP+12]:[ESP+8]:[ESP+4]:[ESP]

 mov esi, eax        // esi:=@res
 mov eax, [esp+16]   // eax:=y1
 mul [esp]           // edx:eax:=y1*x1
 mov ebx, edx        // ebx:=(y1*x1).hi,ebx now stores res2
 mov [esi], eax      // @res^.data1:=(y1*x1).lo

 mov eax, [esp+16]   // eax:=y1
 xor ecx, ecx        // ecx:=0
 mul [esp+4]         // edx:eax:=y1*x2
 add ebx, eax        // ebx:=(y1*x1).hi+(y1*x2).lo
 mov eax, [esp+20]   // eax:=y2
 adc ecx, edx        // ecx:=(y1*x2).hi+carry

 mul [esp]           // edx:eax:=y2*x1
 add ebx, eax        // ebx:=(y1*x1).hi+(y1*x2).lo
                     //   +(y2*x1).lo
 adc ecx, 0          // ecx:=(y1*x2).hi+carry, o.f. impos.
 mov [esi+4], ebx    // @res^.data2:=(y1*x1).hi+(y1*x2).lo
                     //   +(y2*x1).lo
 xor ebx, ebx        // ebx:=0, ebx is now storage for res4
 mov eax, [esp+16]   // eax:=y1
 add ecx, edx        // ecx:=(y1*x2).hi+(x1*y2).hi
 adc ebx, 0          // ebx:=0+carry

 mul [esp+8]         // edx:eax:=y1*x3
 add ecx, eax        // ecx:=(y1*x2).hi+(x1*y2).hi
                     //   +(y1*x3).lo
 mov eax, [esp+20]   // eax:=y2
 adc ebx, edx        // ebx:=(y1*x3).hi+carry

 mul [esp+4]         // edx:eax:=y2*x2
 add ecx, eax        // ecx:=(y1*x2).hi+(x1*y2).hi
                     //   +(y1*x3).lo+(y2*x2).lo
 mov eax, [esp+24]   // eax:=y3
 adc ebx, edx        // ebx:=(y1*x3).hi+(y2*x2).hi+carry
                     //   (Ignore overflow)
 mul [esp]           // edx:eax:=y3*x1
 add ecx, eax        // ecx:=(y1*x2).hi+(x1*y2).hi
                     //   +(y1*x3).lo+(y2*x2).lo+(y3*x1).lo
 adc ebx, edx        // ecx:=(y1*x3).hi+(y2*x2).hi
                     //   +(y3*x1).hi+Carry (Ign. overflow)
 mov eax, [esp+12]   // eax:=x4
 mov [esi+8], ecx    // @res^.data3:=(y1*x2).hi+(x1*y2).hi
                     //   +(y1*x3).lo+(y2*x2).lo+(y3*x1).lo

{$IFOPT Q+}          // If overflow check is on
 mov ecx, eax        // ecx:=x4, check sign using ecx
{$ENDIF}
 mul [esp+16]        // edx:eax:=x4*y1
 add ebx, eax        // ebx:=(y1*x3).hi+(y2*x2).hi
                     //   +(y3*x1).hi+(x4*y1).lo (Ign o.f.)
 mov eax, [esp+20]   // eax:=y2
 mul [esp+8]         // edx:eax:=y2*x3
 add ebx, eax        // ebx:=(y1*x3).hi+(y2*x2).hi
                     //   +(y3*x1).hi+(x4*y1).lo+(y2*x3).lo
                     // (Ignore overflow)
 mov eax, [esp+24]   // eax:=y3
 mul [esp+4]         // edx:eax:=y3*x2
 add ebx, eax        // ebx:=(y1*x3).hi+(y2*x2).hi
                     //   +(y3*x1).hi+(x4*y1).lo+(y2*x3).lo
                     //   +(y3*x2).lo (Ignore overflow)
 mov eax, [esp+28]   // eax:=y4
{$IFOPT Q+}          // If overflow check is on
 xor ecx, eax        // ecx:=x4 xor y4
{$ENDIF}
 mul [esp]           // edx:eax:=y4*x1
 add ebx, eax        // ebx:=(y1*x3).hi+(y2*x2).hi
                     //   +(y3*x1).hi+(x4*y1).lo+(y2*x3).lo
                     //   +(y3*x2).lo+(y4*x1).lo (Ign o.f.)
 mov [esi+12], ebx   // @res^.data4:=(y1*x3).hi+(y2*x2).hi
                     //   +(y3*x1).hi+(x4*y1).lo+(y2*x3).lo
                     //   +(y3*x2).lo+(y4*x1).lo (Ign o.f.)
{$IFOPT Q+}          // If overflow check is on
 mov eax, [esi]      // eax:=res1
 test ebx, ebx       // Check whether ebx=res4 is nonzero
 jnz @resNonZero
 mov edx, [esi+4]    // edx:=res2
 test eax, eax       // Check whether eax=res1 is nonzero
 jnz @resNonZero
 mov eax, [esi+8]    // eax:=res3
 test edx, edx       // Check whether edx=res2 is nonzero
 jnz @resNonZero
 test eax, eax       // Check whether eax is zero
 jz @maybeNoOverflow
@resNonZero:         // res is nonzero and the sign bit
                     //   is sgn(x) xor sqn(y)
 xor ecx, ebx        // esi:=res4 xor x4 xor y4
 jns @maybeNoOverflow // if and odd number of minuses in x,
                     //   y,res an error must have occured
 call IntMulSignOverflow
@maybeNoOverflow:
{$ENDIF}
 add esp, 32         // Reset the stack pointer
 pop ebx
 pop esi
end;
\end{verbatim}
}
\end{multicols}

\subsection{Quotient reduction}
If you thought the multiplication was tedious you should probably rip out the remaining pages, the division will not be as easy.
Since we chose to represent the coefficients as quotients between two 128 bit integers we can implement divisions as multiplication with the denominator. 
To avoid overflow we need to reduce the quotient coefficients by their greatest common divisor (gcd), i.e. if the coefficient is $x/y$ we want to find the largest factor $z=\gcd(x,y)$, shared by both x and y, so $x\rightarrow x/z$ and $y\rightarrow y/z$ becomes the reduced quotient.

We base our reduction algorithm on an GCD algorithm made up by J. Stein in 1967\cite{knuth}, which is based on the following facts
\begin{itemize}
\item If $x$ and $y$ are even, then $\gcd(x,y)=2\cdot\gcd(x/2,y/2)$, meaning we can start by shifting x and shifting y N bits to the right, i.e. dividing by $2^N$, where N is the least significant nonzero bit in either x or y (remember that the bits are numbered 0,1,2,\dots).
\item If $x$ even and $y$ odd, then because 2 is not common to an even and an odd number $\gcd(x,y)=\gcd(x/2,y)$, meaning we can reduce the gcd from $\gcd(x,y)=\gcd(x',y)$, where M is the least significant nonzero bit in x and $x'=x/2^M$.    
\item If $x$ and $y$ are odd, then $\gcd(x,y)=\gcd(x',y')$, where $x'=\max(x,y)-\min(x,y)$ and $y'=\min(x,y)$ (this is because any number that is a factor of both $x$ and $y$ must also be a factor of $x-y$ and vice versa) so we can come back to the previous step with $x'$ even and $y'$ odd. 
\item If $x = y$, then $\gcd(x,y) = x = y$. 
\end{itemize}
Thus the algorithm is to first divide $x$ and $y$ by $2^N$, using fact 2 and 3 successively until the transformed variables becomes equal, $x'=y'$. Thus $z=\gcd(x,y)=2^N\gcd(x',y')=2^Nx'$ should be divided from both x and y. The factor $2^N$ can easily be shifted away already in the gcd-algorithm.   

\subsubsection{Division without remainder}
Name the bits of the 128 bit integers $x$, $y$ and $q$ as 
\begin{align}
x&=x_{127}:x_{126}:\dots:x_1:x_0\nn\\
y&=y_{127}:y_{126}:\dots:y_1:y_0\nn\\
q&=q_{127}:q_{126}:\dots:q_1:q_0
\end{align}
To perform a division $q=x/y$, where we know that y is a factor in x, i.e. the remainder is zero, we use
\begin{align}
0 = qy - x &= \lp q_{127}2^{127} + \dots + q_{1}2^{1} + q_{0}2^{0}\rp \lp y_{127}2^{127} + \dots + y_{1}2^{1} + y_{0}2^{0}\rp\nn\\
& - \lp x_{127}2^{127} + \dots + x_{1}2^{1} + x_{0}2^{0}\rp 
\eqnlab{int128_zerorem}
\end{align}
and our mission is to find $q$.
We know $x$ and $y$ so we can scan them from the most significant bit to find the first non zero entries $x_i$ and $y_j$, where $i\ge j$ at all times for a no remainder division.
For the relation to hold ($x$ can only cancel cross terms up to order $i$) we must then have
\begin{align}
q_{k} = \left\{\begin{array}{ll} 
0 , & k > i-j'\cr
1 , & k = i-j'
\end{array}\right. 
\end{align}
where $j'\ge j$ depends on $x$ and $y$. In the case $j'=j$ it is trivial to see that the first relation should hold and that the second relation can fail from carry bits in lower order multiplications between $y$ and q and thus we introduce $j'$ which might me greater than $j$. 
The relation \eqnref{int128_zerorem} is
\begin{align}
0 &= \lp 2^{i-j'} + q_{i-j'-1}2^{i-j'-1} + \dots \rp y - \lp 2^{i} + x_{i-1}2^{i-1} + \dots \rp\nn\\ 
& = \lp q_{i-j'-1}2^{i-j'-1} + \dots \rp y - \lp 2^{i} + x_{i-1}2^{i-1} + \dots \rp + 2^{i-j'}y\nn\\ 
& = \lp q_{i-j'-1}2^{i-j'-1} + \dots \rp y - \lp x'_{i-1}2^{i-1} + \dots + x'_02^0 \rp 
\end{align}
where we have used that the first term in $2^{i-j'}y$ cancels $2^i$ in x and
\begin{align}
x'_{i-i'}=\left\{\begin{array}{ll}
x_{i-i'} - y_{j'-i'}, & j'-i' \ge 0\cr
x_{i-i'} , & j'-i' < 0\cr
\end{array}\right.
\end{align}
where $i'=1,2,\dots,i$.
We are now back in the same situation as in \eqnref{int128_zerorem} but with $x'$ with most significant bit of order less than $i$. Repeating this procedure until nothing remains completely determines all coefficients $q_k$ and we are done.
The only problem left is thus to find a method to decide $j'$ given $j$, $x$ and $y$. This can be done by checking bits after the most significant bit in $x$ and $y$, i.e. let $\Delta_i$ be the number of nonzero bits after $x_i$ and $\Delta_j$ the number after $y_j$. If $\Delta_i=\Delta_j$ we ignore the zero bit and continue the bit scan until $\Delta_i\ne\Delta_j$. 
We have
\begin{align}
j'=\left\{\begin{array}{ll}
j, & \mbox{if $\Delta_i>\Delta_j$}\cr
j+1, & \mbox{if $\Delta_i<\Delta_j$}\cr
\end{array}\right.
\end{align}
Since there are only two options on $j'$ we can comfortably shift $y$ left $i-j$ steps and check if $2^{i-j}y>x$ and if so we simply shift $y$ right one step and set bit $i-j-1$ in $q$. 
In assembler the quotient reduction algorithm becomes 
\begin{multicols}{2}
{\tiny
\begin{verbatim}
procedure ReduceFraction128(frac:pointer);
// frac is a pointer to a 256 bit coefficient
//     fracUp:fracDown
// frac^
// fracDown is assured to be positive if fracUp is nonzero
// ReduceFraction128 divides fracUp and FracDown with their
//   greatest common divisor
procedure helperShr128;
// Helper function, shifts a 128 bit integer to the right
// Nonconventional parameter pass:
//   edi:esi:edx:eax = 128 bit integer to shift
//   cl = number of bits to shift
asm
// Note: Comparing cl to an 8 bit imm. value gives smaller
//   code than comparing ecx to a 32 bit immediate value.
// shrd doesnt pair
 cmp cl, 32          // Compare cl - 32 to 0
 jb  @shiftBelow32   // Do a word shift if cl < 32
 cmp cl, 64          // Compare cl - 64 to 0
 jb  @shiftBelow64   // Do a double word shift if cl < 64
 cmp cl, 96          // Compare cl - 96 to 0
 jb  @shiftBelow96   // Do a triple word shift if cl < 96
 cmp cl, 128         // Compare cl - 128 to 0
 jb  @shiftBelow128  // Do a quad word shift if cl < 128
                     // t > 128 => val = 0
                     // Should never reach here
 ud2                 // Undefined instr. => Raises an error
 jmp @afterShift
@shiftBelow128:
 mov eax, edi        // eax:=data4
 shr eax, cl         // eax:=data4 shifted right
                     //   96 + cl mod 32 bits (no sign)
 xor edx, edx        // edx:=0
 xor esi, esi        // esi:=0
 xor edi, edi        // edi:=0
 jmp @afterShift
@shiftBelow96:
 mov eax, esi        // eax:=data3
 mov edx, edi        // edx:=data4
 shrd eax, edx, cl   // eax:=data3 shifted right
                     //   64 + cl mod 32 bits into data1
 shr edx, cl         // edx:=data4 shifted right
                     //   64 + cl mod 32 bits (no sign)
 xor esi, esi        // esi:=0
 xor edi, edi        // edi:=0
 jmp @afterShift
@shiftBelow64:
 mov eax, edx        // eax:=data2
 mov edx, esi        // edx:=data3
 mov esi, edi        // esi:=data4
 shrd eax, edx, cl   // eax:=data3 shifted right
                     //   32 + cl mod 32 bits into data1
 shrd edx, esi, cl   // edx:=data4 shifted right
                     //   32 + cl mod 32 bits into data2
 shr esi, cl         // esi:=data4 shifted right
                     //   32 + cl mod 32 bits (no sign)
 xor edi, edi        // edi:=0
 jmp @afterShift
@shiftBelow32:       // Right shift less than 32 bits
 shrd eax, edx, cl   // eax:=data2 s.r. cl bits into data1
 shrd edx, esi, cl   // edx:=data3 s.r. cl bits into data2
 shrd esi, edi, cl   // esi:=data4 s.r. cl bits into data3
 shr edi, cl         // edi:=data4 s.r. cl bits (no sign)
@afterShift:         // Now edi:esi:edx:eax = a
end;
procedure helperShl128;
// Helper function, shifts a 128 bit integer to the left
// Works like helperShr128;
asm
 cmp cl, 32          // Compare cl - 32 to 0
 jb  @shiftBelow32   // Do a word shift if cl < 32
 cmp cl, 64          // Compare cl - 64 to 0
 jb  @shiftBelow64   // Do a double word shift if cl < 64
 cmp cl, 96          // Compare cl - 96 to 0
 jb  @shiftBelow96   // Do a triple word shift if cl < 96
 cmp cl, 128         // Compare cl - 128 to 0
 jb  @shiftBelow128  // Do a quad word shift if cl < 128
                     // t > 128 => val = 0
 jmp @afterShift
@shiftBelow128:
 mov edi, eax        // edi:=data1
 shr edi, cl         // edi:=data1 shifted left
                     //   96 + cl mod 32 bits
 xor esi, esi        // esi:=0
 xor edx, edx        // edx:=0
 xor eax, eax        // eax:=0
 jmp @afterShift
@shiftBelow96:
 mov edi, edx        // edi:=data2
 mov esi, eax        // esi:=data1
 shld edi, esi, cl   // edi:=data1 shifted left
                     //   64 + cl mod 32 bits into data4
 shl esi, cl         // esi:=data1 shifted left
                     //   64 + cl mod 32 bits
 xor edx, edx        // edx:=0
 xor eax, eax        // eax:=0
 jmp @afterShift
@shiftBelow64:
 mov edi, esi        // edi:=data3
 mov esi, edx        // esi:=data2
 mov edx, eax        // edx:=data1
 shld edi, esi, cl   // edi:=data2 shifted left
                     //   32 + cl mod 32 bits into data3
 shld esi, edx, cl   // esi:=data1 shifted left
                     //   32 + cl mod 32 bits into data2
 shl edx, cl         // edx:=data1 shifted left
                     //   32 + cl mod 32 bits
 xor eax, eax        // eax:=0
 jmp @afterShift
@shiftBelow32:       // Shift less than 32 bits to the left
 shld edi, esi, cl   // edi:=data3 s.l. cl bits into data4
 shld esi, edx, cl   // esi:=data2 s.l. cl bits into data3
 shld edx, eax, cl   // edx:=data1 s.l. cl bits into data2
 shl eax, cl         // edi:=data1 s.l. cl bits
@afterShift:         // Now edi:esi:edx:eax = a
end;
procedure helperDivZeroRem_128_128(x,y:pInt128);
// Helper function, divides two 128 bit integers, no rem.
// x:=x / y, no remainder, x > 0, y > 0 assured
// Params: eax = @x; edx = @y;
asm
// Check if y is less than 32 bits and push some regs
 mov ecx, [edx+12]   // ecx:=y.data4
 push ebx
 or ecx, [edx+4]     // ecx:=y.data4 or y.data2
 mov ebx, [edx+8]    // ebx:=y.data3
 or ebx, ecx         // ebx:=y.data4 or y.data3 or y.data2
 jz @denomLength32   // Jump if all y.data2-4 are 0
// Perform a full 128 bit division
 push esi
 push edi
 push ebp
 push eax
 mov ebp, edx        // ebp pointer to y
 mov edi, [eax+12]   // Push x to stack
 mov esi, [eax+8]
 push edi
 push esi
 mov edi, [eax+4]
 mov esi, [eax]
 mov [eax], 0        // Store 0 as result
 mov [eax+4], 0
 mov [eax+8], 0
 mov [eax+12], 0
 push edi
 push esi
// Find most significant bit in y and store its pos in ebx
 mov ebx, 96
 bsr ecx, [ebp+12]   // Get most significant bit in y.data4
 jnz @foundMostSignY
 bsr ecx, [ebp+8]    // Get most significant bit in y.data3
 mov ebx, 64
 jnz @foundMostSignY
 bsr ecx, [ebp+4]    // Get most significant bit in y.data2
 mov ebx, 32
 jnz @foundMostSignY
 bsr ecx, [ebp]      // Get most significant bit in y.data1
 mov ebx, 0
 jnz @foundMostSignY // Jump if bit was found
                     // Should never reach here (y = 0)
 ud2                 // Undef. instruct. => Raises an error
@foundMostSignY:
 add ebx, ecx        // ebx:=Pos of first bit = j
 push ebx            // Put ebx=j on bottom of stack
// Current mem layout:
//   regs: edi:esi:edx:eax = y, ebp = pointer to mem y
//   stack: [esp]=j, [esp+4]=x, [esp+20]=pointer to res,
//     [esp+24]=old ebp,edi,esi
// Repeat x':=x-2^(i-j)y until x'=0
@beforeRepeat1:
// Find most significant bit of x and store its pos in ecx
 mov ecx, 96
 bsr ebx, [esp+16]   // Get most significant bit in x.data4
 jnz @foundMostSignX
 bsr ebx, [esp+12]   // Get most significant bit in x.data3
 mov ecx, 64
 jnz @foundMostSignX
 bsr ebx, [esp+8]    // Get most significant bit in x.data2
 mov ecx, 32
 jnz @foundMostSignX
 bsr ebx, [esp+4]    // Get most significant bit in x.data1
 mov ecx, 0
 jnz @foundMostSignX // Jump if bit was found
                     // Should never reach here (x = 0)
 ud2                 // Undef. instruct. => Raises an error
@foundMostSignX:
 add ecx, ebx        // ecx:=Pos of first bit = i
 sub ecx, [esp]      // ecx:=i-j
// Calculate x'=x-2^(i-j)y
 mov eax, [ebp]      // edi:esi:edx:eax:=y
 mov edx, [ebp+4]
 mov esi, [ebp+8]
 mov edi, [ebp+12]
 call helperShl128   // Shift y left i-j steps
 cmp [esp+16], edi   // Check if y*2^(i-j) > x
 ja @doSubtract
 jb @useCarry
 cmp [esp+12], esi
 ja @doSubtract
 jb @useCarry
 cmp [esp+8], edx
 ja @doSubtract
 jb @useCarry
 cmp [esp+4], eax
 jae @doSubtract
@useCarry:           // y*2^(i-j) > x => use j'=j+1
 mov ebx, ecx
 mov ecx, 1
 call helperShr128   // Shift y right 1 step
 mov ecx, ebx
 dec ecx
@doSubtract:
 sub [esp+4], eax    // x.data1:=x.data1-(2^(i-j)y).data1
 sbb [esp+8], edx    // x.data2:=x.data2-(2^(i-j)y).data2
 sbb [esp+12], esi   // x.data3:=x.data3-(2^(i-j)y).data3
 sbb [esp+16], edi   // x.data4:=x.data4-(2^(i-j)y).data4
// Set bit i-j in result
 mov ebx, [esp+20]   // ebx:=pointer to res
 mov eax, 1          // eax:=1 (to be shifted)
 cmp cl, 32          // Compare cl - 32 to 0
 jb  @setBitBelow32
 cmp cl, 64          // Compare cl - 64 to 0
 jb  @setBitBelow64
 cmp cl, 96          // Compare cl - 96 to 0
 jb  @setBitBelow96
                     // Set bit in res.data4
 shl eax, cl         // eax:=1 shl (i-j)mod 32
 or [ebx+12], eax    // res.data4:=res.data4 or eax
 jmp @afterSetBit
@setBitBelow96:      // Set bit in res.data3
 shl eax, cl         // eax:=1 shl (i-j)mod 32
 or [ebx+8], eax     // res.data3:=res.data3 or eax
 jmp @afterSetBit
@setBitBelow64:      // Set bit in res.data2
 shl eax, cl         // eax:=1 shl (i-j)mod 32
 or [ebx+4], eax     // res.data2:=res.data2 or eax
 jmp @afterSetBit
@setBitBelow32:      // Set bit in res.data1
 shl eax, cl         // eax:=1 shl (i-j)mod 32
 or [ebx], eax       // res.data1:=res.data1 or eax
@afterSetBit:
 cmp [esp+4], 0      // Check if x.data1 = 0
 jnz @beforeRepeat1
 cmp [esp+8], 0      // Check if x.data2 = 0
 jnz @beforeRepeat1
 cmp [esp+12], 0     // Check if x.data3 = 0
 jnz @beforeRepeat1
 cmp [esp+16], 0     // Check if x.data4 = 0
 jnz @beforeRepeat1
 add esp, 24         // Throw i, x and pointer to res away
 pop ebp
 pop edi
 pop esi
 jmp @afterDiv
@denomLength32:
// Divide a 128 bitnumerator with a 32 bit numerator
 mov ecx, eax        // ecx:=pointer to x
 mov ebx, [edx]      // ebx:=y1
 mov eax, [ecx+12]   // eax:=x4
 xor edx, edx        // edx:=0
 div ebx             // edx:eax:=rem and quote of 0:x4/y1
 mov [ecx+12], eax    
 mov eax, [ecx+8]    // eax:=x3
 div ebx             // edx:eax:=rem and quote of rem:x3/y1
 mov [ecx+8], eax
 mov eax, [ecx+4]    // eax:=x2
 div ebx             // edx:eax:=rem and quote of rem:x2/y1
 mov [ecx+4], eax
 mov eax, [ecx]      // eax:=x1
 div ebx             // edx:eax:=rem and quote of rem:x1/y1
 mov [ecx], eax
@afterDiv:
 pop ebx
end;
// Now ReduceFraction128 begins!
// Local variables in ReduceFraction128:
var sign:boolean;
    a,b:int128;
begin
// If fracUp = 0 set fracDown to 0 and return
 if IsZero128(pInt128(frac)) then
  begin
   pWord(integer(frac)+16)^:=IMM_0;
   exit;
  end;

{$IFDEF DEBUG}
Assert(IsPos128(pInt128(integer(frac)+16)));
{$ENDIF}

// Set sign = x < 0 
 sign:=pInt128(frac)^.data4 < 0;
// Change fracUp to its absolute value
 Abs128(pInt128(frac));

asm
 push esi            
 push edi
 push ebx
// mov fracUp and fracDown to a and b
 mov eax, [frac]     // Load address of fracUp:fracDown
 mov ecx, [eax]      // edi:esi:edx:ecx:=fracUp
 mov edx, [eax+4]    //
 mov esi, [eax+8]    //
 mov edi, [eax+12]   // 
 mov a.data1, ecx    // a:=fracUp
 mov a.data2, edx
 mov a.data3, esi
 mov a.data4, edi
 mov ecx, [eax+16]   // edi:esi:edx:ecx:=fracDown
 mov edx, [eax+20]   // (edx will be kept for later use)
 mov esi, [eax+24]   // (esi will be kept for later use)
 mov edi, [eax+28]   // (edi will be kept for later use)
 mov b.data1, ecx    // b:=fracDown
 mov b.data2, edx
 mov b.data3, esi
 mov b.data4, edi
// Find least sig. bit in fracDown and store its pos in ecx
 xor ecx, ecx        // ecx:=0
 bsf eax, b.data1    // Get least significant bit in
                     //   fracDown.data1 (bsf doesnt pair)
 jnz @foundFirstBit1 // Jump if bit was found
 bsf eax, edx        // Get l.s. bit in fracDown.data2
 mov cl, 32          // ecx:=32
 jnz @foundFirstBit1 // Jump if bit was found
 bsf eax, esi        // Get l.s. bit in fracDown.data3
 mov cl, 64          // ecx:=64
 jnz @foundFirstBit1 // Jump if bit was found
 bsf eax, edi        // Get l.s. bit in fracDown.data4
 mov cl, 96          // ecx:=96
 jnz @foundFirstBit1 // Jump if bit was found
                     // Should never be here (fracDown = 0)
 ud2                 // Undef. instr. => Raises an error
@foundFirstBit1:
 add ecx, eax        // ecx:=ecx+eax is the global position 
// Find least signi. bit in fracUp and store its pos in ebx
 xor ebx, ebx        // ebx:=0
 bsf eax, a.data1    // Get l.s. bit in fracUp.data1
 jnz @foundFirstBit2 // Jump if bit was found
 bsf eax, a.data2    // Get l.s. bit in fracUp.data2
 mov bl, 32          // ebx:=32
 jnz @foundFirstBit2 // Jump if bit was found
 bsf eax, a.data3    // Get l.s. bit in fracUp.data3
 mov bl, 64          // ebx:=64
 jnz @foundFirstBit2 // Jump if bit was found
 bsf eax, a.data4    // Get l.s. bit in fracUp.data4
 mov bl, 96          // ebx:=96
 jnz @foundFirstBit2 // Jump if bit was found
                     // Should never be here (fracUp = 0)
 ud2                 // Undef. instr. => Raises an error
@foundFirstBit2:
 add ebx, eax        // ebx:=ebx+eax is the global position

// Get minimum number N of bits to shift
// N=ecx<-min(ecx,ebx), ebx<-max(ecx,ebx)-min(ecx,ebx)
 cmp ecx, ebx        // Compare ecx-ebx to 0
 setbe bh            // Use bit 8 in ebx to store which of
                     //   original ecx and ebx was the lar-
                     //   gest, bit is set if ecx <= ebx
                     //   (bh=bit 8..15 in ebx)
 jbe @afterSwap      // Jump if already ecx'=min(ecx,ebx)
                     //             and ebx'=max(ecx,ebx)
 mov al, cl          // eax:=ebx
 mov cl, bl          // ecx':=Min(ebx,ecx)
 mov bl ,al          // ebx':=Max(ebx,ecx)
@afterSwap:

// Divide the actual values of fracUp and fracDown with 2^N
//   by shifting N steps to the right
// Use helper function by putting edi:esi:edx:eax:=fracDown
//   edi,esi,edx already assigned from before
 mov eax, b.data1    // eax:=fracDown.data1
 call helperShr128   // Shift fracDown ecx steps right
 sub bl, cl          // ebx':=Max(ebx,ecx)-Min(ebx,ecx)
                     //   moved to avoid possible memory stall
 push ecx            // Store ecx
 mov ecx, [frac]     // Load address of fracUp:fracDown
 mov [ecx+16], eax   // Update shifted fracDown to memory
 mov [ecx+20], edx
 mov [ecx+24], esi
 mov [ecx+28], edi
 test bh, bh         // Check if it was fracDown (bh = 0)
                     //   that should be shifted extra
 jnz @afterExtraShift1 // Jump if not
 mov cl, bl          // Get number of shifts cl from bl
 call helperShr128   // Shift fracDown ecx steps right
@afterExtraShift1:
 pop ecx             // Restore ecx
 mov b.data1, eax    // Copy possibly shifted fracDown to b
 mov b.data2, edx
 mov b.data3, esi
 mov b.data4, edi
// Use helper function by putting edi:esi:edx:eax:=fracUp
 mov eax, a.data1    // Copy fracUp to edi:esi:edx:eax
 mov edx, a.data2
 mov esi, a.data3
 mov edi, a.data4
 call helperShr128   // Shift fracUp ecx steps (=>ecx free)
 mov ecx, [frac]     // Load address of fracUp:fracDown
 mov [ecx], eax      // Update shifted fracUp to memory
 mov [ecx+4], edx
 mov [ecx+8], esi
 mov [ecx+12], edi
 test bh, bh         // Check if it was fracUp (bh = 1)
 jz @afterExtraShift2//   that should be shifted extra
 mov cl, bl
 call helperShr128   // If so shift fracUp ecx steps right
@afterExtraShift2:

// edi:esi:edx:eax contains a, don't need to update memory
// Both a and b are now odd
// Repeat: a:=a-b and divide by 2 until odd if a>b
//         b:=b-a and divide by 2 until odd if b>a
// until a = b
@beforeWhile1:       // Use edi:esi:edx:eax = a on enter
 cmp edi, b.data4    // Compare a4-b4 to 0
 jnle @aGreaterWhile1 // Jump if a4>b4 => a>b (signed comp)
 jl @bGreaterWhile1  // Jump if a4<b4 => b>a (signed comp)
                     // Fallthrough if a4 = b4
 cmp esi, b.data3    // Compare a3-b3 to 0
 jnbe @aGreaterWhile1 // Jump if a3>b3 => a>b
 jb @bGreaterWhile1  // Jump if a3<b3 => b>a
                     // Fallthrough if a3 = b3
 cmp edx, b.data2    // Compare a2-b2 to 0
 jnbe @aGreaterWhile1 // Jump if a2>b2 => a>b
 jb @bGreaterWhile1  // Jump if a2<b2 => b>a
                     // Fallthrough if a2 = b2
 cmp eax, b.data1    // Compare a1-b1 to 0
 jz @afterWhile1     // Exit loop if a1 = b1 => a = b
 jb @bGreaterWhile1  // Jump if a1<b1 => b>a
                     // Fallthrough if a.data1 < b.data1
@aGreaterWhile1:
// a:=a-b, edi:esi:edx:eax = a
 sub eax, b.data1    // eax:=a1 - b1, sets flags
 sbb edx, b.data2    // edx:=a2 - b2 - Borrow, sets flags
 sbb esi, b.data3    // esi:=a3 - b3 - Borrow, sets flags
 sbb edi, b.data4    // edi:=a4 - b4 - Borrow
// Divide a with 2 until a becomes odd
@repeat1Start:
// a:=a shr 1;       // Run loop at least once
 shrd eax, edx, 1    // eax:=a2 shif. right one bit into a1
 shrd edx, esi, 1    // edx:=a3 shif. right one bit into a2
 shrd esi, edi, 1    // esi:=a4 shif. right one bit into a3
 shr edi, 1          // edi:=a4 shif. right one bit(no sgn)
 test al, 1          // Check if a even with bitwise and
 jz @repeat1Start    // Repeat if a even
 jmp @beforeWhile1   // Repeat big loop

@bGreaterWhile1:
// b:=b-a, edi:esi:edx:eax = a
 mov a.data1, eax    // Store a so we can put b in regs
 mov a.data2, edx
 mov a.data3, esi
 mov a.data4, edi
 mov ebx, eax        // ebx:=a1
 mov eax, b.data1    // eax:=b1
 sub eax, ebx        // eax:=b1 - a1, sets flags
 mov ebx, edx        // ebx:=a2, mov dowesn't affect flags
 mov edx, b.data2    // edx:=b2
 sbb edx, ebx        // edx:=b2 - a2 - Borrow, sets flags
 mov ebx, esi        // ebx:=a3
 mov esi, b.data3    // esi:=b3
 sbb esi, ebx        // edx:=b3 - a3 - Borrow, sets flags
 mov ebx, edi        // ebx:=a4
 mov edi, b.data4    // edi:=b4
 sbb edi, ebx        // edx:=b4 - a4 - Borrow
// Divide b with 2 until b becomes odd
@repeat2Start:
// b:=b shr 1;       // Run loop at least once
 shrd eax, edx, 1    // eax:=b2 shif. right one bit into b1
 shrd edx, esi, 1    // edx:=b3 shif. right one bit into b2
 shrd esi, edi, 1    // esi:=b4 shif. right one bit into b3
 shr edi, 1          // edi:=b4 shif. right one bit(no sgn)
 test al, 1          // Check if b even with bitwise and
 jz @repeat2Start    // Repeat if b even
 mov b.data1, eax    // Store b
 mov b.data2, edx
 mov b.data3, esi
 mov b.data4, edi
 mov eax, a.data1    // Load a for next big loop iteration
 mov edx, a.data2
 mov esi, a.data3
 mov edi, a.data4
 jmp @beforeWhile1   // Repeat big loop
@afterWhile1:
 mov a.data1, eax    // update a
 mov a.data2, edx
 mov a.data3, esi
 mov a.data4, edi

 pop ebx
 pop edi
 pop esi
end;
// Perform the division with pos numbers without remainder
 helperDivZeroRem_128_128(pInt128(frac),@a);
 helperDivZeroRem_128_128(pInt128(integer(frac)+16),@a);
// Dont forget to adjust the sign of the result
 if sign then Neg128(pInt128(frac));
end;
\end{verbatim}
}
\end{multicols}
\noindent
Although looking suspiciously long this actually runs faster than the same algorithm compiled in Delphi with just 64 bit integers.


% Lägger till alla referenser
\nocite{*}

%%%%%%%%%%%%%%%%%%%%%%%%%%%%%%%%%%%%%%%%%%%%
%
% Bibliography
%
%%%%%%%%%%%%%%%%%%%%%%%%%%%%%%%%%%%%%%%%%%%%
%\pagestyle{empty}
\cleardoublepage
\pagestyle{plain}
\def\href#1#2{#2}
\bibliographystyle{biblio/utphysmod2}
%\bibliographystyle{plain}
\addcontentsline{toc}{chapter}{\sffamily\bfseries Bibliography}
\bibliography{biblio/biblio}

\end{document}
